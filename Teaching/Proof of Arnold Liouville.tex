\documentclass{article}
\usepackage{../header}

\title{Proof of the Arnol'd-Liouville Theorem}
\author{Linden Disney-Hogg}

\begin{document}

\section*{Proof of the Arnol'd Liouville Theorem - Required Concepts}
\subsection*{Symplectic Manifold}
Let $M$ be a $2n$-dimensional manifold and $\omega^2$ a differential 2-form on $M$ that is closed and non-degenerate. Then the pair $(M,\omega^2)$ is called a \bam{symplectic manifold}. 


\subsection*{I}
On any symplectic manifold there is a natural isomorphism between the cotangent space and tangent space at a point $x\in M$, $I : T_x^\ast M \to T_x M $, given by for $\omega^1 \in T_x^\ast M$, $\forall v\in T_xM$
\[
\omega^1(v) = \omega^2(v,I\omega^1)
\]
It can be seen that $I$ is a linear map as $\omega^2$ is, and that injectivity follows from the non-degeneracy of $\omega^2$. By counting dimensions we then know $I$ is an isomorphism.


\subsection*{Phase Flow} 
To $F:M\to \mbb{R}$ we may associate the 1-parameter group of diffeomorphisms $\phi_F^t : M \to M$ defined by 
\[
\left. \frac{d}{dt} \right\rvert_{t=0} \phi_F^t x = (IdF)(x)
\]
Such a 1-parameter group is called the \bam{phase flow}.

\subsection*{Poisson Bracket}
Given $F,G : M \to \mbb{R}$ we may define the \bam{Poisson bracket} $\{ F, G \} : M \to \mbb{R}$ by \begin{align*}
    \{ F, G \}(x) &= \left. \frac{d}{dt} \right\rvert_{t=0} F(\phi_G^t x) \\
    &= dF(IdG(x)) \quad \text{by the chain rule} \\
    &= \omega^2(IdF,IdG)(x)
\end{align*}

\subsection*{Canonical Sympletic Form}
Given coordinates $(\bm{q},\bm{p})$ on a manifold, the \bam{canonical symplectic form} is 
\[
\omega^2 = d\bm{p} \wedge d\bm{q}
\]
 

\end{document}
