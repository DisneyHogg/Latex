\documentclass{article}

\usepackage{header}
%%%%%%%%%%%%%%%%%%%%%%%%%%%%%%%%%%%%%%%%%%%%%%%%%%%%%%%%
%Preamble

\title{Universal Configuration Space}
\author{Linden Disney-Hogg}
\date{December 2019}

%%%%%%%%%%%%%%%%%%%%%%%%%%%%%%%%%%%%%%%%%%%%%%%%%%%%%%%%
%%%%%%%%%%%%%%%%%%%%%%%%%%%%%%%%%%%%%%%%%%%%%%%%%%%%%%%%
\begin{document}

\maketitle
\tableofcontents

%%%%%%%%%%%%%%%%%%%%%%%%%%%%%%%%%%%%%%%%%%%%%%%%%%%%%%%%
%%%%%%%%%%%%%%%%%%%%%%%%%%%%%%%%%%%%%%%%%%%%%%%%%%%%%%%%
\section{Introduction}
These notes comes from the paper \cite{Krichever1996OnTheories} and are the first in a series of attempts to understand algebraic geometry. What will a universal configuration space be? It will be the moduli space of all algebraic curves with a fixed pair of abelian integrals. Before jumping in we should detail some results from \cite{Miranda1995AlgebraicSurfaces} (\hl{update this as we go}). 

%%%%%%%%%%%%%%%%%%%%%%%%%%%%%%%%%%%%%%%%%%%%%%%%%%%%%%%%
%%%%%%%%%%%%%%%%%%%%%%%%%%%%%%%%%%%%%%%%%%%%%%%%%%%%%%%%
\section{Let's get cracking}
Let's be precise; let $\Gamma$ be a Riemann surface, genus $g$, with $N$ punctures $P_\alpha$, and two abelian integrals $E,Q$ with with poles of order at most $n=(n_\alpha), m=(m_\alpha)$ respectively at the punctures. 

\begin{definition}
An $n_\alpha$-\bam{jet} $[z_\alpha]_{n_\alpha}$ is an equivalence class of coordinates near $P_\alpha$ with 
\eq{
z_\alpha^\prime \sim z_\alpha \text{ if } z_\alpha^\prime = z_\alpha + O(z_\alpha^{n_\alpha+1})
}
\end{definition}

\begin{fact}
The space of $n$-jets near a puncture $P$ has dimension $n$
\end{fact}

Choose a particular puncture $P_1$ to be a marked puncture. 

\begin{definition}
In presence of $[z]_n$ near $P_1$ an \bam{abelian integral} $Q$ is a pair $(dQ,c_Q)$ where $dQ$ is a meromorphic differential on $\Gamma$ and, given 
\eq{
dQ = d\pround{\sum_{k=-m}^\infty c_k z^k} + R^Q \frac{dz}{z} \, ,
}
we have 
\eq{
Q = \sum_{k=-m}^\infty c_k z^k + c_Q + R^Q \log z
}
\end{definition}

By integration along paths we can analytically extend to a neighbourhood of any point in $\Gamma\setminus\pbrace{P_1, \dots, P_N}$, giving the abelian integral $E$. This will be a path dependent definition, varying by homotopy class. If we then fix $n,m$ we can define the universal configuration space to be 
\eq{
\mc{M}_g(n,m) = \pbrace{\Gamma,P_\alpha,[z_\alpha]_{n_\alpha}; E,Q}
}
The abelian integrals have the expansion
\eq{
E &= z_1^{-n_1} + c_E + R_1^E \log z_1 + O(z_1) \\
dE &= d\pround{z_\alpha^{-n_\alpha} + O(z_\alpha)} + R_\alpha^E \frac{dz_\alpha}{z_\alpha} \\
Q &= \sum_{k=1}^{m_1} c_{1,k} z_1^{-k} + c_Q + R_1^Q \log z_1 + O(z_\alpha) \\
dQ &= d\pround{\sum_{k=1}^{m_\alpha} c_{\alpha,k} z_\alpha^{-k} + O(z_\alpha)} + R_\alpha^Q \frac{dz_\alpha}{z_\alpha}
}
$\mc{M}_g(n,m)$ is a complex manifold with only orbifold singularities (\hl{why is this the case}) with 
\eq{
\dim \mc{M}_g(n,m) &= 5g -3 + 3N + \sum_{\alpha=1}^N (n_\alpha + m_\alpha) \\
&= \psquare{3g-3+N} + \psquare{\sum_{\alpha=1}^N n_\alpha} + \psquare{N+g} + \psquare{N + g + \sum_{\alpha=1}^N m_\alpha} \\
&= \psquare{\Gamma_g\,, P_\alpha} + \psquare{\text{jets}} + \psquare{R_\alpha^N,c_E,g\text{ abelian diffs}} + \psquare{R_\alpha^Q,c_{\alpha,k}, g \text{ abelian diffs}}
}
(\text{understand why there are g abelian differentials which do not change the singularities})To take coordinates, start by taking 
\eq{
R_\alpha^E = \res_{P_\alpha} dE \\
R_\alpha^Q = \res_{P_\alpha} dQ 
}
for $\alpha = 2, \dots, N$. Now choose a homology basis $A_i,B_i$ with canonical intersection matrix 
\eq{
A_i \circ A_j =B_i \circ B_j &= 0\\
A_i \circ B_j &= \delta_{ij}
}
If we choose branch cuts from $P_1$ to $P_\alpha$ we get well defined branches of $E,Q$, and this can be done locally on $\mc{M}_g$. In the case $n_\alpha \geq 1$ there exists coordinates in the jet such that 
\eq{
E = z_\alpha^{-n_\alpha} + R_\alpha^E \log z_\alpha
}
around $P_\alpha$. In his neighbourhood we can then define 
\eq{
T_{\alpha,k} &= \frac{1}{k} \res_{P_\alpha} (z_\alpha^k Q dE) & &\alpha = 1, \dots, N, \, k=1, \dots, n_\alpha+ m_\alpha \\
T_{\alpha,0} &= \res_{P_\alpha}(QdE) & & \alpha= 2, \dots, N
}
Alternatively, if $n_\alpha = 0$, then for $R_\alpha^E \neq 0$ we can choose $z_\alpha$ s.t.
\eq{
E - E(P_\alpha) = R_\alpha^e \log z_\alpha
}
We finally get coordinates defined by 
\eq{
\tau_{A_i,E} &= \oint_{A_i} dE \\
\tau_{B_i,E} &= \oint_{B_i} dE \\
\tau_{A_i,Q} &= \oint_{A_i} dQ \\
\tau_{B_i,Q} &= \oint_{B_i} dQ \\
a_i &= \oint_{A_i} Q dE 
}
Letting $\mc{D}$ be the open nbhd where the zero divisors $\pbrace{\gamma \, | \, dE(\gamma) = 0}$, $\pbrace{\gamma \, | \, dQ(\gamma) = 0}$ do not intersect, we have the following: 

\begin{theorem}
Near each point in $\mc{D}$, the $5g -3 + 3N + \sum_{\alpha=1}^N (n_\alpha + m_\alpha)$ coordinates : $R_\alpha^E$, $R_\alpha^Q$, $T_{\alpha,k}$, $T_{\alpha,0}$ $\tau_{A_i,E}$, $ \tau_{B_i,E}$, $\tau_{A_i,Q}$, $\tau_{B_i,Q}$, $a_i$, have linearly independent differentials, and as such define a local holomorphic coordinate system for $\mc{M}_g(n,m)$
\end{theorem}
%%%%%%%%%%%%%%%%%%%%%%%%%%%%%%%%%%%%%%%%%%%%%%%%%%%%%%%%
%%%%%%%%%%%%%%%%%%%%%%%%%%%%%%%%%%%%%%%%%%%%%%%%%%%%%%%%
\bibliographystyle{plain}
\bibliography{references.bib}

\end{document}