\documentclass{article}

\usepackage{../../header-colourful}
%%%%%%%%%%%%%%%%%%%%%%%%%%%%%%%%%%%%%%%%%%%%%%%%%%%%%%%%
%Preamble

\title{Algebraic Topology: \\ All the results - None of the proofs }
\author{Linden Disney-Hogg}
\date{June 2020}

%%%%%%%%%%%%%%%%%%%%%%%%%%%%%%%%%%%%%%%%%%%%%%%%%%%%%%%%
%%%%%%%%%%%%%%%%%%%%%%%%%%%%%%%%%%%%%%%%%%%%%%%%%%%%%%%%
\begin{document}

\maketitle
\tableofcontents

%%%%%%%%%%%%%%%%%%%%%%%%%%%%%%%%%%%%%%%%%%%%%%%%%%%%%%%%
%%%%%%%%%%%%%%%%%%%%%%%%%%%%%%%%%%%%%%%%%%%%%%%%%%%%%%%%
\section{Introduction}
In every mathematical physicist's life there comes a point when they need to know some algebraic topology, and there are many great resources to learn this from. These will be my personal notes which will accumulate many resources, which I will try to reference, though I doubt I will give when each one was used. The current list is 
\begin{itemize}
	\item \textit{Diferential Forms in Algebraic Topology} (Bott, Tu)
	\item nlab
	\item \textit{Algebraic Topology} and \textit{Vector Bundles and K-Theory} (Hatcher) 
\end{itemize} 
I hope to come back some day and fill in all the proofs, but in the name of current expedience I will avoid this. 
%%%%%%%%%%%%%%%%%%%%%%%%%%%%%%%%%%%%%%%%%%%%%%%%%%%%%%%%
%%%%%%%%%%%%%%%%%%%%%%%%%%%%%%%%%%%%%%%%%%%%%%%%%%%%%%%%
\section{Preliminaries}
This section will contain small, relatively self-contained bits of knowledge which will be useful to know throughout. 
%%%%%%%%%%%%%%%%%%%%%%%%%%%%%%%%%%%%%%%%%%%%%%%%%%%%%%%%
\subsection{General Topology}

\begin{definition}
	A map of topological spaces $f:X \to Y$ is \bam{proper} if $\forall K \subset Y$ compact, $f^{-1}(K)\subset X$ is compact. 
\end{definition}

\begin{prop}
	The image of a proper map in a locally-compact Hausdorff space is closed
\end{prop}

\begin{prop}
	A compact subspace of a Hausdorff space is closed. 
\end{prop}

%%%%%%%%%%%%%%%%%%%%%%%%%%%%%%%%%%%%%%%%%%%%%%%%%%%%%%%%
\subsection{Differential Topology}

\begin{definition}
	A \bam{critical point} of a smooth map of manifolds $f:M\to N$ is $p \in M$ s.t. $(f_\ast)_p:T_pM \to T_{f(p)}N$ is not surjective. A \bam{critical value} is the image of a critical point
\end{definition}

\begin{theorem}[Sard]
	The set of critical values of a smooth map has measure 0.
\end{theorem}

\begin{definition}
An \bam{open cover} of a manifold is a collection of open sets $U_\alpha \subset M$ s.t. $M = \bigcup_\alpha U_\alpha$
\end{definition}

\begin{definition}
	A \bam{good cover} of an $n$-dimensional manifold is an open cover where all finite intersections $U_{\alpha_0} \cap \dots \cap U_{\alpha_p}$ are diffeomorphic to $\mbb{R}^n$. A manifold with a good cover is said to be of \bam{finite type}. 
\end{definition}

\begin{theorem}
	Every manifold is of finite type, and moreover if it is compact, the cover can be chosen to be finite. 
\end{theorem}
\begin{proof}
	Use a cover provided by taking geodesic balls at each point. The second point follows from the definition of compact.
\end{proof}

\begin{definition}
	A \bam{partition of unity} on a manifold $M$ is a collection of non-negative $C^\infty$ functions $\pbrace{\rho_\alpha}$ s.t. 
	\begin{itemize}
		\item Each $p \in M$ has a neighbourhood where $\sum \rho_\alpha$ is a finite sum
		\item $\sum \rho_\alpha$
	\end{itemize}
\end{definition}

\begin{definition}
	Given a manfiold with open cover $\pbrace{U_\alpha}$, a partition of unity $\pbrace{\rho_\alpha}$ s.t $\supp(\rho_\alpha) \subset U_\alpha$ is called \bam{subordinate} to $\pbrace{U_\alpha}$. 
\end{definition}

\begin{prop}
	Given a manfiold with open cover $\pbrace{U_\alpha}$:
	\begin{itemize}
		\item $\exists$ a partition of unity subordinate to it.
		\item $\exists$ a partition of unity $\pbrace{\rho_\beta}$ s.t. each $\rho_\beta$ has compact support and $\exists \alpha \, s.t. \, \supp(\rho_\beta) \subset U_\alpha$ 
	\end{itemize}
\end{prop}

\begin{prop}
	Every manifold is paracompact.
\end{prop}

%%%%%%%%%%%%%%%%%%%%%%%%%%%%%%%%%%%%%%%%%%%%%%%%%%%%%%%%
\subsection{Category Theory}
Here we will cover the basic category theory required to be able to provide a slightly general formalism to many of the concepts in topology. 

\begin{definition}
	A \bam{Category} $\mc{C}$ is a collection of objects $\text{Obj}\mc{C}$ s.t $\forall A,B \in  \text{Obj}\mc{C}$ (for simplicity we will often write $A,B \in \mc{C}$) there is a set of morphisms $\Hom(A,B)$ under the conditions
	\begin{itemize}
		\item $\forall f \in \Hom(A,B), \, g \in \Hom(B,C), \, \exists g \circ f \in \Hom(A,C)$
		\item the composition $\circ$ is associative with identity $1_A \in \Hom(A,A)$
	\end{itemize}
\end{definition}

\begin{example}
	There is a category whose objects are Euclidean spaces $\pbrace{\mbb{R}^n}$, and whose morphisms are smooth maps. It is denoted $\text{CartSp}_{\text{smooth}}$
\end{example}

\begin{example}
	The category of differential commutative-graded algebras (DGCAs) with homomorphisms for morphisms exists. Recall a graded algebra $A$ has a map $\deg : A \to \mbb{Z}$, and to be graded-commutative means 
	\eq{
\forall a,b \in A, \, ab = (-1)^{\deg(a) \deg(b)}ba	
}
The diferenial means we have a map $d:A \to A$ s.t. 
\eq{
d(ab) = (da)b + (-1)^{\deg(a)}a(db)
}
and $d$ is compatible with the grading.
\end{example}

\begin{definition}
	A \bam{covariant functor} is a map between categories $F: \mc{C} \to \mc{D}$ in the sense that for $A \in \mc{C}, \, F(A) \in \mc{D}$, and $f \in \Hom(A,B)$ for $A,B \in \mc{C}$ gives $F(f) \in \Hom(F(A), F(B))$ obeying 
	\begin{itemize}
		\item $F(g \circ f) = F(g) \circ ~F(f)$ 
		\item $F(1_A) = 1_{F(a)}$
	\end{itemize}
A \bam{contravariant functor} is a map $F:\mc{C} \to \mc{D}$ s.t. for $f \in \Hom(A,B), \, F(f) \in \Hom(F(B),F(A))$
\end{definition}

\begin{definition}
	Given objects $X,Y,Z$ and two morphisms $f:X\to Z, \, g:Y \to Z$, the \bam{pullback} of the morphisms is an object $P$ with morphisms $p_1:P \to X, \, p_2 :P \to Y$ s.t. the diagram
\begin{tkz}
	P \arrow[r,"p_2"] \arrow[d,"p_1"'] & Y \arrow[d,"g"] \\
	X \arrow[r,"f"'] & Z
\end{tkz} 
commutes, and moreover that the pullback is universal wrt to this diagram i.e. 
\begin{tkz}
Q \arrow[drr,"q_2", bend left=15] \arrow[ddr,"q_1"', bend right=15] \arrow[dr, dashed, "\exists ! u"]& & \\
	& P \arrow[r,"p_2"] \arrow[d,"p_1"'] & Y \arrow[d,"g"] \\
& X \arrow[r,"f"'] & Z	
\end{tkz}
The pullback is often denoted as $P=X \times_Z Y$
\end{definition}

\begin{lemma}[Five Lemma]
	Given a commutative diagram of abelian groups 
\begin{tkz}
\dots \arrow[r] & A \arrow[r] \arrow[d,"\alpha"] & B \arrow[r] \arrow[d,"\beta"] & C \arrow[r] \arrow[d,"\gamma"] & D \arrow[r] \arrow[d,"\delta"] & E \arrow[r] \arrow[d,"\eps"] & \dots \\
\dots \arrow[r] & A^\prime \arrow[r] & B^\prime \arrow[r] & C^\prime \arrow[r] & D^\prime \arrow[r] & E^\prime \arrow[r] & \dots
\end{tkz}
where the rows are exact, if $\alpha, \beta, \delta, \eps$ are isomorphisms, then so is $\gamma$. 
\end{lemma}

%%%%%%%%%%%%%%%%%%%%%%%%%%%%%%%%%%%%%%%%%%%%%%%%%%%%%%%%
\subsection{Orientation and Integration}

We may use partitions of unity to define the integral of a top form over a manifold $M$. 

\begin{theorem}[Stokes' Threom]
If $\omega$ is an $(n-1)$-form with compact support on an oriented $n$-dimensional manifold $M$
	\eq{
\int_M d\omega = \int_{\del M}\omega	
}
\end{theorem}

%%%%%%%%%%%%%%%%%%%%%%%%%%%%%%%%%%%%%%%%%%%%%%%%%%%%%%%%
\subsection{Bundle Theory}
\begin{definition}
	Given manifolds $M,N$, fibre bundle $E \to M$, and map $f:N \to M$, we define the \bam{pullback bundle of E by f} to be 
	\eq{
f^{\ast}E \equiv \pbrace{(n,e) \, | \, f(n) = \pi(e)} \subset N \times E	
}
the bundle with base $N$, with the natural projection onto the first component $p_1$. 
\end{definition}

\begin{lemma}
The pullback bundle is the unique maximal subset of $N \times E$ s.t. 
\begin{tkz}
	f^{\ast} E \arrow[r,"p_2"] \arrow[d,"p_1"'] & E \arrow[d,"\pi"] \\
	N \arrow[r,"f"'] & M
\end{tkz} 
commutes.
\end{lemma}

\begin{ex}
	Look at how this relates to the categorical concept of a pullback. 
\end{ex}

\begin{example}
	The pullback by the identity map is isomorphic to the bundle itself, i.e. 
	\eq{
	\id^{\ast} E &\leftrightarrow E \\
	(\pi(e), e) &\leftrightarrow e
}
\end{example}

\begin{lemma}
Given $g:M^{\prime\prime}\to M^{\prime}, \, f:M^\prime \to M$, $(f \circ g)^{\ast}E = g^{\ast}\pround{f^{\ast}E}$.
\end{lemma}

\begin{lemma}
	The pullback of a trivial bundle is trivial. i.e. If $E =F \times M$, then for $f:N \to M$, $f^{\ast}E = N \times F$.
\end{lemma}

\begin{remark}
	If we let $\Vect_k(M)$ be the isomorphism classes of rank-$k$ real vector bundles, and $\Vect_k(f) = f^{\ast}$ be the pullback of vector bundles along $f$, then we get a contravariant functor from manifolds with smooth maps to pointed sets with basepoint preserving maps, where the basepoint of $\Vect_k(M)$ is the trivial bundle over $M$.
\end{remark}

\begin{prop}
	If $f,g : M \to N$ are homotopic then $f^{\ast}E$ and $g^{\ast}E$ are isomorphic.
\end{prop}
\begin{remark}
	This result holds true more generally for a  paracompact topological space $M$. As all manifolds are paracompact, this holds in our case.
\end{remark}
\begin{corollary}
	A bundle with contractible base is trivial.
\end{corollary}
\begin{proof}
	Suppose we have 
	\begin{tkz}
		M \arrow[r,"f",shift left] & \ast \arrow[l,"g",shift left]
	\end{tkz}
s.t. $g \circ f$ is homotopic to $\id_M$. Then 
\eq{
E \cong (g\circ f)^{\ast}E  = f^{\ast}\pround{g^{\ast}E}
}
As $g^{\ast}E$ is a bundle over a point it is necessarily trivial, and so $f^{\ast}(g^{\ast}E)$ is also.
\end{proof}

\begin{example}
	We want to work out $\Vect_k(S^1)$. Intuition might tell us that $Vect_1(S^1) = \mbb{Z}$ (number of twists of a mobius like band, where 0 is the trivial bundle) so we have a starting point. This will not turn out to be the case \\
	To specify a isomorphism class of rank $k$-vector bundles, we can take the cover of $S^1$ with only two open sets and then an element of $\Vect_k(S^1)$ is specified by two elements $g,h \in GL(k,\mbb{R})$ up to conjugation. We can use a reduction of structure bundle to only ask about elements of $O(k)$. 
\end{example}

%%%%%%%%%%%%%%%%%%%%%%%%%%%%%%%%%%%%%%%%%%%%%%%%%%%%%%%%
%%%%%%%%%%%%%%%%%%%%%%%%%%%%%%%%%%%%%%%%%%%%%%%%%%%%%%%%
\section{de-Rham Theory}
%%%%%%%%%%%%%%%%%%%%%%%%%%%%%%%%%%%%%%%%%%%%%%%%%%%%%%%%
\subsection{General Cohomology}
We will start with some basic definitions and results, with very few proofs for now. 

\begin{definition}
	A direct sum of vector spaces $C = \oplus_{k \in \mbb{Z}} C^k$ is called a \bam{differential complex} if there are homomorphisms
	\eq{
		\dots \to C^{k-1} \overset{d}{\to} C^k \overset{d}{\to} C^{k+1} \to \dots	
	}
	s.t. $d^2=0$
\end{definition}

\begin{definition}
	The \bam{cohomology} of $C$ is $H(C) = \oplus_{k \in \mbb{Z}} H^k(C)$ where 
	\eq{
		H^k(X) = \faktor{\pround{\ker d \cap C^k}}{\pround{\image d \cap C^k}}	
	}
\end{definition}

\begin{definition}
	A (linear) map $f:A \to B$ between two differential complexes is called a \bam{chain map} if it commutes with the differential operator, i.e. $f \circ d_A = d_B \circ f$. 	
\end{definition}

\begin{prop}
	A short exact sequence of chain maps 
	\eq{
		0 \to A \overset{f}{\to} B \overset{g}{\to} C \to 0	
	}
	induces a long exact sequence of cohomology 
	\begin{center}
		\begin{tikzcd}
			\dots \arrow[r] & H^k(A) \arrow[r,"f^\ast"]
			& H^k(B) \arrow[r,"g^\ast"]
			\arrow[d, phantom, ""{coordinate, name=Z}]
			& H^k(C) \arrow[dll,
			"d^\ast",
			rounded corners,
			to path={ -- ([xshift=2ex]\tikztostart.east)
				|- (Z) [near end]\tikztonodes
				-| ([xshift=-2ex]\tikztotarget.west)
				-- (\tikztotarget)}] \\
			& H^{k+1}(A) \arrow[r]
			& \dots 
			& 
		\end{tikzcd}
	\end{center}
\end{prop}
\begin{proof}
	Consider the diagram obtained 
	\begin{center}
		\begin{tkz}
			\phantom{}& \vdots & \vdots & \vdots &\phantom{} \\
			0 \arrow[r] & A^{k+1} \arrow[r,"f"] \arrow[u] & B^{k+1} \arrow[r,"g"] \arrow[u] & C^{k+1} \arrow[r] \arrow[u] & 0 \\
			0 \arrow[r] & A^{k} \arrow[r,"f"] \arrow[u,"d_A"] & B^{k} \arrow[r,"g"] \arrow[u,"d_B"] & C^{k} \arrow[r] \arrow[u,"d_C"] & 0 \\
			\phantom{}& \vdots \arrow[u] & \vdots \arrow[u] & \vdots \arrow[u] &\phantom{}
		\end{tkz}
	\end{center}
	$f$ induces a well defined map on the cohomology $f^\ast$ as $f(a+d_A\omega) = f(a) + (f \circ d_A)(\omega) = f(a) + (d_B \circ f)(\omega)$, so $[f(a)] = [f(a+d_A\omega)]$ in $H^\bullet(B)$.	Likewise for $g$. \\
	Pick $c \in C^k$, then by surjectivity of $g$, $\exists b \in B^k, \, g(b)=c$. If $dc=0$, we can say $g(db) = dg(b) = dc=0$, so by exactness $db = f(a)$ for some $a \in A^{k+1}$. As such we define $d^\ast : H^k(C) \to H^{k+1}(A)$ by $d^\ast[c] = [a]$. 
\end{proof}

%%%%%%%%%%%%%%%%%%%%%%%%%%%%%%%%%%%%%%%%%%%%%%%%%%%%%%%%
\subsection{The de-Rham complex}
\hl{I have seen a lot of de-Rham definitions in the past, so I will come back and fill this in when I have time.}


\begin{definition}
	The \bam{de-Rham complex} $\Omega^\bullet_{dR}$ is a differential complex where $C^k = \Omega^k_{dR}$ are the $k$-forms and $d$ is the exterior derivative. 
\end{definition}


\begin{remark}
		The de-Rham complex for Euclidean spaces admits a functorial description as $\Omega^\bullet$ is the unique contravariant functor from Euclidean spaces with smooth maps to commutative differential graded algebras s.t. $\Omega^0$ is the pullback of functions. The fact that the de-Rham complex admits this functorial description tells us that pullback commutes with the exterior derivative. This definition can be extended to the category of differentiable manifolds. 
\end{remark}

\begin{remark}
	As there a no $k$ forms on a manifold $M$ when $k > \dim M$, $H^k_{dR}(M)=0$.
\end{remark}

\begin{example}
	The cohomology of the de-Rham complex is sometimes denoted as $H^k_{dR}$. When the context makes clear that we are consider the de-Rham cohomology we omit the dR. \\
	Consider the point space $\ast$. Functions on $\ast$ are specified by points in $\mbb{R}$, and are all constant, so closed. There can be no cohomology higher than the dimension of the space, so we get 
	\eq{
H^k_{dR}(\ast) = \left\lbrace\begin{array}{cc}
	\mbb{R} & k=0 \\ 0 & k > 0 
\end{array}	\right.
}
	Closed $0$-forms on $\mbb{R}$ are again constant functions. Further, any one form $\omega = g(x)dx$ can be written as $df$ for $f(x) = \int_0^x g(u) du$ so we get 
	\eq{
	H^k_{dR}(\mbb{R}) = \left\lbrace\begin{array}{cc}
		\mbb{R} & k=0 \\ 0 & k > 0 
	\end{array}	\right.
}
	If $U\subset \mbb{R}$ is a union of $m$ dijoint open intervals in $\mbb{R}$ we have 
	\eq{
	H^k_{dR}(U) = \left\lbrace\begin{array}{cc}
		\mbb{R}^m & k=0 \\ 0 & k > 0 
	\end{array}	\right.
}
\end{example}

\begin{definition}
The \bam{de-Rham complex with compact support} is the differential complex of the k-forms with compact support, denoted $\Omega_c^\bullet(M)$. The cohomology is denotes $H_c^\bullet(M)$.
\end{definition}

\begin{prop}
	If $M$ is compact, $H_c^k(M) = H_{dR}^k(M)$.
\end{prop}

\begin{example}
 We can consider the same cohomologies as above to get 
	\eq{
H^k_c(\ast)&= \left\lbrace\begin{array}{cc}
	\mbb{R} & k=0 \\ 0 & k > 0 
\end{array}	\right. \\
}
as all functions on $\ast$ are constant. \\
There are no constant functions on $\mbb{R}$ with compact support except for the zero map, so $H^0_c(\mbb{R})=0$. The only other non-trivial cohomolgy is $H^1_c(\mbb{R})$. Our previous construction (starting the integral at $-\infty$ which we can do as $g$ has compact support) of an $f$ s.t. $df = g(x)dx$ works iff $f$ gets compact support, and this happens where the integral $\int_{\mbb{R}} g(x) \, dx = 0$ so 
\eq{
H_c^1(\mbb{R}) = \faktor{\Omega^1_c(\mbb{R})}{\ker\smallint_\mbb{R}}
}
\end{example}

\begin{remark}
$\Omega_c^\bullet$ also admits a functorial description, but we must restrict from all smooth maps as pullbacks of functions with compact support might not have compact support. $\Omega_c^\bullet$ can be made either as 
\begin{itemize}
	\item a contravariant functor when maps are restricted to be proper
	\item a covariant functor when maps are restricted to be inclusions of open sets.
\end{itemize}	
\end{remark}


%%%%%%%%%%%%%%%%%%%%%%%%%%%%%%%%%%%%%%%%%%%%%%%%%%%%%%%%
\subsection{The Poincar\'e lemma}
We will now build up a bit of theory culminating in a full understanding of de-Rham cohomology of $\mbb{R}^n$. We start by considering the maps 
\begin{tkz}
\mbb{R}^n \times \mbb{R} \arrow[r,shift left, "\pi"] & \mbb{R}^n \arrow[l,shift left, "s"]	
\end{tkz}
given by $\pi(x,t) = x, \, s(x) = (x,0)$. Under the functor $\Omega^\bullet$ we get 
\begin{tkz}
	\Omega^\bullet(\mbb{R}^n \times \mbb{R}) \arrow[r,shift right, "s^\ast"'] & \Omega^\bullet(\mbb{R}^n) \arrow[l,shift right, "\pi^\ast"']	
\end{tkz}

\begin{prop}
	The induced maps on cohomology
\begin{tkz}
	H_{dR}^k(\mbb{R}^n \times \mbb{R}) \arrow[r,shift right, "s^\ast"'] & H_{dR}^k(\mbb{R}^n) \arrow[l,shift right, "\pi^\ast"']	
\end{tkz}	
	 are inverse isomorphisms. 
\end{prop}
\begin{proof}
	Certainly $\pi \circ s = \id_{\mbb{R}^n} \Rightarrow s^\ast \circ \pi^\ast = \id_{\Omega^\bullet(\mbb{R}^n)} \Rightarrow s^\ast \circ \pi^\ast = \id_{H_{dR}^\bullet(\mbb{R}^n)}$. It remains to show $\pi^\ast \circ s^\ast = \id$ in cohomoloy. Note that it is not the identity on the complex, as for example $(\pi^\ast \circ s^\ast)f(x,t) = f(x,0)$. However it is sufficient to show $\exists K:\Omega^k(\mbb{R}^n \times \mbb{R}) \to \Omega^{k-1}(\mbb{R}^n \times \mbb{R})$
	\eq{
1 - \pi^\ast \circ s^\ast = \pm(d\circ K \pm K\circ d)
} 
As the RHS maps closed forms to exact ones, it induces $0$ on the cohomology. \hl{finish constructing K}. 
\end{proof}

\begin{corollary}[Poincar\'e lemma]
	We have 
	\eq{
		H_{dR}^k(\mbb{R}^n) = \left\lbrace\begin{array}{cc}
			\mbb{R} & k=0 \\ 0 & k > 0 
		\end{array}	\right.
	}	
\end{corollary}

\begin{corollary}
	Applying the proposition to an atlas for a manifold $M$ we get 
	\eq{
H_{dR}^k(M \times \mbb{R}) \cong H^k(M)	\Rightarrow H^k(M \times \mbb{R}^n) \cong H^k(M)
}

\end{corollary}

\begin{corollary}
	Homotopic maps induce the same cohomology map. 
\end{corollary}
\begin{comment}
\begin{proof}
	$f,g : M \to N$ are homotopic if $\exists F:M \times I \to N$ restricting to $f,g$ at either end of the interval. Extend to a map $M \times \mbb{R}$, construct $f,g$ from $F$ using sections and pullback. 
\end{proof}
\end{comment}

\begin{corollary}
	The de-Rham cohomology is homotopy invariant. 
\end{corollary}

A similar result holds for compact de-Rham cohomology, namely

\begin{lemma}
	$H^{k+1}_c(M \times \mbb{R}) \cong H^k_c(M) \Rightarrow H^{k+l}(M \times \mbb{R}^l) \cong H^k(M)$
\end{lemma}

\begin{corollary}[Poincar\'e lemma for compact support]
	We have 
	\eq{
		H_{c}^k(\mbb{R}^n) = \left\lbrace\begin{array}{cc}
			\mbb{R} & k=0,n \\ 0 & \text{otherwise} 
		\end{array}	\right.
	}	
\end{corollary}


%%%%%%%%%%%%%%%%%%%%%%%%%%%%%%%%%%%%%%%%%%%%%%%%%%%%%%%%
\subsection{Mayer-Vietoris}

Write $M = U \cup V$ where $M$ is a manifold and $U,V \subset M$ are open. We then get the inclusions 
\begin{tkz}
	M & U\coprod V \arrow[l,hook] & U \cap V \arrow[l,shift left, hook, "i_U"] \arrow[l, shift right, hook, "i_V"']
\end{tkz}
Applying the functor $\Omega^\bullet$ to this yields
\begin{tkz}
	\Omega^\bullet(M) \arrow[r] & \Omega^\bullet(U) \oplus \Omega^\bullet(V) \arrow[r,shift left, "i_U^\ast"] \arrow[r, shift right, "i_V^\ast"'] & \Omega^\bullet(U \cap V) 
\end{tkz}

\begin{definition}
	The \bam{Mayer-Vietoris sequence} is the that obtained using the difference of the above two maps, that is 
	\eq{
0 \to \Omega^\bullet(M) \to \Omega^\bullet(U) \oplus \Omega^\bullet(V) &\to \Omega^\bullet(U \cap V) \to 0	\\
(\omega,\tau) &\mapsto \tau - \omega
} 
\end{definition}

\begin{prop}
The Mayer-Vietoris sequence is exact	
\end{prop}
\begin{proof}
	\hl{do this eventually, requires partitions of unity}
\end{proof}

\begin{prop}
	The Mayer-Vietoris sequence induces a long exact sequence of cohomology
		\begin{center}
		\begin{tikzcd}
			\dots \arrow[r] & H^k(M) \arrow[r]
			& H^k(U) \oplus H^k(V) \arrow[r]
			\arrow[d, phantom, ""{coordinate, name=Z}]
			& H^k(U \cap V) \arrow[dll,
			"d^\ast",
			rounded corners,
			to path={ -- ([xshift=2ex]\tikztostart.east)
				|- (Z) [near end]\tikztonodes
				-| ([xshift=-2ex]\tikztotarget.west)
				-- (\tikztotarget)}] \\
			& H^{k+1}(M) \arrow[r]
			& \dots 
			& 
		\end{tikzcd}
	\end{center}
\end{prop}

Let us now see some uses of the machinery we have just developed:

\begin{example}
Consider a circle $S^1$ and cover it with two open sets (north and south, slightly overlapping). The part of the sequence we care about is 
			\begin{center}
		\begin{tikzcd}
			0 \arrow[r] & H^0(S^1) \arrow[r]
			& \mbb{R} \oplus \mbb{R} \arrow[r]
			\arrow[d, phantom, ""{coordinate, name=Z}]
			& \mbb{R} \oplus \mbb{R} \arrow[dll,
			"d^\ast",
			rounded corners,
			to path={ -- ([xshift=2ex]\tikztostart.east)
				|- (Z) [near end]\tikztonodes
				-| ([xshift=-2ex]\tikztotarget.west)
				-- (\tikztotarget)}] \\
			& H^{1}(S^1) \arrow[r]
			& 0 \arrow[r]  
			& 0
		\end{tikzcd}
	\end{center}
Calling the map $\mbb{R} \oplus \mbb{R} \to \mbb{R} \oplus \mbb{R}$ $\delta$, we can see under $\delta, \, (\omega,\tau) \mapsto (\omega-\tau,\omega-\tau)$. Hence $\dim \image \delta = 1 \Rightarrow \dim \ker \delta = 1$. Counting dimensions of the maps we see 
\eq{
H^0(S^1) \cong \mbb{R} \cong H^1(S^1)
}
We can extend this to calculate for any sphere. Covering $S^n$ with $U,V$ the north/south hemisphere respectively extended so they cover the equator, we get $U\cap V$ is homotopic to $S^{n-1}$, and $U,V$ are contractible. Hence in the Mayer-Vietoris we get a sequence
			\begin{center}
	\begin{tikzcd}
		\dots \arrow[r] & H^k(S^n) \arrow[r]
		& (\delta_{k0})(\mbb{R} \oplus \mbb{R}) \arrow[r]
		\arrow[d, phantom, ""{coordinate, name=Z}]
		& H^{k}(S^{n-1}) \arrow[dll,
		"d^\ast",
		rounded corners,
		to path={ -- ([xshift=2ex]\tikztostart.east)
			|- (Z) [near end]\tikztonodes
			-| ([xshift=-2ex]\tikztotarget.west)
			-- (\tikztotarget)}] \\
		& H^{k+1}(S^n) \arrow[r]
		& \dots  
		& 
	\end{tikzcd}
\end{center}
This means that we have for $k>0, \, H^k(S^{n-1})\cong H^{k+1}(S^{n})$ and
			\begin{center}
	\begin{tikzcd}
		0 \arrow[r] & H^0(S^n) \arrow[r]
		& \mbb{R} \oplus \mbb{R} \arrow[r]
		\arrow[d, phantom, ""{coordinate, name=Z}]
		& H^{0}(S^{n-1}) \arrow[dll,
		"d^\ast",
		rounded corners,
		to path={ -- ([xshift=2ex]\tikztostart.east)
			|- (Z) [near end]\tikztonodes
			-| ([xshift=-2ex]\tikztotarget.west)
			-- (\tikztotarget)}] \\
		& H^{1}(S^n) \arrow[r]
		& 0  
		& 
	\end{tikzcd}
\end{center}
We can then prove by induction that for $n \geq 1$ $H^k(S^n) = \mbb{R}$ if $k=0,n$ and 0 otherwise. 
\end{example}

\begin{prop}
	If a manifold has a finite good cover, then its cohomology is finite dimensional.
\end{prop}
\begin{proof}
We will use proof by induction on the cardinality of the good cover, noting that if $M$ is diffeomorphic to $\mbb{R}^n$ then we have a cover given by $M$, and the result is true by the Poincar\' lemma. \\ 
We now note from the part of the Mayer-Vietoris sequence for $U\cup V$ 
	\eq{
\dots \to H^{k-1}(U \cap V) \overset{d^\ast}{\to} H^k(U \cup V)	\overset{r}{\to} H^k(U) \oplus H^k(V) \to \dots
}
that 
\eq{
H^k(U \cup V) \cong \ker r \oplus \image r \cong \image d^\ast \oplus \image r
}
(this is just the first isomorphisms theorem and exactness). Hence, if $H^k(U), H^k(V), $ and $H^k(U \cap V)$ are f.d. then so is $H^k(U \cup V)$. \\
Now suppose $M$ has good cover $\pbrace{U_0, \dots, U_p}$. Then $(U_0 \cup \dots \cup U_{p-1}) \cap U_p$ has a good cover 
\eq{
\pbrace{U_0 \cap U_p, \dots, U_{p-1} \cap U_p}
}
By the induction hypothesis $(U_0 \cup \dots \cup U_{p-1}) \cap U_p$ has f.d. cohomology, and so does $M$ from the Mayer-Vietoris (taking $U = U_0 \cup \dots \cup U_{p-1}, \, V = U_p$).
\end{proof}

With this results, we can define a related quantity

\begin{definition}
	On an $n$-dimensional manifold with f.d. cohomology, the \bam{Euler characterstic} of $M$ is 
	\eq{
\chi(M) = \sum_{k=0}^n (-1)^k \dim	H^k_{dR}(M)
} 
\end{definition}

We can also build a Mayer-Vietoris sequence for the functor $\Omega_c^\bullet$ taken to be covariant when restricted to inclusions. The image of the inclusion $j:U \hookrightarrow M$ under the functor is $j_\ast : \Omega_c^\bullet(U) \to \Omega_c^\bullet(M)$ which extends a form by 0. This gives the sequence 
\begin{tkz}
	\Omega_c^\bullet(M) & \arrow[l,"\text{sum}"] \Omega_c^\bullet(U) \oplus \Omega_c^\bullet(V) & \arrow[l,"-j_\ast \oplus j_\ast"] \Omega_c^\bullet(U \cap V) 
\end{tkz}

\begin{prop}
	The Mayer-Vietoris sequence with compact support
	\begin{tkz}
		0 & \arrow[l] \Omega_c^\bullet(M) & \arrow[l] \Omega_c^\bullet(U) \oplus \Omega_c^\bullet(V) & \arrow[l] \Omega_c^\bullet(U \cap V) & \arrow[l] 0
	\end{tkz}
	 is exact	
\end{prop}

\begin{remark}
	This is in the opposite direction to our other Mayer-Vietories sequence for standard de-Rham cohomology. The difference is from the functor being covariant. 
\end{remark}

\begin{prop}
	The Mayer-Vietoris sequence with compact support induces a long exact sequence of cohomology
	\begin{center}
		\begin{tikzcd}
			\dots \arrow[r] & H_c^k(U \cap V) \arrow[r]
			& H^k_c(U) \oplus H^k_c(V) \arrow[r]
			\arrow[d, phantom, ""{coordinate, name=Z}]
			& H^k_c(M) \arrow[dll,
			"d^\ast",
			rounded corners,
			to path={ -- ([xshift=2ex]\tikztostart.east)
				|- (Z) [near end]\tikztonodes
				-| ([xshift=-2ex]\tikztotarget.west)
				-- (\tikztotarget)}] \\
			& H^{k+1}_c(U \cap V) \arrow[r]
			& \dots 
			& 
		\end{tikzcd}
	\end{center}
\end{prop}

\begin{example}
	We can use this Mayer-Vietoris sequence to calculte $H_c^\bullet(S^1)$, which we can check against $H^\bullet_{dR}(S^1)$ as they must be the same. \\
	Using the same cover as before we get the same part of the sequence 
	\eq{
0 \to H_c^0(S^1) \to \mbb{R} \oplus \mbb{R} \to \mbb{R} \oplus \mbb{R} \to H_c^1(S^1) \to 0	
}
and again the image and kernel in $\mbb{R} \oplus \mbb{R}$ are 1 dimensional. 
\end{example}

Many of our other results have compact support analogues, e.g. 
\begin{prop}
	If a manifold has a finite good cover, then its compact-support cohomology is finite dimensional. 
\end{prop}

We can say more about the relation between cohomology using that integration descends to cohomology, giving on oriented $n$-dimensional manifolds $M$ a pairing
\eq{
\int : H^k(M) \otimes H^{n-k}_c(M) &\to \mbb{R} 
}
given by $(\omega, \tau) \mapsto \int_M \omega \wedge \tau$
\begin{lemma}
	The two Mayer-Vietoris sequences may be paired to gether to form the diagram 
	\begin{tkz}
	 \arrow[r] & H^k(U \cup V) \arrow[r] & H^k(U) \oplus H^k(V) \arrow[r] & H^k(U \cap V) \arrow[r] & H^{k+1}(U \cup V) \arrow[r] & \phantom{\dots} \\
		& \otimes & \otimes & \otimes & \otimes & \\
		\phantom{\dots} & \arrow[l] H_c^{n-k}(U \cup V) \arrow[d,"\int_{U\cup V}"] & \arrow[l] H_c^{n-k}(U) \arrow[d,"\int_U + \int_V"] \oplus H^{n-k}_c(V)  & \arrow[l] H_c^{n-k}(U \cap V) \arrow[d,"\int_{U\cap V}"] & \arrow[l] H_c^{n-k-1}(U \cup V) \arrow[d,"\int_{U\cup V}"] &  \arrow[l] \\
		& \mbb{R} & \mbb{R} & \mbb{R} & \mbb{R}& 
	\end{tkz}
sign-commutative in the sense that 
\eq{
\int_{U \cap V} \omega \wedge d_\ast \tau = \pm \int_{U \cup V} (d^\ast \omega) \wedge \tau
}
\end{lemma}

\begin{remark}
	The above lemma is equivalent to saying we get the sign-commutative diagram
\begin{tkz}
	\arrow[r] & H^k(U \cup V) \arrow[r] \arrow[d] & H^k(U) \oplus H^k(V) \arrow[r] \arrow[d] & H^k(U \cap V) \arrow[r] \arrow[d] & \phantom{\dots} \\
	\arrow[r] & \pround{H_c^{n-k}(U \cup V)}^\ast \arrow[r] & \pround{H_c^{n-k}(U)}^\ast \oplus \pround{H_c^{n-k}( V)}^\ast \arrow[r] & \pround{H_c^{n-k}(U \cap V)}^\ast \arrow[r] & \phantom{\dots}
\end{tkz}
\end{remark}

\begin{prop}[Poincar\'e duality]
If $M$ is an $n$-dimensional orientable manifold and has a finite good cover, 
\eq{
H^k(M) \cong \pround{H^{n-k}_c(M)}^\ast
}	
\end{prop}
\begin{proof}
	Again proceed by induction on the size of the good cover, noting it is true for $\mbb{R}^n$. The above lemma used with the five lemma gives that if Poincar\'e duality holds for $U,V,U\cap V$ then it holds for $U \cup V$. 
\end{proof}

\begin{remark}
	This result can be extended to any orientable manofolds 
\end{remark}

\begin{corollary}
The Euler characteristic any odd-dimensional, compact, orientable manifold is 0
\end{corollary}
\begin{proof}
If $M$ is compact orientable $n$-dimensional, then 
\eq{
\dim H^k(M) &= \dim H_c^{n-k}(M) = \dim H^{n-k}(M)
}
so if $n$ odd
\eq{
\chi(M) &= \sum_{k=0}^n (-1)^k \dim H^k(M) \\
&= \sum_{k=0}^{\frac{n-1}{2}} (-1)^k \dim H^k(M) + \sum_{k=\frac{n+1}{2}}^n (-1)^k \dim H^{n-k}(M) \\
&= \sum_{k=0}^{\frac{n-1}{2}} (-1)^k \dim H^k(M) + \sum_{k=\frac{n-1}{2}}^{0} (-1)^{n-k} \dim H^{k}(M) \\
&= \psquare{1+(-1)^n}\sum_{k=0}^{\frac{n-1}{2}} (-1)^k \dim H^k(M)
}
As $n$ odd, $1+(-1)^n=0$.
\end{proof}
%%%%%%%%%%%%%%%%%%%%%%%%%%%%%%%%%%%%%%%%%%%%%%%%%%%%%%%%
\subsection{The Poincar\'e Dual and Thom Isomorphism. }
The concept can be extended to the idea of a Poincar\'e dual:

\begin{definition}
	Given $M$ an $n$-dimensional oriented manifold, and $i:S \hookrightarrow M$ a closed $k$-dimensional oriented submanifold, the \bam{(closed) Poincar\'e dual} to $S$ is $[\eta_S]\in H^{n-k}(M)$ given by 
	\eq{
\int_S i^\ast \omega = \int_M \omega \wedge \eta_S	
} 
for any $\omega \in H_c^k(M)$. 
\end{definition}

Let us unpack this definition: Given such an $\omega$, $\supp(\ev{\omega}{S}) = \supp(\omega) \cap S$ is closed and compact, and as pullback an $d$ commute we know $\int_S i^\ast \omega$ indeed exists and is well defined. Then by Poincar'e duality the map $H_c^k(M) \to \mbb{R}, \ ,\omega \mapsto \int_S i^\ast \omega$ (which is a linear functional) corresponds to a unique element of $H^{n-k}(M)$. \\
We can define the related notion 
\begin{definition}
	Given $M$ an $n$-dimensional oriented manifold, and $i:S \hookrightarrow M$ a compact $k$-dimensional oriented submanifold, the \bam{(compact) Poincar\'e dual} to $S$ is $[\eta_S^\prime]\in H^{n-k}_c(M)$ given by 
\eq{
	\int_S i^\ast \omega = \int_M \omega \wedge \eta_S^\prime	
} 
for any $\omega \in H^k(M)$. 	
\end{definition}

\begin{remark}
	As manifolds are Hausdorff, any compact submanifold is also closed. Hence a compact submanifold has both an associated closed Poincar\'e dual and an associated compact Poincar\'e dual. These are in general different and so need specifying. 
\end{remark}

\begin{example}
	Consider $M= \mbb{R}^n$ with compact submanifold $P=\ast$. Note $H^n(\mbb{R}^n)=0$ so $[\eta_P]=[0]$. Contrastingly, $[\eta_P^\prime]$ must generate $H^n_c(\mbb{R}^n)\cong \mbb{R}$, and as closed $0$-forms are constant functions, all we require is that $\int_{\mbb{R}^n} \eta_P^\prime=1$, which can be achieved by a normalised bump function. 
\end{example}
%%%%%%%%%%%%%%%%%%%%%%%%%%%%%%%%%%%%%%%%%%%%%%%%%%%%%%%%
\subsection{Cohomology of Bundles}

\begin{prop}
	Let $M$ be a manifold and $\pbrace{U_\alpha}$ a collection of open subsets. Then 
	\eq{
		H_{dR}^k\pround{\coprod_\alpha U_\alpha} &= \prod_\alpha H_{dR}^k(U_\alpha)	\\
		H_c^k \pround{\coprod_\alpha U_\alpha} &= \bigoplus_\alpha H_c^k(U_\alpha)
	}
\end{prop}

\begin{prop}[K\"unneth Formula]
	$H^k(M \times F) = \oplus_{p+q=k} H^p(M) \otimes H^q(F)$
\end{prop}

With this result we can start to develop more specialised results for cohomology on bundles. For important definitions on bundles look at my EKC of Gauge theory notes. \\
The K\"unneth formula has a specialisation for fibre bundles. 

\begin{theorem}[Leray-Hirsch]
Let $E\to M$ be a fibre bundle with fibre $F$. If there are global cohomology classes $e_1, \dots, e_r$ on $E$ which, restricted to each fibre, freely generate the cohomology of $F$, then $H^k(M)$ is a free module over $H^k(M)$ i.e 
\eq{
H^k(E) \cong H^k(M) \otimes \mbb{R}\pbrace{e_1, \dots, e_r} \cong H^k(M) \otimes H^k(F)
} 	
\end{theorem}

\begin{prop}
	If $E\to M$ is a vector bundle, then $	H^k(E) \cong H^k(M)$.
\end{prop}
\begin{proof}
	Deformation retract onto the zero section of the bundle and then use homotopy invariance of cohomolgy
\end{proof}

\begin{remark}
	\hl{how does this agree with Leray Hirsch?}
\end{remark}

\begin{prop}
	Let $E \to M$ be a rank-$k$ vector bundle, where $E,M$ are orientable and of finite type. Then $H_c^p(E) \cong H^{p-k}_c(M)$. 
\end{prop}
\begin{proof}
	Let $\dim M = n$. Then 
	\eq{
H_c^p(E) &\cong \pround{H^{n+k-p}(E)}^\ast \quad \text{(Poincar\'e duality)} \\
&\cong \pround{H^{n+k-p}(M)}^\ast \quad \text{(homotopy invariance)} \\
&\cong H_c^{p-n}(M) \quad \text{(Poincar\'e duality)}
}
\end{proof}

In vector bundles there is an additional type of de-Rham cohomology we can look at 

\begin{definition}
	The \bam{de-Rham complex with compact vertical support} is the differential complex $\Omega_{cv}^\bullet$. of $k$-forms with compact support in the fibres. The associated cohomology is notated as $H^\bullet_{cv}$
\end{definition}

\begin{theorem}[Thom Isomorphism]
	If $E \to M$ is an orientable rank-$k$ vector bundle with base manifold of finite type then 
	\eq{
	H^{p+k}_{cv}(E) \cong H^{p}(M)
}
\end{theorem}

\begin{remark}
	The theorem is actually true for arbitrary manifolds. 
\end{remark}

\begin{definition}
	The image of the constant function $1 \in H^0(M)$ under the Thom isomorphism $\mc{T} : H^p(M) \overset{\cong}{\to} H^{p+k}_{cv}(E)$ is called the \bam{Thom class} of the oriented vector bundle $E$
\end{definition}
%%%%%%%%%%%%%%%%%%%%%%%%%%%%%%%%%%%%%%%%%%%%%%%%%%%%%%%%
%%%%%%%%%%%%%%%%%%%%%%%%%%%%%%%%%%%%%%%%%%%%%%%%%%%%%%%%
\section{Chern Classes}

We start by recalling a definition:

\begin{definition}
	A \bam{complex line bundle} is a complex vector bundle of rank 1.
\end{definition}

\begin{remark}
	Analogously to how real vector bundles have reduction of structure group $GL(r,\mbb{R}) \to O(r)$, complex vector bundles have reduction $GL(r,\mbb{C})\to U(\mbb{C})$
\end{remark}

\begin{lemma}
	There is a bijection between complex line bundles  and oriented rank-2 real vector bundles.
\end{lemma}
\begin{proof}
	Every rank-$r$ $\mbb{C}$-vector space $E$ corresponds to a rank-$2r$ $\mbb{R}$-vector space $E_{\mbb{R}}$  by forgetting the complex structure. Then as $U(1) \cong SO(2)$ each complex line bundle in the case $r=1$ this is a bijection if we give an orientation to the real bundle, which picks out $SO(2) \subset O(2)$.
\end{proof}

\begin{definition}
	The \bam{first Chern class} of a complex line bundle $L$ with base $M$ is the Euler class of $L_{\mbb{R}}$, that is 
	\eq{
c_1(L) = e(L_\mbb{R}) \in H^2(M)	
} 
\end{definition}

\end{document}