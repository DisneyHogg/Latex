\documentclass{article}

\usepackage{../../header}
%%%%%%%%%%%%%%%%%%%%%%%%%%%%%%%%%%%%%%%%%%%%%%%%%%%%%%%%
%Preamble

\title{Linearising Flows and a Cohomological Interpretation of Lax Equations - Unpacking the Paper}
\author{Linden Disney-Hogg}
\date{November 2020}

%%%%%%%%%%%%%%%%%%%%%%%%%%%%%%%%%%%%%%%%%%%%%%%%%%%%%%%%
%%%%%%%%%%%%%%%%%%%%%%%%%%%%%%%%%%%%%%%%%%%%%%%%%%%%%%%%
\begin{document}

\maketitle
\tableofcontents

%%%%%%%%%%%%%%%%%%%%%%%%%%%%%%%%%%%%%%%%%%%%%%%%%%%%%%%%
%%%%%%%%%%%%%%%%%%%%%%%%%%%%%%%%%%%%%%%%%%%%%%%%%%%%%%%%
\section{Introduction}
The purpose of this document is to facilitate the understanding of \cite{Griffiths1985} by discussing the terms and how they fit into the wider picture of algebraic geometry. 
%%%%%%%%%%%%%%%%%%%%%%%%%%%%%%%%%%%%%%%%%%%%%%%%%%%%%%%%
%%%%%%%%%%%%%%%%%%%%%%%%%%%%%%%%%%%%%%%%%%%%%%%%%%%%%%%%
\section{The Paper}

%%%%%%%%%%%%%%%%%%%%%%%%%%%%%%%%%%%%%%%%%%%%%%%%%%%%%%%%
\subsection{Laying out the Ingredients}
\begin{notation}
	We start by laying out some notation that will be necessary for the following section. Let:
	\begin{itemize}
		\item $P=\mbb{CP}^1$ with coordinates $[\xi_0:\xi_1]$. We take $\xi = \frac{\xi_1}{\xi_0}$. 
		\item $O_P$ be the natural structure sheaf on the variety $P$
		\item $V$ be a $m$-dimensional vector space, $\mc{V} = V \otimes O_P$, $\mc{V}(k) = V \otimes \mc{O}_P(k)$ where we view $V$ as either the constant sheaf or trivial bundle over $P$.
		\item $A(t,\xi) = \sum_{k=0}^n A_k(t) \xi^k \in  H^0(P,\Hom(\mc{V},\mc{V}(n)))$ for some $n$, where we see $A_i(t) \in \End(V)$ as a time dependent $m\times m$ matrix and $\xi^k \in H^0(P,\mc{O}(n))$ as
		\eq{
	[\xi_0:\xi_1]^k = \underbrace{\xi_0 \otimes \dots \otimes \xi_0}_{\times (n-k)} \otimes \underbrace{\xi_1 \otimes \dots \otimes \xi_1}_{\times k}	
	}
This is homogeneous of degree $n$, so we allow $A$ to not have a scale\hl{?}
        \item $B(\xi,t) \in  H^0(P,\Hom(\mc{V},\mc{V}(N)))$ for some $N$ likewise. 
        \item $Q(\xi,\eta) = \det\psquare{\eta I - A(\xi,t)}$ be the characteristic polynomial of $A$.
        \item $\sigma$ be the tautological section of $\mc{O}_P(n)$.  
	\end{itemize}
\end{notation}

\begin{lemma}
	$Q(\xi,\sigma) \in H^0(\mc{O}_P(n), \pi^\ast \mc{O}_P(mn))$
\end{lemma}

\begin{definition}
	The pair $A,B$ is a Lax pair if $\dot{A} = \comm[A]{B}$. 
\end{definition}

\begin{prop}
	The Lax equation is invariant under the substitution 
	\eq{
B \mapsto B + p(A,\xi)	
}
for polynomial $p(x,\xi) \in \mbb{C}[x,\xi]$. 
\end{prop}

\begin{definition}
	The \bam{spectral curve} is $C$ given by the solution in $P$ of 
	\eq{
		Q(\xi,\eta) =0	
	} 
\end{definition}

\begin{prop}
	The flow $t \mapsto A(\xi,t)$ is isospectral.
\end{prop}

It will be the understanding of this isospectral flow that we want to gain. We formulate this flow as the family of holomorphic map gained by the eigenvectors 
\eq{
	f_t : C \to \mathbb{P}V \cong \mbb{P}^{m-1}
}
Suppose that $C$ has degree $d$, then we know we can define 
\eq{
L_t  = f_t^\ast \mc{O}_{\mathbb{P}V}(1) \in \Pic^d(C)
}
Lets choose a reference bundle $L_0 =L\in \Pic^d(X)$ 
\begin{lemma}
	The map $\Pic^d(C) \overset{\otimes L^{-1}}{\to} J(C) $ is an isomorphism. 
\end{lemma}
Now knowing our result about the tangent space to the Picard group we can say $\dot{L} = \frac{dL_t}{dt} \in H^1(C,O_C)$. 

%%%%%%%%%%%%%%%%%%%%%%%%%%%%%%%%%%%%%%%%%%%%%%%%%%%%%%%%
\subsection{The Eigenvector Mapping as a Deformation}
Recall we have a 1-parameter family of maps 
\[
f_t : C \to \mathbb{P}V \, .
\]
We want to interpret this a as a deformation of the map 
\[
f_0 : C \to \mathbb{P}V
\]
and characterise it as such, so we need to develop a little theory. \\
Given a map 
\[
f : X \to Y
\]
we think of a deformation as a 1-parameter family of maps 
\[
f_t : X_t \to Y
\]
with $X_0=X, \, f_0 = f$. This gives a point in the tangent space to the moduli space of such arrangements, which by deformation theory is $H^0(X,\mathcal{N}_{X/Y})$ where $\mathcal{N}$ is the normal bundle given by the SES 
\[
0 \to \mathcal{T}_X \to f^\ast \mathcal{T}_Y \to \mathcal{N} \to 0 \, .
\]
This short exact sequence gives rise to the segment of an LES 
\[
H^0(X,\mathcal{T}_X) \to H^0(X,f^\ast \mathcal{T}_Y) \to H^0(X,\mathcal{N}) \overset{\tilde{\delta}}{\to} H^1(X,\mathcal{T}_X) \, .
\]
Again, from deformation theory, we know that $H^1(X,\mathcal{T}_X)$ is the tangent space to the moduli space of $X$, so if we wanted to look at deformations that kept $X$ fixed, we would need the kernel of $\tilde{\delta}$. Hence the eigenvector mapping gives a cohomology class
\[
\dot{f} \in \faktor{H^0(C,f^\ast \mathcal{T}_{\mathbb{P}V})}{H^0(C,\mathcal{T}_C)} \subset H^0(C,\mathcal{N}) \, . 
\]

%%%%%%%%%%%%%%%%%%%%%%%%%%%%%%%%%%%%%%%%%%%%%%%%%%%%%%%%
\subsection{Combining with the Euler Sequence}

Recall we also have the sequence 
\[
0 \to O_{\mathbb{P}V} \to \mathcal{V} \otimes \mathcal{O}_{\mathbb{P}V}(1) \to \mathcal{T}_{\mathbb{P}V} \to 0
\]
which pulls back under $f$ to give 
\[
0 \to O_C \overset{\nu}{\to} \mathcal{V} \otimes L \to f^\ast \mathcal{T}_{\mathbb{P}V} \to 0 \, ,
\]
where
\begin{align*}
    \nu : O_C &\to \mathcal{V} \otimes L \, ,\\
    \phi &\mapsto \phi \nu \, ,
\end{align*} 
and $\nu$ is the vector defined s.t. for $z \in C, \, f_t(z) = \mathbb{C}\nu(z,t)$.
\begin{definition}
We define $\dot{\nu}$ by 
\[
\dot{\nu}(z) = \ev{\frac{\partial \nu(z,t)}{\partial t}}{t=0} \mod \nu 
\]
\end{definition}
\begin{lemma}
	$\dot{\nu}$ is well defined.
\end{lemma}
\begin{proof}
	Suppose we had chosen a different representative $\tilde{\nu}$. Writing $\tilde{\nu} = \rho \nu$ we see 
	\[
	\dot{\tilde{\nu}} = \rho \dot{\nu} + \dot{\rho} \nu = \rho \dot{\nu} \mod \nu \, .
	\]
\end{proof}
Combined with the normal sheaf sequence this gives the cohomology diagram
\begin{center}
	\begin{tikzcd}
		& H^0(\mathcal{V} \otimes L) \arrow[d,"\tau"] & & \\
		H^0(C,\mathcal{T}_C) \arrow[r] & H^0(C,f^\ast \mathcal{T}_{\mathbb{P}V}) \arrow[r,"j"] \arrow[d,"\delta"] & H^0(C,\mathcal{N}) \arrow[r,"\tilde{\delta}"] & H^1(C,\mathcal{T}_C) \\
		& H^1(C,O_C) & &  
	\end{tikzcd}
\end{center}
and we can interpret $\dot{\nu}$ as a cohomology class 
\[
\dot{\nu} \in {H^0(C,\faktor{\mathcal{V} \otimes L}{O_C})} = H^0(C,f^\ast \mathcal{T}_{\mathbb{P}V})
\]
\begin{prop}
	We have 
	\begin{itemize}
		\item $j(\dot{\nu}) = \dot{f}$,
		\item $\delta(\dot{\nu}) = \dot{L}$. 
	\end{itemize}
\end{prop}
\begin{corollary}
	$\dot{L}=0 \Leftrightarrow \exists w \in H^0(C,\mathcal{V} \otimes L), \, \dot{\nu} = \tau(w)$. 
\end{corollary}

%%%%%%%%%%%%%%%%%%%%%%%%%%%%%%%%%%%%%%%%%%%%%%%%%%%%%%%%
\subsection{Divisor at Infinity}


%%%%%%%%%%%%%%%%%%%%%%%%%%%%%%%%%%%%%%%%%%%%%%%%%%%%%%%%
%%%%%%%%%%%%%%%%%%%%%%%%%%%%%%%%%%%%%%%%%%%%%%%%%%%%%%%%
\bibliographystyle{../../bib/custom-bib-style}
\bibliography{../../bib/jabref_library.bib}

\end{document}
