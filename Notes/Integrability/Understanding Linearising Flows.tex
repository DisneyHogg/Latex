\documentclass{article}

\usepackage{../../header}
%%%%%%%%%%%%%%%%%%%%%%%%%%%%%%%%%%%%%%%%%%%%%%%%%%%%%%%%
%Preamble

\title{Linearising Flows and a Cohomological Interpretation of Lax Equations - Unpacking the Paper}
\author{Linden Disney-Hogg}
\date{November 2020}

%%%%%%%%%%%%%%%%%%%%%%%%%%%%%%%%%%%%%%%%%%%%%%%%%%%%%%%%
%%%%%%%%%%%%%%%%%%%%%%%%%%%%%%%%%%%%%%%%%%%%%%%%%%%%%%%%
\begin{document}

\maketitle
\tableofcontents

%%%%%%%%%%%%%%%%%%%%%%%%%%%%%%%%%%%%%%%%%%%%%%%%%%%%%%%%
%%%%%%%%%%%%%%%%%%%%%%%%%%%%%%%%%%%%%%%%%%%%%%%%%%%%%%%%
\section{Introduction}
The purpose of this document is to facilitate the understanding of \cite{Griffiths1985} by discussing the terms and how they fit into the wider picture of algebraic geometry. 
%%%%%%%%%%%%%%%%%%%%%%%%%%%%%%%%%%%%%%%%%%%%%%%%%%%%%%%%
%%%%%%%%%%%%%%%%%%%%%%%%%%%%%%%%%%%%%%%%%%%%%%%%%%%%%%%%
\section{The Preliminaries}

%%%%%%%%%%%%%%%%%%%%%%%%%%%%%%%%%%%%%%%%%%%%%%%%%%%%%%%%
\subsection{Divisors}
\begin{definition}
	A \bam{(Weil) divisor} on $C$ is a formal finite sum of points, i.e. $D = \sum_i n_i p_i$ for $n_i \in \mbb{Z}, \, p_i \in C$. The group of divisors under addition is denoted $\Div(C)$. 
\end{definition}

\begin{definition}
	The \bam{degree} of a divisor $D = \sum_i n_i p_i$ $\deg D = \sum_i n_i $
\end{definition}

\begin{definition}
	Given a meromorphic function $f:C \to \mbb{C}$ we define $(f) \in \Div(C)$ by 
	\eq{
(f) = \sum_{p \in X} \ord_p (f) \cdot p 
}
For $D \in \Div(C)$, if $\exists f$ s.t. $D=(f)$ we say $D$ is a \bam{principal divisor}. 
\end{definition}

\begin{lemma}$(fg) = (f)+(g)$
\end{lemma}
\begin{corollary}
	Principal divisors form a subgroup  $\Lin(C) \leq \Div(C)$. 
\end{corollary}

\begin{lemma}
	If $X$ is a compact Riemann surface and $f:X \to \mbb{C}$ meromorphic then $\deg(f) = 0$. 
\end{lemma}

\begin{prop}
	Let $C$ be compact. Then $\Lin(C) = \pbrace{D \in \Div(C) \, | \deg(D) = 0}$. 
\end{prop}

\begin{definition}
	The \bam{divisor class group} of $C$ is $\Cl(C) = \faktor{\Div(C)}{\Lin(C)}$.
\end{definition}

\begin{remark}
	$\deg: \Div(C) \to \mbb{Z}$ is a group homomorphism and as the kernel is $\Lin(C)$ we see $\Cl(C) \cong \image \deg$
\end{remark}
\begin{corollary}
	$\Cl(\mbb{CP}^n) \cong \mbb{Z}$. 
\end{corollary}

\begin{definition}
	Two divisors $D,E$ are \bam{linearly equivalent} if $D-E$ is a principal.
\end{definition}

\begin{lemma}
	Linear equivalence of divisors is an equivalence relation. 
\end{lemma}

\begin{lemma}
	$f:X \to Y$ induces a group morphism $f:\Div(X) \to \Div(Y)$ by 
	\eq{
	f \pround{\sum_i n_i p_i} = \sum_i n_i f(p_i)
}
\end{lemma}

\begin{prop}
	If $f:X \to Y$ is a map of Riemann surfaces and $D \in \Div(X)$, then $\deg(f(D)) = \deg f \cdot \deg D$.
\end{prop}

\begin{definition}
	A divisor $D = \sum_i n_i p_i$ is \bam{effective} if each $n_i \geq0$.  
\end{definition}

\begin{prop}
	We have a partial ordering on $\Div(C)$ by saying $D\geq D^\prime$ if $D-D^\prime$ is effective. 
\end{prop}

\begin{definition}
	A Weil divisor on $C$ defines a \hl{coherent} sheaf $O_C(D)$ as meromorphic functions $f$ s.t $(f)+D\geq0$. 
\end{definition}

%%%%%%%%%%%%%%%%%%%%%%%%%%%%%%%%%%%%%%%%%%%%%%%%%%%
\subsection{Abel-Jacobi}
Suppose $C$ has genus $g$, then we know that $H_1(C,\mbb{Z}) \cong \mbb{Z}^{2g}$ where the generators are the loops $\pbrace{\gamma_i}_{i=1}^{2g}$. There is an alternative way to say this condition:
\begin{definition}
The \bam{canonical bundle} on a space $X$ with $\dim X = n$ is the line bundle of exterior $n$-forms on $X$.  
\end{definition}
\begin{remark}
Note we know the canonical bundle is a line bundle as there is only 1 basis element of $n$-forms on an $n$-dimensional space. 
\end{remark}
\begin{prop}
If $X=C$ is a Riemann surface of genus $g$ then $H^0(C,K)\cong \mbb{C}^g$. 
\end{prop}
\begin{proof}
\hl{Find this}
\end{proof}
\begin{corollary}
We can take a basis $\pbrace{\omega_i}_{i=1}^g$ of 1-forms on $C$. 
\end{corollary}
\begin{definition}
The \bam{Jacobian} of $C$ is defined to be 
\eq{
	J(C) = \faktor{\mbb{C}^g}{\Lambda}	
}
where $\Lambda$ is the lattice generated over $\mbb{R}$ by the vectors 
\eq{
	\Omega_j = \pround{\int_{\gamma_j} \omega_1, \dots, \int_{\gamma_j} \omega_g }, \quad 1 \leq j \leq 2g 
}
\end{definition}

\begin{definition}
The \bam{Abel-Jacobi map} for $p_0 \in C$ is 
\eq{
	u : C &\to J(C) \\
	p &\mapsto 	\pround{\int_{p_0}^p \omega_1, \dots, \int_{p_0}^p \omega_g } \mod \Lambda 
}
This is independent of the path of integration as we have quotiented by $\Lambda$. 
\end{definition}

\begin{theorem}[Abel's Theorem]
	Let $u$ be the Abel-Jacobi map and $D,E$ effective divisors. Then $u(D) = u(E) \Leftrightarrow D \sim E$. 
\end{theorem}

\begin{theorem}[Jacobi's Theorem]
	The map Abel-Jacobi map is surjective. 
\end{theorem}

\begin{corollary}
	There is an isomorphism from the space of degree-0 divisors to the Jacobian. 
\end{corollary}

%%%%%%%%%%%%%%%%%%%%%%%%%%%%%%%%%%%%%%%%%%%%%%%%%%%%%%%%
\subsection{Bundles and Sheaves}
We recall a few necessary bundle definitions and results:
\begin{definition}
	The tensor product of vector bundles $E,F \to M$ is $E\otimes F \to M$ s.t. $(E \otimes F)_m = E_m \otimes F_m$ for $m \in M$. 
\end{definition}

\begin{lemma}
	If $O$ is the trivial line bundle then $E\otimes O = E$. 
\end{lemma}

\begin{definition}
	The \bam{dual bundle} of a vector bundle $E \to M$ is $E^\ast \to M$ where the fibres of $E^\ast$ are the dual spaces of the fibres of $E$, with the transition functions $g_{ij}^\ast = \pround{g_{ij}^T}^{-1}$.
\end{definition}

\begin{remark}
	We can check the cocycle condition here as 
	\eq{
g_{kj}^\ast g_{ji}^ \ast = \pround{g_{kj}^T}^{-1} \pround{g_{ji}^T}^{-1}= \pround{g_{ji}^T g_{kj}^T}^{-1} = \pround{\psquare{g_{kj}g_{ji}}^T}^{-1}=\pround{g_{ki}^T}^{-1} =g_{ki}^\ast	
}
\end{remark}

\begin{example}
	The dual bundle to the tangent bundle is the cotangent bundle, i.e. $\pround{TM}^\ast = T^\ast M$
\end{example}

\begin{lemma}
	$E \otimes E^\ast \cong \End(E)$. 
\end{lemma}

\begin{lemma}
	Line bundles have tensor inverses, i.e given $L$, $\exists L^{-1}$ s.t. $L \otimes L^{-1} \cong O$ the trivial bundle. 
\end{lemma}
\begin{proof}
	We will show this by showing $L^{-1} = L^\ast$. To trivialise $\End(L)$ we note here the transition maps are $g_{ij} \otimes g_{ij}^{-1} =  1 \otimes 1$ as $g_{ij}, g_{ij}^\ast \in \mbb{F}$. Hence any section is globally defined. 
\end{proof}

\begin{remark}
	\hl{Why is the identity section not global on any other vector bundle.}
\end{remark}

These results motivate the definition of the \bam{Picard group} which we will cover now:

\begin{definition}
	A \bam{ringed space} is a pair $(X,O_X)$ where $X$ is a topological space and $O_X$ is a sheaf of rings on $X$. $O_X$ is called the \bam{structure sheaf}. 
\end{definition}

\begin{example}
	Given a topological space $X$, if we take $O_X$ to be $\mbb{R}$-valued continuous functions on open subsets of $X$ then $(X,O_X)$ is a ringed space. 
\end{example}

\begin{example}
	An example that will be relevant for later discussions is that an affine variety $X$ with sheaf $O_X$ given by $O_X(U)$ being the regular functions on $U$, regular functions being those given locally by polynomials. 
\end{example}

\begin{definition}
	The \bam{Picard group} of a locally ringed space $X$ is $\Pic(X)$ the group of isomorphism classes of line bundles on $X$ with the group operation being $\otimes$. 
\end{definition}

\begin{remark}
	In place of line bundles we can actually say \bam{invertible sheaves} 
\end{remark}

\begin{theorem}
	$\Cl(C) \cong \Pic(C)$ naturally. 
\end{theorem}
\begin{corollary}
	We get a group homomorphism $\deg:\Pic(C) \to \mbb{Z}$ giving the degree of the corresponding divisor in $\Cl(C)$. 
\end{corollary}
\begin{corollary}
	$\Pic(\mbb{CP}^1) \cong \mbb{Z}$. 
\end{corollary}

\begin{notation}
    We denote the isomorphism class of line bundles degree $d$ as $\Pic^d(C)$
\end{notation}

\begin{remark}
	With this new notation we may rephrase the corollary of the Abel-Jacobi theorem to say $J(C) \cong \Pic^0(C)$. 
\end{remark}

\begin{prop}
	There is a canonical isomorphism $\Pic(X) \cong H^1(X, O_X^\times)$. 
\end{prop}
\begin{corollary}
	$T_L(\Pic^d(X)) \cong H^1(X,O_X)$
\end{corollary}
\begin{proof}
	You need to use the \hl{exponential sheaf sequence}. 
\end{proof}

Let us now consider a specific class of bundles: 

\begin{definition}
	The \bam{hyperplane bundle on $\mbb{CP}^n$} is the bundle $\mbb{C}^{n+1}\setminus 0 \to \mbb{CP}^n$ given by the standard projection $(z_0, \dots, z_n) \to [z_0: \dots : z_n]$. It is often denoted $\mc{O}(1)$.  We denote $\mc{O}(n) = \mc{O}(1)^{\otimes n}$.
\end{definition}



\begin{definition}
	The \bam{tautological line bundle} on projective space is $\mc{O}(-1) = \mc{O}(1)^\ast$. We denote $\mc{O}(-n) = \mc{O}(-1)^{\otimes n}$. 
\end{definition}

\begin{prop}
	The canonical bundle on the projective space is $K = \mc{O}(-n-1)$. 
\end{prop}
\begin{prop}
	$\Pic(\mbb{CP}^n)$ is generated by $\mc{O}(\pm1)$.  
\end{prop}

We make a few more useful definitions. 

\begin{definition}
	Let $X$ be an algebraic surface and $\pi:L \to X$ a line bundle. Then the \bam{tautological section} of $\pi^\ast L$ as a bundle over $L$ is given by $\sigma(l) = (l,l)$. 
\end{definition}

\begin{remark}
	Not that the tautological section is indeed valid as we have 
	\eq{
\pi^\ast L = \pbrace{(l,l^\prime) \in L\times L \, | \, \pi(l) = \pi(l^\prime)}	
}
so certainly $(l,l) \in \pi^\ast L$. 
\end{remark}

%%%%%%%%%%%%%%%%%%%%%%%%%%%%%%%%%%%%%%%%%%%%%%%%%%%%%%%%
\subsection{Lax Pairs and Spectral Curves}
\begin{notation}
	We start by laying out some notation that will be necessary for the following section. Let:
	\begin{itemize}
		\item $P=\mbb{CP}^1$ with coordinates $[\xi_0:\xi_1]$. We take $\xi = \frac{\xi_1}{\xi_0}$. 
		\item $O_P$ be the natural structure sheaf on the variety $P$
		\item $V$ be a $m$-dimensional vector space, $\mc{V} = V \otimes O_P$, $\mc{V}(k) = V \otimes \mc{O}_P(k)$ where we view $V$ as either the constant sheaf or trivial bundle over $P$.
		\item $A(t,\xi) = \sum_{k=0}^n A_k(t) \xi^k \in  H^0(P,\Hom(\mc{V},\mc{V}(n)))$ for some $n$, where we see $A_i(t) \in \End(V)$ as a time dependent $m\times m$ matrix and $\xi^k \in H^0(P,\mc{O}(n))$ as
		\eq{
	[\xi_0:\xi_1]^k = \underbrace{\xi_0 \otimes \dots \otimes \xi_0}_{\times (n-k)} \otimes \underbrace{\xi_1 \otimes \dots \otimes \xi_1}_{\times k}	
	}
This is homogeneous of degree $n$, so we allow $A$ to not have a scale\hl{?}
        \item $B(\xi,t) \in  H^0(P,\Hom(\mc{V},\mc{V}(N)))$ for some $N$ likewise . 
        \item $Q(\xi,\eta) = \det\psquare{\eta I - A(\xi,t)}$ be the characteristic polynomial of $A$.
        \item $\sigma$ be the tautological section of $\mc{O}_P(n)$.  
	\end{itemize}
\end{notation}

\begin{lemma}
	$Q(\xi,\sigma) \in H^0(\mc{O}_P(n), \pi^\ast \mc{O}_P(mn))$
\end{lemma}

\begin{definition}
	The pair $A,B$ is a Lax pair if $\dot{A} = \comm[A]{B}$. 
\end{definition}

\begin{prop}
	The Lax equation is invariant under the substitution 
	\eq{
B \mapsto B + p(A,\xi)	
}
for polynomial $p(x,\xi) \in \mbb{C}[x,\xi]$. 
\end{prop}

\begin{definition}
	The \bam{spectral curve} is $C$ given by the solution in $P$ of 
	\eq{
		Q(\xi,\eta) =0	
	} 
\end{definition}

\begin{prop}
	The flow $t \mapsto A(\xi,t)$ is isospectral.
\end{prop}

It will be the understanding of this isospectral flow that we want to gain. We formulate this flow as the family of holomorphic map gained by the eigenvectors 
\eq{
	f_t : C \to \mbb{CP}^{m-1}
}
Suppose that $C$ has degree $d$, then we know we can define 
\eq{
L_t  = f_t^\ast \pround{\mc{O}(1)} \in \Pic^d(C)
}
Lets choose a reference bundle $L_0 \in \Pic^d(X)$ 
\begin{lemma}
	The map 
	\eq{
\Pic^d(C) &\to J(C) \\
L &\mapsto L \otimes L_0^{-1}	
}
is an isomorphism. 
\end{lemma}
Now knowing our result about the tangent space to the Picard group we can say $\frac{dL_t}{dt} \in H^1(C,O_C)$. 
%%%%%%%%%%%%%%%%%%%%%%%%%%%%%%%%%%%%%%%%%%%%%%%%%%%%%%%%
%%%%%%%%%%%%%%%%%%%%%%%%%%%%%%%%%%%%%%%%%%%%%%%%%%%%%%%%
\bibliographystyle{../../bib/custom-bib-style}
\bibliography{../../bib/jabref_library.bib}

\end{document}
