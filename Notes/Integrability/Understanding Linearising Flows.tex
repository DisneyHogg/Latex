\documentclass{article}

\usepackage{../../header}
%%%%%%%%%%%%%%%%%%%%%%%%%%%%%%%%%%%%%%%%%%%%%%%%%%%%%%%%
%Preamble

\title{Linearising Flows and a Cohomological Interpretation of Lax Equations - Unpacking the Paper}
\author{Linden Disney-Hogg}
\date{November 2020}

%%%%%%%%%%%%%%%%%%%%%%%%%%%%%%%%%%%%%%%%%%%%%%%%%%%%%%%%
%%%%%%%%%%%%%%%%%%%%%%%%%%%%%%%%%%%%%%%%%%%%%%%%%%%%%%%%
\begin{document}

\maketitle
\tableofcontents

%%%%%%%%%%%%%%%%%%%%%%%%%%%%%%%%%%%%%%%%%%%%%%%%%%%%%%%%
%%%%%%%%%%%%%%%%%%%%%%%%%%%%%%%%%%%%%%%%%%%%%%%%%%%%%%%%
\section{Introduction}
The purpose of this document is to facilitate the understanding of \cite{Griffiths1985} by discussing the terms and how they fit into the wider picture of algebraic geometry. 
%%%%%%%%%%%%%%%%%%%%%%%%%%%%%%%%%%%%%%%%%%%%%%%%%%%%%%%%
%%%%%%%%%%%%%%%%%%%%%%%%%%%%%%%%%%%%%%%%%%%%%%%%%%%%%%%%
\section{The Preliminaries}

%%%%%%%%%%%%%%%%%%%%%%%%%%%%%%%%%%%%%%%%%%%%%%%%%%%%%%%%
\subsection{Lax Pairs and Spectral Curves}
\begin{notation}
	We start by laying out some notation that will be necessary for the following section. Let:
	\begin{itemize}
		\item $P=\mbb{CP}^1$ with coordinates $[\xi_0:\xi_1]$. We take $\xi = \frac{\xi_1}{\xi_0}$. 
		\item $O_P$ be the natural structure sheaf on the variety $P$
		\item $V$ be a $m$-dimensional vector space, $\mc{V} = V \otimes O_P$, $\mc{V}(k) = V \otimes \mc{O}_P(k)$ where we view $V$ as either the constant sheaf or trivial bundle over $P$.
		\item $A(t,\xi) = \sum_{k=0}^n A_k(t) \xi^k \in  H^0(P,\Hom(\mc{V},\mc{V}(n)))$ for some $n$, where we see $A_i(t) \in \End(V)$ as a time dependent $m\times m$ matrix and $\xi^k \in H^0(P,\mc{O}(n))$ as
		\eq{
	[\xi_0:\xi_1]^k = \underbrace{\xi_0 \otimes \dots \otimes \xi_0}_{\times (n-k)} \otimes \underbrace{\xi_1 \otimes \dots \otimes \xi_1}_{\times k}	
	}
This is homogeneous of degree $n$, so we allow $A$ to not have a scale\hl{?}
        \item $B(\xi,t) \in  H^0(P,\Hom(\mc{V},\mc{V}(N)))$ for some $N$ likewise . 
        \item $Q(\xi,\eta) = \det\psquare{\eta I - A(\xi,t)}$ be the characteristic polynomial of $A$.
        \item $\sigma$ be the tautological section of $\mc{O}_P(n)$.  
	\end{itemize}
\end{notation}

\begin{lemma}
	$Q(\xi,\sigma) \in H^0(\mc{O}_P(n), \pi^\ast \mc{O}_P(mn))$
\end{lemma}

\begin{definition}
	The pair $A,B$ is a Lax pair if $\dot{A} = \comm[A]{B}$. 
\end{definition}

\begin{prop}
	The Lax equation is invariant under the substitution 
	\eq{
B \mapsto B + p(A,\xi)	
}
for polynomial $p(x,\xi) \in \mbb{C}[x,\xi]$. 
\end{prop}

\begin{definition}
	The \bam{spectral curve} is $C$ given by the solution in $P$ of 
	\eq{
		Q(\xi,\eta) =0	
	} 
\end{definition}

\begin{prop}
	The flow $t \mapsto A(\xi,t)$ is isospectral.
\end{prop}

It will be the understanding of this isospectral flow that we want to gain. We formulate this flow as the family of holomorphic map gained by the eigenvectors 
\eq{
	f_t : C \to \mbb{CP}^{m-1}
}
Suppose that $C$ has degree $d$, then we know we can define 
\eq{
L_t  = f_t^\ast \pround{\mc{O}(1)} \in \Pic^d(C)
}
Lets choose a reference bundle $L_0 \in \Pic^d(X)$ 
\begin{lemma}
	The map 
	\eq{
\Pic^d(C) &\to J(C) \\
L &\mapsto L \otimes L_0^{-1}	
}
is an isomorphism. 
\end{lemma}
Now knowing our result about the tangent space to the Picard group we can say $\frac{dL_t}{dt} \in H^1(C,O_C)$. 
%%%%%%%%%%%%%%%%%%%%%%%%%%%%%%%%%%%%%%%%%%%%%%%%%%%%%%%%
%%%%%%%%%%%%%%%%%%%%%%%%%%%%%%%%%%%%%%%%%%%%%%%%%%%%%%%%
\bibliographystyle{../../bib/custom-bib-style}
\bibliography{../../bib/jabref_library.bib}

\end{document}
