\documentclass{article}

\usepackage{../../header}
%%%%%%%%%%%%%%%%%%%%%%%%%%%%%%%%%%%%%%%%%%%%%%%%%%%%%%%%
%Preamble

\title{Monopoles}
\author{Linden Disney-Hogg}
\date{November 2020}

%%%%%%%%%%%%%%%%%%%%%%%%%%%%%%%%%%%%%%%%%%%%%%%%%%%%%%%%
%%%%%%%%%%%%%%%%%%%%%%%%%%%%%%%%%%%%%%%%%%%%%%%%%%%%%%%%
\begin{document}

\maketitle
\tableofcontents

%%%%%%%%%%%%%%%%%%%%%%%%%%%%%%%%%%%%%%%%%%%%%%%%%%%%%%%%
%%%%%%%%%%%%%%%%%%%%%%%%%%%%%%%%%%%%%%%%%%%%%%%%%%%%%%%%
\section{Introduction}
%%%%%%%%%%%%%%%%%%%%%%%%%%%%%%%%%%%%%%%%%%%%%%%%%%%%%%%%
\subsection{Preamble}
I already have notes on Gauge Theory, Algebraic Geometry, Solitons, and Algebraic Topology, but I have yet to actually make any notes on Monopoles. The purpose of these notes is to be a comprehensive cover of the knowledge required to understand \cite{Braden2018}. This will include previous works by Atiyah, Donaldson, Hitchin, Nahm, and more.
%%%%%%%%%%%%%%%%%%%%%%%%%%%%%%%%%%%%%%%%%%%%%%%%%%%%%%%%
\subsection{Preliminaries}
As with all my projects, the preliminaries will undoubtably end up being too long, but I will try keep this minimal this time: 
\begin{definition}
	The \bam{annihilator} of $U \leq V$ is 
\eq{
U^0 = \pbrace{f \in V^\ast \, | \, \forall u \in U, \, f(u)=0} \leq V^\ast
} 
If $V$ has bilinear $\pangle{\cdot, \cdot}$ we can use the isomorphism of $V^\ast\cong V$ to understand 
	\eq{
U^0 = \pbrace{v \in V \, | \, \forall u \in U, \, \pangle{u,v}=0} \leq V	
}
\end{definition}

\begin{lemma}
	THe annihilator is a subspace, $\dim U^0 = \dim V  - \dim U$. 
\end{lemma}

\begin{definition}
	A subspace $U$ is called \bam{isotropic} if $U \subset U^0$. 
\end{definition}
%%%%%%%%%%%%%%%%%%%%%%%%%%%%%%%%%%%%%%%%%%%%%%%%%%%%%%%%
%%%%%%%%%%%%%%%%%%%%%%%%%%%%%%%%%%%%%%%%%%%%%%%%%%%%%%%%
\section{The Monopole Equations}
\begin{definition}
Take a principal $G$-bundle $P\to M$, $\omega_{vol}$ an orientation on $M$, and $\pangle{\cdot, \cdot}$ to be an $\ad$-invariant inner product on $\mf{g}$. Then the \bam{Yang-Mills-Higgs actions} on $M$ is 
\eq{
S_{YMH}[A,\phi] = \int_M \psquare{\abs{F}^2 + \abs{D\phi}^2 + V(\phi)}\omega_{vol}
}
where $F=dA+A\wedge A$ is the curvature associated to a section $A\in \Gamma(T^\ast M \otimes \ad(P))$, $D = d+A$ is the associated covariant derivative, and $\phi \in \Gamma(\ad(P))$.
\end{definition}

\begin{remark}
	A common choice of potential function $V$ is $V(\phi) = \lambda \pround{1-\abs{\phi}^2}^2$, the \bam{$\phi^4$-potential}. 
\end{remark}

\begin{prop}
	The variational equations corresponding to $S_{YMH}$ are the\bam{Yang-Mills-Higgs equations}
	\eq{
DF &= 0 \quad \text{(Bianchi)} \\
D \star F &= -\comm[\phi]{D\phi} \\
D \star D\phi &= -V^\prime(\phi)	
}
\end{prop}
%%%%%%%%%%%%%%%%%%%%%%%%%%%%%%%%%%%%%%%%%%%%%%%%%%%%%%%%
%%%%%%%%%%%%%%%%%%%%%%%%%%%%%%%%%%%%%%%%%%%%%%%%%%%%%%%%
\section{The ADHM construction}
This section follows the work first laid out in \cite{Atiyah1978}. Suppose we have the following information:
\begin{itemize}
	\item $W$ a $k$-dimensional vector space
	\item $V$ a $2k+2$-dimensional vector space with skew, non-degenerate bilinear form $(\cdot, \cdot):\wedge^2 V \to \mbb{C}$. 
	\item $z = (z_i) \in \mbb{C}^4$
	\item $A(z) = \sum_i A_i z_i \in \End(W,V)$ s.t. \
	\eq{
\forall z \neq 0, \; U_z \equiv A(z)W\subset V \text{ is isotropic and $k$-dimensional}	
}
\end{itemize}

We now state some important properties:

\begin{lemma}
	Let $E_z = \faktor{U_z^0}{U_z}$, then 
	\begin{itemize}
		\item $\dim E_z = 2$
		\item $E_z$ inherits a non-degenerate skew bilinear
		\item $\forall \lambda \in \mbb{C}^\times, \, E_z = E_{\lambda z}$. 
	\end{itemize}
\end{lemma}
\begin{proof}
	We go point by point:
	\begin{itemize}
		\item $\dim E_z = \dim U_z^0 - \dim U_z = \pround{\dim V - \dim U_z} - \dim U_z = 2k+2 - 2k = 2$.
		\item The bilinear on $W$ is only degenerate in $U_z^0$ on $U_z$, so by quotienting by this it descends directly to $E_z$. 
		\item $A(\lambda z) = \lambda A(z)$, so $A(\lambda z) (\lambda^{-1} \bm{w}) = A(z)(\bm{w})$. Hence we can see $U_{\lambda z} = U_z$ and so result.  
	\end{itemize}
\end{proof}

\begin{corollary}
	We get a vector bundle $E\to \mbb{CP}^3$ with group $SL(2,\mbb{C})$.
\end{corollary}


%%%%%%%%%%%%%%%%%%%%%%%%%%%%%%%%%%%%%%%%%%%%%%%%%%%%%%%%
%%%%%%%%%%%%%%%%%%%%%%%%%%%%%%%%%%%%%%%%%%%%%%%%%%%%%%%%
\bibliographystyle{../../bib/custom-bib-style}
\bibliography{../../bib/jabref_library.bib}

\end{document}
