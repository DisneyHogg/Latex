\documentclass{article}

\usepackage{../../header}
%%%%%%%%%%%%%%%%%%%%%%%%%%%%%%%%%%%%%%%%%%%%%%%%%%%%%%%%
%Preamble

\title{Monopoles}
\author{Linden Disney-Hogg}
\date{November 2020}

%%%%%%%%%%%%%%%%%%%%%%%%%%%%%%%%%%%%%%%%%%%%%%%%%%%%%%%%
%%%%%%%%%%%%%%%%%%%%%%%%%%%%%%%%%%%%%%%%%%%%%%%%%%%%%%%%
\begin{document}

\maketitle
\tableofcontents

%%%%%%%%%%%%%%%%%%%%%%%%%%%%%%%%%%%%%%%%%%%%%%%%%%%%%%%%
%%%%%%%%%%%%%%%%%%%%%%%%%%%%%%%%%%%%%%%%%%%%%%%%%%%%%%%%
\section{Introduction}
%%%%%%%%%%%%%%%%%%%%%%%%%%%%%%%%%%%%%%%%%%%%%%%%%%%%%%%%
\subsection{Preamble}
I already have notes on Gauge Theory, Algebraic Geometry, Solitons, and Algebraic Topology, but I have yet to actually make any notes on Monopoles. The purpose of these notes is to be a comprehensive cover of the knowledge required to understand \cite{Braden2018}. This will include previous works by Atiyah, Donaldson, Hitchin, Nahm, and more.
%%%%%%%%%%%%%%%%%%%%%%%%%%%%%%%%%%%%%%%%%%%%%%%%%%%%%%%%
\section{Preliminaries}
As with all my projects, the preliminaries will undoubtedly end up being too long, but I will try keep this minimal this time: 
\begin{definition}
	The \bam{annihilator} of $U \leq V$ is 
\eq{
U^0 = \pbrace{f \in V^\ast \, | \, \forall u \in U, \, f(u)=0} \leq V^\ast
} 
If $V$ has bilinear $\pangle{\cdot, \cdot}$ we can use the isomorphism of $V^\ast\cong V$ to understand 
	\eq{
U^0 = \pbrace{v \in V \, | \, \forall u \in U, \, \pangle{u,v}=0} \leq V	
}
\end{definition}

\begin{lemma}
	THe annihilator is a subspace, $\dim U^0 = \dim V  - \dim U$. 
\end{lemma}

\begin{definition}
	A subspace $U$ is called \bam{isotropic} if $U \subset U^0$. 
\end{definition}
%%%%%%%%%%%%%%%%%%%%%%%%%%%%%%%%%%%%%%%%%%%%%%%%%%%%%%%%
\subsection{The Dirac Monopole}
The standard maxwell equations prohibit monopoles, by which we mean point magnetic field sources, as $\bm{\nabla}\cdot \bm{B}=\bm{0}$. Dirac showed in \cite{Dirac1931} that it is possible to escape this conclusion by giving non-trivial topology to the space by allowing $\bm{B} = \frac{g}{4\pi r^2}\hat{\bm{x}}$ to have a singularity at $\bm{x}=\bm{0}$. We can calculate $\bm{\nabla} \cdot \bm{B} = g\delta(\bm{x})$. Removing this circle gives $\mbb{R}^3\setminus 0$, homotopic to $S^2$, and the corresponding magnetic two form on this sphere is
\eq{
f = \frac{g}{4\pi} \sin \theta d\theta \wedge d\phi
} 
and so the flux through a 2-sphere enclosing the origin is $\int_{S_R^2} f = g $. For $g \neq 0$, we know $f \neq da$ for a global $a \in \Omega^1(S^2)$ by Stokes' theorem, but if we take a cover of the sphere $U_{N/S}$ (north/south) and define gauge potentials
\eq{
a_N &= \frac{g}{4\pi}(1-\cos\theta )d\phi \in \Omega^1(U_N) \\
a_S &= \frac{g}{4\pi}(-1-\cos\theta) d\phi \in \Omega^1(U_S)
}
On the intersect $U_N \cap U_S$ we have $da_N = f = da_S$ and $a_N = a_S + \frac{g}{2\pi}d\phi$. \\
Now taking $A = ia, \, F = if$, we have that $g_{NS}(\theta,\phi) = e^{-i\frac{g\phi}{2\pi}}$. Requiring that this is a well-defined transition function gives $g \in \mbb{Z}$. This is equivalent to the integrality of the Chern number. \\
We will not want to consider this as this solution is not solitonic (it has infinite mass), but for further discussion see \cite{Manton2004}. 

%%%%%%%%%%%%%%%%%%%%%%%%%%%%%%%%%%%%%%%%%%%%%%%%%%%%%%%%
\subsection{Pauli Matrices}

\begin{definition}[Pauli Matrices]
	The \bam{Pauli matrices} are
	
	\begin{align*}
		\sigma_1 &= \begin{pmatrix} 0 & 1 \\ 1 & 0\end{pmatrix}, & 
		\sigma_2 &= \begin{pmatrix} 0 & -i \\ i & 0\end{pmatrix}, &  
		\sigma_3 &= \begin{pmatrix} 1 & 0 \\ 0 & -1\end{pmatrix}.  
	\end{align*}
	Note they are all Hermitian and traceless.
\end{definition}

\begin{fact}$\sigma_i \sigma_j = \delta_{ij}I +i\epsilon_{ijk}\sigma_k \Rightarrow \tr(\sigma_i \sigma_j) = 2\delta_{ij}$
\end{fact}

%%%%%%%%%%%%%%%%%%%%%%%%%%%%%%%%%%%%%%%%%%%%%%%%%%%%%%%%
\subsection{\secmath{SU(2)}}
We can write 
\[
SU(2)=\pbrace{    \begin{pmatrix} \alpha & -\overline{\beta} \\ \beta & \overline{\alpha} \end{pmatrix}  : \alpha,\beta\in\mathbb{C} , |\alpha|^2+|\beta|^2=1   }
\]
This can be expressed as, for $A\in SU(2)$
\[
A=a_0 I +i\bm{a}\cdot\bm{\sigma}
\]
where $\bm{a}=(a_1, a_2, a_3)$, $\bm{\sigma}=(\sigma_1, \sigma_2, \sigma_3)$, and $a_0^2+|\bm{a}|^2=1$. Hence $SU(2)\cong S^3$. In addition, by parametrising $SU(2)$ by the $a_i$, it can be seen that $\set{  i\sigma_i }$ forms a basis of $\mf{su}(2)$. It is typical to normalise this basis to $\set{  T^a=-\frac{1}{2} i\sigma_a  }$. 

\begin{lemma}
	The structure constants in this basis $\set{  T^a  }$ are $f^{ab}_c=\epsilon_{abc}$.
\end{lemma}

\begin{corollary}
	The Killing form is given by $\kappa\pround{T^a,T^b} = \kappa^{ab} = -2\delta^{ab} = 4\tr(T^aT^b)$. Hence $\kappa(X,Y) = 4\tr(XY)$
\end{corollary}


%%%%%%%%%%%%%%%%%%%%%%%%%%%%%%%%%%%%%%%%%%%%%%%%%%%%%%%%
\subsection{Degree of a Map}
We will want to consider continuous maps $f:X \to Y$ between connected oriented $n$-dimensional manifold. We state the following lemma:
\begin{lemma}
	$X$ orientable iff $H_n(X) = \mbb{Z}$. 
\end{lemma}
\begin{proof}
	See \cite{Hatcher2002}.
\end{proof}
Denote $[X], [Y]$ to be the generators of $H_n(X), H_n(Y)$ respectively. Recalling that we get an induced homomorphism $f_\ast : H_n(X) \to H_n(Y)$ we make the following definition:
\begin{definition}
	The \bam{degree} of $f$ is defined s.t. 
	\eq{
f_\ast([X]) = \deg(f) [Y]	
}
\end{definition}
Now recall we have a pairing between $k$-forms and $k$-chains on a manifold given by 
\eq{
\pangle{c, \omega} = \int_c \omega
}
and this descends to a pairing between cohomology and homology. With this we have 
\eq{
\pangle{f_\ast [c], \omega} = \pangle{[c], f^\ast [\omega]}
}
Meaning that we can also express the degree via 
\eq{
\deg(f) \int_Y \omega = \int_X f^\ast \omega
}
for some top form on $Y$. This reformulation gives a convenient way to interpret the degree:
\begin{prop}
	Take $y \in Y$ s.t. $f^{-1}(y) \subset X$ is a set of isolated points, say $\pbrace{x_i}_{i=1}^m$. Then 
	\eq{
\deg f = \sum_{i=1}^m \sign \pround{J(x_i)}	
}
\end{prop}
\begin{proof}
	See \cite{Manton2004}, the idea is to choose a volume form on $Y$ localised around $y$, which pulls back to a volume form on $X$ localised around the $x_i$, with the $\pm1$ coming from the change of orientation. 
\end{proof}
%%%%%%%%%%%%%%%%%%%%%%%%%%%%%%%%%%%%%%%%%%%%%%%%%%%%%%%%
%%%%%%%%%%%%%%%%%%%%%%%%%%%%%%%%%%%%%%%%%%%%%%%%%%%%%%%%
\subsection{Spinors}
%%%%%%%%%%%%%%%%%%%%%%%%%%%%%%%%%%%%%%%%%%%%%%%%%%%%%%%%
\subsubsection{Spinor Bundles}
\begin{definition}
	The \bam{complexification} of a vector space $V$ is the vector space $V_\mbb{C} = V \otimes_\mbb{R} \mbb{C}$. 
\end{definition}
\begin{prop}
	$V_\mbb{C}$ comes with a well defined map of conjugation $v \otimes z \to v \otimes \bar{z}$. This is a $\mbb{C}$-linear isomorphism onto $\bar{V_\mbb{C}}$. 
\end{prop}
\begin{lemma}
	$\pround{V^\ast}_\mbb{C} = \pround{V_\mbb{C}}^\ast$. 
\end{lemma}

We will now state a few results about $SL_2(\mbb{C})$ bundles:
\begin{lemma}
	A rank-2 complex vector bundle $S \to M$ has a reduction of structure group to a $SL_2(\mbb{C})$-bundle iff $\exists \varepsilon \in \Gamma(M,\wedge^2 S)$ a non-degenerate form. 
\end{lemma}

\begin{remark}
	Such an $\eps$ will be a symplectic form on $S$.
\end{remark}

\begin{corollary}
	The cotangent bundle on a 2-dimensional manifold $M$ reduces to a $SL_2(\mbb{C})$ bundle iff the canonical bundle is trivial.  
\end{corollary}

\begin{definition}\label{def:spinor bundle}
	Suppose $M$ is a 4-dimensional manifold s.t $\exists S \to M$ a $SL_2(\mbb{C})$-bundle s.t. 
	\eq{
TM_\mbb{C} \cong S \otimes \bar{S}	
}
Then we call $S$ a \bam{spinor bundle} on  $M$. A section $s \in \Gamma(S)$ is called a \bam{spinor}.   
\end{definition}

\begin{remark}
	This is a quite specific definition, so to see this done in more generality check out \cite{Michelsohn1989}. 
\end{remark}

\begin{notation}
	Given a spinor bundle $S \to M$ we will use the notation 
	\eq{
S^A = S^- &= S & S^{A^\prime} = S^+ &= \bar{S} & S_A = S_- &= S^\ast & S_{A^\prime} = S_+ &= \bar{S}^\ast 
}
Each of the above comes with an associated non-degenerate two form $\eps^\pm \in \Gamma(M;\wedge^2 S^\pm)$, $ \eps_\pm \in \Gamma(M; \wedge^2 S_\pm)$, and these provide isomorphisms to the dual as follows:
\eq{
\eps^{A^\prime B^\prime} = \eps^+ : S_+ &\to S^+ & \eps^{AB} = \eps^- : S_- &\to S^- & \eps_{A^\prime B^\prime} = \eps_+ : S^+ &\to S_+ & 
\eps_{AB} = \eps_- : S_- &\to S_- 
}
\end{notation}
If we take an open $U \subset M$ and trivialise s.t $\ev{S}{U} \cong \mbb{C}^2$, we have the standard action of $SL_2(\mbb{C})$ given by $v \mapsto gv$ for $g \in SL_2(\mbb{C})$. We think of this as saying that a spinor in $S$ transforms in the fundamental representation of $SL_2(\mbb{C})$. We can then build the following table:
\begin{center}
\begin{tabular}{c|c|c}
	Bundle & Representation & Actions \\ \hline \hline 
		$S^-$ & Fundamental & $v \mapsto gv$ \\ \hline 
		$S_-$ & Dual & $v \mapsto v^T g^T $ \\ \hline 
		$S^+$ & Congujate & $v \mapsto \bar{g} v$ \\ \hline 
		$S_+$ & Conjugate Dual & $v \mapsto v^T \bar{g}^T$
\end{tabular}
\end{center}
\begin{definition}
	A \bam{spinor frame} is a choice of frame of $S^-$ wrt which 
\eq{
\eps^- = \begin{pmatrix} 0 & 1 \\ -1 & 0 \end{pmatrix} 
}
\end{definition}
%%%%%%%%%%%%%%%%%%%%%%%%%%%%%%%%%%%%%%%%%%%%%%%%%%%%%%%%
\subsubsection{Spinors on Minkowski Space}
One important example of Spinors will be those on Minkowski space, which we will now denote as $M^4$, so we shall cover it in slightly more detail. 
\begin{remark}
	Here, as in the twistor section, we shall follow Ward \& Wells \cite{Ward1991}. They remark that we should really consider Minkwoski space as an affine space in order to not distinguish the origin, and cite \cite{Shirokov1962} for a further discussion. The result of this is that if we choose an origin $0 \in M^4$ we have $M^4 \cong T_0 M^4$. \\
	Note also, \cite{Ward1991} uses the mostly-minus Minkowski metric, so initially this section should be assumed to be using this convention unless stated otherwise. 
\end{remark}

We first introduce a useful coordinate system via the following map to $H_2 = \pbrace{\text{$2\times 2$ Hermitian matrices}}$:
\eq{
M^4  &\overset{\cong}{\to}H_2 \\
x = (x^\mu) &\mapsto \tilde{x} = \frac{1}{\sqrt{2}} \begin{pmatrix} x^0 + x^3 & x^1 -ix^2 \\ x^1 + ix^2 & x^0 - x^3 \end{pmatrix} 
}
Note the correspondence can be written as $\tilde{x} = x^\mu \sigma_\mu$ where $\sigma_\mu = (I,\bm{\sigma})$ corresponds to the Pauli matrices.
\begin{prop}
	$\det \tilde{x} = \frac{1}{2} \abs{x}^2$. 
\end{prop}
\begin{proof}
	 It is simple to calculate 
	 \eq{
\det \tilde{x} = \frac{1}{2} \psquare{(x^0+x^3)(x^0-x^3) - (x^1-ix^2)(x^1+ix^2)} = \frac{1}{2}\psquare{(x^0)^2 - (x^3)^2 -(x^1)^2 - (x^2)^2 }	 
 }
\end{proof}
Note that under the same isomorphism, when we complexify, we get that $M_\mbb{C}^4 \cong M_2(\mbb{C})$. This is as every complex can be decomposed into a Hermitian and anti-Hermitian part. \\
Now let $S = \mbb{C}^2$, where we consider an element $s \in S$ as a column vector. Then $S^\ast = \mbb{C}^2$ with elements $\tilde{s}$ considered as row vectors. 
\begin{prop}
	$S$ is a spinor bundle on $M^4$. 
\end{prop} 
\begin{proof}
\eq{
	T_0 M^4 \otimes \mbb{C} &\cong M^4 \otimes \mbb{C} \cong M_2(\mbb{C}) \cong S \otimes S^\ast 
}
We are then done as $\bar{S} = S^\ast$ here. 
\end{proof}
Under this correspondence, we get the following result
\begin{lemma}
	$v \in T_0M^4_\mbb{C} \otimes \mbb{C}^4$ is null iff $\exists s\in S, \tilde{s} \in S^\ast$ s.t. $v = s \tilde{s}$. Moreover $v$ corresponds to a real vector iff $\tilde{s} = rs$ for $r \in \mbb{R}$. 
\end{lemma}
\begin{remark}
	By the above a vector $v \in T_0M^4_\mbb{C}$ is real iff $\exists s \in S$ s.t. $v = ss^\dagger$. In this sense we can think of spinors as the square root of real null vectors. 
\end{remark}
%%%%%%%%%%%%%%%%%%%%%%%%%%%%%%%%%%%%%%%%%%%%%%%%%%%%%%%%
\subsubsection{Spinor Fields}
We want to extend the above work looking at spinors on Minkowski space to write explicitly the isomorphism claimed in Definition \ref{def:spinor bundle}
\begin{definition}
	We define the mixed spinor tensor $\sigma\indices{_a^A^{A^\prime}}$ by the conditions
	\begin{itemize}
		\item $\bar{\sigma}\indices{_a^A^{A^\prime}} = \sigma\indices{_a^A^{A^\prime}}$
		\item $\sigma\indices{_a^A^{A^\prime}} \sigma\indices{^b_A_{A^\prime}} = \delta^b_a$
		\item $\sigma\indices{_a^A^{A^\prime}} \sigma\indices{^a_B_{B^\prime}} = \eps\indices{_B^A}\eps\indices{_{B^\prime}^{A^\prime}}$.
		\item $\sigma\indices{_[_a^A^{A^\prime}}\sigma\indices{_b_]_A^{B^\prime}} = -\frac{1}{2}i \eps_{abcd} \sigma^{cAA^\prime} \sigma\indices{^d_A^{B^\prime}}$
	\end{itemize}
Given a manifold $M$ with spin bundle $S$ this gives the isomorphism $TM = S \otimes \bar{S}$ as 
\eq{
v^{AA^\prime} = \sigma\indices{_a^A^{A^\prime}} v^a, \quad v^a = \sigma\indices{^a_A_{A^\prime}} v^{AA^\prime}
}
\end{definition}

\begin{lemma}
	Given a metric $\eta_{ab}$ on $M$ the spinor equivalent is 
	\eq{
\eta_{ab} \sigma\indices{^a_A_{A^\prime}} \sigma\indices{^b_B_{B^\prime}} = \eps_{AB} \eps_{A^\prime B^\prime}	
}
\end{lemma}

\begin{remark}
	We will often omit the $\sigma$ to simply write $v^a = v^{AA^\prime}$. 
\end{remark}

\begin{lemma}
	If $\xi_{AB}$ is a skew-spinor then $\xi_{AB} = \eps_{AB} \xi\indices{_C^C}$. 
\end{lemma}

\begin{example}
	If $F_{ab}$ is a Maxwell tensor (i.e. will eventually be a curvature tensor in a gauge theory) then 
	\eq{
F_{AA^\prime BB^\prime} = \phi_{AB} \eps_{A^\prime B^\prime} + \phi_{A^\prime B^\prime} \eps_{AB}
}
where 
\eq{
\phi_{AB} = \frac{1}{2} F\indices{_A_{C^\prime}_B^{C^\prime}}, \quad \phi_{A^\prime B^\prime} = \frac{1}{2} F\indices{_C_{B^\prime}^C_{A^\prime}}
}
are symmetric. We also have the following result:
\begin{lemma}
	$\phi_{A^\prime B^\prime}\eps_{AB}, \, \phi_{AB}\eps_{A^\prime B^\prime}$ are the self dual and anti self dual parts of $F$ respectively. 
\end{lemma}
\end{example}

Now recall a vector $v \in TM$ is null if $v_a v^a=0$. We can translate this result into a comment on spinors:
\begin{prop}
	$v^{AA^\prime}$ is null iff $v^{AA^\prime} = \lambda^A \xi^{A^\prime}$ for spinors $\lambda, \xi$. 
\end{prop}
%%%%%%%%%%%%%%%%%%%%%%%%%%%%%%%%%%%%%%%%%%%%%%%%%%%%%%%%
%%%%%%%%%%%%%%%%%%%%%%%%%%%%%%%%%%%%%%%%%%%%%%%%%%%%%%%%
\subsection{Twistors}
\subsubsection{Grassmannians and Flag Manifolds}
\begin{definition}
	Let $V$ be an $n$-dimensional vector space over $\mbb{F}$. Define the \bam{Grassmannian manifold of $k$-dimensional subspaces} as $G_k(V)$. 
\end{definition}

\begin{lemma}
	If $\mbb{F}= \mbb{R}$, $G_k(V)\cong \faktor{O(n)}{[O(k) \times O(n-k)]}$
\end{lemma}
\begin{proof}
	We can view the Grassmannian as the quotient of $GL(V)$ by the stabiliser of a $k$-dim subspace. 
\end{proof}
\begin{remark}
	Likewise if $\mbb{F}=\mbb{C}, \, G_k(V) \cong \faktor{U(n)}{[U(k) \times U(n-k)]}$,
\end{remark}
\begin{corollary}
	$\dim G_k(V) = k(n-k)$. 
\end{corollary}

\begin{example} 
	$G_1(\mbb{C}) \cong \mbb{CP}^1$. 
\end{example}

\begin{notation}
	If we pick a basis of complex vector space $V$ to make $V \cong \mbb{C}^n$, we identify $G_k(V) \cong G_k(\mbb{C}^n)$ which we will notation $G_{k,n}$. 
\end{notation}

If we denote $M^\ast_{n \times k}(\mbb{C})$ to be the full rank $m \times k$ matrices we can construct $G$ via the projection 
\eq{
M^\ast_{n \times k}(\mbb{C}) &\to G_{k,n} \\
M &\mapsto \spn\pbrace{\text{columns of $M$}}
}
$M^\ast_{n \times k}(\mbb{C})$ carries a transitive left action via multiplication that doesn't affect span, and so we get a bundle structure. We can then consider coordinates on $G_{k,n}$ as $Z \in M_{(n-k) \times k}(\mbb{C})\cong \mbb{C}^{k(n-k)}$ via the map 
\eq{
\phi : M_{(n-k) \times k}(\mbb{C}) &\to G_{k,n} \\
Z &\mapsto \psquare{\begin{pmatrix} Z \\ I_k \end{pmatrix}}
}
It will now be necessary to generalise this concept slightly:
\begin{definition}
	Given a fixed complex $n$-dimensional vector space $V$ and $\pbrace{d_i}_{i=1}^m$ integers satisfying $1 \leq d_1 < d_2 < \dots < d_m < n$ we define a \bam{flag manifold of type $(d_1, \dots, d_m)$} as 
	\eq{
F_{d_1 \dots d_m}(V) = \pbrace{(S_1, \dots, S_m) \, | S_i \text{ a $d_i$-dimensional subspace of $V$, } S_1 \subset S_2 \subset \dots \subset S_m}	
} 
\end{definition}

\begin{lemma}
	The dimension of a flaf manifold of type $(d_1, \dots, d_m)$ is 
	\eq{
d_1(n-d_1) + (d_2-d_1)(n-d_2) + \dots +(d_m = d_{m-1})(n-d_m)	
}
\end{lemma}

\begin{definition}
	If $V$ is a 4-dimensional vector space we define the \bam{Pl\"ucker embedding} to be the map 
	\eq{
\text{pl} : G_2(V) &\to \mbb{P}(\wedge^2 V) \\
\spn(Z,W) &\mapsto \psquare{\spn(Z \wedge W)}	
}
\end{definition}

\begin{theorem}
	$\text{pl}$ is an embedding, with image $Q_4 = \text{pl}(G_{2,4})$ given by \eq{
Q_4 = \pbrace{z^{ij} \, | \, z^{12}z^{34}-z^{13}z^{24}+z^{14}z^{23}=0} = \pbrace{[z] \, | \, z \wedge z=0}	\subset \mbb{P}^5
}
where $[z^{ij}]$ are homogeneous coordinates on $\mbb{P}(\wedge^2 V)$. $Q_4$ is called the \bam{Klein quadric}. 
\end{theorem}

Generically a line $L$ in $\mbb{P}^5$ will intersect $Q_4$ in isolated points, but when this doesn't happen we can give such lines a special name:
\begin{definition}
	A line $L \subset \mbb{P}^5$ is \bam{null} if $L \subset Q_4$. A plane $N \subset \mbb{P}^5$ is null if every line $L\subset N$ is null. 
\end{definition}


\subsubsection{Twistor Space}
Now given $\mbb{T}$ a $4$d $\mbb{C}$ vector space we can get a \bam{double fibration}
\eq{
F_1(\mbb{T}) \overset{\pi_1}{\twoheadleftarrow} F_{12}(\mbb{T}) \overset{\pi_2}{\twoheadrightarrow} F_2(\mbb{T})
}
from the projections of $(S_1,S_2)$ 
\begin{prop}
	The $\pi_i$ are holomorphic.
\end{prop}

\begin{definition}
	Given seta $A,B$ a \bam{correspondence}  is a map $f: A \to \mc{P}(B)$ sending each $a \in A$ to $f(a) \subset B$.
\end{definition}
The double fibration induces natural correspondences $c=\pi_2 \circ \pi_1^{-1}$ and $c^{-1} = \pi_1 \circ \pi_2^{-1}$. 
\begin{notation}
	We shall call $\mbb{T}$ the \bam{twistor space}, and denote 
	\begin{itemize}
		\item $\mbb{PT}=F_1(\mbb{T}) \cong \mbb{CP}^3$ \bam{projective twistor space}
		\item $M_\mbb{C} = F_2(\mbb{T}) \cong G_{2,4}(\mbb{C})$ \bam{compactified complexified Minkowski space}
		\item $F = F_{12}(\mbb{T})$ the \bam{correspondence} between $\mbb{PT}$ and $M_{\mbb{C}}$. 
	\end{itemize}
For $A \subset \mbb{PT}$ we shall let $c(A) = \tilde{A}$, and for $B \subset M_{\mbb{C}}$ let $\hat{B} = c^{-1}(A)$. 
\end{notation}

\begin{idea}
	The point of twistor geometry is understanding how to use the correspondence
	\eq{
\mbb{PT} \leftarrow F \rightarrow M_{\mbb{C}} 	
} 
	to transfer information about (subsets of) $\mbb{PT}$ to (subsets of) $M_{\mbb{C}}$. 
\end{idea}

\begin{remark}
	$F$ is sometimes denoted as $\mbb{PS}$ and called the \bam{projective spinor bundle}. \hl{We will hopefully return to this later}. 
\end{remark}

\begin{prop}\label{prop:twistor correspondence shape}
	For $p \in \mbb{PT}, \, q \in M_{\mbb{C}}$, $\tilde{p} \cong \mbb{P}^2$ and $\hat{q} \cong \mbb{P}^1$
\end{prop}

We now wish to understand the correspondence in coordinates. We consider the map $\phi$ as described before, though now we send 
\eq{
	z &\mapsto \psquare{\begin{pmatrix} iz \\ I_2 \end{pmatrix}}
}
for convenience, let $M^I_\mbb{C}= \phi(M_2(\mbb{C}))$, $\mbb{PT}^I = c^{-1}(M_{\mbb{C}}^I)$ and $F^I = \pi_2^{-1}(M^I)$.  
\begin{prop}
	$F^I \cong M^I \times \mbb{P}^1$. 
\end{prop} 
\begin{proof}
	The isomorphism is given as 
	\eq{
	M^I_\mbb{C} \times \mbb{P}^1 &\to F^I \\
	(z,[v]) &\mapsto \pround{\psquare{\begin{pmatrix} izv \\ v \end{pmatrix}},\psquare{\begin{pmatrix} iz \\ I_2 \end{pmatrix}}}
}
\end{proof}
\begin{corollary}
	In coordinates we can see the double fibration as 
	\eq{ 
\mbb{PT}^I \leftarrow & \; \; \; F^I \; \; \;  \rightarrow M_\mbb{C}^I \\
(izv, v)^T \leftarrow &(z,[v]) \rightarrow z	
}
\end{corollary}

We can further construct the following double fibrations of flag manifolds over $\mbb{T}$ 
\eq{
F_3 \twoheadleftarrow \, &F_{23} \twoheadrightarrow F_2 
 & F_{13} \twoheadleftarrow \, &F_{123} \twoheadrightarrow F_2
}
As $\dim \mbb{T}=4$ we have that canonically $F_3 \cong \mbb{PT}^\ast$, $F_{23} \cong F^\ast$. 
\begin{definition}
 We call $F_3$ the \bam{dual projective twistor space}, $F_{13} = \mbb{A}$ the \bam{ambitwistor space}. We will denote $F_{123} = \mbb{G}$. 
\end{definition}
With this notation the  fibrations look as 
\eq{
	\mbb{PT}^\ast \twoheadleftarrow \, &F^\ast \twoheadrightarrow M_\mbb{C} 
	& \mbb{A} \twoheadleftarrow \, &\mbb{G} \twoheadrightarrow M_\mbb{C}
}
We now can extend the result of prop \ref{prop:twistor correspondence shape}
\begin{prop}
	If $p \in \mbb{PT}, \, q \in \mbb{PT}^\ast,\, r \in \mbb{A}$ then 
	\eq{
\tilde{p} & \cong \mbb{P}^2 & \tilde{q} &\cong \mbb{P}^2 & \tilde{r} &\cong \mbb{P}^1	
}
and for $s \in M_\mbb{C}$
\eq{
\hat{s}_{\mbb{PT}} & \cong \mbb{P}^1 & \hat{s}_{\mbb{PT}^\ast} & \cong \mbb{P}^1 & \hat{s}_{\mbb{A}} & \cong \mbb{P}^1 \times \mbb{P}^1 
}
\end{prop}
\begin{definition}
	Given $p \in \mbb{PT}$, $\tilde{p} \subset M_\mbb{C}$ is called an $\alpha$-\bam{plane}, and given $q \in \mbb{PT}^\ast$, $\tilde{q} \subset M_\mbb{C}$ is called an $\beta$-\bam{plane}.
\end{definition}

By the Pl\"ucker embedding, we have that $M_\mbb{C} \cong Q_4 \subset \mbb{P}^5$, and so we can describe $\alpha,\beta$-planes in terms of null planes via the following result:
\begin{prop}
	We have that
	\begin{itemize}
		\item All $\alpha,\beta$-planes in $M_\mbb{C}$ are null in $\mbb{P}^5$
		\item Any null planes in $\mbb{P}^5$ is either an $\alpha$-plane or $\beta$-plane.
		\item Null lines in $M_\mbb{C}$ are the 5-dimensional family of lines of the form $\tilde{r}$ for $r\in \mbb{A}$.
		\item Any null line in $M_\mbb{C}$ is the intersection of an $\alpha$-plane and $\beta$-plane.  
	\end{itemize}
\end{prop}

%%%%%%%%%%%%%%%%%%%%%%%%%%%%%%%%%%%%%%%%%%%%%%%%%%%%%%%%%%%%%%%%%%%%%%%%%%%%%%%
\subsubsection{Actions on Twistor Space}
We will often want to equip $\mbb{T}$ with a Hermitian form $\Phi$. We state a quick lemma:
\begin{lemma}
	$\Phi$ determines a volume form $\Omega \in \wedge^4 \mbb{T}^\ast$ on $\mbb{T}$ by 
	\eq{
\Omega = \image \Phi \wedge \image \Phi 
}
\end{lemma}
\begin{definition}
	We define $SU(\mbb{T},\Phi) \subset GL(\mbb{T})$ to be the subset preserving $\Phi$ and $\Omega$.  
\end{definition}
\begin{example}
	If $\Phi$ has signature $(2,2)$ then $SU(\mbb{T},\Phi) \cong SU(2,2)$. 
\end{example}
Given a $(2,2)$-signature Hermitian form on $\mbb{T}$ $\exists(Z^\alpha)$ coordinates wrt which we have 
\eq{
\Phi = \begin{pmatrix} 0 & I_2 \\ I_2 & 0 \end{pmatrix} \Leftrightarrow \Phi(Z,Z) = Z^0 \bar{Z^2} + Z^1 \bar{Z^3} + Z^2 \bar{Z^0} + Z^3 \bar{Z^1} 
}
\begin{notation}
	We denote the vector space $\mbb{T}$ with the above coordinates adapted to $\Phi$ as $\mbb{T}^\alpha$ and its dual as $\mbb{T}_\alpha$. We call the corresponding dual coordinates $W_\alpha$. 
\end{notation}
$SL_2(\mbb{C}) \leq SU(2,2)$ will act reducibly on $\mbb{T}^\alpha$ meaning we have $\mbb{T}^\alpha = S_1 \oplus S_2$ where the $S_i$ are spinor bundles in the fundamental and conjugate dual representations respectively. This means we can write our coordinates as $Z^\alpha = (\omega^A, \pi_{A^\prime})$. 

\begin{remark}
 Recalling the double fibration, we can see that in order for a point $Z = (\omega,\pi) \in \mbb{PT}$ to correspond to a $z \in M_{\mbb{C}}$ we must have 
 \eq{
iz\pi = \omega 
}
This is often called the \bam{incidence relation}. From this point of view, we see that $z$ must get the indices $z^{AA^\prime}$ s.t. in coordinates the incidence relation is 
\eq{
\omega^A = iz^{AA^\prime} \pi_{A^\prime}
}
\end{remark}

\subsubsection{Complexified Minkowski Space}
It is at this point useful to discuss slightly more concretely what is meant by complexified Minkowski space. The resource to use here is \cite{Adamo2018}. 
\begin{definition}
	Let $(M,g)$ be a smooth real manifold, and take a coordinate system $x^a$ s.t. $g= g_{ab} dx^a dx^b$. The \bam{complexification} is defined by allowing the $x^a$ to take complex values and extending $g_{ab}(x)$ holomorphically.  
\end{definition}
%%%%%%%%%%%%%%%%%%%%%%%%%%%%%%%%%%%%%%%%%%%%%%%%%%%%%%%%
%%%%%%%%%%%%%%%%%%%%%%%%%%%%%%%%%%%%%%%%%%%%%%%%%%%%%%%%
\section{The Monopole Equations}
%%%%%%%%%%%%%%%%%%%%%%%%%%%%%%%%%%%%%%%%%%%%%%%%%%%%%%%%
\subsection{Yang-Mills-Higgs equations}
\begin{definition}
Take a principal $G$-bundle $P\to M$, $\omega_{vol}$ an orientation on $M$, and $\pangle{\cdot, \cdot}$ to be an $\ad$-invariant inner product on $\mf{g}$. Then the \bam{Yang-Mills-Higgs actions} on $M$ is 
\eq{
S_{YMH}[A,\phi] = \int_M \psquare{-\abs{F}^2 - \abs{D\phi}^2 - V(\phi)}\omega_{vol}
}
where $F=dA+A\wedge A$ is the curvature associated to a section $A\in \Gamma(T^\ast M \otimes \ad(P))$, $D = d+A$ is the associated covariant derivative, and $\phi \in \Gamma(\ad(P))$.
\end{definition}

To connect with physical theory we want our Lagrangian to be of the form kinetic-potential. This will manifest itself in our choice of signs by requiring that 
\eq{
-\abs{D\phi}^2 &= (\del_0 \phi)^2 + \dots \\
-\abs{F}^2 &= E_i^2 + \dots
}
This is the reason for the somewhat strange looking sign choice at this point. Obviously in the end it will be equivalent to take 
\eq{
S_{YMH} = \int \abs{F}^2 + \abs{D\phi}^2 + V
}
when it comes to the variational equations, but this will not be true when we consider the energy functional.
\begin{remark}
	A common choice of potential function $V$ is $V(\phi) = \lambda \pround{1-\abs{\phi}^2}^2$, the \bam{$\phi^4$-potential}. By our choice of signs, we want $\lambda >0$. We can check $V^\prime(\phi) = -4\lambda(1-\abs{\phi}^2)\abs{\phi}$. 
\end{remark}

\begin{definition}
	A \bam{monopoles} will be a soliton-like solution to the Yang-Mills-Higgs equations when $G=SU(2)$, $M=\mbb{R}^4$ with the Minkowski metric, the principal bundle is $P = M \times G$, and the potential is $\phi^4$. 
\end{definition}

\begin{prop}
	The variational equations corresponding to $S_{YMH}$ in Minkowski $\mbb{R}^4$ are the\bam{Yang-Mills-Higgs equations}
	\eq{
DF &= 0 \quad \text{(Bianchi)} \\
 \star D \star F &= -\comm[\phi]{D\phi} \\
 \star D \star D\phi &= -\frac{1}{2\abs{\phi}}V^\prime(\phi)\phi	
}
\end{prop}
\begin{proof}
	We first consider the equation that comes from the variation of $A$. Let $A_t = A + t\beta$, then $F_t = F + t\pround{d\beta + \beta \wedge A + A \wedge \beta} + \mc{O}(t^2)$ and $D_t \phi = D \phi + t\comm[\beta]{\phi}$. Hence 
	\eq{
S_t = S + 2t\int_M \psquare{-\pangle{F,D\beta} - \pangle{D\phi,\comm[\beta]{\phi}}} \omega_{vol} + \mc{O}(t^2)
}
Hence to be at a stationary point of the action variation we want
\eq{
\int_M \psquare{-\pangle{F,D\beta} - \pangle{D\phi,\comm[\beta]{\phi}}} \omega_{vol} = 0
}
Using the fact that inner product is $\ad$-invariant and letting $D^\ast$ be the formal adjoint of $D$ wrt to inner product $\pangle{\pangle{\eta,\omega}} = \int_M \pangle{\eta,\omega} \omega_{vol}$ we can rewrite this as 
\eq{
\int_M \pangle{-D^\ast F + \comm[D\phi]{\phi},\beta} \omega_{vol} = 0
}
Using results on the dual of the covariant derivative we can say that 
\eq{
D^\ast F = (-1)^{4(2-1)+1}(-1)\star D \star F = \star D \star F 
}
Hence as $\beta$ was a generic variation we must have $\star D \star F - \comm[D\phi]{\phi} =0$. \\
We now consider a $\phi$ variation so $\phi_t = \phi + t\psi$. Note 
\eq{
\abs{\phi_t} = \sqrt{\pangle{\phi_t,\phi_t}} = \sqrt{\abs{\phi}^2 + 2t\pangle{\phi,\psi} + \mc{O}(t^2)} = \abs{\phi}\sqrt{1 + \frac{2t\pangle{\phi,\psi}}{\abs{\phi}^2} + \mc{O}(t^2)} = \abs{\phi} + t\frac{\pangle{\phi,\psi}}{\abs{\phi}} + \mc{O}(t^2)
}
so if we consider $V$ as $V(\phi)=V(\abs{\phi})$ (i.e. as a function of a real variable) then 
\eq{
V(\phi_t) = V(\phi) + t\abs{\phi}^{-1} V^\prime(\phi) \pangle{\phi,\psi} + \mc{O}(t^2)
}
Then a variational argument as before means that we need to set 
\eq{
 && \int_M \psquare{-2\pangle{D\phi, D\psi} - \abs{\phi}^{-1} V^\prime(\phi) \pangle{\phi,\psi}} \omega_{vol} &= 0 \\
 \Rightarrow& &  -2D^\ast D \phi- \abs{\phi}^{-1} V^\prime(\phi) \phi &= 0 \\
\Rightarrow& &  (-1)^{4(1-1)+1}(-1)(-1)\star D \star D \phi - \frac{1}{2\abs{\phi}}V^\prime(\phi) \phi &= 0 \\
\Rightarrow& &  \star D \star D \phi + \frac{1}{2\abs{\phi}}V^\prime(\phi) \phi &= 0
}
\end{proof}

\begin{remark}
	The Dirac monopole is a solution in the case of $G=U(1)$.
\end{remark}
We may make these equations explicit in coordinates. The first approach is to try and substitute in coordinate expressions into the YMH equations. Taking coordinates $x^\mu$ on $\mbb{R}^4$ and writing $F = \frac{1}{2}F_{\mu\nu} dx^\mu \wedge dx^\nu$
\eq{
\star F &= \frac{1}{4} F_{\mu\nu} \epsilon\indices{^\mu^\nu_\rho_\sigma} dx^\rho \wedge dx^\sigma \\
\Rightarrow D\star F &= \frac{1}{4} D_\tau F_{\mu\nu} \epsilon\indices{^\mu^\nu_\rho_\sigma} dx^\tau \wedge dx^\rho \wedge dx^\sigma \\
\Rightarrow \star D \star F &= \frac{1}{4} D_\tau F_{\mu\nu} \epsilon\indices{^\mu^\nu_\rho_\sigma} \epsilon\indices{^\tau^\rho^\sigma_\lambda} dx^\lambda \\
&= \frac{1}{4} D_\tau F_{\mu\nu} \eps\indices{_\rho_\sigma^\mu^\nu}\eps\indices{^\rho^\sigma^\tau_\lambda} dx^\lambda \\
&= -D_\tau F\indices{^\mu_\nu} \delta^\tau_{[\mu}\delta^\nu_{\lambda]}dx^\lambda \\
&= -D_\tau F\indices{^\tau_\lambda} dx^\lambda
}
Hence the first monopole equation reads 
\eq{
D_\mu F\indices{^\mu^\nu} +\comm[D^\nu \phi]{\phi} = 0
}
Next we have 
\eq{
\star D \phi &= \frac{1}{6} (D_\mu \phi) \eps\indices{^\mu_\nu_\rho_\sigma} dx^\nu \wedge dx^\rho \wedge dx^\sigma \\
\Rightarrow D\star D \phi &= \frac{1}{6} (D_\tau D_\mu \phi) \eps\indices{^\mu_\nu_\rho_\sigma} dx^\tau \wedge dx^\nu \wedge dx^\rho \wedge dx^\sigma \\
\Rightarrow \star D \star D \phi &= \frac{1}{6} (D_\tau D_\mu \phi) \eps\indices{^\mu_\nu_\rho_\sigma} \epsilon^{\tau \nu \rho \sigma} \\
&=-(D_\tau D^\mu \phi) \delta_\mu^\tau = -D_\mu D^\mu \phi
}
which yields (taking the $\phi^4$ potential)
\eq{
D_\mu D^\mu \phi +2\lambda (1-\abs{\phi}^2)\phi = 0
}
Collecting this with the Bianchi identity (taking it as $\star DF=0$) we have 
\eq{
\epsilon^{\rho\mu\nu\tau} D_\rho F_{\mu\nu} &=0 \\
D_\mu F\indices{^\mu^\nu} +\comm[D^\nu \phi]{\phi} &= 0 \\
D_\mu D^\mu \phi +2\lambda (1-\abs{\phi}^2)\phi &= 0
}
An alternative approach to deriving these equations is to first write the Lagrangian in coordinate form and then derive the variational equations. We take the inner product on $\mf{g} = \mf{su}(2)$ to be $\pangle{X,Y} = -\frac{1}{2}\kappa(X,Y) = -2\tr(XY)$ for concreteness, which gives $\pangle{t^a,t^b} = \delta^{ab}$, and as such we need the mostly-positive Minkowski metric. 
\begin{remark}
	It is possible to change around the signs here in order to assure that we use the mostly-minus metric, as Manton \& Sutcliffe do. We will want to avoid this as the mostly positive makes more sense when reducing from Minkowski $\mbb{R}^{n+1}$ to Euclidean $\mbb{R}^n$. 
\end{remark}
We have 
\eq{
\pangle{F,F} &= \frac{1}{4}\pangle{F_{\mu\nu},F_{\rho\sigma}} \pangle{dx^\mu \wedge dx^\nu, dx^\rho \wedge dx^\sigma} \\
&= \frac{1}{4}\pangle{F_{\mu\nu},F_{\rho\sigma}}\pround{\eta^{\mu\rho}\eta^{\nu\sigma} - \eta^{\mu\sigma}\eta^{\nu\rho}} \\
&= \frac{1}{2}\pangle{F_{\mu\nu},F^{\mu\nu}} \\
&= -\tr(F_{\mu\nu}F^{\mu\nu}) \\
\pangle{D\phi,D\phi} &= \pangle{D_\mu\phi,D_\nu\phi}\pangle{dx^\mu, dx^\nu} \\
&= \pangle{D_\mu \phi, D_\nu\phi} \eta^{\mu\nu} \\
&= -2\tr(D_\mu \phi D^\mu\phi)
}
Hence the corresponding Lagrangian density is 
\eq{
\mc{L} = \tr(F_{\mu\nu}F^{\mu\nu}) + 2\tr(D_\mu \phi D^\mu \phi)- V(\phi) = \sum_a \psquare{- \frac{1}{2}F_{\mu\nu}^{(a)} F^{(a)\mu\nu} - \pround{D_\mu \phi}^{(a)} \pround{D^\mu \phi}^{(a)}} - \lambda \psquare{1-\sum_a (\phi^{(a)})^2}^2 
}
To check that we have the correct signs in this Lagrangian we verify that it takes the form kinetic-potenial with 
\eq{
\text{kinetic} &= -2\tr(E_i E_i) -2\tr(D_0 \phi D_0 \phi) \\
\text{potential} &= -\tr(F_{ij} F_{ij}) -2\tr(D_i \phi D_i \phi) + V(\phi)  
}
where $E_i = F_{0i}$.
\begin{remark}
	Something we can immediately recognise is that in order to get finite energy solutions when $\lambda \neq 0$, we need $\abs{\phi} \to 1$.
\end{remark}
We now recall the Euler-Lagrange equations for Lagrangian with field $\psi$
\eq{
\pd[\mc{L}]{\psi} - \del_\mu \pd[\mc{L}]{(\del_\mu \psi)}=0
}
The fields here are really the coefficients in $\mf{su}(2)$ of $A_\mu, \phi$, that is they are $A_\mu^{(a)}, \phi^{(a)}$, so we expand   
\eq{
	F_{\mu\nu}^{(a)} &= \del_\mu A_\nu^{(a)} - \del_\nu A_\mu^{(a)} + A_\mu^{(b)}A_\nu^{(c)}\eps\indices{_a_b_c} \\
	(D_\mu \phi)^{(a)} &= \del_\mu \phi^{(a)} + A_\mu^{(b)} \phi^{(c)} \eps\indices{_a_b_c}
}
giving 
\eq{
\pd[\mc{L}]{(\del_\mu A_\nu^{(a)})} &= -2F^{(a)\mu\nu} \\
\pd[\mc{L}]{A_\mu^{(a)}} &= -2\eps\indices{_b_a_c}A^{(c)}_\nu F^{(b)\mu\nu}-2\phi^{(c)}\eps\indices{_b_a_c}(D^\mu \phi)^{(b)} \\
\pd[\mc{L}]{(\del_\mu \phi^{(a)})} &= -2(D^\mu \phi)^{(a)} \\
\pd[\mc{L}]{\phi^{(a)}} &= -2\eps\indices{_c_b_a}A_\mu^{(b)} (D^\mu \phi)^{(c)} +4\lambda \phi^{(a)}\psquare{1-\sum_b (\phi^{(b)})^2}
}
We can now write the Euler-Lagrange equations 
\eq{
0 &= -\del_\nu F^{(a)\nu\mu } - \psquare{-\eps\indices{_b_a_c}A^{(c)}_\nu F^{(b)\mu\nu}-\phi^{(c)}\eps\indices{_b_a_c}(D^\mu \phi)^{(b)}} \\
&= -\psquare{\del_\nu F^{(a)\nu\mu} + A_\nu^{(c)}F^{(b)\nu\mu}\eps_{cba}} - (D^\mu \phi)^{(b)}\phi^{(c)}\eps_{bca} \\
\Rightarrow 0 &= D_\nu F^{\nu \mu} + \comm[D^\mu \phi]{\phi}  \\
}
and 
\eq{
0 &= -\del_\mu (D^\mu \phi)^{(a)} - \psquare{-\eps\indices{_c_b_a}A_\mu^{(b)} (D^\mu \phi)^{(c)} +2\lambda \phi^{(a)}\psquare{1-\sum_b (\phi^{(b)})^2}} \\
&= -\psquare{\del_\mu (D^\mu \phi)^{(a)} + A_\mu^{(b)} (D^\mu \phi)^{(c)} \eps_{bca}} - \lambda \phi^{(a)}\psquare{1-\sum_b (\phi^{(b)})^2} \\
\Rightarrow 0 &= D_\mu D^\mu \phi + 2\lambda  (1-\abs{\phi}^2)\phi
}
We happily see that these two approaches agree, and we should see that these are indeed the sort of equations we want (e.g the $\phi$ equation looks like Klein-Gordon if we linearise around $\abs{\phi}=1$. )
%%%%%%%%%%%%%%%%%%%%%%%%%%%%%%%%%%%%%%%%%%%%%%%%%%%%%%%%
\subsection{BPS limit}
The monopole equations we have found so far are second order, but we want to apply the classic strategy when working with topological solitons: write the energy functional of a static configuration as the integral of a square term plus a topological term, and then we locally must have a minimising solution by setting the squared term to 0. This will be possible if we set $\lambda=0$ but retain that $\abs{\phi}=1$ at infinity. More specifically we take the conditions 
\eq{
\abs{\phi} & = 1 - \frac{m}{2r} + \mc{O}\pround{\frac{1}{r^2}} \\
\pd[\abs{\phi}]{\Omega} &= \mc{O}\pround{\frac{1}{r^2}} \\
\abs{D\phi} &= \mc{O}\pround{\frac{1}{r^2}} 
}
With $\lambda=0$ we can rewrite the energy functional of static configurations at any time as 
\eq{
E[A,\phi] = \int_M \psquare{-\tr(F_{ij} F_{ij}) -2\tr(D_i \phi D_i \phi)} d^3 x
}
If we keep the same inner product on $\mf{su}(2)$, but now take Euclidean $\mbb{R}^3$ we can express
\eq{
E[A,\phi] = \int_{\mbb{R}^3} \psquare{\abs{F}^2 + \abs{D\phi}^2} d^3x
}
As we are in Euclidean $\mbb{R}^3$ we can write 
\eq{
\abs{\star D \phi}^2 \omega_{vol} = \pangle{\star D \phi \wedge \star^2 D \phi} = (-1)^{1(3-1)}\pangle{\star D \phi \wedge D \phi} =  (-1)^{2\times1} \pangle{D \phi \wedge \star D \phi} = \abs{D \phi}^2 \omega_{vol}
}
and, using the Bianchi identity,
\eq{
	\pangle{F, \star D \phi}\omega_{vol}  = \pangle{F \wedge \star^2 D \phi} &= (-1)^{1(3-1)}\pangle{F \wedge D \phi} = (-1)^2\psquare{d\pangle{F \wedge \phi} -\pangle{DF \wedge \phi}} = d\pangle{F \wedge \phi}
}
so 
\eq{
E[A,\phi] &= \int_{\mbb{R}^3} \psquare{\pangle{F \mp \star D\phi, F\mp \star D\phi} \pm 2\pangle{F,\star D\phi}}d^3x \\
&= \int_{\mbb{R}^3} \abs{F\mp \star D\phi}^2 d^3x  \pm 2 \lim_{R \to \infty} \int_{S_R^2} \pangle{F \wedge \phi}
}
\begin{remark}
	In the above discussion, make sure to check that as this the $3d$ Hodge star, our inner products are well defined in the sense that the are the inner product of a $k$-form with a $k$-form. 
\end{remark}
This boundary term turns out to be a topological contribution, and we can see this in two ways. We first need to prove the following lemma
\begin{lemma}
	In the $\mbb{R}^3$ bulk 
	\eq{
		\pangle{\phi \wedge \pround{D \phi \wedge D \phi}} = \pangle{\phi \wedge \pround{d \phi \wedge d\phi}} + \abs{\phi}^2 \psquare{\pangle{F \wedge \phi} - d\pangle{A \wedge \phi}} - \frac{1}{2}\pangle{A \wedge \phi} \wedge d\abs{\phi}^2 
	}
\end{lemma}
\begin{proof}
	\hl{Exercise}. 
\end{proof}
\begin{corollary}
	Given the decay conditions, on $S_\infty^2$ we have 
	\eq{
		\pangle{F \wedge \phi} - d\pangle{A \wedge \phi} = -\pangle{\phi \wedge \pround{d\phi \wedge d\phi}}	
	}
	and hence the topological term is $\mp 2\int_{S_\infty^2} \pangle{\phi \wedge (d\phi \wedge d\phi)}$.
	\begin{comment}
	Now the inverse stereographic projection $f : \mbb{R}^2 \to S^2 \subset \mbb{R}^{3}$ allows us to pullback the standard metric on $\mbb{R}^{3}$. This can be done in Sage:
	\begin{lstlisting}[language=Python,frame=single]
	sage: M = Manifold(2, 'R^2')
	sage: U.<u,v> = M.chart()
	sage: N = Manifold(3, 'R^3')
	sage: X.<x,y,z> = N.chart()
	sage: f = M.diff_map(N,(2*u/(1+u^2+v^2), 2*v/(1+u^2+v^2),
	(-1+u^2+v^2)/(1+u^2+v^2)))
	sage: g = N.sym_bilin_form_field(name='g')
	sage: g[1,1], g[2,2], g[3,3] = 1, 1, 1
	sage: fg = f.pullback(g)
	sage: f.display()
	4/(u^4 + v^4 + 2*(u^2 + 1)*v^2 + 2*u^2 + 1) du*du 
	+ 4/(u^4 + v^4 + 2*(u^2 + 1)*v^2 + 2*u^2 + 1) dv*dv
	\end{lstlisting}
	\end{comment}
\end{corollary}
The two interpretations are then as follows: 
\begin{enumerate}
	\item Degree of a map of spheres - We will want to recall some standard vector calculus results stated in the language of forms. Firstly recall that if we have a manifold $M$ with volume form $\omega_M$ and orientable submanifold $\Sigma$ with normal $N$ we get a volume form on $\Sigma$ given by 
\eq{
\omega_\Sigma = i_N \omega_M 
}
If we parametrise $\Sigma$ on an open patch $U$, that is find $\psi:\mbb{R}^d \to U \subset \Sigma$, and do an area integral over $U \subset \Sigma$ in these coordinates, that is equivalent to pulling back $\omega_\Sigma$ by $\psi$. \\
By noting that, on $S_\infty^2$, $\phi:S_\infty^2 \to S_1^2 \subset \mf{su}(2)$, where on this $S^2$ $\phi$ is also the normal, and that we have chosen the metric on $\mf{su}(2)$ to be Euclidean, we have that 
\eq{
\int_{S_\infty^2} \pangle{\phi \wedge (d\phi \wedge d\phi)} = \int_{S_\infty^2} \bm{\phi} \cdot \pround{\del_u \bm{\phi} \times \del_v \bm{\phi}} du dv = \int_{S_\infty^2} \phi^\ast \omega_{S^2} = 4\pi \deg \phi 
}
where we have the used the notation $\phi = \bm{\phi} \cdot \bm{t}$, and the degree here refers to of $\phi$ as a map of spheres. This degree is a topological invariant.
\item Chern class of a line bundle - Consider the matrix $\bm{\phi} \cdot \bm{\sigma}$. We give the following lemma
\begin{lemma}
	Let $P = \frac{1}{2}\pround{I + \bm{\phi} \cdot \bm{\sigma}}$. Then $P$ is a projection operator, i.e. $P^2=P$. 
\end{lemma}
\begin{proof}
	Do the multiplication
\end{proof}
\begin{corollary}
	$\bm{\phi} \cdot\bm{\sigma}$ has eigenvalues $\pm 1$. 
\end{corollary}
\begin{proof}
	Projection operators have eigenvalues $0,1$. Hence we get the result, and as we can never have $\bm{\phi} \cdot\bm{\sigma} = \pm I$, both eigenvalues must occur. 
\end{proof}
Now we can consider the eigenvector bundle 
\eq{
L = \pbrace{(\bm{x}, \psi) \in S_\infty^2 \times \mbb{C}^2 \, | \, \psquare{\bm{\phi}(\bm{x}) \cdot\bm{\sigma}} \psi = \psi}
}
\begin{remark}
	This will be a complex line bundle, so the only possible Chern class we can relate to it will be $c_1$. 
\end{remark}
We can get a connection on the bundle by viewing it as a subbundle of $S_\infty^2 \times \mbb{C}^2$, and getting covariant derivative on vector fields by projection, that is $DX = PdX$.
\begin{remark}
	Note that here the variations in $\psi$ that preserve the the eigenvector condition are exactly those that scale $\psi$ by some element of $\mbb{C}^\times$. This really makes the bundle look like the tangent bundle to the sphere $S_\infty^2$ in the sense that there is a plane attached at every point of $S^2$. 
\end{remark} 
Now at every point $\bm{x} \in S_\infty^2$ if $\hat{\psi}(\bm{x})$ is a normalised eigenvector we have that a local section $\sigma : U \subset S_\infty^2 \to E$ is given by $\sigma = h \hat{\psi}$ where $h:U \to \mbb{C}$ is some scale. Then as we are dealing with $2\times 2$ non-trivial projection matrices we can write $P = \hat{\psi} \hat{\psi}^\dagger$ giving 
\eq{
D(h\hat{\psi}) &= \hat{\psi} \hat{\psi}^\dagger \psquare{(dh)\hat{\psi}+h(d\hat{\psi})} \\
&= (dh)\hat{\psi}+h\hat{\psi} \hat{\psi}^\dagger(d\hat{\psi})
}
What we have done by choosing a local section $\hat{\psi}$ is given a local trivialisation of the bundle. To find the connection locally we need to see how the covariant derivaitve acts locally on $\mbb{C}$, that is how it acts on $h$. By writing 
\eq{
	D(h \hat{\psi}) = \hat{\psi} \psquare{dh + h \hat{\psi}^\dagger d\hat{\psi}}
}
We can read off that $A = \hat{\psi}^\dagger d\hat{\psi}$ locally, giving $F_L = d\hat{\psi}^\dagger \wedge d\hat{\psi}$. \\
We now want to relate this to our map $\phi$. As we are dealing with $2\times 2$ non-trivial projection matrices we can write 
\eq{
P = \hat{\psi} \hat{\psi}^\dagger \Rightarrow \phi = -i\hat{\psi} \hat{\psi}^\dagger + \frac{i}{2}I \Rightarrow d\phi = -i\psquare{(d\hat{\psi})\hat{\psi}^\dagger +\hat{\psi} (d\hat{\psi}^\dagger)}
}
We can see that, as the inner product corresponds to the trace, $\pangle{d\phi \wedge d\phi} = \pangle{(\del_\mu \phi)(\del_\nu \phi)} dx^\mu \wedge dx^\nu = \frac{1}{2}\pangle{\comm[\del_\mu \phi]{\del_\nu \phi}} dx^\mu \wedge dx^\nu = 0$ which means 
\eq{
	\pangle{\phi \wedge (d\phi \wedge d\phi)} &= -i\pangle{\pround{\hat{\psi} \hat{\psi}^\dagger -\frac{1}{2}I}(d\phi \wedge d\phi)} \\ 
	&= -i \pangle{\hat{\psi} \hat{\psi}^\dagger\psquare{(d\hat{\psi})\hat{\psi}^\dagger + \hat{\psi}(d\hat{\psi}^\dagger)}^{\wedge 2}}
}
We expand out the latter, using the cyclicity of the trace and Leibniz' rule with $\hat{\psi}^\dagger \hat{\psi}=1$, for example
\eq{
\pangle{\hat{\psi} \hat{\psi}^\dagger (d\hat{\psi})\hat{\psi}^\dagger \wedge (d\hat{\psi})\hat{\psi}^\dagger} &= \pangle{\hat{\psi}^\dagger\hat{\psi} \hat{\psi}^\dagger (d\hat{\psi})\hat{\psi}^\dagger \wedge (d\hat{\psi})} \\
&= \pangle{\hat{\psi}^\dagger (d\hat{\psi})\hat{\psi}^\dagger \wedge (d\hat{\psi})} \\
&= \pangle{\psquare{d(\hat{\psi}^\dagger \hat{\psi} ) - (d\hat{\psi}^\dagger)\hat{\psi} }\hat{\psi}^\dagger \wedge (d\hat{\psi})} \\
&= - \pangle{(d\hat{\psi}^\dagger)\hat{\psi} \wedge \hat{\psi}^\dagger (d\hat{\psi})} \\
&= - \pangle{\hat{\psi} \hat{\psi}^\dagger \hat{\psi} (d\hat{\psi}^\dagger) \wedge \hat{\psi} (d\hat{\psi}^\dagger)}
} 
\eq{
 \pangle{\hat{\psi} \hat{\psi}^\dagger (d\hat{\psi}) \wedge (d\hat{\psi}^\dagger)}&= -\pangle{\hat{\psi}(d\hat{\psi}^\dagger) \hat{\psi} \wedge (d\hat{\psi}^\dagger) } \\
 &= \pangle{\hat{\psi}(d\hat{\psi}^\dagger) \wedge (d\hat{\psi})\hat{\psi}^\dagger } \\
 &= \pangle{d\hat{\psi}^\dagger \wedge d\hat{\psi}} \\
 &= \pangle{\hat{\psi} \hat{\psi}^\dagger \hat{\psi} (d\hat{\psi}^\dagger) \wedge (d\hat{\psi})\hat{\psi}^\dagger}
}
and as a result we get 
\eq{
\pangle{\phi \wedge (d\phi \wedge d\phi)} = -2i \pangle{ d\hat{\psi}^\dagger \wedge d\hat{\psi}} = -4\pi \pround{\frac{i}{2\pi} \pangle{F_L}}
}
We can then see that 
\eq{
	\int_{S_\infty^2} \pangle{\phi \wedge (d\phi \wedge d\phi)} = -4 \pi c_1(L)
}
\end{enumerate} 
\begin{remark}
	A point to be made about the above is that, a priori, the connection is not related to the Higgs field on $S_\infty^2$. It is the decay condition on $D\phi$ which enforces that on $S_\infty^2$ we have $\del_\mu \phi = \comm[\phi]{A_\mu}$.
\end{remark}


Through either discussion we have $E\geq \pm 8\pi k$ for some $k \in \mbb{Z}$ with equality iff $F=\mp \star D \phi$ where we choose the sign to make the bound positive. This is the \bam{BPS equation}.
\begin{remark}
	We might wonder what is the connection between these two viewpoints, where we have that $\deg \phi = -c_1(L)$. To make this connection we recall a result stated in \cite{Babelon2003}:
	\begin{prop}
		If $L$ is a complex line bundle on a Riemann surface and $f$ is a meromorphic section then 
		\eq{
	\deg(D(f))	= c_1(L)
	}
	\end{prop}
\begin{proof}
	See the Riemann Surfaces notes by Joel Robbin, University of Wisconsin, which in turn reference \cite{Griffiths2014}. 
\end{proof}
We can think of $L$ as a complex line bundle over $\mbb{P}^1$ viewed as the Riemann sphere. If we take a section $\sigma = h \hat{\psi}$ where we choose $h=\hat{\psi}_2$, this section has poles exactly where the eigenvector is $(1,0) \Rightarrow \bm{\phi} = (1,0,0)$, and has no zeros. Hence the Chern class is counting the number of preimages of $(1,0,0)$, that is 
\eq{
-c_1(L) = \abs{\pbrace{\text{preimages of $(1,0,0)$}}}
}
This gives $c_1(L)=\deg \phi$, \hl{provided we have an argument to tell us that these all should be counted with sign 1}. 
\end{remark}

%%%%%%%%%%%%%%%%%%%%%%%%%%%%%%%%%%%%%%%%%%%%%%%%%%%%%%%%
\subsection{Self-Dual Reduction}
Suppose now we consider pure Yang-Mills on $\mbb{R}^{n+1}$ with contant diagonal metric $g$ (i.e we are considering it to be either Minkowski or Euclidean). Take coordinates $x^\mu$ and ask that the connection $A = A_\mu dx^\mu$ is $x^0$-independent. Then writing $A = \phi dx^0 + A_i dx^i$ we have 
\eq{
F &= \frac{1}{2} F_{ij} dx^i \wedge dx^j + (D_i \phi) dx^i \wedge dx^0 \\
&= {\indices{^3}F} + D\phi \wedge dx^0 
}
as $F_{i0} = \del_i \phi + \comm[A_i]{\phi} = D_i \phi$ and denoting $\indices{^3}F = \frac{1}{2}F_{ij} dx^i \wedge dx^j$. We calculate
\eq{
\pangle{\indices{^3}F,D\phi \wedge dx^0} &= \frac{1}{2}\pangle{F_{ij}, D_k \phi} \underbrace{\pangle{dx^i \wedge dx^j , dx^k \wedge dx^0}}_{=0} \\
\pangle{D\phi \wedge dx^0,D\phi \wedge dx^0} &= \pangle{D_i \phi, D_j \phi} \underbrace{\pangle{dx^i \wedge dx^0, dx^j \wedge dx^0} }_{=g^{ij}g^{00}}
}
and so on Euclidean $\mbb{R}^{n+1}$, $\abs{F}^2 = \abs{\indices{^3}F}^2 + \abs{D\phi}^2 $. This means we recover the action for Yang-Mills-Higgs with $0$ potential from this reduction. 
\begin{remark}
	It is not absurd to consider Euclidean $\mbb{R}^4$, as we should view this as performing a Wick rotation from Minkowski space, which is natural when quantising the theory as it means that the path integral is now well defined (see \cite{Ward1991} for more of a discussion on this). 
\end{remark}
The Yang-Mills equation for $F$ reads 
\eq{
D \star F = 0 \Rightarrow g^{\mu\nu}D_\mu F_{\nu\rho} &= 0 
}
This splits to give 
\eq{
g^{\mu\nu}D_\mu F_{\nu 0} = 0 &\Rightarrow g^{ij} D_i D_j \phi = D_i D^i \phi = 0 \\
g^{\mu\nu} D_\mu F_{\nu k} = 0 &\Rightarrow -g^{00}D_0D_k \phi + g^{ij} D_i F_{jk} =0 \\
& \phantom{==} \;  -g^{00}\comm[\phi]{D_k \phi} +D_i F\indices{^i_k} = 0   
}
Moreover we have the stronger result:
\begin{prop}
	$F$ is (anti-)self-dual iff $(\indices{^3}F,\phi)$ satisfy the Bogomolny equations. 
\end{prop}
\begin{proof}
	The (anti-)self-duality equations for $F$ say $\star_4 F = (-)F$, where we are now making explicit the dimension wrt which $\star$ is acting. We can calculate 
	\eq{
\star_4 F &= \frac{1}{4} F_{ij} \eps\indices{^i^j_\mu_\nu} dx^\mu \wedge dx^\nu + \frac{1}{2} (D_i \phi) \eps\indices{^i^0_\mu_\nu} dx^\mu \wedge dx^\nu \\
&= -\frac{1}{2}(D_k\phi) \epsilon\indices{^k_i_j} dx^i \wedge dx^j - \frac{1}{2} F_{ij}\epsilon\indices{^i^j_k} dx^k \wedge dx^0 \\
&= -\star_3 D\phi - \star_3 F \wedge dx^0	
}
which we see to mean, using $\star_3^2=1$.  
\eq{
\star_4 F = \pm F \Leftrightarrow {}\indices{^3}F\pm \star_3 D \phi = 0
}
\end{proof}
%%%%%%%%%%%%%%%%%%%%%%%%%%%%%%%%%%%%%%%%%%%%%%%%%%%%%%%%
%%%%%%%%%%%%%%%%%%%%%%%%%%%%%%%%%%%%%%%%%%%%%%%%%%%%%%%%
\section{Self-Dual Gauge Fields}
We now want to study self-dual gauge fields more, using Ward \& Wells section 8 as our reference. For this section we will denote the gauge potential as $\Phi$ as $A$ will be a spinor index eventually, so this should not be confused with a Higgs field. The gauge fields we will be studying we now extend to be those over $M_\mbb{C}$. As such we will be using the machinery of the twistor correspondence. 
\begin{definition}
	An open set $U \subset M_\mbb{C}$ is \bam{elementary} if $\forall \tilde{Z}$ a self-dual plane (i.e. an $\alpha$-plane) s.t. $\tilde{Z} \cap U \neq \emptyset$, $\tilde{Z} \cap U$ is connected and simply connected. 
\end{definition}
This lets us give a novel relation to self-dual curvature:
\begin{prop}
	The curvature $F$ on a given elementary $U$ is anti-self-dual iff $\forall$ self dual $\tilde{Z}$ that intersects $U$, the restriction of the covariant derivative $D$ to $\tilde{Z} \cap U$ is integrable. 
\end{prop}
\begin{proof}
	Because $\tilde{Z} \cap U$ is connected and simply connected, the covariant derivative being integrable is equivalent to the curvature being flat, that is 
	\eq{
\forall v,w \in T\tilde{Z}, \, v^a w^b F_{ab} = 0	
}
Suppose $\tilde{Z}$ corresponds to a projective twistor $Z = (\omega^A, \pi_{A^\prime})$. Then as $\tilde{Z}$ is null, a tangent vector to $\tilde{Z}$ will be null hence of the form $v^a = \lambda^A \pi^{A^\prime}$ for some spinor $\lambda$. This means we get the condition 
\eq{
(\phi_{A^\prime B^\prime}\eps_{AB} + \phi_{A B} \eps_{A^\prime B^\prime}) \lambda_1^{A} \lambda_2^B \pi^{A^\prime} \pi^{B^\prime} = 0
}
The term $\eps_{A^\prime B^\prime}\pi^{A^\prime} \pi^{B^\prime}$ vanishes, and then for this to vanish $\forall \lambda_1, \lambda_2, \tilde{Z}$, we have that $\phi_{A^\prime B^\prime}=0$, i.e. the self dual part of $F$ is 0. 
\end{proof}
This in turn leads to a useful equivalence coming from the twistor geometry. 
\begin{theorem}
	Let $\hat{U} \subset \mbb{PT}$ be the open subset s.t $\forall Z \in \hat{U}, \tilde{Z} \cap U \neq \emptyset$, then there is a bijection (up to equivalence) between
	\begin{enumerate}
		\item anti-self-dual $GL_n(\mbb{C})$-bundles on $U$ 
		\item holomorphic rank-n vector bundles $E\to \hat{U}$ s.t. $E$ is trivial over $\hat{x}$ for each $x \in U$. 
	\end{enumerate}
\end{theorem}
\begin{proof}
	Ward and Wells \cite{Ward1991} gives a very thorough proof of this, we will only sketch it. \\
	We define the map $1 \mapsto 2$ by making the fibres of $E$ to be 
	\eq{
E_Z = \pbrace{\psi \, | \, D\psi=0 \text{ on }\tilde{Z}}	
}
As the bundle is ASD, the covariant derivative is integrable and so a value of $\psi$ is defined by its value at a point $x\in U$. Hence $E_z \cong \mbb{C}^n$. Moreover, any $Z^\prime$ s.t $\tilde{Z}^\prime$ intersects $x$ will also have its value of $\psi$ fixed by the value of $\psi$ on $x$. Hence $\ev{E}{\hat{x}}$ is trivial. \\
To map $2 \mapsto 1$ note as the vector bundle $E$ is trivial over $\hat{x}$ we can get the rank $n$ vector bundle $\tilde{E} \to U$ by taking the copy of $\mbb{C}^n$ over $\hat{x}$ in $\ev{E}{\hat{x}}$. To define a connection we need a way to parallel transport. Given $x,y \in \tilde{Z}$ we know they have corresponding lines $\hat{x},\hat{y} \subset U$ s.t. $\hat{x} \cap \hat{y} = Z$. As $E$ is trivial over $\hat{x},\hat{y}$, we may transport along these lines under the connection induced from the base $M$ (sometimes called the \bam{spacetime connection}, denoted $\nabla$) via $Z$ to transport from $x$ to $y$. This connection will turn out to satisfy the conditions we need. 
\end{proof}

\begin{remark}
	There are many versions of this theorem that consider either bundles over $S^4$ or bundles with different gauge groups. To see all of these check out Ward \& Wells \S 8.1
\end{remark}

The theorem offers a potential way to construct solutions to the ASD Yang Mills equations via holomorphic vector bundles. These are vector bundles on $\hat{U} \subset \mbb{PT}$ which can be covered with two charts $W, \underline{W}$ (analogous to the north/south charts on the Riemann sphere, s.t. on any line $\hat{x}$ they are are these chats so their intersection contains the line $W \cap \underline{W} \cap \hat{x}$), and so they are defined by a single holomorphic $n\times n$ matrix transition function $F(Z) = F(\omega^A,\pi_{A^\prime})$ on $W \cap \underline{W}$. This acts as
\eq{
\underline{\xi} = F(Z) \xi
} 
where $\xi,\underline{\xi}$ are coordinate vectors on $W,\underline{W}$ respectively. To do the transport along $\hat{x}$ we need to restrict to the line, that is take 
\eq{
G(x,\pi) = F(ix^{AA^\prime}\pi_{A^\prime},\pi_{A^\prime})
}
and then find $H, \underline{H}$ non-singular $n\times n$ matrices holomorphic on $W \cap \hat{x}, \, \underline{W} \cap \hat{x}$ respectively s.t. on $W \cap \underline{W} \cap \hat{x}$  
\begin{equation}\label{eq:splitting formula}
G = \underline{H} H^{-1} 
\end{equation}
This is called the \bam{spliting formula}. Such $H,\underline{H}$ must exists as the triviality of $\ev{E}{\hat{x}}$ means that we know 
\eq{
\xi = H \psi, \quad \underline{\xi} = \underline{H} \psi
}
for some constant vector $\psi$. 
\begin{prop}
	With the above notation we find the connection on the bundle to be given by 
	\eq{
\pi^{A^\prime} \Phi_{AA^\prime} = H^{-1} \pi^{A^\prime} \nabla_{AA^\prime}H	
}
\end{prop}
%%%%%%%%%%%%%%%%%%%%%%%%%%%%%%%%%%%%%%%%%%%%%%%%%%%%%%%%
%%%%%%%%%%%%%%%%%%%%%%%%%%%%%%%%%%%%%%%%%%%%%%%%%%%%%%%%
\section{Constructions}
%%%%%%%%%%%%%%%%%%%%%%%%%%%%%%%%%%%%%%%%%%%%%%%%%%%%%%%%
\subsection{Nahm's Construction}
Relevant reading for this section is Manton \& Sutcliffe \cite{Manton2004} and Hitchin \cite{Hitchin1983}. The original paper is \cite{Nahm1983}. 

%%%%%%%%%%%%%%%%%%%%%%%%%%%%%%%%%%%%%%%%%%%%%%%%%%%%%%%%
\subsection{The ADHM construction}
This section follows the work first laid out in \cite{Atiyah1978}. Suppose we have the following information:
\begin{itemize}
	\item $W$ a $k$-dimensional vector space
	\item $V$ a $2k+2$-dimensional vector space with skew, non-degenerate bilinear form $(\cdot, \cdot):\wedge^2 V \to \mbb{C}$. 
	\item $z = (z_i) \in \mbb{C}^4$
	\item $A(z) = \sum_i A_i z_i \in \End(W,V)$ s.t. \
	\eq{
\forall z \neq 0, \; U_z \equiv A(z)W\subset V \text{ is isotropic and $k$-dimensional}	
}
\end{itemize}

We now state some important properties:

\begin{lemma}
	Let $E_z = \faktor{U_z^0}{U_z}$, then 
	\begin{itemize}
		\item $\dim E_z = 2$
		\item $E_z$ inherits a non-degenerate skew bilinear
		\item $\forall \lambda \in \mbb{C}^\times, \, E_z = E_{\lambda z}$. 
	\end{itemize}
\end{lemma}
\begin{proof}
	We go point by point:
	\begin{itemize}
		\item $\dim E_z = \dim U_z^0 - \dim U_z = \pround{\dim V - \dim U_z} - \dim U_z = 2k+2 - 2k = 2$.
		\item The bilinear on $W$ is only degenerate in $U_z^0$ on $U_z$, so by quotienting by this it descends directly to $E_z$. 
		\item $A(\lambda z) = \lambda A(z)$, so $A(\lambda z) (\lambda^{-1} \bm{w}) = A(z)(\bm{w})$. Hence we can see $U_{\lambda z} = U_z$ and so result.  
	\end{itemize}
\end{proof}

\begin{corollary}
	We get a vector bundle $E\to \mbb{CP}^3$ with group $SL(2,\mbb{C})$.
\end{corollary}


\hl{A break to introduce Bernd's notes}

\subsubsection{Connection from Projection}
Given an rank-$n$ vector bundle $E \to M$ a subbundle of $\mbb{R}^{n+k} \times M$. We can decompose $E_x + E_x^\perp = \mbb{R}^{n+k}$ and define projectors $P,Q$ onto $E,E^\perp$ respectively. Then 
\eq{
E = \pbrace{(x,v) \in M\times \mbb{R}^{n+k} \, | \, P_x(v) = v}
}
and sections are maps $x \mapsto (x,f(x))$ s,t $P_xf(x) = f(x)$. We define a connection on $E$ via 
\eq{
Df = P df
}
Now we want to pick a gauge via $u_x: \mbb{R}^n \to \mbb{R}^{n+k}$, i.e $u^\dagger u = I$, and so we can write $P = uu^\dagger$. Then if we write our section as $f=ug$ for $g:U \subset M \to \mbb{R}^n$. Then
\eq{
Df = uu^\dagger \psquare{(du)g + u(dg)} = u\psquare{dg + u^\dagger (du)g}
}
and we can read off that the connection must be $A = u^\dagger du$. and so 
\eq{
D = du^\dagger \wedge du + u^\dagger du \wedge u^\dagger du 
}
Now in terms of the other projection, we have the following results:
\begin{lemma}
	$dQ = Q(dQ) + (dQ)Q$ and $(dQ)f = -Q(df)$. 
\end{lemma} 
If we now define $B = QdQ$ we get for $f: M \to \mbb{R}^{n+k}$ 
\eq{
D_B f = df + Bf = df + (dQ)f - (dQ)Q f
}
On sections of $E$ we have $Qf=0$ and so $D_B f = Df$. As we can calculate $F_b = dQ \wedge dQ$ we must get 
\eq{
F = P(dQ) \wedge (dQ)P
}
Choose a gauge fo $E^\perp$ $v:\mbb{R}^k \to \mbb{R}^{n+k}$ with $u^\dagger v= 0$. Letting $\rho = v^\dagger v$ we get 
\eq{
Q = v \rho^{-2} v^\dagger
}
giving 
\eq{
F = \dots = P(dv)\rho^{-2} \wedge (dv^\dagger)P 
}
\subsubsection{Quaternionic Bundles}
We now take $M = S^4 = \mbb{HP}^1$ with matrix multiplication form the left but scalar $\mbb{H}$ action on the right. Our plan will be to try find good $v$ s.t. we can get $F$ (anti)-self dual, and $\bar{F} = -F$ to ensure $F$ is $\mf{su}(2)$-valued. \\
The ADHM idea is
\begin{enumerate}
	 \item take $v(x,y) = Cx+Dy$ for $x,y \in \mbb{H}, \, C,D \in \mbb{H}^{(k+n)k}$.  
	 \item Assume maximal rank for $(x,y) \neq 0$. Then image of $v$ is a $k$-dimensional subspace of $\mbb{H}^{n+k}$ depending on $xy^{-1}$. 
	 \item Project to $\mbb{R}^4 \subset S^4$ where $y \neq 0 $ and take affine coordinate $(x,1)$
	 \item assume $\rho^2 = (\bar{x}C^\dagger + D^\dagger)(Cx+d)$ is real for $x \in \mbb{H}$.  
\end{enumerate}
We now write $L$ for the tautological quaternionic line bundle over $\mbb{HP}^1$ with $c_2 = -1$. Take 
\eq{
E^\perp = \underbrace{L \oplus \dots \oplus L}_{\times k}
}
Then since $E \oplus E^\perp$ is trivial and Chern number adds we have $c_2(E) = k$. Thus taking $n=1$,
\eq{
F = PC(dx) \rho^{-2} \wedge (d\bar{x})C^\dagger P
} 
is an $SU(2)$-instanton with charge $k$.
\begin{remark}
	We needed $n=1$ as we actually get an $Sp(n)$ bundle
\end{remark}
Now we can pick a gauge where 
\eq{
v(x) = \begin{pmatrix} \Lambda \\ B-xI_k \end{pmatrix}
}
with $n\times k$ quaternionic matrix $\Lambda$ and $k\times k$ constant quaternionic matrix $B$. We have to solve the non-linear condition 
\eq{
\bar{\Lambda^\dagger \Lambda + B^\dagger B} = \Lambda^\dagger \Lambda + B^\dagger B
}
for symmetric $B$. We then solve the linear, $x$-dependent matrix equation $v^\dagger u = 0$ for u to find an explicit formula for $A,F$. 
%%%%%%%%%%%%%%%%%%%%%%%%%%%%%%%%%%%%%%%%%%%%%%%%%%%%%%%%
%%%%%%%%%%%%%%%%%%%%%%%%%%%%%%%%%%%%%%%%%%%%%%%%%%%%%%%%
\bibliographystyle{../../bib/custom-bib-style}
\bibliography{../../bib/jabref_library.bib}

\end{document}
