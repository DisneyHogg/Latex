\documentclass{article}

\usepackage{../../header}
%%%%%%%%%%%%%%%%%%%%%%%%%%%%%%%%%%%%%%%%%%%%%%%%%%%%%%%%
%Preamble

\title{Monopoles}
\author{Linden Disney-Hogg}
\date{November 2020}

%%%%%%%%%%%%%%%%%%%%%%%%%%%%%%%%%%%%%%%%%%%%%%%%%%%%%%%%
%%%%%%%%%%%%%%%%%%%%%%%%%%%%%%%%%%%%%%%%%%%%%%%%%%%%%%%%
\begin{document}

\maketitle
\tableofcontents

%%%%%%%%%%%%%%%%%%%%%%%%%%%%%%%%%%%%%%%%%%%%%%%%%%%%%%%%
%%%%%%%%%%%%%%%%%%%%%%%%%%%%%%%%%%%%%%%%%%%%%%%%%%%%%%%%
\section{Introduction}
%%%%%%%%%%%%%%%%%%%%%%%%%%%%%%%%%%%%%%%%%%%%%%%%%%%%%%%%
\subsection{Preamble}
I already have notes on Gauge Theory, Algebraic Geometry, Solitons, and Algebraic Topology, but I have yet to actually make any notes on Monopoles. The purpose of these notes is to be a comprehensive cover of the knowledge required to understand \cite{Braden2018}. This will include previous works by Atiyah, Donaldson, Hitchin, Nahm, and more.
%%%%%%%%%%%%%%%%%%%%%%%%%%%%%%%%%%%%%%%%%%%%%%%%%%%%%%%%
\subsection{Preliminaries}
As with all my projects, the preliminaries will undoubtably end up being too long, but I will try keep this minimal this time: 
\begin{definition}
	The \bam{annihilator} of $U \leq V$ is 
\eq{
U^0 = \pbrace{f \in V^\ast \, | \, \forall u \in U, \, f(u)=0} \leq V^\ast
} 
If $V$ has bilinear $\pangle{\cdot, \cdot}$ we can use the isomorphism of $V^\ast\cong V$ to understand 
	\eq{
U^0 = \pbrace{v \in V \, | \, \forall u \in U, \, \pangle{u,v}=0} \leq V	
}
\end{definition}

\begin{lemma}
	THe annihilator is a subspace, $\dim U^0 = \dim V  - \dim U$. 
\end{lemma}

\begin{definition}
	A subspace $U$ is called \bam{isotropic} if $U \subset U^0$. 
\end{definition}
%%%%%%%%%%%%%%%%%%%%%%%%%%%%%%%%%%%%%%%%%%%%%%%%%%%%%%%%
\subsection{The Dirac Monopole}
The standard maxwell equations prohibit monopoles, by which we mean point magnetic field sources, as $\bm{\nabla}\cdot \bm{B}=\bm{0}$. Dirac showed in \cite{Dirac1931} that it is possible to escape this conclusion by giving non-trivial topology to the space by allowing $\bm{B} = \frac{g}{4\pi r^2}\hat{\bm{x}}$ to have a singularity at $\bm{x}=\bm{0}$. We can calculate $\bm{\nabla} \cdot \bm{B} = g\delta(\bm{x})$. Removing this circle gives $\mbb{R}^3\setminus 0$, homotopic to $S^2$, and the corresponding magnetic two form on this sphere is
\eq{
f = \frac{g}{4\pi} \sin \theta d\theta \wedge d\phi
} 
and so the flux through a 2-sphere enclosing the origin is $\int_{S_R^2} f = g $. For $g \neq 0$, we know $f \neq da$ for a global $a \in \Omega^1(S^2)$ by Stokes' theorem, but if we take a cover of the sphere $U_{N/S}$ (north/south) and define gauge potentials
\eq{
a_N &= \frac{g}{4\pi}(1-\cos\theta )d\phi \in \Omega^1(U_N) \\
a_S &= \frac{g}{4\pi}(-1-\cos\theta) d\phi \in \Omega^1(U_S)
}
On the intersect $U_N \cap U_S$ we have $da_N = f = da_S$ and $a_N = a_S + \frac{g}{2\pi}d\phi$. \\
Now taking $A = ia, \, F = if$, we have that $g_{NS}(\theta,\phi) = e^{-i\frac{g\phi}{2\pi}}$. Requiring that this is a well-defined transition function gives $g \in \mbb{Z}$. This is equivalent to the integrality of the Chern number. \\
We will not want to consider this as this solution is not solitonic (it has infinite mass), but for further discussion see \cite{Manton2004}. 

%%%%%%%%%%%%%%%%%%%%%%%%%%%%%%%%%%%%%%%%%%%%%%%%%%%%%%%%
\subsection{Pauli Matrices}

\begin{definition}[Pauli Matrices]
	The \bam{Pauli matrices} are
	
	\begin{align*}
		\sigma_1 &= \begin{pmatrix} 0 & 1 \\ 1 & 0\end{pmatrix}, & 
		\sigma_2 &= \begin{pmatrix} 0 & -i \\ i & 0\end{pmatrix}, &  
		\sigma_3 &= \begin{pmatrix} 1 & 0 \\ 0 & -1\end{pmatrix}.  
	\end{align*}
	Note they are all Hermitian and traceless.
\end{definition}

\begin{fact}$\sigma_i \sigma_j = \delta_{ij}I +i\epsilon_{ijk}\sigma_k \Rightarrow \tr(\sigma_i \sigma_j) = 2\delta_{ij}$
\end{fact}

%%%%%%%%%%%%%%%%%%%%%%%%%%%%%%%%%%%%%%%%%%%%%%%%%%%%%%%%
\subsection{\secmath{SU(2)}}
We can write 
\[
SU(2)=\pbrace{    \begin{pmatrix} \alpha & -\overline{\beta} \\ \beta & \overline{\alpha} \end{pmatrix}  : \alpha,\beta\in\mathbb{C} , |\alpha|^2+|\beta|^2=1   }
\]
This can be expressed as, for $A\in SU(2)$
\[
A=a_0 I +i\bm{a}\cdot\bm{\sigma}
\]
where $\bm{a}=(a_1, a_2, a_3)$, $\bm{\sigma}=(\sigma_1, \sigma_2, \sigma_3)$, and $a_0^2+|\bm{a}|^2=1$. Hence $SU(2)\cong S^3$. In addition, by parametrising $SU(2)$ by the $a_i$, it can be seen that $\set{  i\sigma_i }$ forms a basis of $\mf{su}(2)$. It is typical to normalise this basis to $\set{  T^a=-\frac{1}{2} i\sigma_a  }$. 

\begin{lemma}
	The structure constants in this basis $\set{  T^a  }$ are $f^{ab}_c=\epsilon_{abc}$.
\end{lemma}

\begin{corollary}
	The Killing form is given by $\kappa\pround{T^a,T^b} = \kappa^{ab} = -2\delta^{ab} = -\tr(T^aT^b)$. Hence $\kappa(X,Y) = -\tr(XY)$
\end{corollary}

%%%%%%%%%%%%%%%%%%%%%%%%%%%%%%%%%%%%%%%%%%%%%%%%%%%%%%%%
\subsection{The Hodge star operator}
\begin{definition}
	A \bam{volume form} of a differentiable manifold $M$ is a top-degree form (i.e. form of degree $\dim M$). 
\end{definition}

\begin{definition}
	An \bam{oriented} manifold is one equipped with a nowhere-vanishing volume form. 
\end{definition}

\begin{lemma}
	Every Riemannian manifold $(M,g)$ has a canonical choice of volume form given in local coordinates $x^1, \dots, x^n$ by 
	\eq{
		\omega_{vol} = \sqrt{\abs{g}} dx^1 \wedge \dots \wedge dx^n
	} 
\end{lemma}

\begin{lemma}
	If $V$ is a vector space with bilinear $\pangle{\cdot, \cdot}$ then $\bigwedge^k V$ inherits a bilinear by the Grammian determinant. 
	\eq{
		\pangle{\alpha_1 \wedge \dots \wedge \alpha_k, \beta_1 \wedge \dots \wedge \beta_k} = \det \pround{\pangle{\alpha_i,\beta_j}_{i,j=1}^k}	
	}
\end{lemma}

\begin{definition}
	If $M$ is a $n$-dimensional Riemannian manifold, the \bam{Hodge star} is the operator $\star : \Omega^r(M) \to \Omega^{n-r}(M)$ given by 
	\eq{
		\alpha \wedge (\star \beta) = \pangle{\alpha,\beta} \omega_{vol}
	}
	for $\alpha,\beta \in \Omega^r(M)$, where $\pangle{\cdot,\cdot}$ is the Euclidean structure on $\Omega^r(M)$. 
\end{definition}

\begin{lemma}
	This property defines $\star$ completely. 
\end{lemma}


\begin{prop}
	If $M$ has determinantal sign $\Delta$ of the inner product, then 
	\eq{
		\star \star \alpha = (-1)^{r(n-r)} \Delta \alpha	
	}
\end{prop}

%%%%%%%%%%%%%%%%%%%%%%%%%%%%%%%%%%%%%%%%%%%%%%%%%%%%%%%%
\subsection{Degree of a Map}
Consider a smooth map $n:S_s^2 \to S_t^2$. The number of preimages of $p \in S_t^2$ computed with sign is calculate via pullback to be 
\eq{
	\deg(n) = \frac{1}{\int_{S_t^2} \omega} \int_{S_s^2} n^\ast \omega
}
for some normalisable volume form on $S_t$. Then 
%%%%%%%%%%%%%%%%%%%%%%%%%%%%%%%%%%%%%%%%%%%%%%%%%%%%%%%%
%%%%%%%%%%%%%%%%%%%%%%%%%%%%%%%%%%%%%%%%%%%%%%%%%%%%%%%%
\section{The Monopole Equations}
%%%%%%%%%%%%%%%%%%%%%%%%%%%%%%%%%%%%%%%%%%%%%%%%%%%%%%%%
\subsection{Yang-Mills-Higgs equations}
\begin{definition}
Take a principal $G$-bundle $P\to M$, $\omega_{vol}$ an orientation on $M$, and $\pangle{\cdot, \cdot}$ to be an $\ad$-invariant inner product on $\mf{g}$. Then the \bam{Yang-Mills-Higgs actions} on $M$ is 
\eq{
S_{YMH}[A,\phi] = \int_M \psquare{\abs{F}^2 + \abs{D\phi}^2 + V(\phi)}\omega_{vol}
}
where $F=dA+A\wedge A$ is the curvature associated to a section $A\in \Gamma(T^\ast M \otimes \ad(P))$, $D = d+A$ is the associated covariant derivative, and $\phi \in \Gamma(\ad(P))$.
\end{definition}

\begin{remark}
	A common choice of potential function $V$ is $V(\phi) = \lambda \pround{1-\abs{\phi}^2}^2$, the \bam{$\phi^4$-potential}. 
\end{remark}

\begin{definition}
	A \bam{monopoles} will be a soliton-like solution to the Yang-Mills-Higgs equations when $G=SU(2)$, $M=\mbb{R}^4$ with the Minkowski metric, the principal bundle is $P = M \times G$, and the potential is $\phi^4$. 
\end{definition}

\begin{prop}
	The variational equations corresponding to $S_{YMH}$ are the\bam{Yang-Mills-Higgs equations}
	\eq{
DF &= 0 \quad \text{(Bianchi)} \\
 \star D \star F &= -\comm[\phi]{D\phi} \\
 \star D \star D\phi &= -V^\prime(\phi)	
}
\end{prop}
\begin{proof}
	We first consider the equation that comes from the variation of $A$. Let $A_t = A + t\beta$, then $F_t = F + t\pround{d\beta + \beta \wedge A + A \wedge \beta} + \mc{O}(t^2)$ and $D_t \phi = D \phi + t\comm[\beta]{\phi}$. Hence 
	\eq{
S_t = S + 2t\int_M \psquare{\pangle{F,D\beta} + \pangle{D\phi,\comm[\beta]{\phi}}} \omega_{vol} + \mc{O}(t^2)
}
Hence to be at a stationary point of the action variation we want
\eq{
\int_M \psquare{\pangle{F,D\beta} + \pangle{D\phi,\comm[\beta]{\phi}}} \omega_{vol} = 0
}
Using the fact that inner product is $\ad$-invariant and letting $D^\ast$ be the formal adjoint of $D$ wrt to inner product $\pangle{\pangle{\eta,\omega}} = \int_M \pangle{\eta,\omega} \omega_{vol}$ we can rewrite this as 
\eq{
\int_M \pangle{D^\ast F - \comm[D\phi]{\phi},\beta} \omega_{vol} = 0
}
Using results on the dual of the covariant derivative we can say that 
\eq{
D^\ast F = (-1)^{4(2-1)+1}(-1)\star D \star F = \star D \star F 
}
Hence as $\beta$ was a generic variation we must have $\star D \star F - \comm[D\phi]{\phi} =0$. \\
An alternative way to do the last step is to use the following:
\begin{lemma} $ d\pangle{\beta \wedge \star F} = \pangle{D\beta \wedge \star F} - \pangle{\beta \wedge D(\star F)} $
\end{lemma}
This means 
\eq{
\int_M \pangle{F,D\beta} \omega_{vol} &= \int_M \pangle{D\beta,F} \omega_{vol} \\
&= \int_M \pangle{D \beta \wedge \star F} \\
&= \int_M d\pangle{\beta \wedge \star F} + \pangle{\beta \wedge D \star F} \\
&= \pangle{\beta,\star D \star F}\omega_{vol} 
}
where we have used the (abelian) Stokes' theorem and that on $4d$ Minkowski space $(\star)^2 = (-1)^{3\times(4-3)+1}=1$ on 3-forms (e.g $D\star F$). \\
We now consider a $\phi$ variation so $\phi_t = \phi + t\psi$. Note 
\eq{
\abs{\phi_t} = \sqrt{\pangle{\phi_t,\phi_t}} = \sqrt{\abs{\phi}^2 + 2t\pangle{\phi,\psi} + \mc{O}(t^2)} = \abs{\phi}\sqrt{1 + \frac{2t\pangle{\phi,\psi}}{\abs{\phi}^2} + \mc{O}(t^2)} = \abs{\phi} + t\frac{\pangle{\phi,\psi}}{\abs{\phi}} + \mc{O}(t^2)
}
so if we consider $V$ as $V(\phi)=V(\abs{\phi})$ (i.e. as a function of a real variable) then 
\eq{
V(\phi_t) = V(\phi) + t\abs{\phi}^{-1} V^\prime(\phi) \pangle{\phi,\psi} + \mc{O}(t^2)
}
Then a variational argument as before means that we need to set 
\eq{
2D^\ast D \phi + \abs{\phi}^{-1} V^\prime(\phi) \phi &= 0 \\
\Rightarrow \star D \star D \phi + \frac{1}{2\abs{\phi}}V^\prime(\phi) \phi &= 0
}
With the $\phi^4$-potential, $V^\prime(\phi) = -4\lambda(1-\abs{\phi}^2)\abs{\phi}$. 
\end{proof}
\begin{remark}
	The Dirac monopole is a solution in the case of $G=U(1)$.
\end{remark}
We may make these equations explicit in coordinates. The first approach is to try and substitue in coordinate expressions into the YMH equations. We use that in general
\eq{
\star (dx^{\mu_1} \wedge \dots \wedge dx^{\mu_p}) = \frac{1}{(n-p)!}\epsilon\indices{^{\mu_1}^\dots^{\mu_p}_{\mu_{p+1}}_\dots_{\mu_n}} dx^{\mu_{p+1}} \wedge dx^{\mu_n} 
}
using summation notation, and so taking coordinates $x^\mu$ on $\mbb{R}^4$ and writing $F = \frac{1}{2}F_{\mu\nu} dx^\mu \wedge dx^\nu$
\eq{
\star F &= \frac{1}{4} F_{\mu\nu} \epsilon\indices{^\mu^\nu_\rho_\sigma} dx^\rho \wedge dx^\sigma \\
\Rightarrow D\star F &= \frac{1}{4} D_\tau F_{\mu\nu} \epsilon\indices{^\mu^\nu_\rho_\sigma} dx^\tau \wedge dx^\rho \wedge dx^\sigma \\
\Rightarrow \star D \star F &= \frac{1}{4} D_\tau F_{\mu\nu} \epsilon\indices{^\mu^\nu_\rho_\sigma} \epsilon\indices{^\tau^\rho^\sigma_\lambda} dx^\lambda \\
&= \frac{1}{4} D_\tau F_{\mu\nu} \eps\indices{_\rho_\sigma^\mu^\nu}\eps\indices{^\rho^\sigma^\tau_\lambda} dx^\lambda \\
&= D_\tau F\indices{^\mu_\nu} \delta^\tau_{[\mu}\delta^\nu_{\lambda]}dx^\lambda \\
&= D_\tau F\indices{^\tau_\lambda} dx^\lambda
}
Hence the first monopole equation reads 
\eq{
D_\mu F\indices{^\mu^\nu} -\comm[D^\nu \phi]{\phi} = 0
}
Next we have 
\eq{
\star D \phi &= \frac{1}{6} (D_\mu \phi) \eps\indices{^\mu_\nu_\rho_\sigma} dx^\nu \wedge dx^\rho \wedge dx^\sigma \\
\Rightarrow D\star D \phi &= \frac{1}{6} (D_\tau D_\mu \phi) \eps\indices{^\mu_\nu_\rho_\sigma} dx^\tau \wedge dx^\nu \wedge dx^\rho \wedge dx^\sigma \\
\Rightarrow \star D \star D \phi &= \frac{1}{6} (D_\tau D_\mu \phi) \eps\indices{^\mu_\nu_\rho_\sigma} \epsilon^{\tau \nu \rho \sigma} \\
&=(D_\tau D^\mu \phi) \delta_\mu^\tau = D_\mu D^\mu \phi
}
which yields 
\eq{
D_\mu D^\mu \phi -2\lambda (1-\abs{\phi}^2)\phi = 0
}
Collecting this with the Bianchi identity (taking it as $\star DF=0$) we have 
\eq{
\epsilon^{\rho\mu\nu\tau} D_\rho F_{\mu\nu} &=0 \\
D_\mu F\indices{^\mu^\nu} -\comm[D^\nu \phi]{\phi} &= 0 \\
D_\mu D^\mu \phi -2\lambda (1-\abs{\phi}^2)\phi &= 0
}
An alternative approach to deriving these equations is to first write the Lagrangian in coordinate form and then derive the variational equations. We take the inner product on $\mf{g} = \mf{su}(2)$ to be $\pangle{X,Y} = -\frac{1}{2}\kappa(X,Y) = -\frac{1}{2}\tr(XY)$ for concreteness, but note that it should not matter (\hl{why?}). We have 
\eq{
\pangle{F,F} &= \frac{1}{4}\pangle{F_{\mu\nu},F_{\rho\sigma}} \pangle{dx^\mu \wedge dx^\nu, dx^\rho \wedge dx^\sigma} \\
&= \frac{1}{4}\pangle{F_{\mu\nu},F_{\rho\sigma}}\pround{\eta^{\mu\nu}\eta^{\rho\sigma} - \eta^{\mu\sigma}\eta^{\nu\rho}} \\
&= \frac{1}{2}\pangle{F_{\mu\nu},F^{\mu\nu}} \\
&= -\frac{1}{4}\tr(F_{\mu\nu}F^{\mu\nu}) \\
\pangle{D\phi,D\phi} &= \pangle{D_\mu\phi,D_\nu\phi}\pangle{dx^\mu, dx^\nu} \\
&= \pangle{D_\mu \phi, D_\nu\phi} \eta^{\mu\nu} \\
&= -\frac{1}{2}\tr(D_\mu \phi D^\mu\phi)
}
Hence the corresponding Lagrangian density is 
\eq{
\mc{L} = -\frac{1}{4}\tr(F_{\mu\nu}F^{\mu\nu}) - \frac{1}{2}\tr(D_\mu \phi D^\mu \phi)+ V(\phi) = \sum_a \psquare{ \frac{1}{2} F_{\mu\nu}^{(a)} F^{(a)\mu\nu} + \pround{D_\mu \phi}^{(a)} \pround{D^\mu \phi}^{(a)}} + \lambda \psquare{1-\sum_a (\phi^{(a)})^2}^2 
}
We now recall the Euler-Lagrange equations for Lagrangian with field $\psi$
\eq{
\pd[\mc{L}]{\psi} - \del_\mu \pd[\mc{L}]{(\del_\mu \psi)}=0
}
The fields here are really the coefficients in $\mf{su}(2)$ of $A_\mu, \phi$, that is they are $A_\mu^{(a)}, \phi^{(a)}$, so we expand   
\eq{
	F_{\mu\nu}^{(a)} &= \del_\mu A_\nu^{(a)} - \del_\nu A_\mu^{(a)} + A_\mu^{(b)}A_\nu^{(c)}\eps\indices{_a_b_c} \\
	(D_\mu \phi)^{(a)} &= \del_\mu \phi^{(a)} + A_\mu^{(b)} \phi^{(c)} \eps\indices{_a_b_c}
}
giving 
\eq{
\pd[\mc{L}]{(\del_\mu A_\nu^{(a)})} &= 2F^{(a)\mu\nu} \\
\pd[\mc{L}]{A_\mu^{(a)}} &= 2\eps\indices{_b_a_c}A^{(c)}_\nu F^{(b)\mu\nu}+2\phi^{(c)}\eps\indices{_b_a_c}(D^\mu \phi)^{(b)} \\
\pd[\mc{L}]{(\del_\mu \phi^{(a)})} &= 2(D^\mu \phi)^{(a)} \\
\pd[\mc{L}]{\phi^{(a)}} &= 2\eps\indices{_c_b_a}A_\mu^{(b)} (D^\mu \phi)^{(c)} -4\lambda \phi^{(a)}\psquare{1-\sum_b (\phi^{(b)})^2}
}
We can now write the Euler-Lagrange equations 
\eq{
0 &= \del_\nu F^{(a)\nu\mu } - \psquare{\eps\indices{_b_a_c}A^{(c)}_\nu F^{(b)\mu\nu}+\phi^{(c)}\eps\indices{_b_a_c}(D^\mu \phi)^{(b)}} \\
&= \psquare{\del_\nu F^{(a)\nu\mu} + A_\nu^{(c)}F^{(b)\nu\mu}\eps_{cba}} + (D^\mu \phi)^{(b)}\phi^{(c)}\eps_{bca} \\
\Rightarrow 0 &= D_\nu F^{\nu \mu} + \comm[D^\mu \phi]{\phi}  \\
}
and 
\eq{
0 &= \del_\mu (D^\mu \phi)^{(a)} - \psquare{\eps\indices{_c_b_a}A_\mu^{(b)} (D^\mu \phi)^{(c)} -2\lambda \phi^{(a)}\psquare{1-\sum_b (\phi^{(b)})^2}} \\
&= \psquare{\del_\mu (D^\mu \phi)^{(a)} + A_\mu^{(b)} (D^\mu \phi)^{(c)} \eps_{bca}} + 2\lambda \phi^{(a)}\psquare{1-\sum_b (\phi^{(b)})^2} \\
\Rightarrow 0 &= D_\mu D^\mu \phi + 2\lambda  (1-\abs{\phi}^2)\phi
}
\hl{These do not match up - sort this out}. 
%%%%%%%%%%%%%%%%%%%%%%%%%%%%%%%%%%%%%%%%%%%%%%%%%%%%%%%%
\subsection{BPS limit}
The idea is now to set $\lambda=0$ but retain that $\abs{\phi}=1$ at infinity. More specifically we take the conditions 
\eq{
\abs{\phi} & = 1 - \frac{m}{2r} + \mc{O}\pround{\frac{1}{r^2}} \\
\pd[\abs{\phi}]{\Omega} &= \mc{O}\pround{\frac{1}{r^2}} \\
\abs{D\phi} &= \mc{O}\pround{\frac{1}{r^2}} 
}
With $\lambda=0$ we can rewrite the energy of the configuration as 
\eq{
E[A,\phi] &= \int_M \pangle{F \mp \star D\phi, F\mp \star D\phi} \pm 2\pangle{F,D\phi} \\
&= \int_M \abs{F-\star D\phi}^2 + \int_{\del M} \pangle{F,\phi}
}
using that $\pangle{F,D\phi} = d\pangle{F,\phi}$. Then as 
\eq{
\int_{\del M} \pangle{F,\phi} = - \int_{\del M} \pangle{\phi, d\phi \wedge d\phi}
}
by our discussion of degree we have $E\geq \pm4\pi k$ for some $K \in \mbb{Z}$ with equality iff $F=\mp \star D \phi$ where we choose the sign to make the bound positive. This is the \bam{BPS equation}. 
%%%%%%%%%%%%%%%%%%%%%%%%%%%%%%%%%%%%%%%%%%%%%%%%%%%%%%%%
%%%%%%%%%%%%%%%%%%%%%%%%%%%%%%%%%%%%%%%%%%%%%%%%%%%%%%%%
\section{Constructions}
%%%%%%%%%%%%%%%%%%%%%%%%%%%%%%%%%%%%%%%%%%%%%%%%%%%%%%%%
\section{The ADHM construction}
This section follows the work first laid out in \cite{Atiyah1978}. Suppose we have the following information:
\begin{itemize}
	\item $W$ a $k$-dimensional vector space
	\item $V$ a $2k+2$-dimensional vector space with skew, non-degenerate bilinear form $(\cdot, \cdot):\wedge^2 V \to \mbb{C}$. 
	\item $z = (z_i) \in \mbb{C}^4$
	\item $A(z) = \sum_i A_i z_i \in \End(W,V)$ s.t. \
	\eq{
\forall z \neq 0, \; U_z \equiv A(z)W\subset V \text{ is isotropic and $k$-dimensional}	
}
\end{itemize}

We now state some important properties:

\begin{lemma}
	Let $E_z = \faktor{U_z^0}{U_z}$, then 
	\begin{itemize}
		\item $\dim E_z = 2$
		\item $E_z$ inherits a non-degenerate skew bilinear
		\item $\forall \lambda \in \mbb{C}^\times, \, E_z = E_{\lambda z}$. 
	\end{itemize}
\end{lemma}
\begin{proof}
	We go point by point:
	\begin{itemize}
		\item $\dim E_z = \dim U_z^0 - \dim U_z = \pround{\dim V - \dim U_z} - \dim U_z = 2k+2 - 2k = 2$.
		\item The bilinear on $W$ is only degenerate in $U_z^0$ on $U_z$, so by quotienting by this it descends directly to $E_z$. 
		\item $A(\lambda z) = \lambda A(z)$, so $A(\lambda z) (\lambda^{-1} \bm{w}) = A(z)(\bm{w})$. Hence we can see $U_{\lambda z} = U_z$ and so result.  
	\end{itemize}
\end{proof}

\begin{corollary}
	We get a vector bundle $E\to \mbb{CP}^3$ with group $SL(2,\mbb{C})$.
\end{corollary}


%%%%%%%%%%%%%%%%%%%%%%%%%%%%%%%%%%%%%%%%%%%%%%%%%%%%%%%%
%%%%%%%%%%%%%%%%%%%%%%%%%%%%%%%%%%%%%%%%%%%%%%%%%%%%%%%%
\bibliographystyle{../../bib/custom-bib-style}
\bibliography{../../bib/jabref_library.bib}

\end{document}
