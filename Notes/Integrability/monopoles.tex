\documentclass{article}

\usepackage{../../header}
%%%%%%%%%%%%%%%%%%%%%%%%%%%%%%%%%%%%%%%%%%%%%%%%%%%%%%%%
%Preamble

\title{Monopoles}
\author{Linden Disney-Hogg}
\date{November 2020}

%%%%%%%%%%%%%%%%%%%%%%%%%%%%%%%%%%%%%%%%%%%%%%%%%%%%%%%%
%%%%%%%%%%%%%%%%%%%%%%%%%%%%%%%%%%%%%%%%%%%%%%%%%%%%%%%%
\begin{document}

\maketitle
\tableofcontents

%%%%%%%%%%%%%%%%%%%%%%%%%%%%%%%%%%%%%%%%%%%%%%%%%%%%%%%%
%%%%%%%%%%%%%%%%%%%%%%%%%%%%%%%%%%%%%%%%%%%%%%%%%%%%%%%%
\section{Introduction}
%%%%%%%%%%%%%%%%%%%%%%%%%%%%%%%%%%%%%%%%%%%%%%%%%%%%%%%%
\subsection{Preamble}
I already have notes on Gauge Theory, Algebraic Geometry, Solitons, and Algebraic Topology, but I have yet to actually make any notes on Monopoles. The purpose of these notes is to be a comprehensive cover of the knowledge required to understand \cite{Braden2018}. This will include previous works by Atiyah, Donaldson, Hitchin, Nahm, and more.
%%%%%%%%%%%%%%%%%%%%%%%%%%%%%%%%%%%%%%%%%%%%%%%%%%%%%%%%
\section{Preliminaries}
As with all my projects, the preliminaries will undoubtedly end up being too long, but I will try keep this minimal this time: 
\begin{definition}
	The \bam{annihilator} of $U \leq V$ is 
\eq{
U^0 = \pbrace{f \in V^\ast \, | \, \forall u \in U, \, f(u)=0} \leq V^\ast
} 
If $V$ has bilinear $\pangle{\cdot, \cdot}$ we can use the isomorphism of $V^\ast\cong V$ to understand 
	\eq{
U^0 = \pbrace{v \in V \, | \, \forall u \in U, \, \pangle{u,v}=0} \leq V	
}
\end{definition}

\begin{lemma}
	THe annihilator is a subspace, $\dim U^0 = \dim V  - \dim U$. 
\end{lemma}

\begin{definition}
	A subspace $U$ is called \bam{isotropic} if $U \subset U^0$. 
\end{definition}
%%%%%%%%%%%%%%%%%%%%%%%%%%%%%%%%%%%%%%%%%%%%%%%%%%%%%%%%
\subsection{The Dirac Monopole}
The standard maxwell equations prohibit monopoles, by which we mean point magnetic field sources, as $\bm{\nabla}\cdot \bm{B}=\bm{0}$. Dirac showed in \cite{Dirac1931} that it is possible to escape this conclusion by giving non-trivial topology to the space by allowing $\bm{B} = \frac{g}{4\pi r^2}\hat{\bm{x}}$ to have a singularity at $\bm{x}=\bm{0}$. We can calculate $\bm{\nabla} \cdot \bm{B} = g\delta(\bm{x})$. Removing this circle gives $\mbb{R}^3\setminus 0$, homotopic to $S^2$, and the corresponding magnetic two form on this sphere is
\eq{
f = \frac{g}{4\pi} \sin \theta d\theta \wedge d\phi
} 
and so the flux through a 2-sphere enclosing the origin is $\int_{S_R^2} f = g $. For $g \neq 0$, we know $f \neq da$ for a global $a \in \Omega^1(S^2)$ by Stokes' theorem, but if we take a cover of the sphere $U_{N/S}$ (north/south) and define gauge potentials
\eq{
a_N &= \frac{g}{4\pi}(1-\cos\theta )d\phi \in \Omega^1(U_N) \\
a_S &= \frac{g}{4\pi}(-1-\cos\theta) d\phi \in \Omega^1(U_S)
}
On the intersect $U_N \cap U_S$ we have $da_N = f = da_S$ and $a_N = a_S + \frac{g}{2\pi}d\phi$. \\
Now taking $A = ia, \, F = if$, we have that $g_{NS}(\theta,\phi) = e^{-i\frac{g\phi}{2\pi}}$. Requiring that this is a well-defined transition function gives $g \in \mbb{Z}$. This is equivalent to the integrality of the Chern number. \\
We will not want to consider this as this solution is not solitonic (it has infinite mass), but for further discussion see \cite{Manton2004}. 

%%%%%%%%%%%%%%%%%%%%%%%%%%%%%%%%%%%%%%%%%%%%%%%%%%%%%%%%
\subsection{Pauli Matrices}

\begin{definition}[Pauli Matrices]
	The \bam{Pauli matrices} are
	
	\begin{align*}
		\sigma_1 &= \begin{pmatrix} 0 & 1 \\ 1 & 0\end{pmatrix}, & 
		\sigma_2 &= \begin{pmatrix} 0 & -i \\ i & 0\end{pmatrix}, &  
		\sigma_3 &= \begin{pmatrix} 1 & 0 \\ 0 & -1\end{pmatrix}.  
	\end{align*}
	Note they are all Hermitian and traceless.
\end{definition}

\begin{fact}$\sigma_i \sigma_j = \delta_{ij}I +i\epsilon_{ijk}\sigma_k \Rightarrow \tr(\sigma_i \sigma_j) = 2\delta_{ij}$
\end{fact}

%%%%%%%%%%%%%%%%%%%%%%%%%%%%%%%%%%%%%%%%%%%%%%%%%%%%%%%%
\subsection{\secmath{SU(2)}}
We can write 
\[
SU(2)=\pbrace{    \begin{pmatrix} \alpha & -\overline{\beta} \\ \beta & \overline{\alpha} \end{pmatrix}  : \alpha,\beta\in\mathbb{C} , |\alpha|^2+|\beta|^2=1   }
\]
This can be expressed as, for $A\in SU(2)$
\[
A=a_0 I +i\bm{a}\cdot\bm{\sigma}
\]
where $\bm{a}=(a_1, a_2, a_3)$, $\bm{\sigma}=(\sigma_1, \sigma_2, \sigma_3)$, and $a_0^2+|\bm{a}|^2=1$. Hence $SU(2)\cong S^3$. In addition, by parametrising $SU(2)$ by the $a_i$, it can be seen that $\set{  i\sigma_i }$ forms a basis of $\mf{su}(2)$. It is typical to normalise this basis to $\set{  T^a=-\frac{1}{2} i\sigma_a  }$. 

\begin{lemma}
	The structure constants in this basis $\set{  T^a  }$ are $f^{ab}_c=\epsilon_{abc}$.
\end{lemma}

\begin{corollary}
	The Killing form is given by $\kappa\pround{T^a,T^b} = \kappa^{ab} = -2\delta^{ab} = 4\tr(T^aT^b)$. Hence $\kappa(X,Y) = 4\tr(XY)$
\end{corollary}


%%%%%%%%%%%%%%%%%%%%%%%%%%%%%%%%%%%%%%%%%%%%%%%%%%%%%%%%
\subsection{Degree of a Map}
Consider a smooth map $f:S^2 \to S^2$. The number of preimages of $p \in S^2$ computed with sign is calculate via pullback to be 
\eq{
	\deg(f) = \frac{1}{\int_{S^2} \omega} \int_{S^2} f^\ast \omega
}
for some normalisable volume form $\omega$ on the target $S^2$. Then \hl{NOT COMPLETE}.
%%%%%%%%%%%%%%%%%%%%%%%%%%%%%%%%%%%%%%%%%%%%%%%%%%%%%%%%
\subsection{Properties of Curves}
Here we want to take an example curve and consider the kind of constructs on it we will be looking at. To this we will effectively go through Miranda \cite{Miranda1995} and apply the relevant sections. \\
Define the degree-$6$ homogeneous polynomial $P:\mbb{C}^3 \to \mbb{C}$ by 
\eq{
P(x,y,z) = y^6 - x^6 + z^2 x^4
}
Define the corresponding projective plane curve 
\eq{
X = \pbrace{[x:y:z] \in \mbb{P}^2 \, | \, P(x,y,z) = 0}
}
Throughout we will want to compare our by-hand calculations with numerics in Sage, so we start by initialising $X$. 
\begin{lstlisting}[language=Python,frame=single]
sage: x,y,z = QQ['x,y,z'].gens()
sage: X = Curve(y^6-x^6+z^2*x^4)
\end{lstlisting}
The corresponding affine plane curve on the intersection with the open set $U_x = \pbrace{x \neq 0}$ is $y^6 = 1-z^2$. Hence 
\eq{
X = \pbrace{[1:y:z] \, | \, y^6 = 1-z^2} \cup \pbrace{[0:0:1]}
}
\begin{prop}
	$X$ is singular at $[0:0:1]$ only. 
\end{prop}
\begin{proof}
	We have 
	\eq{
\pd[P]{x} &= 2x^3(2z^2 -3x^2), & \pd[P]{y} &= 6y^5, & \pd[P]{z} &= 2zx^4	
}
It can be seen that a common solution to these must have $y=0$ from $\del_y P=0$, and then either $z=0$  or $x=0$ from $\del_zP=0$. If $z=0$, $\del_zP=0$ enforces $x=0$ which isn't in $\mbb{P}^2$. If $x=0$, $z$ is arbitrary and we find the point $[0:0:1] \in \mbb{P}^2$. This is calculated in Sage as follows:
\begin{lstlisting}[language=Python,frame=single]
sage: X.singular_points()
[(0 : 0 : 1)]
\end{lstlisting}
\end{proof}
We need to get an understanding of this singularity. If we take the neighbourhood $U_z = \pbrace{z \neq 0}$ the corresponding affine plane curve on the intersection is 
\eq{
p(x,y) = y^6-x^6+x^4 = 0
}
as 
\eq{
X = \pbrace{[x:y:1] \, | \, y^6-x^6+x^4=0} \cup \pbrace{[x:y:0] \, | \, y^6-x^6=0}
}
\begin{prop}
	The only singular point on the affine plane curve $p(x,y)=0$ is $(0,0)$ and it is $2$-monomial. 
\end{prop}
\begin{proof}
	$\del_x p(x,y) = 2x^3(2-3x^2), \, \del_y p(x,y) = 6y^5$. It can be seen that the only simultaneous solution to these is  $(0,0)$. Here we write $p(x,y)=g(x,y)^4 - h(x,y)^6$ where
	\eq{	
g(x,y) &= x\psquare{1 - \frac{1}{4}x^2 + \dots} = x\psquare{1-x^2}^\frac{1}{4} \\
h(x,y) &= iy
}
We then note $\gcd(4,6) = 2$. 
\end{proof}
As such by general theory we know that we can resolve this singularity to make $X$ into a compact Riemann surface. This process involves removing the singularity at $[0:0:1]$ and patching the two holes created. \\
We now check some topology using Riemann-Hurwitz. 
\begin{prop}
	$g(X) = 2$.
\end{prop}
\begin{proof}
	As standard we consider the map $X\to \mbb{P}^1$ taking the coordinate $y$. Then $z$ is generically double valued so the degree of the map is 2. This is ramified at the the 6 roots of $1-y^6=0$, so using $g(\mbb{P}^1)$
	\eq{
	g(X) = 1+2(0-1)+\frac{1}{2} \times 6 \times (2-1) = 1-2+3=2
}
This is calculated in Sage as follows:
\begin{lstlisting}[language=Python,frame=single]
sage: X.genus()
2
\end{lstlisting}
\end{proof}
\begin{remark}
	This is to be expected, as here we are dealing with a hyperelliptic curve, of the form $z^2=h(y)$ where $h$ is a polynomial of degree $2g+2$. 
\end{remark}
\begin{corollary}
	Through Hurwitz' theorem we know $\abs{\Aut(X)} \leq 84$.
\end{corollary}
This is a very lax bound, we know many groups of order $\leq 84$. We now try to calculate $\Aut(X)$ more systematically. Firstly we note that as a hyperelliptic curve it comes with the hyperelliptic involution $(y,z) \mapsto(y,-z)$ generating a $C_2$ symmetry. We also have a $D_6$ dihedral group of automorphisms generated by $r:(y,z) \mapsto (\zeta y, z)$ and $s:(y,z) \mapsto (1/y,iz/y^3)$ where $\zeta = e^{\frac{i\pi}{3}}$. We can check that as $\zeta^3=-1$ 
\eq{
srs:(y,z) \mapsto (1/y,iz/y^3) \mapsto (\zeta/y,i z/y^3) \mapsto (\zeta^{-1}y,i^2 \zeta^{-3}z) = (\zeta^{-1}y,z)
}
giving $srs=r^{-1}$ as required from the dihedral group. The hyperelliptic involution commutes with this dihedral action, so in total we have found a $\Aut(X) \geq C_2 \times D_6$. As $\abs{C_2 \times D_6}=24$, by Hurwitz' theorem we know that we could have at most one other automorphism of order either 2 or 3. This turns out not to be the case, and to see a general classification see for example \cite{Muller2017}. \\
To do the calculation in Sage, not we can use a result in \cite{Lauter2001} (translated \href{https://www.arxiv-vanity.com/papers/1811.07007/}{here}):
\begin{lemma}
	If $X$ is a hyperelliptic curve then $\Aut(J(X),a) \cong \Aut(X)$ where $a$ is the canonical principal polarisation of the Jacobian. 
\end{lemma}
This allows us to calculate in Sage:
\begin{lstlisting}[language=Python,frame=single]
sage: A.<u,v> = QQ[]
sage: Mod = sage.schemes.riemann_surfaces.riemann_surface
sage: S = Mod.RiemannSurface(u^2-1+v^6)
sage: G = S.symplectic_automorphism_group()
sage: G.structure_description()
'(C6 x C2) : C2'
\end{lstlisting}
\begin{remark}
	Note that above Sage is acting as a GAP wrapper. The formatting translates to say $G \cong (C_6 \times C_2) \rtimes C_2$
\end{remark}
We now want to consider forms on $X$. General theory says that we have a basis of $\Omega^1(X)$ given by 
\eq{
\pbrace{\omega_j = \frac{y^j dy}{z}}_{j=0}^{g-1}
}
and a generating set of $H_1(X,\mbb{Z})$ given by a suitable choice of independent closed paths $c = \gamma_1 - \gamma_2$ where $\gamma_i$ are the two lifts of a path in $\mbb{P}^1$ between branch points of the map $X \to \mbb{P}^1$. In order to find the period matrices explicitly we need to calculate 
\eq{
\int_c \omega_j 
}
for these loops c. To be explicit for the paths in the base that we choose we take 
\eq{
a_1 &= \pbrace{e^{\frac{it\pi}{3}} \, | \, t \in [0,1]} \\
b_1 &= \pbrace{e^{\frac{i(t+1)\pi}{3}} \, | \, t \in [0,1]} \\
a_2 &= \pbrace{e^{\frac{i(t+3)\pi}{3}} \, | \, t \in [0,1]} \\
b_2 &= \pbrace{e^{\frac{i(t+5)\pi}{3}} \, | \, t \in [0,1]}  
}
This can be calculated numerically in Sage as below (output formatted to fit in the document):
\begin{lstlisting}[language=Python,frame=single]
sage: S.period_matrix()
[ -0.6-1.1*I  0.6+1.1*I 1.2+0.0*I -1.2+0.0*I]
[ -1.1-0.1*I -1.1-0.6*I 0.0-1.2*I  0.0-1.2*I]
\end{lstlisting}
%%%%%%%%%%%%%%%%%%%%%%%%%%%%%%%%%%%%%%%%%%%%%%%%%%%%%%%%
%%%%%%%%%%%%%%%%%%%%%%%%%%%%%%%%%%%%%%%%%%%%%%%%%%%%%%%%
\subsection{Spinors}
\subsubsection{Spinor Bundles}
\begin{definition}
	The \bam{complexification} of a vector space $V$ is the vector space $V_\mbb{C} = V \otimes_\mbb{R} \mbb{C}$. 
\end{definition}
\begin{prop}
	$V_\mbb{C}$ comes with a well defined map of conjugation $v \otimes z \to v \otimes \bar{z}$. This is a $\mbb{C}$-linear isomorphism onto $\bar{V_\mbb{C}}$. 
\end{prop}
\begin{lemma}
	$\pround{V^\ast}_\mbb{C} = \pround{V_\mbb{C}}^\ast$. 
\end{lemma}

We will now state a few results about $SL_2(\mbb{C})$ bundles:
\begin{lemma}
	A rank-2 complex vector bundle $S \to M$ has a reduction of structure group to a $SL_2(\mbb{C})$-bundle iff $\exists \varepsilon \in \Gamma(M,\wedge^2 S)$ a non-degenerate form. 
\end{lemma}
\begin{corollary}
	The cotangent bundle on a 2-dimensional manifold $M$ reduces to a $SL_2(\mbb{C})$ bundle iff the canonical bundle is trivial.  
\end{corollary}

\begin{definition}
	Suppose $M$ is a 4-dimensional manifold s.t $\exists S \to M$ a $SL_2(\mbb{C})$-bundle s.t. 
	\eq{
TM_\mbb{C} \cong S \otimes \bar{S}	
}
Then we call $S$ a \bam{spinor bundle} on  $M$.  
\end{definition}

\begin{remark}
	This is a quite specific definition, so to see this done better check out \cite{Michelsohn1989}. 
\end{remark}

%%%%%%%%%%%%%%%%%%%%%%%%%%%%%%%%%%%%%%%%%%%%%%%%%%%%%%%%
%%%%%%%%%%%%%%%%%%%%%%%%%%%%%%%%%%%%%%%%%%%%%%%%%%%%%%%%
\subsection{Twistors}
\subsubsection{Grassmannians and Flag Manifolds}
\begin{definition}
	Let $V$ be an $n$-dimensional vector space over $\mbb{F}$. Define the \bam{Grassmannian manifold of $k$-dimensional subspaces} as $G_k(V)$. 
\end{definition}

\begin{lemma}
	If $\mbb{F}= \mbb{R}$, $G_k(V)\cong \faktor{O(n)}{[O(k) \times O(n-k)]}$
\end{lemma}
\begin{proof}
	We can view the Grassmannian as the quotient of $GL(V)$ by the stabiliser of a $k$-dim subspace. 
\end{proof}
\begin{remark}
	Likewise if $\mbb{F}=\mbb{C}, \, G_k(V) \cong \faktor{U(n)}{[U(k) \times U(n-k)]}$,
\end{remark}
\begin{corollary}
	$\dim G_k(V) = k(n-k)$. 
\end{corollary}

\begin{example} 
	$G_1(\mbb{C}) \cong \mbb{CP}^1$. 
\end{example}

\begin{notation}
	If we pick a basis of complex vector space $V$ to make $V \cong \mbb{C}^n$, we identify $G_k(V) \cong G_k(\mbb{C}^n)$ which we will notation $G_{k,n}$. 
\end{notation}

If we denote $M^\ast_{n \times k}(\mbb{C})$ to be the full rank $m \times k$ matrices we can construct $G$ via the projection 
\eq{
M^\ast_{n \times k}(\mbb{C}) &\to G_{k,n} \\
M &\mapsto \spn\pbrace{\text{columns of $M$}}
}
$M^\ast_{n \times k}(\mbb{C})$ carries a transitive left action via multiplication that doesn't affect span, and so we get a bundle structure. We can then consider coordinates on $G_{k,n}$ as $Z \in M_{(n-k) \times k}(\mbb{C})\cong \mbb{C}^{k(n-k)}$ via the map 
\eq{
\phi : M_{(n-k) \times k}(\mbb{C}) &\to G_{k,n} \\
Z &\mapsto \psquare{\begin{pmatrix} Z \\ I_k \end{pmatrix}}
}
It will now be necessary to generalise this concept slightly:
\begin{definition}
	Given a fixed complex $n$-dimensional vector space $V$ and $\pbrace{d_i}_{i=1}^m$ integers satisfying $1 \leq d_1 < d_2 < \dots < d_m < n$ we define a \bam{flag manifold of type $(d_1, \dots, d_m)$} as 
	\eq{
F_{d_1 \dots d_m}(V) = \pbrace{(S_1, \dots, S_m) \, | S_i \text{ a $d_i$-dimensional subspace of $V$, } S_1 \subset S_2 \subset \dots \subset S_m}	
} 
\end{definition}

\subsubsection{Twistor Space}
Now given $\mbb{T}$ a $4$d $\mbb{C}$ vector space we can get a \bam{double fibration}
\eq{
F_1(\mbb{T}) \overset{\pi_1}{\twoheadleftarrow} F_{12}(\mbb{T}) \overset{\pi_2}{\twoheadrightarrow} F_2(\mbb{T})
}
from the projections of $(S_1,S_2)$ 
\begin{prop}
	The $\pi_i$ are holomorphic.
\end{prop}

\begin{definition}
	Given seta $A,B$ a \bam{correspondence}  is a map $f: A \to \mc{P}(B)$ sending each $a \in A$ to $f(a) \subset B$.
\end{definition}
The double fibration induces natural correspondences $c=\pi_2 \circ \pi_1^{-1}$ and $c^{-1} = \pi_1 \circ \pi_2^{-1}$. 
\begin{notation}
	We shall denote 
	\begin{itemize}
		\item $\mbb{PT}=F_1(\mbb{T}) \cong \mbb{CP}^3$ \bam{projective twistor space}
		\item $M_\mbb{C} = F_2(\mbb{T}) \cong G_{2,4}(\mbb{C})$ \bam{compactified complexified Minkowski space}
		\item $F = F_{12}(\mbb{T})$ the \bam{correspondence} between $\mbb{PT}$ and $M_{\mbb{C}}$. 
	\end{itemize}
For $A \subset \mbb{PT}$ we shall let $c(A) = \tilde{A}$, and for $B \subset M_{\mbb{C}}$ let $\hat{B} = c^{-1}(A)$. 
\end{notation}

\begin{idea}
	The point of twistor geometry is understanding how to use the correspondence
	\eq{
\mbb{PT} \leftarrow F \rightarrow M_{\mbb{C}} 	
} 
	to transfer information about (subsets of) $\mbb{PT}$ to (subsets of) $M_{\mbb{C}}$. 
\end{idea}

\begin{remark}
	$F$ is sometimes denoted as $\mbb{PS}$ and called the \bam{projective spinor bundle}. \hl{We will hopefully return to this later}. 
\end{remark}

\begin{prop}
	For $p \in \mbb{PT}, \, q \in M_{\mbb{C}}$, $\tilde{p} \cong \mbb{P}^2$ and $\hat{q} \cong \mbb{P}^1$
\end{prop}

We now wish to understand the correspondence in coordinates. We consider the map $\phi$ as described before, though now we send 
\eq{
	z &\mapsto \psquare{\begin{pmatrix} iz \\ I_2 \end{pmatrix}}
}
for convenience, let $M^I_\mbb{C}= \phi(M_2(\mbb{C}))$, $\mbb{PT}^I = c^{-1}(M_{\mbb{C}}^I)$ and $F^I = \pi_2^{-1}(M^I)$.  
\begin{prop}
	$F^I \cong M^i \times \mbb{P}^1$. 
\end{prop} 
\begin{proof}
	The isomorphism is given as 
	\eq{
	M^I_\mbb{C} \times \mbb{P}^1 &\to F^I \\
	(z,[v]) &\mapsto \pround{\psquare{\begin{pmatrix} izv \\ v \end{pmatrix}},\psquare{\begin{pmatrix} iz \\ I_k \end{pmatrix}}}
}
\end{proof}
\begin{corollary}
	In coordinates we can see the double fibration as 
	\eq{ 
\mbb{PT}^I \leftarrow & \; \; \; F^I \; \; \;  \rightarrow M_\mbb{C}^I \\
[izv, v] \leftarrow &(z,[v]) \rightarrow z	
}
\end{corollary}
\subsubsection{Complexified Minkowski Space}
It is at this point useful to discuss slightly more concretely what is meant by complexified Minkowski space. The resource to use here is \cite{Adamo2018}. 
\begin{definition}
	Let $(M,g)$ be a smooth real manifold, and take a coordinate system $x^a$ s.t. $g= g_{ab} dx^a dx^b$. The \bam{complexification} is defined by allowing the $x^a$ to take complex values and extending $g_{ab}(x)$ holomorphically.  
\end{definition}
%%%%%%%%%%%%%%%%%%%%%%%%%%%%%%%%%%%%%%%%%%%%%%%%%%%%%%%%
%%%%%%%%%%%%%%%%%%%%%%%%%%%%%%%%%%%%%%%%%%%%%%%%%%%%%%%%
\section{The Monopole Equations}
%%%%%%%%%%%%%%%%%%%%%%%%%%%%%%%%%%%%%%%%%%%%%%%%%%%%%%%%
\subsection{Yang-Mills-Higgs equations}
\begin{definition}
Take a principal $G$-bundle $P\to M$, $\omega_{vol}$ an orientation on $M$, and $\pangle{\cdot, \cdot}$ to be an $\ad$-invariant inner product on $\mf{g}$. Then the \bam{Yang-Mills-Higgs actions} on $M$ is 
\eq{
S_{YMH}[A,\phi] = \int_M \psquare{-\abs{F}^2 - \abs{D\phi}^2 - V(\phi)}\omega_{vol}
}
where $F=dA+A\wedge A$ is the curvature associated to a section $A\in \Gamma(T^\ast M \otimes \ad(P))$, $D = d+A$ is the associated covariant derivative, and $\phi \in \Gamma(\ad(P))$.
\end{definition}

To connect with physical theory we want our Lagrangian to be of the form kinetic-potential. This will manifest itself in our choice of signs by requiring that 
\eq{
-\abs{D\phi}^2 &= (\del_0 \phi)^2 + \dots \\
-\abs{F}^2 &= E_i^2 + \dots
}
This is the reason for the somewhat strange looking sign choice at this point. Obviously in the end it will be equivalent to take 
\eq{
S_{YMH} = \int \abs{F}^2 + \abs{D\phi}^2 + V
}
when it comes to the variational equations, but this will not be true when we consider the energy functional.
\begin{remark}
	A common choice of potential function $V$ is $V(\phi) = \lambda \pround{1-\abs{\phi}^2}^2$, the \bam{$\phi^4$-potential}. By our choice of signs, we want $\lambda >0$. We can check $V^\prime(\phi) = -4\lambda(1-\abs{\phi}^2)\abs{\phi}$. 
\end{remark}

\begin{definition}
	A \bam{monopoles} will be a soliton-like solution to the Yang-Mills-Higgs equations when $G=SU(2)$, $M=\mbb{R}^4$ with the Minkowski metric, the principal bundle is $P = M \times G$, and the potential is $\phi^4$. 
\end{definition}

\begin{prop}
	The variational equations corresponding to $S_{YMH}$ in Minkowski $\mbb{R}^4$ are the\bam{Yang-Mills-Higgs equations}
	\eq{
DF &= 0 \quad \text{(Bianchi)} \\
 \star D \star F &= -\comm[\phi]{D\phi} \\
 \star D \star D\phi &= -\frac{1}{2\abs{\phi}}V^\prime(\phi)\phi	
}
\end{prop}
\begin{proof}
	We first consider the equation that comes from the variation of $A$. Let $A_t = A + t\beta$, then $F_t = F + t\pround{d\beta + \beta \wedge A + A \wedge \beta} + \mc{O}(t^2)$ and $D_t \phi = D \phi + t\comm[\beta]{\phi}$. Hence 
	\eq{
S_t = S + 2t\int_M \psquare{-\pangle{F,D\beta} - \pangle{D\phi,\comm[\beta]{\phi}}} \omega_{vol} + \mc{O}(t^2)
}
Hence to be at a stationary point of the action variation we want
\eq{
\int_M \psquare{-\pangle{F,D\beta} - \pangle{D\phi,\comm[\beta]{\phi}}} \omega_{vol} = 0
}
Using the fact that inner product is $\ad$-invariant and letting $D^\ast$ be the formal adjoint of $D$ wrt to inner product $\pangle{\pangle{\eta,\omega}} = \int_M \pangle{\eta,\omega} \omega_{vol}$ we can rewrite this as 
\eq{
\int_M \pangle{-D^\ast F + \comm[D\phi]{\phi},\beta} \omega_{vol} = 0
}
Using results on the dual of the covariant derivative we can say that 
\eq{
D^\ast F = (-1)^{4(2-1)+1}(-1)\star D \star F = \star D \star F 
}
Hence as $\beta$ was a generic variation we must have $\star D \star F - \comm[D\phi]{\phi} =0$. \\
We now consider a $\phi$ variation so $\phi_t = \phi + t\psi$. Note 
\eq{
\abs{\phi_t} = \sqrt{\pangle{\phi_t,\phi_t}} = \sqrt{\abs{\phi}^2 + 2t\pangle{\phi,\psi} + \mc{O}(t^2)} = \abs{\phi}\sqrt{1 + \frac{2t\pangle{\phi,\psi}}{\abs{\phi}^2} + \mc{O}(t^2)} = \abs{\phi} + t\frac{\pangle{\phi,\psi}}{\abs{\phi}} + \mc{O}(t^2)
}
so if we consider $V$ as $V(\phi)=V(\abs{\phi})$ (i.e. as a function of a real variable) then 
\eq{
V(\phi_t) = V(\phi) + t\abs{\phi}^{-1} V^\prime(\phi) \pangle{\phi,\psi} + \mc{O}(t^2)
}
Then a variational argument as before means that we need to set 
\eq{
 && \int_M \psquare{-2\pangle{D\phi, D\psi} - \abs{\phi}^{-1} V^\prime(\phi) \pangle{\phi,\psi}} \omega_{vol} &= 0 \\
 \Rightarrow& &  -2D^\ast D \phi- \abs{\phi}^{-1} V^\prime(\phi) \phi &= 0 \\
\Rightarrow& &  (-1)^{4(1-1)+1}(-1)(-1)\star D \star D \phi - \frac{1}{2\abs{\phi}}V^\prime(\phi) \phi &= 0 \\
\Rightarrow& &  \star D \star D \phi + \frac{1}{2\abs{\phi}}V^\prime(\phi) \phi &= 0
}
\end{proof}

\begin{remark}
	The Dirac monopole is a solution in the case of $G=U(1)$.
\end{remark}
We may make these equations explicit in coordinates. The first approach is to try and substitute in coordinate expressions into the YMH equations. Taking coordinates $x^\mu$ on $\mbb{R}^4$ and writing $F = \frac{1}{2}F_{\mu\nu} dx^\mu \wedge dx^\nu$
\eq{
\star F &= \frac{1}{4} F_{\mu\nu} \epsilon\indices{^\mu^\nu_\rho_\sigma} dx^\rho \wedge dx^\sigma \\
\Rightarrow D\star F &= \frac{1}{4} D_\tau F_{\mu\nu} \epsilon\indices{^\mu^\nu_\rho_\sigma} dx^\tau \wedge dx^\rho \wedge dx^\sigma \\
\Rightarrow \star D \star F &= \frac{1}{4} D_\tau F_{\mu\nu} \epsilon\indices{^\mu^\nu_\rho_\sigma} \epsilon\indices{^\tau^\rho^\sigma_\lambda} dx^\lambda \\
&= \frac{1}{4} D_\tau F_{\mu\nu} \eps\indices{_\rho_\sigma^\mu^\nu}\eps\indices{^\rho^\sigma^\tau_\lambda} dx^\lambda \\
&= -D_\tau F\indices{^\mu_\nu} \delta^\tau_{[\mu}\delta^\nu_{\lambda]}dx^\lambda \\
&= -D_\tau F\indices{^\tau_\lambda} dx^\lambda
}
Hence the first monopole equation reads 
\eq{
D_\mu F\indices{^\mu^\nu} +\comm[D^\nu \phi]{\phi} = 0
}
Next we have 
\eq{
\star D \phi &= \frac{1}{6} (D_\mu \phi) \eps\indices{^\mu_\nu_\rho_\sigma} dx^\nu \wedge dx^\rho \wedge dx^\sigma \\
\Rightarrow D\star D \phi &= \frac{1}{6} (D_\tau D_\mu \phi) \eps\indices{^\mu_\nu_\rho_\sigma} dx^\tau \wedge dx^\nu \wedge dx^\rho \wedge dx^\sigma \\
\Rightarrow \star D \star D \phi &= \frac{1}{6} (D_\tau D_\mu \phi) \eps\indices{^\mu_\nu_\rho_\sigma} \epsilon^{\tau \nu \rho \sigma} \\
&=-(D_\tau D^\mu \phi) \delta_\mu^\tau = -D_\mu D^\mu \phi
}
which yields (taking the $\phi^4$ potential)
\eq{
D_\mu D^\mu \phi +2\lambda (1-\abs{\phi}^2)\phi = 0
}
Collecting this with the Bianchi identity (taking it as $\star DF=0$) we have 
\eq{
\epsilon^{\rho\mu\nu\tau} D_\rho F_{\mu\nu} &=0 \\
D_\mu F\indices{^\mu^\nu} +\comm[D^\nu \phi]{\phi} &= 0 \\
D_\mu D^\mu \phi +2\lambda (1-\abs{\phi}^2)\phi &= 0
}
An alternative approach to deriving these equations is to first write the Lagrangian in coordinate form and then derive the variational equations. We take the inner product on $\mf{g} = \mf{su}(2)$ to be $\pangle{X,Y} = -\frac{1}{2}\kappa(X,Y) = -2\tr(XY)$ for concreteness, which gives $\pangle{t^a,t^b} = \delta^{ab}$, and as such we need the mostly-positive Minkowski metric. 
\begin{remark}
	It is possible to change around the signs here in order to assure that we use the mostly-minus metric, as Manton \& Sutcliffe do. We will want to avoid this as the mostly positive makes more sense when reducing from Minkowski $\mbb{R}^{n+1}$ to Euclidean $\mbb{R}^n$. 
\end{remark}
We have 
\eq{
\pangle{F,F} &= \frac{1}{4}\pangle{F_{\mu\nu},F_{\rho\sigma}} \pangle{dx^\mu \wedge dx^\nu, dx^\rho \wedge dx^\sigma} \\
&= \frac{1}{4}\pangle{F_{\mu\nu},F_{\rho\sigma}}\pround{\eta^{\mu\rho}\eta^{\nu\sigma} - \eta^{\mu\sigma}\eta^{\nu\rho}} \\
&= \frac{1}{2}\pangle{F_{\mu\nu},F^{\mu\nu}} \\
&= -\tr(F_{\mu\nu}F^{\mu\nu}) \\
\pangle{D\phi,D\phi} &= \pangle{D_\mu\phi,D_\nu\phi}\pangle{dx^\mu, dx^\nu} \\
&= \pangle{D_\mu \phi, D_\nu\phi} \eta^{\mu\nu} \\
&= -2\tr(D_\mu \phi D^\mu\phi)
}
Hence the corresponding Lagrangian density is 
\eq{
\mc{L} = \tr(F_{\mu\nu}F^{\mu\nu}) + 2\tr(D_\mu \phi D^\mu \phi)- V(\phi) = \sum_a \psquare{- \frac{1}{2}F_{\mu\nu}^{(a)} F^{(a)\mu\nu} - \pround{D_\mu \phi}^{(a)} \pround{D^\mu \phi}^{(a)}} - \lambda \psquare{1-\sum_a (\phi^{(a)})^2}^2 
}
To check that we have the correct signs in this Lagrangian we verify that it takes the form kinetic-potenial with 
\eq{
\text{kinetic} &= -2\tr(E_i E_i) -2\tr(D_0 \phi D_0 \phi) \\
\text{potential} &= -\tr(F_{ij} F_{ij}) -2\tr(D_i \phi D_i \phi) + V(\phi)  
}
where $E_i = F_{0i}$.
\begin{remark}
	Something we can immediately recognise is that in order to get finite energy solutions when $\lambda \neq 0$, we need $\abs{\phi} \to 1$.
\end{remark}
We now recall the Euler-Lagrange equations for Lagrangian with field $\psi$
\eq{
\pd[\mc{L}]{\psi} - \del_\mu \pd[\mc{L}]{(\del_\mu \psi)}=0
}
The fields here are really the coefficients in $\mf{su}(2)$ of $A_\mu, \phi$, that is they are $A_\mu^{(a)}, \phi^{(a)}$, so we expand   
\eq{
	F_{\mu\nu}^{(a)} &= \del_\mu A_\nu^{(a)} - \del_\nu A_\mu^{(a)} + A_\mu^{(b)}A_\nu^{(c)}\eps\indices{_a_b_c} \\
	(D_\mu \phi)^{(a)} &= \del_\mu \phi^{(a)} + A_\mu^{(b)} \phi^{(c)} \eps\indices{_a_b_c}
}
giving 
\eq{
\pd[\mc{L}]{(\del_\mu A_\nu^{(a)})} &= -2F^{(a)\mu\nu} \\
\pd[\mc{L}]{A_\mu^{(a)}} &= -2\eps\indices{_b_a_c}A^{(c)}_\nu F^{(b)\mu\nu}-2\phi^{(c)}\eps\indices{_b_a_c}(D^\mu \phi)^{(b)} \\
\pd[\mc{L}]{(\del_\mu \phi^{(a)})} &= -2(D^\mu \phi)^{(a)} \\
\pd[\mc{L}]{\phi^{(a)}} &= -2\eps\indices{_c_b_a}A_\mu^{(b)} (D^\mu \phi)^{(c)} +4\lambda \phi^{(a)}\psquare{1-\sum_b (\phi^{(b)})^2}
}
We can now write the Euler-Lagrange equations 
\eq{
0 &= -\del_\nu F^{(a)\nu\mu } - \psquare{-\eps\indices{_b_a_c}A^{(c)}_\nu F^{(b)\mu\nu}-\phi^{(c)}\eps\indices{_b_a_c}(D^\mu \phi)^{(b)}} \\
&= -\psquare{\del_\nu F^{(a)\nu\mu} + A_\nu^{(c)}F^{(b)\nu\mu}\eps_{cba}} - (D^\mu \phi)^{(b)}\phi^{(c)}\eps_{bca} \\
\Rightarrow 0 &= D_\nu F^{\nu \mu} + \comm[D^\mu \phi]{\phi}  \\
}
and 
\eq{
0 &= -\del_\mu (D^\mu \phi)^{(a)} - \psquare{-\eps\indices{_c_b_a}A_\mu^{(b)} (D^\mu \phi)^{(c)} +2\lambda \phi^{(a)}\psquare{1-\sum_b (\phi^{(b)})^2}} \\
&= -\psquare{\del_\mu (D^\mu \phi)^{(a)} + A_\mu^{(b)} (D^\mu \phi)^{(c)} \eps_{bca}} - \lambda \phi^{(a)}\psquare{1-\sum_b (\phi^{(b)})^2} \\
\Rightarrow 0 &= D_\mu D^\mu \phi + 2\lambda  (1-\abs{\phi}^2)\phi
}
We happily see that these two approaches agree, and we should see that these are indeed the sort of equations we want (e.g the $\phi$ equation looks like Klein-Gordon if we linearise around $\abs{\phi}=1$. )
%%%%%%%%%%%%%%%%%%%%%%%%%%%%%%%%%%%%%%%%%%%%%%%%%%%%%%%%
\subsection{BPS limit}
The monopole equations we have found so far are second order, but we want to apply the classic strategy when working with topological solitons: write the energy functional of a static configuration as the integral of a square term plus a topological term, and then we locally must have a minimising solution by setting the squared term to 0. This will be possible if we set $\lambda=0$ but retain that $\abs{\phi}=1$ at infinity. More specifically we take the conditions 
\eq{
\abs{\phi} & = 1 - \frac{m}{2r} + \mc{O}\pround{\frac{1}{r^2}} \\
\pd[\abs{\phi}]{\Omega} &= \mc{O}\pround{\frac{1}{r^2}} \\
\abs{D\phi} &= \mc{O}\pround{\frac{1}{r^2}} 
}
With $\lambda=0$ we can rewrite the energy functional of static configurations at any time as 
\eq{
E[A,\phi] = \int_M \psquare{-\tr(F_{ij} F_{ij}) -2\tr(D_i \phi D_i \phi)} d^3 x
}
If we keep the same inner product on $\mf{su}(2)$, but now take Euclidean $\mbb{R}^3$ we can express
\eq{
E[A,\phi] = \int_{\mbb{R}^3} \psquare{\abs{F}^2 + \abs{D\phi}^2} d^3x
}
As we are in Euclidean $\mbb{R}^3$ we can write 
\eq{
\abs{\star D \phi}^2 \omega_{vol} = \pangle{\star D \phi \wedge \star^2 D \phi} = (-1)^{1(3-1)}\pangle{\star D \phi \wedge D \phi} =  (-1)^{2\times1} \pangle{D \phi \wedge \star D \phi} = \abs{D \phi}^2 \omega_{vol}
}
and, using the Bianchi identity,
\eq{
	\pangle{F, \star D \phi}\omega_{vol}  = \pangle{F \wedge \star^2 D \phi} &= (-1)^{1(3-1)}\pangle{F \wedge D \phi} = (-1)^2\psquare{d\pangle{F \wedge \phi} -\pangle{DF \wedge \phi}} = d\pangle{F \wedge \phi}
}
so 
\eq{
E[A,\phi] &= \int_{\mbb{R}^3} \psquare{\pangle{F \mp \star D\phi, F\mp \star D\phi} \pm 2\pangle{F,\star D\phi}}d^3x \\
&= \int_{\mbb{R}^3} \abs{F\mp \star D\phi}^2 d^3x  \pm 2 \lim_{R \to \infty} \int_{S_R^2} \pangle{F \wedge \phi}
}
\begin{remark}
	In the above discussion, make sure to check that as this the $3d$ Hodge star, our inner products are well defined in the sense that the are the inner product of a $k$-form with a $k$-form. 
\end{remark}
This boundary term turns out to be a topological contribution, and we can see this in two ways. We first need to prove the following lemma
\begin{lemma}
	In the $\mbb{R}^3$ bulk 
	\eq{
		\pangle{\phi \wedge \pround{D \phi \wedge D \phi}} = \pangle{\phi \wedge \pround{d \phi \wedge d\phi}} + \abs{\phi}^2 \psquare{\pangle{F \wedge \phi} - d\pangle{A \wedge \phi}} - \frac{1}{2}\pangle{A \wedge \phi} \wedge d\abs{\phi}^2 
	}
\end{lemma}
\begin{proof}
	\hl{Exercise}. 
\end{proof}
\begin{corollary}
	Given the decay conditions, on $S_\infty^2$ we have 
	\eq{
		\pangle{F \wedge \phi} - d\pangle{A \wedge \phi} = -\pangle{\phi \wedge \pround{d\phi \wedge d\phi}}	
	}
	and hence the topological term is $\mp 2\int_{S_\infty^2} \pangle{\phi \wedge (d\phi \wedge d\phi)}$.
	\begin{comment}
	Now the inverse stereographic projection $f : \mbb{R}^2 \to S^2 \subset \mbb{R}^{3}$ allows us to pullback the standard metric on $\mbb{R}^{3}$. This can be done in Sage:
	\begin{lstlisting}[language=Python,frame=single]
	sage: M = Manifold(2, 'R^2')
	sage: U.<u,v> = M.chart()
	sage: N = Manifold(3, 'R^3')
	sage: X.<x,y,z> = N.chart()
	sage: f = M.diff_map(N,(2*u/(1+u^2+v^2), 2*v/(1+u^2+v^2),
	(-1+u^2+v^2)/(1+u^2+v^2)))
	sage: g = N.sym_bilin_form_field(name='g')
	sage: g[1,1], g[2,2], g[3,3] = 1, 1, 1
	sage: fg = f.pullback(g)
	sage: f.display()
	4/(u^4 + v^4 + 2*(u^2 + 1)*v^2 + 2*u^2 + 1) du*du 
	+ 4/(u^4 + v^4 + 2*(u^2 + 1)*v^2 + 2*u^2 + 1) dv*dv
	\end{lstlisting}
	\end{comment}
\end{corollary}
The two interpretations are then as follows: 
\begin{enumerate}
	\item Degree of a map of spheres - We will want to recall some standard vector calculus results stated in the language of forms. Firstly recall that if we have a manifold $M$ with volume form $\omega_M$ and orientable submanifold $\Sigma$ with normal $N$ we get a volume form on $\Sigma$ given by 
\eq{
\omega_\Sigma = i_N \omega_M 
}
If we parametrise $\Sigma$ on an open patch $U$, that is find $\psi:\mbb{R}^d \to U \subset \Sigma$, and do an area integral over $U \subset \Sigma$ in these coordinates, that is equivalent to pulling back $\omega_\Sigma$ by $\psi$. \\
By noting that, on $S_\infty^2$, $\phi:S_\infty^2 \to S_1^2 \subset \mf{su}(2)$, where on this $S^2$ $\phi$ is also the normal, and that we have chosen the metric on $\mf{su}(2)$ to be Euclidean, we have that 
\eq{
\int_{S_\infty^2} \pangle{\phi \wedge (d\phi \wedge d\phi)} = \int_{S_\infty^2} \bm{\phi} \cdot \pround{\del_u \bm{\phi} \times \del_v \bm{\phi}} du dv = \int_{S_\infty^2} \phi^\ast \omega_{S^2} = 4\pi \deg \phi 
}
where we have the used the notation $\phi = \bm{\phi} \cdot \bm{t}$, and the degree here refers to of $\phi$ as a map of spheres. This degree is a topological invariant.
\item Chern class of a line bundle - Consider the matrix $\bm{\phi} \cdot \bm{\sigma}$. We give the following lemma
\begin{lemma}
	Let $P = \frac{1}{2}\pround{I + \bm{\phi} \cdot \bm{\sigma}}$. Then $P$ is a projection operator, i.e. $P^2=P$. 
\end{lemma}
\begin{proof}
	Do the multiplication
\end{proof}
\begin{corollary}
	$\bm{\phi} \cdot\bm{\sigma}$ has eigenvalues $\pm 1$. 
\end{corollary}
\begin{proof}
	Projection operators have eigenvalues $0,1$. Hence we get the result, and as we can never have $\bm{\phi} \cdot\bm{\sigma} = \pm I$, both eigenvalues must occur. 
\end{proof}
Now we can consider the eigenvector bundle 
\eq{
L = \pbrace{(\bm{x}, \psi) \in S_\infty^2 \times \mbb{C}^2 \, | \, \psquare{\bm{\phi}(\bm{x}) \cdot\bm{\sigma}} \psi = \psi}
}
\begin{remark}
	This will be a complex line bundle, so the only possible Chern class we can relate to it will be $c_1$. 
\end{remark}
We can get a connection on the bundle by viewing it as a subbundle of $S_\infty^2 \times \mbb{C}^2$, and getting covariant derivative on vector fields by projection, that is $DX = PdX$.
\begin{remark}
	Note that here the variations in $\psi$ that preserve the the eigenvector condition are exactly those that scale $\psi$ by some element of $\mbb{C}^\times$. This really makes the bundle look like the tangent bundle to the sphere $S_\infty^2$ in the sense that there is a plane attached at every point of $S^2$. 
\end{remark} 
Now at every point $\bm{x} \in S_\infty^2$ if $\hat{\psi}(\bm{x})$ is a normalised eigenvector we have that a local section $\sigma : U \subset S_\infty^2 \to E$ is given by $\sigma = h \hat{\psi}$ where $h:U \to \mbb{C}$ is some scale. Then as we are dealing with $2\times 2$ non-trivial projection matrices we can write $P = \hat{\psi} \hat{\psi}^\dagger$ giving 
\eq{
D(h\hat{\psi}) &= \hat{\psi} \hat{\psi}^\dagger \psquare{(dh)\hat{\psi}+h(d\hat{\psi})} \\
&= (dh)\hat{\psi}+h\hat{\psi} \hat{\psi}^\dagger(d\hat{\psi})
}
What we have done by choosing a local section $\hat{\psi}$ is given a local trivialisation of the bundle. To find the connection locally we need to see how the covariant derivaitve acts locally on $\mbb{C}$, that is how it acts on $h$. By writing 
\eq{
	D(h \hat{\psi}) = \hat{\psi} \psquare{dh + h \hat{\psi}^\dagger d\hat{\psi}}
}
We can read off that $A = \hat{\psi}^\dagger d\hat{\psi}$ locally, giving $F_L = d\hat{\psi}^\dagger \wedge d\hat{\psi}$. \\
We now want to relate this to our map $\phi$. As we are dealing with $2\times 2$ non-trivial projection matrices we can write 
\eq{
P = \hat{\psi} \hat{\psi}^\dagger \Rightarrow \phi = -i\hat{\psi} \hat{\psi}^\dagger + \frac{i}{2}I \Rightarrow d\phi = -i\psquare{(d\hat{\psi})\hat{\psi}^\dagger +\hat{\psi} (d\hat{\psi}^\dagger)}
}
We can see that, as the inner product corresponds to the trace, $\pangle{d\phi \wedge d\phi} = \pangle{(\del_\mu \phi)(\del_\nu \phi)} dx^\mu \wedge dx^\nu = \frac{1}{2}\pangle{\comm[\del_\mu \phi]{\del_\nu \phi}} dx^\mu \wedge dx^\nu = 0$ which means 
\eq{
	\pangle{\phi \wedge (d\phi \wedge d\phi)} &= -i\pangle{\pround{\hat{\psi} \hat{\psi}^\dagger -\frac{1}{2}I}(d\phi \wedge d\phi)} \\ 
	&= -i \pangle{\hat{\psi} \hat{\psi}^\dagger\psquare{(d\hat{\psi})\hat{\psi}^\dagger + \hat{\psi}(d\hat{\psi}^\dagger)}^{\wedge 2}}
}
We expand out the latter, using the cyclicity of the trace and Leibniz' rule with $\hat{\psi}^\dagger \hat{\psi}=1$, for example
\eq{
\pangle{\hat{\psi} \hat{\psi}^\dagger (d\hat{\psi})\hat{\psi}^\dagger \wedge (d\hat{\psi})\hat{\psi}^\dagger} &= \pangle{\hat{\psi}^\dagger\hat{\psi} \hat{\psi}^\dagger (d\hat{\psi})\hat{\psi}^\dagger \wedge (d\hat{\psi})} \\
&= \pangle{\hat{\psi}^\dagger (d\hat{\psi})\hat{\psi}^\dagger \wedge (d\hat{\psi})} \\
&= \pangle{\psquare{d(\hat{\psi}^\dagger \hat{\psi} ) - (d\hat{\psi}^\dagger)\hat{\psi} }\hat{\psi}^\dagger \wedge (d\hat{\psi})} \\
&= - \pangle{(d\hat{\psi}^\dagger)\hat{\psi} \wedge \hat{\psi}^\dagger (d\hat{\psi})} \\
&= - \pangle{\hat{\psi} \hat{\psi}^\dagger \hat{\psi} (d\hat{\psi}^\dagger) \wedge \hat{\psi} (d\hat{\psi}^\dagger)}
} 
\eq{
 \pangle{\hat{\psi} \hat{\psi}^\dagger (d\hat{\psi}) \wedge (d\hat{\psi}^\dagger)}&= -\pangle{\hat{\psi}(d\hat{\psi}^\dagger) \hat{\psi} \wedge (d\hat{\psi}^\dagger) } \\
 &= \pangle{\hat{\psi}(d\hat{\psi}^\dagger) \wedge (d\hat{\psi})\hat{\psi}^\dagger } \\
 &= \pangle{d\hat{\psi}^\dagger \wedge d\hat{\psi}} \\
 &= \pangle{\hat{\psi} \hat{\psi}^\dagger \hat{\psi} (d\hat{\psi}^\dagger) \wedge (d\hat{\psi})\hat{\psi}^\dagger}
}
and as a result we get 
\eq{
\pangle{\phi \wedge (d\phi \wedge d\phi)} = -2i \pangle{ d\hat{\psi}^\dagger \wedge d\hat{\psi}} = -4\pi \pround{\frac{i}{2\pi} \pangle{F_L}}
}
We can then see that 
\eq{
	\int_{S_\infty^2} \pangle{\phi \wedge (d\phi \wedge d\phi)} = -4 \pi c_1(L)
}
\end{enumerate} 
\begin{remark}
	A point to be made about the above is that, a priori, the connection is not related to the Higgs field on $S_\infty^2$. It is the decay condition on $D\phi$ which enforces that on $S_\infty^2$ we have $\del_\mu \phi = \comm[\phi]{A_\mu}$.
\end{remark}


Through either discussion we have $E\geq \pm 8\pi k$ for some $k \in \mbb{Z}$ with equality iff $F=\mp \star D \phi$ where we choose the sign to make the bound positive. This is the \bam{BPS equation}.
\begin{remark}
	We might wonder what is the connection between these two viewpoints, where we have that $\deg \phi = -c_1(L)$. To make this connection we recall a result stated in \cite{Babelon2003}:
	\begin{prop}
		If $L$ is a complex line bundle on a Riemann surface and $f$ is a meromorphic section then 
		\eq{
	\deg(D(f))	= c_1(L)
	}
	\end{prop}
\begin{proof}
	See the Riemann Surfaces notes by Joel Robbin, University of Wisconsin, which in turn references \cite{Griffiths2014}. 
\end{proof}
We can think of $L$ as a complex line bundle over $\mbb{P}^1$ viewed as the Riemann sphere. If we take a section $\sigma = h \hat{\psi}$ where we choose $h=\hat{\psi}_2$, this section has poles exactly where the eigenvector is $(1,0) \Rightarrow \bm{\phi} = (1,0,0)$. Hence the Chern class is counting the number of preimages of $(1,0,0)$. \hl{We require an argument to tell us that these all should be counted with sign -1}. 
\end{remark}

%%%%%%%%%%%%%%%%%%%%%%%%%%%%%%%%%%%%%%%%%%%%%%%%%%%%%%%%
\subsection{Self-Dual Reduction}
Suppose now we consider pure Yang-Mills on $\mbb{R}^{n+1}$ with contant diagonal metric $g$ (i.e we are considering it to be either Minkowski or Euclidean). Take coordinates $x^\mu$ and ask that the connection $A = A_\mu dx^\mu$ is $x^0$-independent. Then writing $A = \phi dx^0 + A_i dx^i$ we have 
\eq{
F &= \frac{1}{2} F_{ij} dx^i \wedge dx^j + (D_i \phi) dx^i \wedge dx^0 \\
&= {\indices{^3}F} + D\phi \wedge dx^0 
}
as $F_{i0} = \del_i \phi + \comm[A_i]{\phi} = D_i \phi$ and denoting $\indices{^3}F = \frac{1}{2}F_{ij} dx^i \wedge dx^j$. We calculate
\eq{
\pangle{\indices{^3}F,D\phi \wedge dx^0} &= \frac{1}{2}\pangle{F_{ij}, D_k \phi} \underbrace{\pangle{dx^i \wedge dx^j , dx^k \wedge dx^0}}_{=0} \\
\pangle{D\phi \wedge dx^0,D\phi \wedge dx^0} &= \pangle{D_i \phi, D_j \phi} \underbrace{\pangle{dx^i \wedge dx^0, dx^j \wedge dx^0} }_{=g^{ij}g^{00}}
}
and so on Euclidean $\mbb{R}^{n+1}$, $\abs{F}^2 = \abs{\indices{^3}F}^2 + \abs{D\phi}^2 $. This means we recover the action for Yang-Mills-Higss with $0$ potential from this reduction. The Yang-Mills equation for $F$ reads 
\eq{
D \star F = 0 \Rightarrow g^{\mu\nu}D_\mu F_{\nu\rho} &= 0 
}
This splits to give 
\eq{
g^{\mu\nu}D_\mu F_{\nu 0} = 0 &\Rightarrow g^{ij} D_i D_j \phi = D_i D^i \phi = 0 \\
g^{\mu\nu} D_\mu F_{\nu k} = 0 &\Rightarrow -g^{00}D_0D_k \phi + g^{ij} D_i F_{jk} =0 \\
& \phantom{==} \;  -g^{00}\comm[\phi]{D_k \phi} +D_i F\indices{^i_k} = 0   
}
Moreover we have the stronger result:
\begin{prop}
	$F$ is (anti-)self-dual iff $(\indices{^3}F,\phi)$ satisfy the Bogomolny equations. 
\end{prop}
\begin{proof}
	The (anti-)self-duality equations for $F$ say $\star_4 F = (-)F$, where we are now making explicit the dimension wrt which $\star$ is acting. We can calculate 
	\eq{
\star_4 F &= \frac{1}{4} F_{ij} \eps\indices{^i^j_\mu_\nu} dx^\mu \wedge dx^\nu + \frac{1}{2} (D_i \phi) \eps\indices{^i^0_\mu_\nu} dx^\mu \wedge dx^\nu \\
&= -\frac{1}{2}(D_k\phi) \epsilon\indices{^k_i_j} dx^i \wedge dx^j - \frac{1}{2} F_{ij}\epsilon\indices{^i^j_k} dx^k \wedge dx^0 \\
&= -\star_3 D\phi - \star_3 F \wedge dx^0	
}
which we see to mean, using $\star_3^2=1$.  
\eq{
\star_4 F = \pm F \Leftrightarrow {}\indices{^3}F\pm \star_3 D \phi = 0
}
\end{proof}
%%%%%%%%%%%%%%%%%%%%%%%%%%%%%%%%%%%%%%%%%%%%%%%%%%%%%%%%
%%%%%%%%%%%%%%%%%%%%%%%%%%%%%%%%%%%%%%%%%%%%%%%%%%%%%%%%
\section{Constructions}
%%%%%%%%%%%%%%%%%%%%%%%%%%%%%%%%%%%%%%%%%%%%%%%%%%%%%%%%
\subsection{Nahm's Construction}
Relevant reading for this section is Manton \& Sutcliffe \cite{Manton2004} and Hitchin \cite{Hitchin1983}. The original paper is \cite{Nahm1983}. 

%%%%%%%%%%%%%%%%%%%%%%%%%%%%%%%%%%%%%%%%%%%%%%%%%%%%%%%%
\subsection{The ADHM construction}
This section follows the work first laid out in \cite{Atiyah1978}. Suppose we have the following information:
\begin{itemize}
	\item $W$ a $k$-dimensional vector space
	\item $V$ a $2k+2$-dimensional vector space with skew, non-degenerate bilinear form $(\cdot, \cdot):\wedge^2 V \to \mbb{C}$. 
	\item $z = (z_i) \in \mbb{C}^4$
	\item $A(z) = \sum_i A_i z_i \in \End(W,V)$ s.t. \
	\eq{
\forall z \neq 0, \; U_z \equiv A(z)W\subset V \text{ is isotropic and $k$-dimensional}	
}
\end{itemize}

We now state some important properties:

\begin{lemma}
	Let $E_z = \faktor{U_z^0}{U_z}$, then 
	\begin{itemize}
		\item $\dim E_z = 2$
		\item $E_z$ inherits a non-degenerate skew bilinear
		\item $\forall \lambda \in \mbb{C}^\times, \, E_z = E_{\lambda z}$. 
	\end{itemize}
\end{lemma}
\begin{proof}
	We go point by point:
	\begin{itemize}
		\item $\dim E_z = \dim U_z^0 - \dim U_z = \pround{\dim V - \dim U_z} - \dim U_z = 2k+2 - 2k = 2$.
		\item The bilinear on $W$ is only degenerate in $U_z^0$ on $U_z$, so by quotienting by this it descends directly to $E_z$. 
		\item $A(\lambda z) = \lambda A(z)$, so $A(\lambda z) (\lambda^{-1} \bm{w}) = A(z)(\bm{w})$. Hence we can see $U_{\lambda z} = U_z$ and so result.  
	\end{itemize}
\end{proof}

\begin{corollary}
	We get a vector bundle $E\to \mbb{CP}^3$ with group $SL(2,\mbb{C})$.
\end{corollary}


%%%%%%%%%%%%%%%%%%%%%%%%%%%%%%%%%%%%%%%%%%%%%%%%%%%%%%%%
%%%%%%%%%%%%%%%%%%%%%%%%%%%%%%%%%%%%%%%%%%%%%%%%%%%%%%%%
\bibliographystyle{../../bib/custom-bib-style}
\bibliography{../../bib/jabref_library.bib}

\end{document}
