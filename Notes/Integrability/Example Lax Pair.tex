\documentclass{article}

\usepackage{../../header}

\geometry{
 a4paper,
 total={170mm,257mm},
 left=20mm,
 top=20mm,
 }

%%%%%%%%%%%%%%%%%%%%%%%%%%%%%%%%%%%%%%%%%%%%%%%%%%%%%%%%
%Preamble

\title{An example of a Lax pair}
\author{Linden Disney-Hogg}
\date{June 2020}

%%%%%%%%%%%%%%%%%%%%%%%%%%%%%%%%%%%%%%%%%%%%%%%%%%%%%%%%
%%%%%%%%%%%%%%%%%%%%%%%%%%%%%%%%%%%%%%%%%%%%%%%%%%%%%%%%
\begin{document}
\maketitle

\section{1 dimension}
Consider the $\mf{su}(2)$ algebra generated by $\pbrace{H,E,F}$ with the relations
\eq{
\comm[H]{E} &= 2E \\
\comm[H]{F} &= -2F \\
\comm[E]{F} &= H 
}
Let us re-write this by introducing $X=E+F, \, Y=E-F$, which now have the commutation relations 
\eq{
\comm[H]{X} &= 2Y \\
\comm[H]{Y} &= 2X \\
\comm[X]{Y} &= -2H 
}
Consider the Hamiltonian system with phase space $q,p \in \mbb{R}$ and Hamiltonian 
\eq{
\mc{H} = \frac{1}{2}p^2 + V(q)
}
\begin{prop}
	The algebra elements 
	\eq{
L(\zeta) &= pH + W(q)X + \zeta Y , \quad \zeta \in \mbb{C} \\
M &= \frac{1}{2}W_q Y 	
}
for a Lax pair for the system if $V_q = WW_q$ (i.e. $V = \frac{1}{2}W^2 + c$) and $\cdot_{q} = \del_q \cdot$. 
\end{prop}
\begin{proof}
	Hamilton's equations for this system are 
	\eq{
\dot{q} &= p \\
\dot{p} &= -V_q 	
}
so 
\eq{
\dot{L} &= -V_q H + pW_q X 
}
whereas 
\eq{
\comm[L]{M} &= pW_q X -WW_q H 
}
\end{proof}
Now we want to think of generalisations of this Lax pair. A generic Hamiltonian for a $2d$ phase space can be written in the form 
\eq{
\mc{H} = \frac{p^2}{2\lambda(q)} + V(q)
}
\begin{prop}
	The algebra elements 
	\eq{
		L(\zeta) &= \frac{1}{\sqrt{\lambda}}pH + W(q)X + \zeta Y , \quad \zeta \in \mbb{C} \\
		M &= \frac{1}{2\sqrt{\lambda}}W_q Y 	
	}
	form a Lax pair for the system.
\end{prop}
\begin{proof}
	We repeat a similar calculation: 
	\eq{
\dot{q} &= \frac{p}{\lambda} \\
\dot{p} &= -V_q + \frac{p^2 \lambda_q}{2\lambda^2}	
}
so 
\eq{
\dot{L} &= \frac{1}{\sqrt{\lambda}} \psquare{\pround{-V_q + \frac{p^2 \lambda_q}{2\lambda^2}} - \frac{p^2\lambda_q}{2\lambda^2}}H  + \frac{pW_q}{\lambda}X \\
\comm[L]{M} &= \frac{pW_q}{\lambda}X - \frac{WW_q}{\sqrt{\lambda}} H
}
\end{proof}
Now if, like me, you do not see the spitting obvious thing that scaling $p \to \frac{p}{\sqrt{\lambda}}$ is a sensible thing to do, how might you approach this? Start by supposing a more general form related to our original pair 
\eq{
L(\zeta) &= f(q,p)H + g(q,p)X + \zeta Y \\
M &= h(q,p)Y
}
We then get that for $L,M$ to be a Lax pair we have the equations 
\eq{
f_q \pround{\frac{p}{\lambda}} + f_p \pround{-V_q + \frac{p^2 \lambda_q}{2\lambda^2}} &= -2gh \\
g_q \pround{\frac{p}{\lambda}} + g_p \pround{-V_q + \frac{p^2 \lambda_q}{2\lambda^2}} &= 2fh
}
Let's make the ansatz that $g,h$ are functions of $q$ only. Then we get from the second equation 
\eq{
g_q \cdot \frac{p}{\lambda} = 2fh
}
Equation the order of $p$ on each side we must get $f(q,p) = pF(q)$ and then 
\eq{
g_q = 2Fh \lambda
}
Substituting our new form of $f$ into the first equation gives 
\eq{
\frac{p^2F_q}{\lambda} + F\pround{-V_q + \frac{p^2 \lambda_q}{2\lambda^2}} = -2gh
}
Again equating orders of $p$ we have 
\eq{
\frac{F_q}{\lambda} + \frac{F\lambda_q}{2\lambda^2} = 0 \Rightarrow F_q \lambda^\frac{1}{2} + \frac{1}{2} F \lambda^{-\frac{1}{2}} \lambda_q &= 0 \\
\Rightarrow \pround{F\lambda^\frac{1}{2}}_q &= 0 \\
\Rightarrow F &= \frac{\alpha}{\sqrt{\lambda}}, \quad \alpha \in \mbb{R}
}
Subbing this back into the first equation gives 
\eq{
-\alpha \lambda^{-\frac{1}{2}} V_q = -2gh
}
so 
\eq{
g g_q &= \frac{\alpha \lambda^{-\frac{1}{2}} V_q}{2h} 2 \alpha h \lambda^{\frac{1}{2}} = \alpha^2 V_q 
}
and 
\eq{
h = \frac{\lambda^{-\frac{1}{2}}g_q}{2\alpha}
}
We recognise taking $g=W, \, \alpha=1$, this is the solution from above.

%%%%%%%%%%%%%%%%%%%%%%%%%%%%%%%%%%%%%%%%%%%%%%%%%%%%%%%%
%%%%%%%%%%%%%%%%%%%%%%%%%%%%%%%%%%%%%%%%%%%%%%%%%%%%%%%%
\section{2 dimensions}
We want to now try to see if we can expand upon this result. Note that our first systems has a natural generalisation of the following form:
\begin{prop}
The algebra elements 
\eq{
L(\bm{\zeta}) &= \sum_{i=1}^n p_i H_i + W_i X_i + \zeta_iY_i \\
M &= \frac{1}{2}  W_i^\prime Y_i 
}
form a Lax pair for the evolution of the Hamiltonian 
\eq{
\mc{H} = \sum_{i=}^n \frac{1}{2}p_i^2 + V_i(q^i)
}
where $\pangle{H_i,X_i,Y_i}$ are distinct copies of the previous algebra that commute with each other. 
\end{prop}
This, however, covers only a small class of Hamiltonians, so we might try the next simplest non-trivial case; that of the 2d in Liouville form.
\begin{definition}
	An $n$-dimensional \bam{Liouville system} is one whose Hamiltonian is of the form 
	\eq{
\mc{H} = \frac{1}{\lambda}\psquare{\sum_{i=1}^n \frac{1}{2}\sigma_i p_i^2 + V_i	}
}
where $\lambda = \sum_{i=1}^n \lambda_i$ and $\lambda_i,\sigma_i,V_i$ are functions of $q^i$ only.
\end{definition}
We have the following theorem that says that in 2d, considering Liouville form is sufficiently general:
\begin{theorem}
	On a $2d$ Riemannian manifold, any separable metric can be written locally in Liouville form. 
\end{theorem}
\begin{proof}
	Historically, this predates St\"ackel's theorem, but using St\"ackel the proof becomes very easy. Write the St\"ackel matrix as 
	\eq{
U &= \begin{pmatrix}
	\sfrac{\lambda_1}{\sigma_1} & -\sfrac{1}{\sigma_1} \\ 
		\sfrac{\lambda_2}{\sigma_2} & \sfrac{1}{\sigma_2}
\end{pmatrix} \\
\Rightarrow U^{-1} &= \frac{1}{\lambda}\begin{pmatrix}
	\sigma_1 & \sigma_2 \\ -\lambda_2\sigma_1 & \lambda_1 \sigma_2
\end{pmatrix}
}
We can then read off the top row. 
\end{proof}
Hamilton's equations for a Liouville system can be read off as 
\eq{
\dot{q}^i &= \frac{\sigma_i p_i}{\lambda} \\
\dot{p}_i &= \frac{\lambda_i^\prime}{\lambda}\mc{H} - \frac{1}{\lambda}\psquare{\frac{1}{2}\sigma_i^\prime p_i^2 + V_i^\prime}
}
Suppose we naively tried to port our Lax pair from the 1d system using the rough scaling argument, that is have 
\eq{
L(\bm{\zeta}) &= \sum_{i=1}^2 \sqrt{\frac{\sigma_i}{\lambda}}p_i H_i +\frac{1}{\sqrt{\lambda}}W_i X_i + \zeta_i Y_i \\
M &= \sum_{i=1}^2 \frac{1}{2}\frac{\sqrt{\sigma_i}}{\lambda}\pround{W_i^\prime-\frac{\lambda_i^\prime}{2\lambda}W_i} Y_i 
}
To calculate the terms in $\dot{L}$ it is necessary to calcualte
\eq{
\frac{d}{dt}\sqrt{\frac{\sigma_i}{\lambda}} &= \sqrt{\frac{\sigma_i}{\lambda}}\pbrace{\frac{\sigma_i p_i}{\lambda}\psquare{\frac{\sigma_i^\prime}{2\sigma_i} - \frac{\lambda_i^\prime}{2\lambda}}- \sum_{j \neq i}\frac{\sigma_j p_j \lambda_j^\prime}{2\lambda^2}} \\
\Rightarrow \frac{d}{dt}p_i\sqrt{\frac{\sigma_i}{\lambda}} &= \sqrt{\frac{\sigma_i}{\lambda}}\pbrace{\frac{\sigma_i p_i^2}{\lambda}\psquare{\frac{\sigma_i^\prime}{2\sigma_i} - \frac{\lambda_i^\prime}{2\lambda}}- p_i\sum_{j \neq i}\frac{\sigma_j p_j \lambda_j^\prime}{2\lambda^2} + \dot{p}_i} \\
&= \frac{1}{\lambda} \sqrt{\frac{\sigma_i}{\lambda}}\pbrace{\sigma_i p_i^2\psquare{\frac{\sigma_i^\prime}{2\sigma_i} - \frac{\lambda_i^\prime}{2\lambda}}- p_i\sum_{j \neq i}\frac{\sigma_j p_j \lambda_j^\prime}{2\lambda} + \lambda_i^\prime \mc{H} -\psquare{\frac{1}{2}\sigma_i^\prime p_i^2 + V_i^\prime}} \\
&= \frac{1}{\lambda} \sqrt{\frac{\sigma_i}{\lambda}}\pbrace{\psquare{\frac{1}{2\lambda}\sum_{j \neq i }\sigma_j p_j (\lambda_i^\prime p_j- \lambda_j^\prime p_i)} + \frac{\lambda_i^\prime}{\lambda}\sum_j V_j - V_i^\prime } 
}

\end{document}
