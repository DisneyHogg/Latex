\documentclass{article}

\usepackage{../../header-colourful}
%%%%%%%%%%%%%%%%%%%%%%%%%%%%%%%%%%%%%%%%%%%%%%%%%%%%%%%%
%Preamble

\title{Gauge Theory Notes}
\author{Linden Disney-Hogg}
\date{January 2020}

%%%%%%%%%%%%%%%%%%%%%%%%%%%%%%%%%%%%%%%%%%%%%%%%%%%%%%%%
%%%%%%%%%%%%%%%%%%%%%%%%%%%%%%%%%%%%%%%%%%%%%%%%%%%%%%%%
\begin{document}

\maketitle
\tableofcontents

%%%%%%%%%%%%%%%%%%%%%%%%%%%%%%%%%%%%%%%%%%%%%%%%%%%%%%%%
%%%%%%%%%%%%%%%%%%%%%%%%%%%%%%%%%%%%%%%%%%%%%%%%%%%%%%%%
\section{Introduction}
These are lecture notes taken from "Topics in Mathematical Physics" lectured by Martens and the SMSTC Gauge theory. It may go poorly trying to type this live but let's see. The course will be covering essentially differential geometric aspects of gauge theory. After the introduction, the first part will correspond to the EKC lectures given by Jos\'{e}, indiscriminately folded in.  
%%%%%%%%%%%%%%%%%%%%%%%%%%%%%%%%%%%%%%%%%%%%%%%%%%%%%%%%
\subsection{Historical Overview}
\subsubsection{Physics}
Starting with Maxwell's equations in the 19th century for describing classical electromagnetism, these are differential equations for some fields. These were reformulated by Weyl in the 1920-30s, and the term gauge theory was coined. Namely, electromagnetism was reformulated as a classical $U(1)$ gauge theory. The Aharonov-Bohm theory gave that this extra degree of freedom corresponding to the vector potential has a physical significance for quantum mechanics. In the 1950s, Yang \& Mills gave the equations for gauge theories with arbitrary non-abelian Lie groups. The physical relevance of the commutativity is that non-abelian groups - when quantised - lead to particles which self interact. Note that this is not the case for photons (which correspond to $U(1)$). In the 1960-70s it was shown that non-abelian gauge theories can be renormalised, so correctly quantised, and this allows the standard model to be developed. This is a quantum gauge theory with group $U(1)\times SU(2) \times SU(3)$.

\subsubsection{Maths}
In the 1930-60s the theory of the principal bundle was developed, and this included the concept of connections. In the 1970s it was realised that these are the same constructs as being developed in physics. In the late 70s the concept of moduli space of instantons was being developed, and in the 80s the Donaldson invariant theory was being worked on. These are invariants of differential manifolds, and led to the discovery of exotic $\mbb{R}^4$. This continued to develop ideas of Seiberg-Witten invariants in the 90s. 
%%%%%%%%%%%%%%%%%%%%%%%%%%%%%%%%%%%%%%%%%%%%%%%%%%%%%%%%
\subsection{Course direction}
The course will be slightly dictated by the will of the class, but the basic structure will be as follows:
\begin{itemize}
    \item Review of manifolds and De Rham cohomology
    \item Riemannian and symplectic geometry, including symplectic reduction
    \item Lie groups, Lie algebras, and basic rep theory
    \item Principal bundles, associated vector bundles, fibre bundles
    \item Connections from different points of view 
    \item Chern-Weil theory, characteristic classes, and classifying spaces
    \item Yang Mills functional, electromagnetism as a gauge theory, and the Aharonov-Bohm experiment
    \item Gauge theory in dimensions 2,3 and 4, instantons, Chern-Simons theory
\end{itemize}
%%%%%%%%%%%%%%%%%%%%%%%%%%%%%%%%%%%%%%%%%%%%%%%%%%%%%%%%
\subsection{Suggested Reading}
Possible texts to look at include: \\
Mathematically sound, though written for physicists - 
\begin{itemize}
    \item Nakahara, "Geometry, topology \& physics" 
    \item Nash, "Topology and geometry for physicists" 
    \item Frankel, "Geometry of physics" 
\end{itemize}
Written for mathematicians - 
\begin{itemize}
    \item Marcathe \& Martucci, "The mathematical formulation of gauge theory"
    \item Naber, "Topology, geometry, and gauge fields"
\end{itemize}
Other options:
\begin{itemize}
	\item Morgan, "Intro to gauge theory" in Gauge theory and the topology of 4-manifolds
	\item Nicolaescu, "Lectures on the geometry of manifolds". 
\end{itemize}
%%%%%%%%%%%%%%%%%%%%%%%%%%%%%%%%%%%%%%%%%%%%%%%%%%%%%%%%
%%%%%%%%%%%%%%%%%%%%%%%%%%%%%%%%%%%%%%%%%%%%%%%%%%%%%%%%
\subsection{Prelim Tools}
%%%%%%%%%%%%%%%%%%%%%%%%%%%%%%%%%%%%%%%%%%%%%%%%%%%%%%%%
\subsubsection{Quaternions}
Here we will cover just some separate material that will be necessary for some comprehension, but is very separate. 

\begin{definition}
	The \bam{quaternions}, denoted $\mbb{H}$, is the real associative algebra spanned by $\pbrace{1,i,j,k}$ subject to 
	\eq{
i^2=-1=j^2, \; ij=k = -ji	
}
For $q=xi+yj+zk+w\in \mbb{H}$ we define its \bam{real part} as $\real(q) = w$ and as such its \bam{quaternionic conjugate} by 
\eq{
\bar{q} = -xi-yj-zk+w
}
\end{definition}

\begin{lemma}
	We can deduce the following relations:
	\begin{itemize}
		\item $k^2=-1$
		\item $jk = i = -kj$
		\item $ki = j = -ik$
		\item $\bar{q_1 q_2} = \bar{q}_2 \bar{q}_1$. 
	\end{itemize}
\end{lemma}
\begin{proof}
	These will all be calculations:
	\begin{itemize}
		\item $k^2 = -ij^2i = i^2=-1$
		\item $jk = -j^2i = i$ and likewise for $kj$
		\item $ki = -ji^2 = j$ and likewise for $ik$.
		\item If we let $q_a = x_a i + y_a j + z_a k + w_a$ then 
		\eq{
q_1 q_2 &= x_1x_2 i^2 + y_1y_2 j^2 +	z_1 z_2 k^2 + w_1 w_2 + ij(x_1y_2 - x_2y_1) + jk(y_1 z_2 - y_2 z_1) + ki(z_1 x_2 - z_2 x_1) \\
&\phantom{=} + i(x_1 w_2 + x_2w_1) + j(y_1 w_2 + y_2 w_1) + k(z_1 w_2 + z_2 w_1) \\
&= w_1w_2 - x_1 x_2 - y_1 y_2 - z_1 z_2 + i [(x_1 w_2 + x_2w_1)+(y_1 z_2 - y_2 z_1)] \\
&\phantom{=} + j[(y_1 w_2 + y_2 w_1) + (z_1 x_2 - z_2 x_1)] + k[(z_1 w_2 + z_2 w_1) + (x_1y_2 - x_2y_1)]
	}
We can see that taking the negative of the $i,j,k$ coefficients is equivalent to taking the negative of the $x_a,y_a,z_a$ and swapping $1 \leftrightarrow 2$. 
	\end{itemize}
\end{proof}
\begin{lemma}
	$(xi+yj+zk)^2 = -(x^2+y^2+z^2)$.
\end{lemma}
\begin{corollary}
	$\forall \zeta \in \mbb{C}^\infty$ we get a complex structure $I_\zeta$ given by $xi+yj+zk$ where 
	\eq{
(x,y,z) = \pround{\frac{1-\zeta\bar{\zeta}}{1+\zeta\bar{\zeta}}, \frac{\zeta+\bar{\zeta}}{1+\zeta\bar{\zeta}}, \frac{i(\zeta-\bar{\zeta})}{1+\zeta\bar{\zeta}}}	
}
This is as we are sending $\mbb{C}^\infty$ to the Riemann sphere. 
\end{corollary}

\begin{definition}
	Define an inner product on the quaternion space by $\pangle{q_1,q_2} = \real(\bar{q}_1 q_2)$ and the norm induced of the quaternions as $\norm{q} = \sqrt{\pangle{q,q}}=\sqrt{\bar{q}q}$ 
\end{definition}

\begin{lemma}
	$\pangle{q_1,q_2} = x_1x_2 + y_1y_2 + z_1z_2 + w_1w_2$. 
\end{lemma}
\begin{proof}
	Using the above calculation of $q_1,q_2$ the answer is immediate. 
\end{proof}

\begin{lemma}
If $q = xi + yj + zk + w$ then $\norm{q} = \sqrt{x^2 + y^2 + z^2 + w^2}$. Further, $\norm{q_1 q_2} = \norm{q_1}\norm{q_2}$. 
\end{lemma}
\begin{proof}
	Using the proof of the above lemma the first result is immediate. To prove the latter note $\norm{q_1 q_2} = \sqrt{\bar{q_2} \bar{q}_1 q_1 q_2}$, and as $\bar{q}_1 q_1$ is a positive real it will pull out of the square root. 
\end{proof}

\begin{remark}
	We can extend the inner product to one on the vector space $\mbb{H}^n$. In this v.sp. we will have our quaternion action to be on the right. 
\end{remark}

\begin{definition}
	We can give $\mbb{H}^n$ 3 skew-symmetric forms 
	\eq{
\omega_1(v,w) &= \pangle{iv,w}, & \omega_2(v,w) &= \pangle{jv,w}, & \omega_3(v,w) &= \pangle{kv,w}	
}
\end{definition}

Now $\mbb{H}^n$ has a left action of quaternionic matrices:

\begin{definition} 
	We define the group $\Sp(n)$ to be the \bam{quaternionic unitary matrices} that preserve the inner product on $\mbb{H}^n$ 
\end{definition}

\begin{remark}
	\hl{What is the relation of these matrices to the symplectic matrices?}
\end{remark}

As we have a group action preserving the inner product, it also preserves the skev-symmetric forms. Hence for any subgroup $G \leq \Sp(n)$ we get $3$ moment maps $\mu_i$ corresponding to $\omega_i$, or said otherwise a map 
\eq{
\mu : \mbb{H}^n \to \mf{g}^\ast \otimes \mbb{R}^3
}
\begin{example}[Important, see \cite{Hitchin1987}]\label{ex:quaternionic moment map equation example}
	Take $M = \mbb{H}^{n^2}$ the space of $n\times n$ quaternionic matrices, and let $G=O(n)$ act on $M$ via conjugation. Write $A \in M$ as 
	\eq{
A &= A_0 + iA_1 + jA_2 + kA_3 \\
&= (A_0+iA_1) + k(A_3 + iA_2) \\
&= Z+kW	
}
For simplifying reasons we will assume the $A_i$ are self-adjoint. Now this conjugation action is an exercise you may read in my CQIS notes, and the end result is that 
\eq{
\mu_1(A) &= \frac{i}{2}\comm[Z]{Z^\dagger} + \frac{i}{2} \comm[W]{W^\dagger} \\
&= \comm[A_2]{A_3}-\comm[A_0]{A_1} \\
\intertext{and likewise} \mu_2(A) &= \comm[A_3]{A_1} - \comm[A_0]{A_2} \\
\mu_3(A) &= \comm[A_1]{A_2} - \comm[A_0]{A_3}
}
We are cycling $A_{1,2,3}$ to get these. 
\end{example}

\begin{definition}
	We will define the \bam{quaternionic moment map equation} to be $\mu =0$. 
\end{definition}

\begin{lemma}
	$(\omega_3+i\omega_2) -2i\zeta\omega_1 + \zeta^2 (\omega_3-i\omega_2)$ is a complex symplectic form wrt to the complex structure $I_\zeta$. 
\end{lemma}
\begin{lemma}
	If $G \lact \mbb{H}^n$ preserves each $\omega_i$, then the moment map corresponding to $(\omega_3+i\omega_2) -2i\zeta\omega_1 + \zeta^2 (\omega_3-i\omega_2)$ is 
	\eq{
(\mu_3+i\mu_2) -2i\zeta\mu_1 + \zeta^2 (\mu_3-i\mu_2)	
}
(\hl{check factors of $i$ here})
\end{lemma}

\begin{prop}
	The quaternionic moment map equation $\mu=0$ is equivalent to 
	\eq{
\forall \zeta, \, 	(\mu_3+i\mu_2) -2\zeta\mu_1 + \zeta^2 (\mu_3-i\mu_2)=0
}
\end{prop}

\begin{example}
	In the same situation as example \ref{ex:quaternionic moment map equation example} we get that the moment map equation is 
	\eq{
\comm[(A_0+iA_1) +\zeta(A_3-iA_2)]{(A_3+iA_2) + 2i\zeta A_1 + \zeta^2(A_3-iA_2)}=0	
}
This gives a ``Lax pair" formulation of these equations. 
\end{example}

We can find a matrix representation of the quaternions as 
\[
I = \begin{pmatrix}
i & 0 \\ 0 & -i\end{pmatrix}, \, J = \begin{pmatrix}
0 & 1 \\ -1 & 0 \end{pmatrix}, \, K = \begin{pmatrix}
0 & i \\ i & 0  \end{pmatrix}
\]
\begin{lemma}
	This identification gives a group isomorphism $\operatorname{Sp}(1) \cong \operatorname{SU(2)}$. 
\end{lemma}
%%%%%%%%%%%%%%%%%%%%%%%%%%%%%%%%%%%%%%%%%%%%%%%%%%%%%%%%
%%%%%%%%%%%%%%%%%%%%%%%%%%%%%%%%%%%%%%%%%%%%%%%%%%%%%%%%
%%%%%%%%%%%%%%%%%%%%%%%%%%%%%%%%%%%%%%%%%%%%%%%%%%%%%%%%
%%%%%%%%%%%%%%%%%%%%%%%%%%%%%%%%%%%%%%%%%%%%%%%%%%%%%%%%
\part{EKC}
%%%%%%%%%%%%%%%%%%%%%%%%%%%%%%%%%%%%%%%%%%%%%%%%%%%%%%%%
%%%%%%%%%%%%%%%%%%%%%%%%%%%%%%%%%%%%%%%%%%%%%%%%%%%%%%%%
\section{Fibre Bundles}
%%%%%%%%%%%%%%%%%%%%%%%%%%%%%%%%%%%%%%%%%%%%%%%%%%%%%%%%
%%%%%%%%%%%%%%%%%%%%%%%%%%%%%%%%%%%%%%%%%%%%%%%%%%%%%%%%
\subsection{Introduction}
%%%%%%%%%%%%%%%%%%%%%%%%%%%%%%%%%%%%%%%%%%%%%%%%%%%%%%%%
\begin{definition}[Fibre Bundle]
	A \bam{fibre bundle} consists of a smooth surjection $\pi : E \to M$ between manifolds $E$ (the \bam{total space}) and $M$ (the \bam{base space}) and such that $\forall a \in M$ there exists a neighbourhood $U \ni a$ and a diffeomorphism $\varphi : \pi^{-1} \to U \times F$ (a \bam{local trivialisation}) for some manifold $F$ (the \bam{typical fibre}) such that the following triangle commutes 
	\begin{center}
		\begin{tikzcd}
			\pi^{-1}(U) \arrow[r,"\varphi"] \arrow[d,"\pi"] & U \times F \arrow[dl,"pr_2"] \\ U 
		\end{tikzcd}
	\end{center}
\end{definition}

We often write $F \to E \overset{\pi}{\to} M$. If we can take $U = M$ we say that $E$ is a \bam{trivial bundle}. Now suppose that $(U,\varphi), (V,\psi)$ are local trivialisations with $U \cap V \neq \emptyset$. Then we have two ways to view $\pi^{-1}(U \cap V)$ as a product. 
\begin{tkz}
	(U\cap V) \times F \arrow[dr,"pr_2"] & \pi^{-1}(U \cap V) \arrow[l,"\psi"] \arrow[d,"\pi"] \arrow[r,"\varphi"] & (U \cap V) \times F \arrow[dl,"pr_2"] \\ & U\cap V & 
\end{tkz}
and hence 
\eq{
	\psi \circ \varphi^{-1} : (U \cap V) \times F &\to (U\cap V) \times F \\
	(a,p) &\mapsto (a,\Phi(a,p)) 
}
where $\Phi(a,\cdot) : F \to F$ is a diffeomorphism, and hence it defines a \bam{transition function} $g : U \cap V \to Diff(F)$. 

\begin{definition}
	Let $F \to E \overset{\pi}{\to} M$. A collection $\pbrace{(U_\alpha,\varphi_\alpha)}$ of local trivialisations where $M = \cup_{\alpha} U_\alpha$ is called a \bam{trivialising atlas} for $E \overset{\pi}{\to} M$. 
\end{definition}

Let us introduce the notation $U_{\alpha\beta} = U_\alpha \cap U_\beta$, etc. and $g_{\alpha\beta}$ the transition function defined by $\varphi_\alpha \circ \varphi_\beta^{-1}$. 

\begin{fact}
	The transition functions satisfy the \bam{cocycle conditions}
	\begin{itemize}
		\item $\forall a \in U_\alpha, \, g_{\alpha\alpha}(a) = \id_F$
		\item $\forall a \in U_{\alpha\beta}, \, g_{\alpha\beta}(a)g_{\beta\alpha}(a) = \id_F$
		\item $\forall a \in U_{\alpha\beta\gamma}, \, g_{\alpha\beta}(a)g_{\beta\gamma}(a)=g_{\alpha\gamma}(a)$.
	\end{itemize}
\end{fact}

\begin{definition}
	Let $E \overset{\pi}{\to} M$, $E^\prime \overset{\pi^\prime}{\to} N$ be fibre bundles. A bundle map is a pair $(\Phi,\phi)$ of smooth maps $\Phi:E \to E^\prime$, $\phi:M \to N$ such that the following commutes
	\begin{tkz}
		E \arrow[r,"\Phi"] \arrow[d,"\pi"'] & E^\prime \arrow[d,"\pi^\prime"] \\ M \arrow[r,"\phi"] & N
	\end{tkz}
	Since $\pi$ is surjective, $\phi$ is uniquely determined by $\Phi$, which is said to \bam{cover} $\phi$. Notice that $\Phi$ is \bam{fibre preserving}. 
\end{definition}

\begin{definition}
	Let $f:M \to  N$ be smooth and $E \overset{\pi_E}{\to} N$ a fibre bundle. Then we can define the \bam{pullback bundle} $f^\ast E \to M$ as the categorical pullback, i.e. 
	\eq{
		f^\ast E \equiv \pbrace{(a,e) \in M \times E \, | \, \pi_E(e) = f(a)}
	}
\end{definition}
Restricting the canonical projections from $M \times E$ we get maps $\pi : f^\ast E \to M$, $\Phi:f^\ast E \to E$ making the following commute 
\begin{tkz}
	f^\ast E \arrow[r,"\Phi"] \arrow[d,"\pi"'] & E \arrow[d,"\pi_E"] \\ M \arrow[r,"f"] & N
\end{tkz}
Taking $ a \in M$, and $(V, \psi)$ a local trivialisation for $E \to N$ with $f(a) \in V$, then $(f^{-1}(V),\varphi)$ with $\varphi: \pi^{-1}(f^{-1}(V)) \to f^{-1}(V) \times F$ defined by $\varphi(b,e) = (b,pr_2(\psi(e))$ is a local trivialisation for $f^\ast E \to M$. This shows that $f^\ast E \to M$ is a fibre bundle, and it has fibres $(f^\ast E)_a = E_{f(a)}$. 

\begin{definition}
	A \bam{section} of a fibre bundle $F \to E \overset{\pi}{\to} M$ is a smooth map $s:M \to E$ such that $\pi\circ s = \id_M$. 
\end{definition}

Sections \emph{may not exist}, but if the fibre bundle is trivial, then any smooth map $\sigma : M \to F$ defines a sections by $s(a) = (a,\sigma(a))$. Since fibres are locally trivial, they admit local sections $s_\alpha : U_\alpha \to \pi^{-1}(U_\alpha)$ via local smooth maps $\sigma_\alpha : U_\alpha \to F$. A section $s:N \to E$ can be pulled back via $f:M \to N$ to give a section $f^\ast s : M \to f^\ast E$ via $(f^\ast s)(a) = (a,s(f(a)))$. 

\begin{definition}
	Consider $F \to E \overset{\pi}{\to} M$. Then the fibres $E_a = \pi^{-1}(a) \subset E$ are submanifolds of $E$. The tangent space at $e \in E_a$ is $\vartheta_e = \ker((\pi_\ast)_e : T_e E \to T_e M)$ and is called the \bam{vertical subspace} of $T_eE$
\end{definition}
In the absence of any additional structure, there is no preferred complementary subspace of $T_eE$.

\begin{definition}
	A \bam{connection} on $E \to M$ is a smooth choice of complementary subspace $\ms{H}_e \subset T_e E$ i.e. $T_eE = \vartheta_e \oplus \ms{H}_e$. That is, a connection is a distribution $\ms{H}\subset TE$
\end{definition}

Note $\ev{(\pi_\ast)_e}{\ms{H}_e} : \ms{H}_e \overset{\cong}{\to} T_{\pi(e)}M$, so $\ms{H}$ gives a choice of how to lift tangent vectors, and so curves, from $M$ to $E$. \\
Given a distribution one can ask whether it is integrable (in the sense of Frobenius), i.e. is $E$ foliated by submanifolds whose tangent spaces are $\ms{H}$. We shall see that the obstruction to the integrability of $\ms{H}$ can be interpreted as the `curvature' of the connection. 
%%%%%%%%%%%%%%%%%%%%%%%%%%%%%%%%%%%%%%%%%%%%%%%%%%%%%%%%
%%%%%%%%%%%%%%%%%%%%%%%%%%%%%%%%%%%%%%%%%%%%%%%%%%%%%%%%
%%%%%%%%%%%%%%%%%%%%%%%%%%%%%%%%%%%%%%%%%%%%%%%%%%%%%%%%
%%%%%%%%%%%%%%%%%%%%%%%%%%%%%%%%%%%%%%%%%%%%%%%%%%%%%%%%
\subsection{Principal fibre bundles}
We now specialise to principal fibre bundles, so called because the typical fibre is a principally homogeneous space for a lie group. 

\begin{definition}
	A \bam{Lie group} consists of a manifold $G$ which is also a group such that group multiplication $G \times G \to G$, $(g,h)\mapsto gh$, and group inversion $G \to G$, $g \mapsto g^{-1}$, are smooth maps
\end{definition}

For $g \in G$ a Lie group, we define diffeomorphisms $L_g : G \to G$, $L_g(h) = gh$, and $R_g : G \to G$, $R_g(h) = hg$, call \bam{left} \& \bam{right} multiplication. 

\begin{definition}
	Recall that given a diffeomorphism $F:M \to N$ we define the \bam{pushforward} $F_\ast : \mf{X}(M) \to \mf{X}(N)$ by, for $\xi \in \mf{X}(M), \, f \in C^\infty(N), \, (F_\ast \xi)(f) = \xi(f \circ F)$.
\end{definition}

\begin{remark}
	Note that given any smooth map of manifolds $F:M \to N$, the derivative $dF:TM \to TN$ gives a map $\forall a \in M, \, dF_a : T_a M \to T_{F(a)}N$ which for $\xi \in T_a M, \, f \in C^\infty(N)$ acts as $(dF_a(\xi))(f) = \xi(f \circ F)$. This is often written as $F_\ast$, but the two concepts are subtly different.
\end{remark}

\begin{definition}
	A vector field $\xi\in \mf{X}(G)$ is \bam{left invariant} if $\forall g \in G, \, (L_g)_\ast \xi = \xi$. Similarly we define right invariant.
\end{definition}

\begin{lemma}
	If $\xi$ is a LIVF, $\xi_g = (L_g)_\ast \xi_e$, where $e\in G$ is the identity.
\end{lemma}
\begin{proof}
	Let $f \in C^\infty(G)$. Then
	\eq{
		(L_g)_\ast \xi = \xi \Rightarrow \xi(f \circ L_g) = \xi(f)
	}
	Now evaluating at $g \in G$, $\xi_g \in T_gG$ so $\xi_g(f \circ L_g)= ((L_g)_\ast \xi_e)(f)$. Result follows. 
\end{proof}

It can be shown that the lie bracket of two left invariant vector fields is also left invariant. 
\begin{definition}
	The vector space of left invariant vector fields is the \bam{Lie algebra} $\mf{g}$ of $G$. 
\end{definition}
Since a LIVF is uniquely determined by its value at the identity, we have that $\mf{g} \cong T_e G$ as a vector space, but we can also transport the Lie bracket from $\mf{g}$ to $T_e G$ so they are isomorphic as algebras. 
\begin{definition}
	The maps $(L_{g^{-1}})_\ast : T_g G \to T_e G \cong \mf{g}$ define a $\mf{g}$-valued one form $\theta$ called the \bam{left invariant Maurer-Cartan one-form}. If $\xi$ is a LIVF, $\theta(\xi) = \xi_e$. 
\end{definition}
By definition, $\theta$ is left invariant. 
\begin{theorem}
	The MC one form satisfies the \bam{structure equation} 
	\eq{
		d\theta = - \frac{1}{2}\comm[\theta]{\theta}
	}
	i.e. for $\xi,\eta \in \mf{X}(G)$, $d\theta(\xi,\eta) = -\comm[\theta(\xi)]{\theta(\eta)}$
\end{theorem}
\begin{proof}
	We will need the following result:
	\begin{claim}
		For $\theta \in \Omega^1(M)$, $X,Y \in \mf{X}(M)$
		\eq{
			d\theta(X,Y) = X(\theta(Y)) - Y(\theta(X)) - \theta(\comm[X]{Y})
		}
	\end{claim}
	To show this take coordinates such that $\theta = \theta_a dx^a, X = X^a \del_a, Y=Y^a \del_a$. Then 
	\eq{
		d\theta(X,Y) &= (\del_b \theta_a X^c Y^d) (dx^b \wedge dx^a)(\del_c,\del_d) \\
		&= \del_b \theta_a (X^b Y^a - X^a Y^b) \\
		&= X^b \del_b (\theta_a Y^a) - Y^b \del_b (\theta_a X^a) - \theta_a (X^b \del_b Y^a - Y^b \del_b X^a) \\
		&= X(\theta(Y)) - Y(\theta(X)) - \theta(\comm[X]{Y})
	}
	Now if $X,Y$ are LIVFs, $\theta(X), \theta(Y)$ are constant, so on these 
	\eq{
		d\theta(X,Y) +\theta(\comm[X]{Y}) = 0
	}
	Moreover for LIVFs $\theta(\comm[X]{Y}) = \comm[\theta(X)]{\theta(Y)}$. Now LIVFs span the space of vector fields, and all the operations are linear, so we are done. 
\end{proof}

\begin{prop}
	If $G$ is a matrix Lie group, $\theta_g = g^{-1}dg$. 
\end{prop}
\begin{proof}
	In a matrix group, we have the correspondence $X \in \mf{g} \Leftrightarrow \exp(tX) \in G$. Take a basis $\pbrace{T_a}$ of $T_eG$ and give $g\in G$ coordinates $x^a$ if $g = \exp(\sum_a x^a T_a)$. Then let $g$ be constant and take a curve through $g$, $\gamma:\mbb{R}\to G$, $\gamma(t) = \exp\psquare{\sum_a (x^a + t\xi^a)T_a}$ with tangent vector $g\pround{\sum_a \xi^a T_a} \in T_gG$. Under $L_{g^{-1}}$, this is a curve through $e$ with tangent vector $\pround{\sum_a \xi^a T_a} \in T_eG$. Hence if we write $\xi = \sum_a \xi^a \pd{x^a}$ for the the vector generating $\gamma$ we get 
	\eq{
		\theta_g = \sum_a T_a dx^a = g^{-1} dg
	}
\end{proof}



Every $g \in G$ defines a diffeomorphism $L_g R_{g^{-1}} : G \to G$, $h \mapsto ghg^{-1}$. Since $e = geg^{-1}$ its derivative belongs to $GL(T_e G) = GL(\mf{g})$. 
\begin{definition}
	The \bam{adjoint representation} of $G$ on $\mf{g}$ is given by $\Ad_g = (L_g)_\ast (R_g^{-1})_\ast$
\end{definition}

\begin{lemma}
	$R_g^\ast \theta = \Ad_{g^{-1}} \theta$
\end{lemma}
\begin{proof}
	\eq{
		R_g^\ast \theta_{hg} &= \theta_{hg}(R_g)_\ast \\
		&= (L_{(hg)^{-1}})_\ast (R_g)_\ast \\
		&= (L_{g^{-1}})_\ast (L_{h^{-1}})_\ast (R_g)_\ast \\
		&= (L_{g^{-1}})_\ast  (R_g)_\ast (L_{h^{-1}})_\ast \\
		&= \Ad_{g^{-1}} \theta_h
	}
\end{proof}

\begin{definition}
	The \bam{left action} of a Lie group $G$ on a manifold $M$ is a smooth map $G \times M \to M$, $(g,a) \mapsto ga$ satisfying the axioms $\forall g,h \in G, \, \forall a \in M$
	\begin{itemize}
		\item $g(ha) = (gh)a$
		\item $ea = a$
	\end{itemize}
	Right action is defined equivalently. 
\end{definition}

Left and right actions are equivalent if we take $ga = ag^{-1}$. 
\begin{definition}
	An action is \bam{transitive} if the $G$-orbit of any point is $M$, equivalently $\forall a,b in M, \, \exists g \in G, \, b = ga$
\end{definition}

\begin{definition}
	An action is \bam{free} if the only element which fixes any point is the identity. 
\end{definition}

\begin{definition}
	A \bam{G-torsor} (or principally homogeneous $G$-space) is a manifold $M$ on which $G$ acts freely and transitively
\end{definition}
Given a $G$-torsor $M$, any point in $M$ defines a diffeomorphism $g \cong M$, and as such $G$-torsors are said to be like a Lie group where we have 'forgotten' the identity. 

\begin{definition}
	A \bam{principal G-bundle} is a fibre bundle $P\overset{\pi}{\to}M$ together with a smooth rights $G$-action $(p,g)\mapsto r_g(p)$ which preserves fibres ($\pi\circ r_g = \pi$) and acts freely and transitively. 
\end{definition}
It follows that fibres are $G$-orbits and hence $M = \faktor{P}{G}$. The condition of local triviality now says that the local trivialisation $\pi^{-1}(U) \overset{\varphi}{\to} U \times G$ are $G$-equivariant, i.e. where $\varphi(p) = (\pi(p),\gamma(p))$, $\gamma:\pi^{-1}(U) \to G$ a $G$-equivariant ($\gamma \circ r_g = R_g \circ \gamma$) fibrewise diffeomorphism

\begin{definition}
	A principal $G$-bundle is \bam{trivial} is $\exists$ a $G$-equivariant diffeomorphism $P\overset{\psi}{\to} M \times G$.
\end{definition}

\begin{prop}
	A principal $G$-bundle $P\overset{\pi}{\to}M$ admits a section iff it is trivial
\end{prop}
\begin{proof}
	If $P\overset{\pi}{\to}M$ is trivial, $\psi : P \to M \times G$ defines a section $s : M \to P$ by $s(a) = \psi^{-1}(a,e)$. \\
	Conversely, is $s$ is a section, define $\psi$ by $\psi(p) = (\pi(p),\chi(p))$ where $\chi(p)$ is uniquely defined by $p = s(\pi(p))\chi(p)$. Notice that since $pg = s(\pi(p))\chi(p)g = s(\pi(pg))\chi(p)g $ so $\chi(pg) = \chi(p)g$. 
\end{proof}

\begin{example}
	Let $G$ be a Lie group and $H \leq G$ a closed subgroup. Then $G \overset{\pi}{\to}\faktor{G}{H}$ is a principal $H$-bundle. Therefore homogeneous spaces are examples of principal bundles. 
\end{example}

Since principal fibre bundles are locally trivial, they admit local sections. Let $\pbrace{(U_\alpha, \varphi_\alpha)}$ be a trivialising atlas for $G \to P \overset{\pi}{\to} M$. The canonical local sections $s_\alpha : U_\alpha \to \pi^{-1}(U_\alpha)$ are given by $s_\alpha(a) = \varphi_\alpha^{-1}(a,e)$. On $U_{\alpha\beta}$ we have sections $s_\alpha, \, s_\beta$. Writing $\varphi_\alpha(p) = (\pi(p),g_\alpha(p))$ for $g_\alpha: U_\alpha \to G$ equivariant we have that for $p\in \pi^{-1}(U_{\alpha\beta})$.
\eq{
	(\pi(p), g_\alpha(p)) = \varphi_\alpha(p) = (\varphi_\alpha \circ \varphi_\beta^{-1} \circ \varphi_\beta)(p) = (\varphi_\alpha \circ \varphi_\beta^{-1})(\pi(p),g_\beta(p)) \\
	\Rightarrow (\pi(p), \underbrace{g_\alpha(p)g_\beta^{-1}(p)}_{\equiv \hat{g}_{\alpha\beta}(p)}g_\beta(p)) = (\varphi_\alpha \circ \varphi_\beta^{-1})(\pi(p),g_\beta(p))
}
Note that $\hat{g}_{\alpha\beta}(pg) = g_\alpha(pg) g_\beta^{-1}(pg) = g_\alpha(p)  g g^{-1} g_\beta(p) = \hat{g}_{\alpha\beta}(p)$ and so is constant along the fibres. Hence $\exists g_{\alpha\beta}:U_{\alpha\beta} \to G $ s.t. $\hat{g}_{\alpha\beta} = \pi^\ast g_{\alpha\beta}$ and $(\varphi_\alpha \circ \varphi_\beta^{-1})(a,g) = (a,g_{\alpha\beta}(a)g)$. It follows that the $g_{\alpha\beta}$ obey the cocycle conditions. \\
Now note $g_\alpha \circ s_\alpha : U_\alpha \to G$ is a constant map taking value $e$, and so letting $p = s_\beta(a)$
\eq{
	g_\alpha(p) = \hat{g}_{\alpha\beta}(p) g_\beta(p) \Rightarrow g_\alpha(s_\beta(a)) &= g_{\alpha\beta}(a)(g_\beta \circ s_\beta)(a) \\
	&= (g_\alpha \circ s_\alpha)(a)g_{\alpha\beta}(a) \\
	&= g_\alpha(s_\alpha(a) g_{\alpha\beta}(a)) \\
	\Rightarrow s_\beta(a) &= s_\alpha(a) g_{\alpha\beta}(a) \quad \text{ as $g_\alpha$ a diffeomorphism}
}
%%%%%%%%%%%%%%%%%%%%%%%%%%%%%%%%%%%%%%%%%%%%%%%%%%%%%%%%
%%%%%%%%%%%%%%%%%%%%%%%%%%%%%%%%%%%%%%%%%%%%%%%%%%%%%%%%
%%%%%%%%%%%%%%%%%%%%%%%%%%%%%%%%%%%%%%%%%%%%%%%%%%%%%%%%
%%%%%%%%%%%%%%%%%%%%%%%%%%%%%%%%%%%%%%%%%%%%%%%%%%%%%%%%
\section{Connections}
\subsection{Ehresmann Connections}
Let $P \overset{\pi}{\to} M$ be a principal $G$-bundle. Taking $p \in P$, the derivative $(\pi_\ast)_p : T_p P \to T_{\pi(p)}M$ is a surjective map. 
\begin{definition}
	The kernel $V_p$ is called the \bam{vertical subspace}. A vector field $\xi \in \mf{X}(P)$ is called \bam{vertical} if $\forall p \in P, \, \xi_p \in V_p$. 
\end{definition}

\begin{lemma}
	The Lie bracket of two vertical vector fields is vertical
\end{lemma}

\begin{lemma}
	The vertical subspaces span a $G$-invariant integrable distribution
\end{lemma}
\begin{proof}
	Note $\pi \circ r_g = \pi \Rightarrow \pi_\ast (r_g)_\ast = \pi_\ast \Rightarrow (r_h)_\ast V_p = V_{pg}$ so $G$-invariant. Integrable by the previous lemma.
\end{proof}

\begin{definition}
	An \bam{Ehresmann connection} on $P$ is a smooth choice of horizontal subspaces $H_p \subset T_p P$ s.t. $T_p P = V_p \oplus H_p$ and $(r_g)_\ast H_p = H_{pg}$. Equivalently an Ehresmann connection is a $G$-invariant distribution $H \subset TP$ complementary to $V$. 
\end{definition}

\begin{example}
	A $G$-invariant Riemannian metric on $P$ defines an Ehresmann connection by $H_p = V_p^\perp$. 
\end{example}

The $G$ action on P defines a smooth map $\mf{g} \to \mf{X}(P)$ assigning to every $X\in \mf{g}$ the \bam{fundamental vector field}  $\xi_X$ defined at $p \in P$ by  
\eq{
	(\xi_X)_p = \ev{\frac{d}{dt} \pround{p e^{tX}}}{t=0}
}

\begin{lemma}
	$\xi_X$ is vertical
\end{lemma}
\begin{proof}
	\eq{
		\pi_\ast \ev{\xi_X}{p} = \ev{\frac{d}{dt} \pi\pround{p e^{tX}}}{t=0} = \ev{\frac{d}{dt} \pi\pround{p}}{t=0} = 0
	}
\end{proof}
As the $G$ action is free, $\forall p \in P$ the map $X \mapsto (\xi_X)_p$ is an isomorphism $\mf{g} \overset{\cong}{\to} V_p$. 

\begin{lemma}
	$(r_g)_\ast \xi_X = \xi_{\Ad_{g^{-1}}(X)}$
\end{lemma}
\begin{proof}
	\eq{
		(r_g)_\ast (\xi_X)_p = \ev{\frac{d}{dt} r_g \pround{p e^{tX}}}{t=0} = \ev{\frac{d}{dt} \pround{p e^{tX}g}}{t=0} = \ev{\frac{d}{dt} \pround{pgg^{-1} e^{tX}g}}{t=0} = \pround{ \xi_{\Ad_{g^{-1}}(X)}}_{pg}
	}
\end{proof}

\begin{definition}
	The \bam{connection one form} of a connection $H \subset TP$ is the $\mf{g}$-valued one form $\omega \in \Omega^1(P;\mf{g})$ defined by 
	\eq{
		\omega(\xi) = \left \lbrace \begin{array}{cc} X & \xi = \xi_X \\ 0 & \xi \in H \end{array} \right.
	}
\end{definition}

\begin{prop}
	The connection one form obeys $r_g^\ast \omega = \Ad_{g^{-1}} \circ \omega$
\end{prop}
\begin{proof}
	Let $\xi$ be horizontal. Then $(r_g)_\ast \xi$ is also horizontal as $H$ $G$-invariant. Then $(r_g^\ast \omega)(\xi) = \omega((r_g)_\ast \xi) = 0$. Note in this case $(\Ad_{g^{-1}} \circ \omega)(\xi) = 0$ too. \\
	Now if $\xi = \xi_X$, $(\Ad_{g^{-1}} \circ \omega)(\xi) = \Ad_{g^{-1}}(X) = \omega(\xi_{\Ad_{g^{-1}}(X)}) = \omega((r_g)_\ast \xi_X) = (r_g^\ast \omega)(\xi)$
\end{proof}
It turn out we also have a converse:
\begin{prop}
	If $\omega \in \Omega^1(P;\mf{g})$ satisfies $r_g^\ast \omega = \Ad_{g^{-1}} \circ \omega$ and $\omega(\xi_X) = X$, then $H\equiv \ker \omega$ is a connection on $P$. 
\end{prop}

Now define the pullback of $\omega$ along local sections to be $A_\alpha \equiv s_\alpha^\ast \omega \in \Omega^1(U_\alpha;\mf{g})$. 
\begin{prop}
	Let $\omega_\alpha \equiv \Ad_{g_\alpha^{-1}} \circ \pi^\ast A_\alpha + g_\alpha^\ast \theta $ where $\theta$ is the LI Maurer-Cartan one form on $G$. Then $\omega_\alpha = \ev{\omega}{\pi^{-1}U_\alpha}$
\end{prop}
\begin{proof}
	The proof will have two steps:
	\begin{claim}
		$\omega$ and $\omega_\alpha$ agree on the image of $s_\alpha$
	\end{claim}
	Since $\pi\circ s_\alpha = \ev{\id}{U_\alpha}$, $T_pP = \image(s_\alpha \circ \pi)_\ast \oplus V_p$ for $p = s_\alpha(a)$. Hence $\forall \xi \in T_pP, \, \exists! \, \xi^v \in V_p $ s.t. $\xi = (s_\alpha)_\ast \pi_\ast \xi + \xi^v$. Then using $g_\alpha(p) = (g_\alpha \circ s_\alpha)(a) = e$
	\eq{
		\omega_\alpha(\xi) &= (\pi^\ast s_\alpha^\ast \omega)(\xi) + (g_\alpha^\ast \theta_e)(\xi) \; (\text{at } p, \, \Ad_{g_\alpha^{-1}}=\id) \\
		&= \omega((s_\alpha)_\ast \pi_\ast \xi) + \theta_e ((g_\alpha)_\ast \xi) \\
		&= \omega((s_\alpha)_\ast \pi_\ast \xi) + \theta_e((g_\alpha)_\ast \xi^v) \; \text{as }(g_\alpha)_\ast (s_\alpha)_\ast = (g_\alpha \circ s_\alpha)_\ast = 0  \\
		&= \omega((s_\alpha)_\ast \pi_\ast \xi) + \omega(\xi^v) \\
		&= \omega(\xi)
	}
	\begin{claim}
		$\omega$ and $\omega_\alpha$ transform in the same way under the right $G$ action. 
	\end{claim}
	\eq{
		r_g^\ast (\omega_\alpha)_{pg} &= \Ad_{g_\alpha(pg)^{-1}} \circ r_g^\ast \pi^\ast s_\alpha^\ast \omega + r_g^\ast g_\alpha^\ast \theta \\
		&= \Ad_{(g_\alpha(p)g)^{-1}}\circ r_g^\ast \pi^\ast s_\alpha^\ast \omega + g_\alpha^\ast R_g^\ast \theta \\
		&= \Ad_{g^{-1}g_\alpha(p)^{-1}} \circ \pi^\ast s_\alpha^\ast \omega + g_\alpha^\ast (\Ad_{g^{-1}}\circ\theta) \\
		&= \Ad_{g^{-1}} \pround{\Ad_{g_\alpha(p)^{-1}} \circ \pi^\ast s_\alpha^\ast \omega + g_\alpha^\ast\theta} \\
		&= \Ad_{g^{-1}} \circ (\omega_\alpha)_p
	}
	Hence we are done. 
\end{proof}

Now as $\omega$ is a global one form, $\omega_\alpha$ and $\omega_\beta$ must agree on $U_{\alpha\beta}$, allowing us to relate $A_\alpha$ and $A_\beta$, namely on $U_{\alpha\beta}$
\eq{
	A_\alpha = s_\alpha^\ast \omega_\alpha = s_\alpha^\ast \omega_\beta &= s_\alpha^\ast \pround{\Ad_{g_\beta(s_\alpha)^{-1}}\circ \pi^\ast A_\beta + g_\beta^\ast \theta} \\
	&=\Ad_{g_{\alpha\beta}}\circ A_\beta + g_{\beta\alpha}^\ast \theta
}

\begin{example}
	For matrix Lie groups, $g_{\beta\alpha}^\ast \theta = g_{\beta\alpha^{-1}}dg_{\alpha\beta} = -dg_{\alpha\beta}g_{\alpha\beta}^{-1}$, so 
	\eq{
		A_\alpha = g_{\alpha\beta} A_\beta g_{\alpha\beta}^{-1} -dg_{\alpha\beta}g_{\alpha\beta}^{-1}
	}
\end{example}

Similarly, one can ask how $\pbrace{A_\alpha}$ depends on the choice of local section. 
\begin{fact}
	If $s_\alpha^\prime$ is another local section for $U_\alpha$, $\exists h_\alpha : U_\alpha \to G$ s.t. $s_\alpha^\prime(a) = s_\alpha(a) h_\alpha(a)$ and then 
	\eq{
		A_\alpha^\prime = \Ad_{h_\alpha^{-1}} \circ A_\alpha + h_\alpha^\ast \theta
	}
\end{fact}

\begin{idea}
	We now have three different ways to understand connections on a principal $G$-bundle $P\overset{\pi}{\to} M$, namely;
	\begin{enumerate}
		\item a $G$-invariant horizontal distribution $H\subset TP$ 
		\item a one form $\omega \in \Omega^1(P;\mf{g})$ satisfying $\omega(\xi_X) = X$ and $r_g^\ast \omega = \Ad_{g^{-1}} \circ \omega$
		\item a family of one forms $\pbrace{A_\alpha \in \Omega^1(U_\alpha;\mf{g}}$ satisfying $A_\alpha = \Ad_{g_{\alpha\beta}}\circ A_\beta + g_{\beta\alpha}^\ast \theta$ on $U_{\alpha\beta}\neq \emptyset$
	\end{enumerate}
\end{idea}

If $P \overset{\pi}{\to} M$ is a principal $G$-bundle, $G$-equivariant bundle diffeomorphisms are called \bam{gauge transformations} and one can ask how an Ehresmann connection transforms. Let $H \subset TP$ be a $G$-invariant horizontal distribution. Then let $H^\Phi \equiv \Phi_\ast H$ be the gauge-transformed distribution. 

\begin{lemma}
	$H^\Phi \subset TP$ is an Ehresmann connection
\end{lemma}
\begin{proof}
	\eq{
		(r_g)_\ast H^\Phi_{\Phi(p)} = (r_g)_\ast \Phi_\ast H_p = \Phi_\ast (r_g)_\ast H_p = \Phi_\ast H_{pg} = H_{(\Phi(pg)}^\Phi = H_{\Phi(p)g}^\Phi
	}
	and $H^\Phi$ is complementary to $V$ because $\Phi_\ast T_pP \overset{\cong}{\to} T_{\Phi(p)}P$ and $\Phi_\ast$ preserves $V = \ker \pi_\ast$ because $\pi\circ\Phi=\pi$
\end{proof}

\begin{ex}
	Let $\Phi$ be a gauge transformation in a principal $G$-bundle $P\overset{\pi}{\to}M$. Let $\xi_X$ denote a fundamental vector fields for the $G$-action on $P$. Show that $\xi_X$ is gauge invariant, i.e. $\Phi_\ast \xi_X = \xi_X$. Further, show that if $\omega$ is the connection one form for an Ehresmann connection $H$ then $(\Phi^{-1})^\ast \omega$ is the connection one form for $H^\Phi$. 
\end{ex}

Let $\pbrace{A_\alpha}, \, \pbrace{A_\alpha^\Phi}$ be the gauge fields corresponding to the Ehresmann connections $H, \, H^\Phi$. Since $\Phi$ preserves fibres it makes sense to restrict to $\pi^{-1}U_\alpha$. Applying the trivialisation $\varphi_\alpha(\Phi(p)) = (\pi(p),g_\alpha(\Phi(p)))$ which defines $\bar{\phi}_\alpha : \pi^{-1} U_\alpha \to G$ by $\bar{\phi}_\alpha(p)=g_\alpha(\Phi(p))g_\alpha(p)^{-1}$. \\

\begin{lemma}
	$\bar{\phi}_\alpha$ is constant on the fibres
\end{lemma}
\begin{proof}
	\eq{
		\bar{\phi}_\alpha(pg) &= g_\alpha(\Phi(pg))g_\alpha(pg)^{-1} \\
		&= g_\alpha(\Phi(p)g)g_\alpha(pg)^{-1} \\
		&= g_\alpha(\Phi(p))g(g_\alpha(p)g)^{-1} \\
		&= g_\alpha(\Phi(p))g_\alpha(p)^{-1} \\
		&= \bar{\phi}_\alpha(p)
	}
\end{proof}
Hence $\bar{\phi}_\alpha$ defines a smooth map $\phi_\alpha:U_\alpha \to G$. On overlaps $U_{\alpha\beta}\neq \phi$ we have that $\forall a \in U_{\alpha\beta}, \, p \in \pi^{-1}(a)$, hence 
\eq{
	\phi_\alpha(a) &= g_\alpha(\Phi(p)) g_\alpha(p)^{-1} \\
	&= g_\alpha(\Phi(p)) \cdot \underbrace{g_\beta(\Phi(p))^{-1}g_\beta(\Phi(p))}_{e} \underbrace{g_\beta(p)^{-1}g_\beta(p)}_{e} g_\alpha(p)^{-1} \\
	&= g_{\alpha\beta}(a) \phi_\beta(a) g_{\alpha\beta}(a)^{-1} \; \text{since }\pi(p) = \pi(\Phi(p)) = a
}

\begin{remark}
	We will see later that $\pbrace{\phi_\alpha}$ defines a section of a fibre bundle $\Ad P$ on $M$ associated to the principal bundle $P$. 
\end{remark}

\begin{ex}
	Show that on $U_\alpha$, $A_\alpha^\Phi = \Ad_{\phi_\alpha} \circ\pround{A_\alpha - \phi_\alpha^\ast \theta} = \phi_\alpha A_\alpha \phi_\alpha^{-1} - d\phi_\alpha \phi_\alpha^{-1}$, which is a gauge transform
\end{ex}

%%%%%%%%%%%%%%%%%%%%%%%%%%%%%%%%%%%%%%%%%%%%%%%%%%%%%%%%
%%%%%%%%%%%%%%%%%%%%%%%%%%%%%%%%%%%%%%%%%%%%%%%%%%%%%%%%
%%%%%%%%%%%%%%%%%%%%%%%%%%%%%%%%%%%%%%%%%%%%%%%%%%%%%%%%
%%%%%%%%%%%%%%%%%%%%%%%%%%%%%%%%%%%%%%%%%%%%%%%%%%%%%%%%
\subsection{Kozul Connections}
\begin{definition}
	A real, rank $k$, \bam{vector bundle} $E \overset{\pi}{\to}M$ is a fibre bundle whose fibres are $k$-dimensional real vector spaces and whose local tirivialisations $\psi:\pi^{-1}U \to U \times \mbb{R}^k$ restrict fibrewise to isomorphisms $\psi : E_a \to \pbrace{a} \times \mbb{R}^k$ of real vector spaces. 
\end{definition}

Let $P\overset{\pi}{\to}M$ be a principal $G$-bundle and let $\rho: G \to GL(V)$ be a Lie group homomorphism (i.e. a representation of $G$), where $V$ is a f.d. vector space. Since $G$ acts freely on $P$, it also acts freely on $P\times V$ via the right action 
\eq{
	(p,v)g = (pg,\rho(g^{-1})v)
}
We let $E \equiv P \times_G V$ denote the quotient $\faktor{(P\times V)}{G}$ via the above action. It is the total space of a vector bundle $E\overset{\varpi}{\to}M$ where 
\eq{
	\varpi: P \times_G V &\to M \\
	[(p,v)] &\mapsto \pi(p)
}
\begin{definition}
	$E\overset{\varpi}{\to}M$ is called an \bam{associated vector bundle} to the PFB $P \to M$, associated via the representation $\rho$. 
\end{definition}
Let $\pbrace{(U_\alpha,\varphi_\alpha)}$ be a trivialising atlas for $P$ with transition function $\pbrace{g_{\alpha\beta}:U_{\alpha\beta} \to G}$ obeying the cocycle conditions. We may then trivialise $P\times_G V$ on each $U_\alpha$, and the transition functions are $\pbrace{\rho \circ g_{\alpha\beta}:U_{\alpha\beta}\to GL(V)}$. More concretely we define $P \times_G V \equiv \sqcup_\alpha \faktor{U_\alpha \times V}{\sim}$ where $(a,v) \sim (a,\rho(g_{\alpha\beta}(a))v)$ \\
Let $P\overset{\pi}{\to} M$ be a $G$-PFB and $E\equiv P \times_G V \overset{\varpi}{\to} M$ an associated VB with $\rho: G \to GL(V)$. Let $\Gamma(E) = \pbrace{s:M \to E \, | \, \varpi \circ s = \id_M}$ denote the $C^\infty(M)$-module of sections of $E$, and $C^\infty_G(P,V) = \pbrace{\zeta : P \to V \, | \, \forall g \in G, \, r_g^\ast \zeta = \rho(g)^{-1} \circ \zeta }$ the $G$-equivariant functions $P \to V$. We can give $C^\infty_G(P,V)$ the structure of a $C^\infty(M)$-module by declaring that for $f \in C^\infty(M), \, f \zeta = \pi^\ast f \zeta$ 
\begin{prop}
	There is a $C^\infty(M)$-module isomorphism 
	\eq{
		\Gamma(E) \cong C^\infty_G(P,V)
	}
\end{prop}
\begin{proof}
	Let $\sigma \in \Gamma(E)$. Let $\psi_\alpha : \varpi^{-1} U_\alpha \to U_\alpha \times V$ be a local trivialisation and define $\sigma_\alpha : U_\alpha \to V, \, (\psi_\alpha \circ \sigma)(a) = (a,\sigma_\alpha(a))$. On overlaps the local functions $\sigma_\alpha, \sigma_\beta$, are related by $\sigma_\alpha(a) = \rho(g_{\alpha\beta}(a)) \sigma_\beta(a)$, where $g_{\alpha\beta}$ are the transition functions of $P \to M$ . We now define $\zeta_\alpha : \pi^{-1} U_\alpha \to V$ by $\zeta_\alpha((\pi^\ast s_\alpha)(p)) = \sigma_\alpha(\pi(p))$ and extend by $\zeta_\alpha((\pi^\ast s_\alpha)(p)g) = \rho(g)^{-1} \sigma_\alpha(\pi(p))$. \\
	Let $\pi(p) = a \in U_{\alpha\beta}$. Then 
	\eq{
		\zeta_\beta(p) = \zeta(s_\alpha(a) g_\alpha(p)) &= \zeta(s_\beta(a) g_{\beta\alpha}(a) g_\alpha(p)) \\
		&= \rho(g_{\beta\alpha}(a) g_\alpha(p))^{-1}\circ \sigma_\beta(a) \\
		&= \rho(g_\alpha(p))^{-1} \circ \rho(g_{\alpha\beta}(a)) \circ \sigma_\beta(a) \\
		&= \rho(g_\alpha(p))^{-1} \circ \sigma_\alpha(a) \\
		&= \rho(g_\alpha(p))^{-1} \zeta_\alpha(s_\alpha(a)) \\
		&= \zeta_\alpha(s_\alpha(a) g_\alpha(p)) = \zeta_\alpha(p)
	}
	The $\pbrace{\zeta_\alpha}$ are constructed to define a function $\zeta:P \to V$ such that $r_g^\ast \zeta = \rho(g)^{-1} \circ \zeta$. If $f\in C^\infty(M)$, then $f\sigma \in \Gamma(E)$ and $(f\sigma)_\alpha = f\sigma_\alpha$ since $\psi_\alpha$ is fibrewise linear. Then by definition 
	\eq{
		\rho(g_\alpha(p))^{-1} \circ \pi^\ast(f\sigma_\alpha) &= \rho(g_\alpha(p))^{-1} \circ (\pi^\ast f) (\pi^\ast \sigma_\alpha) \\
		&= (\pi^\ast f) \rho(g_\alpha(p))^{-1} \circ (\pi^\ast \sigma_\alpha) \\
		&= (\pi^\ast f)\zeta_\alpha(p)
	}
	so the map $\Gamma(E) \to C^\infty_G(P,V)$, thus defined, is $C^\infty(M)$-linear. \\
	Conversely, given a $G$-equivariant $\zeta:P \to V$, we define $\sigma\in \Gamma(E)$ as follows: let $s_\alpha : U_\alpha \to P$ be the canonical local sections. Then let $\sigma_\alpha = s_\alpha^\ast \zeta$. For $a \in U_{\alpha\beta}$, 
	\eq{
		\sigma_\beta(a) = \zeta(s_\beta(a)) = \zeta(s_\alpha(a) g_{\alpha\beta}(a)) = \rho(g_{\alpha\beta}(a))^{-1} \zeta(s_\alpha(a)) = \rho(g_{\beta\alpha}(a)) \sigma_\alpha(a)
	}
\end{proof}

\begin{example}
	Let $\omega, \omega^\prime$ be connection one forms for Ehresmann connections $\ms{H}, \ms{H}^\prime$ on $P \to M$. Then $r_g^\ast \omega = \Ad_{g^{-1}} \circ \omega$ and similarly for $\omega^\prime$. Now if $\xi$ is vertical, $\omega(\xi) = \omega^\prime(\xi)$, and hence $\tau \equiv \omega - \omega^\prime \in \Omega^1(P;\mf{g})$ is \bam{horizontal} (i.e. $\tau(\xi) = 0$ if $\xi$ vertical). \\
	Now let $\tau_\alpha = s_\alpha^\ast \tau \in \Omega^1(U_\alpha;\mf{g})$. Then $\tau_\alpha = s_\alpha^\ast \omega - s_\alpha^\ast \omega^\prime = A_\alpha - A_\alpha^\prime$. On $U_{\alpha\beta}$, $A_\alpha = \Ad_{g_{\alpha\beta}} \circ A_\beta + g_{\beta\alpha}^\ast \theta$, and likewise for $A_\alpha^\prime$, $\Rightarrow \tau_\alpha = \Ad_{g_{\alpha\beta}}\circ\tau_\beta$. Hence $\pbrace{\tau_\alpha}$ defines $\tau \in \Omega^1(M;\ad P)$ where $\ad P \equiv P \times_G \mf{g}$. 
\end{example}

\begin{example}
	Take $H \leq G$ closed and $M= \faktor{G}{H}$. Then $G\overset{\pi}{\to}M$ is a principal $H$-bundle. Let $\rho : H \to GL(V)$ be a representation. Then $E\equiv G \times_H V \to M$ is a \bam{homogeneous vector bundle}. Then $\Gamma(E) \cong \pbrace{f:G \to V \, | \, f(ph) = \rho(h)^{-1} f(p)}$ as $C^\infty(M)$-modules. On $\Gamma(E)$ we have a rep of $G$ given by $(g \cdot f)(g_1) = f(g^{-1} g_1)$.
\end{example}

There is a sort of converse to the associated VB construction. If $E \overset{\pi}{\to}M$ is a real rank $k$ vector bundle, we may associate with it a principal $GL(k,\mbb{R})$-bundle in one of two ways as follows:
\begin{enumerate}
	\item Let $\pbrace{(U_\alpha,\psi_\alpha)}$ be a trivialising atlas for $E$, with $\psi_\alpha : \pi^{-1} U_\alpha \to U_\alpha \times \mbb{R}^k$ and transition functions $g_{\alpha\beta}:U_{\alpha\beta} \to GL(k,\mbb{R})$. We can then glue $U_\alpha \times GL(k,\mbb{R})$ and 
	$U_\beta \times GL(k,\mbb{R})$ along $U_{\alpha\beta}$ by 
	\eq{
		(a,A) \sim (a,g_{\alpha\beta}(a) A)
	}
	which is equivariant under right multiplication by $GL(k,\mbb{R})$. The resulting principal $GL(k,\mbb{R})$-bundle is denoted $GL(E) \overset{\varpi}{\to} M$ and it follows that $E \to M$ is the vector bundle associated to $GL(E)$ view the identity rep
	\item The PFB $GL(E) \overset{\varpi}{\to} M $ can understood as the \bam{bundle of frames} of $E \overset{\pi}{\to} M$. Let $GL(E)_a = \pbrace{\text{ordered bases for }E_a}$. Let $u = (u_1, \dots, u_n)$ be a frame for $E_a$. Then $\varpi(u) = a$ defines $\varpi : GL(E) \to M$. If $A \in GL(k,\mbb{R})$, $uA$ defined by $(uA)_i = \sum_j u_j A_{ji}$ is another frame for $E_a$. Given frames $u,u^\prime$ for $E_a$, $\exists ! \, A \in GL(k,\mbb{R})$ s.t. $u^\prime = uA$. Let $(U,\psi)$ be a local trivialisation for $E$. We define a reference frame $\bar{u}(a)$ for each $a \in U$ by $\psi(\bar{u}_i(a)) = (a,e_i)$, where $\pbrace{e_i}$ is the standard bases for $\mbb{R}^k$. This defines a trivialisation $\Psi : \varpi^{-1} U \to U \times GL(k,\mbb{R})$ by $\Psi(u) = (a,A(u))$ where $u$ is a frame for $E_a$ and $A(u)\in GL(k,\mbb{R})$ is the unique element sending $u$ to $\bar{u}(a)$. Now for $B \in GL(k,\mbb{R})$, we have 
	\eq{
		\bar{u}(a) A(uB) = uB = (\bar{u}(a)A(u))B \Rightarrow A(uB) = A(u)B
	}
	Hence $\Psi$ is $GL(k,\mbb{R})$-equivariant. Let $\pbrace{(U_\alpha,\Psi_\alpha)}$ denote the reslting trivialising atlas. Then if $a \in U_{\alpha\beta}$ and $u$ is a frame for $E_a$, then $\Psi_\alpha(u) = (a,A_\alpha(u))$ where $\bar{u}_\alpha(u) A_\alpha(u) = u$. Now note 
	\eq{
		\bar{u}_\beta(a)_i &= \psi^{-1}(a,e_i) \\
		&= \psi^{-1}_\alpha \circ \psi_\alpha \circ \psi_\beta^{-1}(a,e_i) \\
		&= \psi^{-1}_\alpha (a,g_{\alpha\beta}(a)e_i) \\
		&= \psi_\alpha^{-1}(a, \sum_j e_j(g_{\alpha\beta}(a))_{ji}) \\
		&= \sum_j \psi_\alpha^{-1}(a,e_j) g_{\alpha\beta}(a)_{ji} \\
		&= \sum_j \bar{u}_\alpha(a)_j g_{\alpha\beta}(a)_{ji} \\
		\Rightarrow \bar{u}_\beta(a) &= \bar{u}_\alpha(a) g_{\alpha\beta}(a) \\
		\Rightarrow A_\alpha(u) &= g_{\alpha\beta}(a)A_\beta(u)
	}
\end{enumerate}

\begin{definition}
	Let $E \overset{\pi}{\to} M$ be a vector bundle. A \bam{Kozul connection} on $E$ is an $\mbb{R}$-bilinear map
	\eq{
		\nabla : \mf{X}(M) \times \Gamma(E)&\to \Gamma(E) \\
		(X,s)&\mapsto \nabla_X s 
	}
	satisfying that, $\forall f \in C^\infty(M), X \in \mf{X}(M), s \in \Gamma(E)$
	\begin{enumerate}
		\item $\nabla_{fX}s = f \nabla_X s $
		\item $\nabla_X(fs) = X(f) s + f \nabla_X s $
	\end{enumerate}
\end{definition}

Suppose that $E = P \times_G V$ for some $G$-PFB $P \overset{\pi}{\to}M$. Then an Ehresmann connection on $P$ induces a Kozul connection on $E$. For this it is convenient to use the $C^\infty(M)$-module isomorphism $\Gamma(E) \cong C_G^\infty(P,V)$ and we will define $\nabla$ on $C_G^\infty(P,V)$:\\
Let $\ms{H} \subset TP$ be an Ehresmann connection. We define $h: T_pP \to T_p P $ to be the projector onto $\ms{H}$ along $\ker(\pi_\ast)$. If we write $\xi \in T_pP$ as $\xi^h + \xi^v$ where $\xi^h \in \ms{H}_p$ and $\pi_\ast(\xi^v) = 0$, then $h(\xi) = \xi^h$. Let $h^\ast : T_p^\ast P \to T_p^\ast P$ be the dual (i.e $(h^\ast \alpha)(\xi) = \alpha(h(\xi))$). Let $X \in \mf{X}(M)$. Then given $p \in P_a$ let $\xi \in T_pP$ be s.t. $\pi_\ast \xi = X(a)$. We define $\ev{\nabla_X \psi}{p} = (d\psi)_p(h\xi)$, i.e. $d^\nabla \psi = h^\ast d\psi$. This is well defined because if $\pi_\ast \xi = \pi_\ast \xi^\prime$, $h\xi = h\xi^\prime$. Further, $\nabla_X \psi \in C^\infty_G(P,V)$ because the split $TP = \mc{V} \oplus \ms{H}$ is $G$-invariant, and hence $r_g^\ast h^\ast= h^\ast r_g^\ast $. Hence 
\eq{
	r_g^\ast d^\nabla \psi &= r_g^\ast h^\ast d\psi \\
	&= h^\ast r_g^\ast d\psi \\
	&= h^\ast d(\rho(g)^{-1} \circ \psi) \\
	&= \rho(g)^{-1} \circ h^\ast d\psi = \rho(g)^{-1} d^\nabla \psi
}

\begin{prop}
	$\nabla$ defines a Kozul connection on $E$ 
\end{prop}
\begin{proof}
	\eq{
		\nabla_{fX} \psi &= d\psi(h(f \xi)) \\
		&= d\psi (h[(\pi^\ast f) \xi]) \\
		&= \pi^\ast f d\psi(h\xi) \\
		&= f \nabla_X \psi \\
		\nabla_X(f\psi) &= \nabla_X[(\pi^\ast f) \psi ] \\
		&= d\psquare{(\pi^\ast f) \psi}(h \xi) \\
		&= (\pi^\ast df)(h\xi) + (\pi^\ast f) \nabla_X \psi \\
		&= \pi^\ast (df(\pi_\ast h\xi)) \psi + f \nabla_X \psi \\
		&= \pi^\ast(df(\pi_\ast \xi)) \psi + f \nabla_X \psi \\
		&= \pi^\ast(Xf) \psi + f \nabla_X \psi \\
		&= X(f) \psi  + f \nabla_X \psi
	}
\end{proof}

We will now define a more calculationally useful formula for the Kozul connection of $P \times_G V$ induced by the Ehresmann connection on $P$. Let $\psi \in C^\infty_G(P,V)$ and let $\xi \in \mf{X}(P)$. We decompose $\xi = h\xi + \xi^v$ where $\pi_\ast \xi^v = 0$. Then 
\eq{
	d\psi(h\xi) = d\psi(\xi - \xi^v) = d\psi(\xi) - d\psi(xi^v)
}
The derivative $\xi^v \psi$ only depends on the value of $\xi^v$ at a point, so we can take $\xi^v$ to be the fundamental vector field $\xi_{\omega(\xi^v)} = \xi_{\omega(\xi)}$ corresponding to the $G$-action. Therefore 
\eq{
	\xi^v \psi = \xi_{\omega(\xi)} \psi &= \ev{\frac{d}{dt}\psi \circ r_{\exp(t\omega(\xi))}}{t=0} \\
	&= \ev{\frac{d}{dt} \rho(\exp(-t\omega(\xi))) \circ \psi}{t=0} \\
	&= -\rho(\omega(\xi)) \circ \psi
}
Therefore $d\psi(h\xi) = d\psi(\xi) + \rho(\omega(\xi)) \circ \psi$, or abstracting $\xi$, 
\eq{
	d^\nabla \psi = d\psi + \rho(\omega) \cdot \psi
}
Finally, we give a formula for $\nabla_X \sigma$, where $\sigma \in \Gamma(P \times_G V)$, now viewed as a family $\pbrace{\sigma_\alpha : U_\alpha \to V}$ of functions transforming in overlaps as $\sigma_\alpha(A) = \rho(g_{\alpha\beta}(a)) \sigma_\beta(a)$;
\eq{
	d^\nabla \sigma_\alpha &= d^\nabla s_\alpha^\ast \psi = d^\nabla(\psi \circ s_\alpha) = d(\psi \circ s_\alpha) \circ h \\
	&= d(s_\alpha^\ast \psi) \circ h = s_\alpha^\ast (d\psi) \circ h \\
	&= s_\alpha^\ast d^\nabla \psi = s_\alpha^\ast (d\psi + \rho(\omega) \circ \psi) \\
	&= d s_\alpha^\ast \psi + \rho(s_\alpha^\ast \omega) \circ s_\alpha^\ast \psi \\
	&= d\sigma_\alpha + \rho(A_\alpha) \circ \sigma_\alpha
}
Hence, if $X \in \mf{X}(M)$, 
\eq{
	\nabla_X \sigma_\alpha \equiv X(\sigma_\alpha) + \rho(A_\alpha(X)) \cdot \sigma_\alpha
}

\begin{ex}
	Show that $\nabla_X \sigma_\alpha$ transforms like $\sigma_ \alpha$ on overlaps, that is 
	\eq{
		\nabla_X \sigma_\alpha = \rho(g_{\alpha\beta}) \circ \nabla_X \sigma_\beta
	}
	Note this justifies the name \bam{covariant derivative}. 
\end{ex}

In summary, given a $G$-PFB, $P \to M$, and a f.d. rep $\rho : G \to GL(V)$, we construct a VB $P\times_G V \to M$. Every VB is obtained in this way from its frame bundle. We then introduced the notion of a Kozul connection on a VB and showed that an Ehresmann connection on $P$ induces a Kozul connection on $P \times_G V$. The converse is also true: a Kozul connection on $E$ induces an Ehresmann connection on $GL(E)$. 

%%%%%%%%%%%%%%%%%%%%%%%%%%%%%%%%%%%%%%%%%%%%%%%%%%%%%%%%
%%%%%%%%%%%%%%%%%%%%%%%%%%%%%%%%%%%%%%%%%%%%%%%%%%%%%%%%
%%%%%%%%%%%%%%%%%%%%%%%%%%%%%%%%%%%%%%%%%%%%%%%%%%%%%%%%
%%%%%%%%%%%%%%%%%%%%%%%%%%%%%%%%%%%%%%%%%%%%%%%%%%%%%%%%
\subsection{Curvature}

Let $P \overset{\pi}{\to} M$ be a principal $G$-bundle and $\rho: G \to GL(V)$ a Lie group homomorphism. Let $E \equiv P \times_G V \overset{\varpi}{\to} M$ be the associated VB. We saw in the last lecture that we have a $C^\infty(M)$-module isomorphism 
\eq{
	\pbrace{s:M \to E \, | \, \varpi \circ s = \id_M} = \Gamma(E) \cong C_G^\infty(P,V) = \pbrace{\zeta : P \to V \, | \, r_g^\ast \zeta = \rho(g^{-1}) \circ \zeta}
}
with module actions $f \cdot \zeta = (\pi^\ast f) \zeta$. \\
We wish to generalise this from functions to forms. We define $\Omega^k(P,V)$ to be the $k$-forms on $P$ with values in $V$. If $p \in P, \,  \omega \in \Omega^k(P,V)$, then $\omega_p : \Lambda^k T_pP \to V$ is linear. Let $\Omega^k_G(P,V) \subset \Omega^k(P,V)$ denote those $V$-valued $k$-forms $\omega$ which are both 
\begin{itemize}
	\item \bam{horizontal}: $\forall \xi$ vertical, $i_\xi \omega = 0$
	\item \bam{invariant}: $\forall g \in G$, $r_g^\ast \omega = \rho(g^{-1}) \circ \omega$. 
\end{itemize} 
Forms $\omega\in \Omega^k(P,V)$ are said to be basic since they come from bundle valued forms on the base. Indeed, we have 
\begin{prop}
	There is an isomorphism of $C^\infty(M)$-modules 
	\eq{
		\Omega^K_G(P,V) \cong \Omega^k(M,P\times_G V)
	}
	where for $\omega \in \Omega^k_G(P,V), \, f \cdot \omega = (\pi^\ast f) \omega$
\end{prop}
\begin{proof}
	Similar to $k=0$ case. Define $\sigma \in \Omega^k(M,P\times_G V)$ locally by $\pbrace{\sigma_\alpha \in \Omega^k(U_\alpha,V)}$ obeying $\sigma_\alpha(a) = \rho(g_{\alpha\beta}(a)) \sigma_\beta(a)$. Then $\zeta_\alpha(p) = \rho(g_\alpha(p))^{-1} \circ \pi^\ast \sigma_\alpha$ is clearly horizontal. It can be shown to be invariant and that $\forall p \in \pi^{-1} U_{\alpha\beta}, \, \zeta_\alpha(p) = \zeta_\beta(p)$. Conversely, if $\zeta \in \Omega^k_G(P,V)$, we define $\sigma_\alpha = s_\alpha^\ast \zeta$ and one can show that $\forall a \in U_{\alpha\beta}, \, \sigma_\alpha(a) = \rho(g_{\alpha\beta}(a)) \sigma_\beta(a)$ 
\end{proof}

If $\sigma \in \Gamma(P \times_G V)$, $d^\nabla \sigma_\alpha = \rho(g_{\alpha\beta}) d^\nabla \sigma_\beta$, and hence $d^\nabla \sigma \in \Omega^1(M, P\times_G V)$. 

\begin{lemma}
	Let $\alpha \in \Omega^k_G(P,V)$. Then $h^\ast d\alpha \in \Omega^{k+1}_G(P,V)$. 
\end{lemma}
\begin{proof}
	$h^\ast d\alpha$ is horizontal by construction, so we check invariance; 
	\eq{
		r_g^\ast h^\ast d\alpha = h^\ast r_g^\ast d\alpha = h^\ast d(r_g^\ast \alpha) = h^\ast d(\rho(g)^{-1} \circ \alpha) = \rho(g)^{-1} \circ h^\ast d\alpha
	}
\end{proof}

\begin{definition}
	Let $\omega \in \Omega^1(P,\mf{g})$ be the connection one form of an Ehresmann connection $\ms{H} \subset TP$. Its \bam{curvature} is $\Omega \equiv h^\ast d\omega$. 
\end{definition}

\begin{lemma}
	$\Omega \in \Omega^2_G(P,V)$. 
\end{lemma}
\begin{proof}
	Horizontal by construction, and by the same calculation as the lemma above it is invariant because $\omega$ is. 
\end{proof}

\begin{prop}
	$\Omega=0$ iff $\ms{H} \subset TP$ is (Frobenius) integrable.  
\end{prop}
\begin{proof}
	we see
	\eq{
		\Omega(\xi,\eta) &= d\omega(h\xi, h\eta) = h\xi \underbrace{\omega(h\eta)}_{=0} - h\eta \underbrace{\omega (h\xi)}_{=0} - \omega (\comm[h\xi]{h\eta}) \\
		&= \omega(\comm[h\xi]{h\eta}) \\
	}
	Hence 
	\eq{
		\Omega = 0 \Leftrightarrow & \forall \xi, \eta \, \comm[h\xi]{h\eta} \text{ is horizontal } \\
		\Leftrightarrow & \comm[\ms{H}]{\ms{H}} \subset \ms{H} \\
		\Leftrightarrow & \ms{H} \subset TP \text{ is integrable}. 
	}
\end{proof}

\begin{prop}[Structure equation]
	$\Omega = d\omega + \frac{1}{2}\comm[\omega]{\omega}$
\end{prop}
\begin{proof}
	We need to show $\Omega(\xi,\eta) = d\omega(\xi,\eta) + \comm[\omega(\xi)]{\omega(\eta)}$. \\
	Let $\xi, \eta$ be horizontal. Then $h\xi = \xi$ and $h\eta =\eta$, . hence $\Omega(\xi,\eta) = d\omega(\xi,\eta)$ and $\omega(\xi) = 0 = \omega(\eta)$. \\
	Let $\eta$ be horizontal and $\xi = \xi_X$ be vertical. Then $h\xi=0$, $h\eta = \eta$, and $\omega(\eta)$. Hence we need 
	\eq{
		0 = d\omega(\xi_X,\eta) = -\eta \omega(\xi_X) - \omega(\comm[\xi_X]{\eta}) = -\underbrace{\eta X}_{=0} - \omega(\comm[\xi_X]{\eta})
	}
	i.e that $\comm[\xi_X]{\ms{H}} \subset \ms{H}$. This is the case as $\ms{H}$ is invariant. \\
	Let $\xi=\xi_X, \eta = \xi_Y$ vertical. Then $h\xi_X = 0 = h\xi_Y$ and $\omega(\xi_X), \omega(\xi_Y) = Y$. So we must show that 
	\eq{
		0 &= d\omega(\xi_X,\xi_Y) + \comm[\omega(\xi_X)]{\omega(\xi_Y)} \\
		&= \xi_X Y - \xi_Y X - \omega(\comm[\xi_X]{\xi_Y}) + \comm[X]{Y} \\
		&= -\omega(\xi_{\comm[X]{Y}}) + \comm[X]{Y}
	}
	so done. 
\end{proof}

\begin{corollary}[Bianchi Identity]
	$h^\ast d\Omega = 0$
\end{corollary}
\begin{proof}
	\eq{
		h^\ast d\Omega = h^\ast d(d\omega + \frac{1}{2}\comm[\omega]{\omega}) = h^\ast \comm[d\omega]{\omega} = \comm[h^\ast d\omega]{h^\ast \omega} = 0
	}
	since $h^\ast \omega = 0$
\end{proof}
Let's define $d^\nabla : \Omega^k_G(P,V) \to \Omega^{k+1}_G(P,V)$ by $d^\nabla = h^\ast d$. Then, unlike $d$, $d^\nabla$ need not be a differential, and the obstruction is the curvature:

\begin{prop}
	$\forall \alpha \in \Omega_G^k(P,V), \, d^\nabla(d^\nabla \alpha) = \rho(\Omega) \wedge \alpha$
\end{prop}
\begin{proof}
	\eq{
		d^\nabla \alpha &= d\alpha + \rho(\omega) \wedge \alpha \\
		\Rightarrow d^\nabla(d^\nabla \alpha) &= d(d\alpha + \rho(\omega) \wedge \alpha) + \rho(\omega) \wedge (d\alpha + \rho(\omega) \wedge \alpha) \\
		&= \rho(d\omega) \wedge \alpha - \rho(\omega) \wedge d\alpha + \rho(\omega) \wedge d\alpha + \rho(\omega) \wedge \rho(\omega) \wedge \alpha \\
		&= \rho(d\omega) \wedge \alpha + \frac{1}{2} \comm[\rho(\omega)]{\rho(\omega)} \wedge \alpha\\
		&= \rho(d\omega + \frac{1}{2} \comm[\omega]{\omega}) \wedge \alpha  \\
		&= \rho(\Omega) \wedge \alpha
	}
\end{proof}

\begin{ex}
	Write $F_\alpha = s_\alpha^\ast \Omega$. Express $F_\alpha$ in terms of $A_\alpha = s_\alpha^\ast \omega$ and relate $F_\alpha, F_\beta$ on $U_{\alpha\beta} \neq \emptyset $
\end{ex}

%%%%%%%%%%%%%%%%%%%%%%%%%%%%%%%%%%%%%%%%%%%%%%%%%%%%%%%%
%%%%%%%%%%%%%%%%%%%%%%%%%%%%%%%%%%%%%%%%%%%%%%%%%%%%%%%%
%%%%%%%%%%%%%%%%%%%%%%%%%%%%%%%%%%%%%%%%%%%%%%%%%%%%%%%%
%%%%%%%%%%%%%%%%%%%%%%%%%%%%%%%%%%%%%%%%%%%%%%%%%%%%%%%%
\subsection{Homogeneous spaces and Invariant Connections I}
Let $G$ be a Lie group acting transitively on a manifold $M$. Pick $a \in M$ and let $H \subset G$ be the stabiliser subgroup. It is a closed subgroup, and then $M \cong \faktor{G}{H}$, where the diffeomorphism is $G$-equivariant and $G \lact \faktor{G}{H}$ is induced by left multiplication in $G$. If $g \in G$, we let $\phi_g : M \to M$ denote the corresponding diffeomorphism. If $X \in \mf{g}$, we define a vector field $\xi_X \in \mf{X}(M)$ by 
\eq{
	(\xi_X f)(m) = \ev{\frac{d}{dt} f \pround{\phi_{\exp(-tX)}(m)}}{t=0}
}
Then $\comm[\xi_X]{\xi_Y} = \xi_{\comm[X]{Y}}$. \\
Since $H$ stabilises $a \in M$, $\forall h \in H, \, (\phi_h)_\ast : T_aM \to T_a M$, and we get a Lie group homomorphism $\lambda : H \to GL(T_aM)$ called the \bam{linear isotropy representation}. We will use the same notation for the induced Lie algebra rep $\lambda: \mf{h} \to \mf{gl}(T_aM)$. Evaluating at $a \in M$, we get a surjective linear map $\mf{g} \to T_aM, \, X \mapsto \ev{\xi_X}{a}$ whose kernel is $\mf{h}$. 

\begin{definition}
	We say that $\faktor{G}{H}$ is \bam{reductive} if the short exact sequence 
	\eq{
		0 \to \mf{h} \to \mf{g} \to T_a M \to 0
	}
	splits as $H$-modules. In other words if $\exists \mf{m} \subset \mf{g}$ such that $\mf{g} \oplus \mf{m}$ and $\forall h \in H, \, \Ad_h : \mf{m} \to \mf{m} $. In that case $T_aM \cong \mf{m}$ as $H$-modules. 
\end{definition}

If $g \in G$ and $\phi_g \in \Diff(M)$, we define $\phi_g \cdot f = f \circ \phi_{g^{-1}}$ and $\phi_g \cdot \xi = (\phi_g)_\ast \xi$ where 
\eq{
	((\phi_g)_\ast \xi)_a = ((\phi_g)_\ast)_{\phi_g^{-1}(a)} \xi_{\phi_g^{-1}(a)}
}
It follows that 
\eq{
	\phi_g \cdot (Xf) &= (\phi_g \cdot X) (\phi_g \cdot f) \\
	\phi_g \cdot (fX) &= (\phi_g \cdot f)(\phi_g \cdot X)
}
Now let $\nabla $ be an affine connection, (i.e. $\nabla_{fX}Y = f \nabla_XY$, $\nabla_X(fY) = X(f)Y + f\nabla_X Y$) . Let $\phi \in \Diff(M)$. Define $\nabla^\phi$ by 
\eq{
	\nabla_X^\phi Y = \phi \cdot \nabla_{\phi^{-1}\cdot X} (\phi^{-1} \cdot Y)
}

\begin{lemma}
	$\nabla^\phi$ is an affine connection
\end{lemma}
\begin{proof}
	\eq{
		\nabla_{fX}^\phi Y &= \phi \cdot \nabla_{\phi^{-1}\cdot(fX)}(\phi^{-1}Y) \\
		&= \phi \cdot \nabla_{(\phi^{-1}\cdot f)(\phi^{-1}\cdot X)}(\phi^{-1}Y) \\
		&= \phi \cdot \pround{ \phi^{-1} \cdot f \nabla_{\phi^{-1}\cdot X}(\phi^{-1}\cdot Y} \\
		&= (\phi \cdot \phi^{-1}f) (\phi \cdot \nabla_{\phi^{-1}X}(\phi^{-1}\cdot Y) \\
		&= f \nabla_X^\phi Y \\
		\nabla_X^\phi (fY) &= \phi \cdot \pround{\nabla_{\phi^{-1}\cdot X}\phi^{-1}(fY)} \\
		&= \phi  \cdot \pround{\nabla_{\phi^{-1}\cdot X}(\phi^{-1}f)(\phi^{-1}Y)} \\
		&= \phi \cdot \pround{(\phi^{-1}X)(\phi^{-1}f) (\phi^{-1}Y) + (\phi^{-1}f) \nabla_{\phi^{-1}X}(\phi^{-1}Y)} \\
		&= (\phi \cdot \phi^{-1} \cdot X(f))(\phi \cdot \phi^{-1} \cdot Y) + (\phi \cdot \phi^{-1} \cdot f) \nabla_{\phi^{-1}X}(\phi^{-1}Y) \\
		&= X(f) Y + f \nabla_X^\phi Y
	}
\end{proof}

\begin{definition}
	An affine connection $\nabla$ on a reductive homogeneous $M=\faktor{G}{H}$ is said to be \bam{G-invariant} if $\forall g \in G, \, \nabla^{\phi_g} = \nabla$. i.e 
	\eq{
		\phi_g \cdot \nabla_X Y = \nabla_{\phi_g X} (\phi_g Y)
	}
	If $H = \pbrace{e}, M=G$, then $\nabla$ is \bam{left invariant} if 
	\eq{
		L_g \cdot \nabla_XY = \nabla_{L_g \cdot X}(L_g \cdot Y)
	}
\end{definition}

Suppose that $X,Y$ are left invariant , so that $L_g \cdot X = X, L_g \cdot Y = Y$. In that case, the left invariance of $\nabla$ implies that $\nabla_XY$ is also left invariant. Now, on a Lie group we may trivialise the tangent bundle via left translations. That means that we have a global frame $(X_1, \dots, X_n)$ consisting of left invariant vector fields. The connection is therefore uniquely determined by $n^3$ numbers $\Gamma^k_{ij}$ defined by 
\eq{
	\nabla_{X_i} X_j = \sum_k \Gamma^k_{ij} X_k
}
These are the components relative to the basis $\pbrace{X_i}$ of a linear map $\Lambda : \mf{g} \to \mf{gl}(\mf{g})$. The torsion and curvature tensors are also left-invariant and are given in terms of $\Lambda$ by 
\eq{
	T(X,Y) &= \Lambda_X Y - \Lambda_Y X - \comm[X]{Y} \\
	R(X,Y)Z &= \comm[\Lambda_X]{\Lambda_Y}Z - \Lambda_{\comm[X]{Y}}Z
}
for LI $X,Y,Z \in \mf{X}(G)$. We see that curvature measures the failure of $\Lambda$ to be a Lie algebra homomorphism. \\
In particular, taking $\Lambda = 0$, we see that there exists a flat connection with torsion given by $T(X,Y) = -\comm[X]{Y}$ relative to which LI vf on $G$ are \bam{parallel} (i.e. $\nabla X = 0$). Of course, there exists another flat connection annihilating the right-invariant vector fields. 

%%%%%%%%%%%%%%%%%%%%%%%%%%%%%%%%%%%%%%%%%%%%%%%%%%%%%%%%
%%%%%%%%%%%%%%%%%%%%%%%%%%%%%%%%%%%%%%%%%%%%%%%%%%%%%%%%
%%%%%%%%%%%%%%%%%%%%%%%%%%%%%%%%%%%%%%%%%%%%%%%%%%%%%%%%
%%%%%%%%%%%%%%%%%%%%%%%%%%%%%%%%%%%%%%%%%%%%%%%%%%%%%%%%
\subsection{Invariant Connections} 
What did we do last time? We were looking at Homogeneous spaces $M \equiv \faktor{G}{H}$, $H\leq G$ a closed subgroup. We had fibre 
\eq{
	H \to G \overset{\pi}{\to} M
}
and for $g \in G$ we have $\phi_g : M \to M$ acting by multiplication, i.e. $\phi_g(a) = g \cdot a$. As a result of the quotient have $eH = o \in M$ s.t 
\eq{
	\forall h \in H \; \phi_h(o) = o \\
}
Then 
\eq{
	(\phi_h)_\ast : T_oM \to T_o M 
}
Hence we may make the following def:
\begin{definition}
	The \bam{linear isotropy representation}
	\eq{
		\lambda: H \to GL(T_oM)
	}
	is given by $\lambda_h = (\phi_h)_\ast$. 
\end{definition}
We also have the map
\eq{
	\xi : \mf{g} &\to \mf{X}(M) \\
	X &\mapsto \xi_X
}
s.t. $\comm[\xi_X]{\xi_Y} = \xi_{\comm[X]{Y}}$. Composing with evaluation yields
\eq{
	ev_o \circ \xi : \mf{g} \to T_o M 
}
where $\ker(ev_o \circ \xi) = \mf{h} \subset \mf{g}$. This is bijective so in fact 
\eq{
	T_o M \cong \faktor{\mf{g}}{\mf{h}}
}
and we get commuting diagram 
\begin{center}
	\begin{tikzcd}
		T_o M  \arrow[r, "\lambda_h"] \arrow[d, "\cong"']
		& T_oM \arrow[d, "\cong"] \\
		\faktor{\mf{g}}{\mf{h}} \arrow[r, "\Ad(h)"]
		& \faktor{\mf{g}}{\mf{h}}
	\end{tikzcd}
\end{center}

\begin{definition}
	An affine connection $\nabla$ on $TM$ is \bam{G-invariant} if 
	\eq{
		\forall g \in G, \, \nabla^{\phi_g} = \nabla \\
		\phi_g \nabla_\xi \eta = \nabla_{\phi_g \xi} (\phi_g \eta)
	}
\end{definition}

If $H = \pbrace{e}, M=G$, then $\nabla$ is left invariant if for all left invariant vector fields $\xi_X,\xi_Y$ $\nabla_{\xi_X} \xi_Y$ is also LI. This is then uniquely determined by its value at $e$. Hence $\nabla $ defines a bilinear map 
\eq{
	\mf{g} \times \mf{g} &\overset{\alpha}{\to} \mf{g} \\
	(X,Y) &\mapsto \ev{\nabla_{\xi_X} \xi_Y}{e}
}
We can then \bam{curry} a map as given $\alpha : \mf{g} \times \mf{g} \to \mf{g}$ we can get 
\eq{
	\Lambda : \mf{g} &\to \End(\mf{g}) \\
	X &\mapsto \Lambda_X
}
where $\Lambda_X(Y) = \alpha(X,Y)$. 

\begin{ex}
	Show that the torsion $T$ and curvature $R$ of $\Lambda$ are left invariant and given by 
	\eq{
		T(X,Y) &= \Lambda_X Y - \Lambda_Y X - \comm[X]{Y} \\ 
		R(X,Y)Z &= \comm[\Lambda_X]{\Lambda_Y}Z - \Lambda_{\comm[X]{Y}}Z
	}
	Note $R$ is the obstruction to $\Lambda$ being a Lie algebra homomorphism. 
\end{ex}

\begin{claim}
	$\exists$ a LI connection $\nabla$ corresponding to $\Lambda=0$. 
\end{claim} 
Such $\Lambda$ is flat, but has torsion $T(X,Y) = - \comm[X]{Y}$. As such $\nabla$ is characterised by $\forall \text{ LI } \xi , \, \nabla\xi = 0$. Now let $H \neq \pbrace{e}$ be closed and reductive: $\mf{g} = \mf{h} \oplus \mf{m}$ where $\Ad_H(\mf{m}) \subset \mf{m}$. Note $\mf{m}\cong \faktor{\mf{g}}{\mf{h}}$, so in the previous case $\mf{m}\cong T_oM$

\begin{aside}
	There is a "holonomy principle" that 
	\eq{
		\pbrace{G\text{-invariant tensor fields on }\faktor{G}{H}} \overset{ev_o}{\leftrightarrow} \pbrace{\Ad(H)\text{-invariant tensors on }\mf{m}}
	}
	This comes about, as if we take a tensor $T$ at $o$, we can define a tensor field on $\faktor{G}{H}$ by 
	\eq{
		\mc{T}(a) = \phi_g T
	}
	for and $g \in G$ s.t. $\phi_g o = a$. Then if we have another representative $g^\prime$ then 
	\eq{
		g^{-1}g^\prime \in H \Leftrightarrow \phi_{g^{-1}g^\prime} o = o
	}
	so 
	\eq{
		T = \phi_{g^{-1}g^\prime} T
	}
\end{aside}

\begin{claim}
	An invariant connection $\nabla$ is determined by a bilinear map 
	\eq{
		\alpha : \mf{m} \times \mf{m} \to \mf{m}
	}
	which is $H$ invariant, the \bam{Nomizu map}. 
\end{claim}

We can take natural coordinates for $M$ in the neighbourhood $V\subset \mf{m}$ of $o$ by exponentiating $\mf{m}$. The projection $\pi$ is a local diffeo on $U = \exp(V)$. \\
In a basis for $\mf{m}$, $\pbrace{e_i}$,  
\eq{
	V & \to U \\
	\sum x^i e_i &\mapsto \exp\pround{\sum x^i e_i}
}
Now $\forall g \in U$, $\pi(g) = \phi_g \cdot o $, Let $\overline{V} = \pbrace{\phi_g \cdot o | g \in U}$. For $X \in \mf{m}$ define $\xi_X \in \mf{X}(\overline{V})$ by 
\eq{
	(\xi_X)_{\phi_g o } &\equiv ((\phi_g)_\ast)_o (\pi_\ast)_e X  \\
	&= ((\phi_g \circ \pi)_\ast)_e X \\
	&= ((\pi \circ L_g)_\ast)_e X = (\pi_\ast)_g X^L_g 
}
where $X^L$ is the LIVF defined by $\ev{X^L}{e} = X$. Hence $\xi_X$ is $\pi$-related to $X^L$. Then $\comm[\xi_X]{\xi_Y}$ is $\pi$-related to $\comm[X^L]{Y^L} = \comm[X]{Y}^L$. \\  
Now let $W\subset V$ s.t. $\forall h \in H , \, \Ad_hW \subset V$, and def $\overline{W}$ accordingly. Then for $h\in H$, $\phi_h : \overline{W} \to \overline{V}$. As such 
\eq{
	\phi_h \phi_g \cdot o &= \phi_h \phi_g \phi_{h^{-1}} \phi_h \cdot o \\
	&= \phi_{hgh^{-1}} \cdot o \in \overline{V}. 
}

We will now need the following lemma 

\begin{lemma}
	$\forall g \in \exp(W), \, h \in H$, 
	\eq{
		(\phi_h)_\ast \xi_X = \xi_{\Ad_hX}
	}
	at $\phi_g o $, i.e. at all point in $\overline{V}$. 
\end{lemma}
\begin{proof}
	\eq{
		\psquare{(\phi_h)_\ast \xi_X}_{\phi_h \phi_g o} &= (\phi_h)_\ast (\xi_X)_{\phi_g o} \\ 
		&= (\phi_h)_\ast (\phi_g)_\ast \pi_\ast X \\
		&= (\phi_{hg})_\ast \pi_\ast X \\
		&= (\phi_{hgh^{-1}})_\ast (\phi_h)_\ast \pi_\ast X \\ 
		&= (\xi_{\Ad(h)X})_{\phi_{hgh^{-1}} o} = (\xi_{\Ad_hX})_{\phi_{h}\phi_g o} 
	}
	recalling the commuting diagram 
	\begin{center}
		\begin{tikzcd}
			T_o M  \arrow[r, "(\phi_h)_\ast"] 
			& T_oM  \\
			\mf{m} \arrow[r, "\Ad(h)"] \arrow[u, "\pi_\ast"]
			& \mf{m} \arrow[u, "\pi_\ast"']
		\end{tikzcd}
	\end{center}
\end{proof}

\begin{lemma}
	Let $X,Y \in \mf{m}$, and $\xi_X,\xi_Y \in \mf{X}(\overline{V})$. Then $\ev{\comm[\xi_X]{\xi_Y}}{o} = \pi_\ast \comm[X]{Y}_{\mf{m}}$
\end{lemma}
\begin{proof}
	We saw above that $\comm[\xi_X]{\xi_Y}$ is $\pi$-related to $\comm[X^L]{Y^L} = \comm[X]{Y}^L$. Hence $\comm[\xi_X]{\xi_Y} = \xi_{\comm[X]{Y}}$ and evaluating at $o \in M$ gives 
	\eq{
		\ev{\comm[\xi_X]{\xi_Y}}{o} = \ev{\xi_{\comm[X]{Y}_{\mf{h}}}}{o} + \ev{\xi_{\comm[X]{Y}_{\mf{m}}}}{o} = \ev{\xi_{\comm[X]{Y}_{\mf{m}}}}{o} = \pi_\ast \comm[X]{Y}_{\mf{m}}
	}
\end{proof}

\begin{theorem}[Nomizu]
	There is a bijective correspondence
	\eq{
		\pbrace{G\text{-invariant affine connections on }M} \leftrightarrow \pbrace{\Ad(h)\text{-invariant bilinear maps } \alpha:\mf{m}\times\mf{m}\to\mf{m}}
	}
	given by 
	$\alpha(X,Y) = \ev{\nabla_{\xi_X} \xi_Y}{o}
	$
\end{theorem}
Note $\exists !$ $G$-invariant connection $\nabla$ with $\alpha=0$, and this is called the \bam{canonical connection}. If you curry this map again you can show 
\eq{
	T(X,Y) &= \alpha(X,Y) = \alpha(Y,X) - \comm[X]{Y}_\mf{m} \\
	R(X,Y)Z &= \alpha(X,\alpha(Y,Z)) - \alpha(Y,\alpha(X,Z)) - \alpha(\comm[X]{Y}_{\mf{m}},Z) - \comm[X]{Y}_\mf{h} Z 
}
If $\alpha=0$ we get 
\eq{
	T(X,Y) &= - \comm[X]{Y}_\mf{m} \\
	R(X,Y) &= -\comm[X]{Y}_\mf{h} 
}
If $T=0$, $M$ is said to be \bam{symmetric}. 

%%%%%%%%%%%%%%%%%%%%%%%%%%%%%%%%%%%%%%%%%%%%%%%%%%%%%%%%%
%%%%%%%%%%%%%%%%%%%%%%%%%%%%%%%%%%%%%%%%%%%%%%%%%%%%%%%%%
\subsection{Cartan Connections}
Again consider homogeneous reductive spaces 
\begin{center}
	\begin{tikzcd}
		H \arrow[r] & G \arrow[d,"\pi"] \\ 
		& M= \faktor{G}{H} 
	\end{tikzcd}
\end{center}
With a local section $\sigma : U \to G$ we can pull back the LI MC 1-form $\vartheta_G \in \Omega^1(G;\mf{g})$
\eq{
	\sigma^\ast \vartheta_G \in \Omega^1(U;\mf{g})
}
Recall the MC 1-form satisfies structure equation s
\eq{
	d\vartheta_G + \frac{1}{2} \comm[\vartheta_G]{\vartheta_G} = 0
}
Then given two such sections $\sigma_i$ we have 
\eq{
	\forall a \in U, \, \sigma_2(a) = \sigma_1(a) h(a)
}
for some $h:U \to H$, a uniquely defined function. 
\begin{lemma}
	\eq{
		\sigma_2^\ast \vartheta_G = \Ad(h^{-1}) \cdot \sigma_1^\ast \vartheta_G + h^\ast \vartheta_H
	}
\end{lemma}
\begin{proof}
	We will notationally use the idea of matrix groups but in general the proof works. Then 
	\eq{
		\sigma^\ast \vartheta_g = \sigma^{-1} d\sigma \,.
	}
	Then 
	\eq{
		\sigma_2^\ast \vartheta_G &= \sigma_2^{-1} d\sigma_2 \\
		&= (\sigma_1 h)^{-1} d(\sigma_1 h) \\
		&= h^{-1} \sigma_1^{-1} ( d\sigma_1 h + \sigma_1 dh) \\
		&= h^{-1}(\sigma_1^{-1}d\sigma_1)h + h^{-1}dh 
	}
	so done. 
\end{proof}
As we are in the reductive case, $\mf{g} = \mf{h} \oplus \mf{m}$ and we can decompose. Write $\sigma_1^\ast \vartheta_G = \theta_1 + \omega_1 $ for $\theta_1 \in \Omega^1(U,\mf{m}), \, \omega_1 \in \Omega^1(U,\mf{h})$. Then 
\eq{
	\theta_2 + \omega_2 &= \Ad(h)^{-1}(\theta_1 + \omega_1) + h^\ast \vartheta_H 
}
so 
\eq{
	\theta_2 &= \Ad(h)^{-1} \theta_1 \\
	\omega_2 &= \Ad(h)^{-1} \omega_1 + h^\ast \vartheta_H
}
decomposing. Hence $\theta_2$ transforms as a tensor, $\omega_2$ as a gauge field. Now if we let $\sigma = \sigma_1$ and the structure equation becomes 
\eq{
	d(\theta + \omega) + \frac{1}{2} \comm[\theta+\omega]{\theta + \omega} &= 0 \\
	d\theta + d\omega + \frac{1}{2}\comm[\theta]{\theta} + \frac{1}{2}\comm[\omega]{\omega} + \comm[\omega]{\theta} &= 0 
}
As such decomposing 
\eq{
	d\theta + \frac{1}{2} \comm[\theta]{\theta}_\mf{m} + \comm[\omega]{\theta} &= 0  &\Rightarrow& &\Theta &\equiv d\theta + \comm[\omega]{\theta} = - \frac{1}{2}\comm[\theta]{\theta}_\mf{m}\\
	d\omega + \frac{1}{2} \comm[\theta]{\theta}_\mf{h} + \frac{1}{2} \comm[\omega]{\omega} &= 0  &\Rightarrow& &\Omega &\equiv d\omega + \frac{1}{2}\comm[\omega]{\omega} = -\frac{1}{2} \comm[\theta]{\theta}_\mf{h}
}
As such 
\eq{
	\Theta(\xi_X,\xi_Y) &= -\comm[X]{Y}_\mf{m} \\
	\Omega(\xi_X,\xi_Y) &= -\comm[X]{Y}_\mf{h}
}
Gauge fields for the canonical invariant connection on $\faktor{G}{H}$ are $\sigma^\ast \vartheta_G$. \\
With this motivation with us, the Cartan connections are going to be generalisation of these where in the gauge field descriptions these are local 1-forms on the base. The Cartan viewpoint is to view $TM$ not as a linear rep of $GL(n,\mbb{R})$, but as a homogeneous space of the affine group $\mbb{A}(n,\mbb{R})\cong GL(n,\mbb{R}) \ltimes \mbb{R}^n$ such that $T_a M \cong \faktor{\mbb{A}(n,\mbb{R})}{GL(n,\mbb{R})}$.

\begin{definition}
	A \bam{Cartan gauge} (def from Sharpe, Jose doesn't like) with model $\faktor{G}{H}$ on $M$ is a pair $(U,\theta)$ where $U \subset M$ open and $\theta \in \Omega^1(U,\mf{g})$ satisfying \bam{regularity}
	\eq{
		T_a M \overset{\theta_a}{\to} \mf{g} \overset{pr}{\to} \faktor{\mf{g}}{\mf{h}}
	}
	is an isomorphism $\forall a \in U$. 
\end{definition}
This is the analogue of a chart
\begin{definition}
	A \bam{Cartan atlas} is a collection of Cartan gauges $\pbrace{(U_\alpha,\theta_\alpha)}$ s.t 
	\begin{itemize}
		\item $\bigcup_{\alpha} U_\alpha = M $
		\item on $U_{\alpha\beta}$ 
		\eq{
			\theta_\beta = \Ad(h_{\alpha\beta}^{-1}) \theta_\alpha + h_{\alpha\beta}^\ast \vartheta_H
		}
		for some $h_{\alpha\beta}:U_{\alpha\beta} \to H$. 
	\end{itemize}
\end{definition}
This is very analogous to atlases. 
\begin{definition}
	Two atlases are \bam{equivalent} if their union is an atlas. 
\end{definition}

\begin{definition}
	A \bam{Cartan structure} on $M$ is an equivalence class (equivalently maximal atlas) of Cartan atlases. A \bam{Cartan geometry} is a manifold $M$ together with a Cartan structure.  
\end{definition}

\begin{definition}
	The \bam{curvature} of a Cartan gauge $(U,\theta)$ is $\Omega \in \Omega^2(U,\mf{g})$ given by 
	\eq{
		\Omega = d\theta + \frac{1}{2} \comm[\theta]{\theta}
	}
\end{definition}
If I have a Cartan atlas, I can ask how respective curvatures $\Omega_\alpha$ change on overlaps. 
\begin{lemma}
	On $U_{\alpha\beta}$
	\eq{
		\Omega_\beta = \Ad(h_{\alpha\beta}^{-1}) \Omega_\alpha
	}
\end{lemma}
\begin{proof}
	\eq{
		\theta_\beta &= \Ad(h_{\alpha\beta}^{-1})\theta_\alpha + h_{\alpha\beta}^\ast \vartheta_H \\ 
		\Rightarrow d\theta_\beta + \frac{1}{2}\comm[\theta_\beta]{\theta_\beta} &= d\pround{\underbrace{\Ad(h_{\alpha\beta}^{-1})\theta_\alpha}_{h_{\alpha\beta}^{-1} \theta_\alpha h_{\alpha\beta}} + h_{\alpha\beta}^\ast \vartheta_H} + \frac{1}{2}\comm[\Ad(h_{\alpha\beta}^{-1})\theta_\alpha + h_{\alpha\beta}^\ast \vartheta_H]{\Ad(h_{\alpha\beta}^{-1})\theta_\alpha + h_{\alpha\beta}^\ast \vartheta_H} \\
		&= \Ad(h_{\alpha\beta}^{-1}) d\theta_\alpha - \comm[\Ad(h_{\alpha\beta}^{-1})\theta_\alpha]{h_{\alpha\beta}^\ast\vartheta_H} - \frac{1}{2} h_{\alpha\beta}^\ast \comm[\vartheta_H]{\vartheta_H} + \frac{1}{2} \Ad(h_{\alpha\beta}^{-1}) \comm[\theta_\alpha]{\theta_\alpha} \\
		& \, +\frac{1}{2} \comm[h_{\alpha\beta}^\ast \vartheta_H]{h_{\alpha\beta}^\ast \vartheta_H} + \comm[\Ad(h_{\alpha\beta}^{-1}) \theta_\alpha]{h_{\alpha\beta}^\ast \vartheta_H} \\
		&= \Ad(h_{\alpha\beta}^{-1}) \pround{d\theta_\alpha + \frac{1}{2} \comm[\theta_\alpha]{\theta_\alpha}}
	}
\end{proof}
Hence setting $\Omega_\alpha=0$ is an \emph{extrinsic} statement of an atlas. 

\begin{definition}
	A Cartan structure is \bam{flat} if $\forall \alpha , \, \Omega_\alpha = 0$
\end{definition}

\begin{example}
	Flat Cartan structures: 
	\begin{itemize}
		\item $G \to \faktor{G}{H}$ with $(U_\alpha,\sigma_\alpha^\ast \vartheta_G)$
		\item an open subset $V\subset \faktor{G}{H}$ as above.
		\item $\Gamma \subset G$ acting by covering transformations, locally like $\faktor{G}{H}$. 
	\end{itemize}
\end{example}

\begin{definition}
	A Klein geometry $\faktor{G}{H}$ has \bam{kernel} $K$: the largest subgroup of $H$ that is normal in $G$. If $K=1$ we say that $\faktor{G}{H}$ is \bam{effective}. If $K$ is discrete we say the geometry is \bam{locally effective}.
\end{definition}

\begin{lemma}
	If $K \neq 1$ then $\faktor{{(\faktor{G}{K})}}{{(\faktor{H}{K})}}$ is effective. 
\end{lemma}

\begin{prop}
	If $\faktor{G}{H}$ is effective, and $\exists k:U \to H$ s.t. $\theta = \Ad(k^{-1})\cdot \theta + k^\ast \vartheta_H$, then $k=1$. 
\end{prop}

This means that, given a Cartan atlas $\pbrace{(U_\alpha,\theta_\alpha)}$ modelled on an effective $\faktor{G}{H}$, then in overlaps $U_{\alpha\beta}$, $\theta_\beta = \Ad(h_{\alpha\beta}^{-1}) \circ \theta_\alpha + h_{\alpha\beta}^\ast \vartheta_H$ for a unique $h_{\alpha\beta}:U_{\alpha\beta} \to H$. Indeed if $\theta_\beta = \Ad(\tilde{h}_{\alpha\beta}^{-1}) \circ \theta_\alpha + \tilde{h}_{\alpha\beta}^\ast \vartheta_H$, then letting $k = \tilde{h}_{\alpha\beta}^{-1} h_{\alpha\beta}$ we would have
\eq{
	\theta_\alpha &= \Ad(\tilde{h}_{\beta\alpha}^{-1}) \circ \theta_\beta + \tilde{h}_{\beta\alpha}^\ast \vartheta_H \\
	\Rightarrow \theta_\beta &= \Ad(h_{\alpha\beta}^{-1}) \circ \psquare{\Ad(\tilde{h}_{\beta\alpha}^{-1}) \circ \theta_\beta + \tilde{h}_{\beta\alpha}^\ast \vartheta_H} + h_{\alpha\beta}^\ast \vartheta_H \\
	&= \Ad(k^{-1}) \circ \theta_\beta + \underbrace{\Ad(h_{\alpha\beta}^{-1}) \circ \tilde{h}_{\beta\alpha}^\ast \vartheta_H + h_{\alpha\beta}^\ast \vartheta_H}_{k^\ast \vartheta_H}
}
It follows from uniqueness then that $\pbrace{h_{\alpha\beta} : U_{\alpha\beta} \to H}$ defines a (Cech) cocycle. Therefore they are the transition functions of a principle $H$-bundle $P\overset{\pi}{\to} M$, where $P = \sqcup_{\alpha} \faktor{(\pbrace{\alpha}\times U_\alpha \times H)}{\sim}$, $(\alpha,a,h)\sim (\beta,a,h_{\alpha\beta}^{-1}(a)h)$, and $\pi(\alpha,a,h) = a$. The right action is given by $r_h[(\alpha,a,\tilde{h})] = [(\alpha,a,\tilde{h}h)]$. This is well defined since the identification uses left multiplication. 

Let $X \in \mf{h}$. Then $X^L\in\mf{X}(H)$ is the corresponding LIVF. We extend it to $U \times H$ as $(0,X^L) \equiv \xi_X \in\mf{X}(U \times H)$. Since $X^L$ is LI and the identifications involve left multiplication the vector fields $\xi_X$ glue to give a well defined vector field $\xi_X \in \mf{X}(P)$. We then have 

\begin{lemma}
	Let $r_h : P \to  P$ denote the right action of $h \in H$ on $P$. Then $\forall X \in \mf{h}, \, (r_h)_\ast \xi_X = \xi_{\Ad(h)^{-1}X}$. 
\end{lemma}
\begin{proof}
	It is sufficient to check locally on $U \times H$. Here $r_h = \id \times R_h$ where $R_h : H \to H$ is right multiplication by $h$. Let $L_h : H \to H$ be left multiplication and then on $U \times H$ we have 
	\eq{
		(r_h)_\ast \xi_X &= (\id \times R_h)_\ast (0,X^L) \\
		&= (0,(R_h)_\ast X^L) \\
		&= (0, (r_h)_\ast (L_{h^{-1}})_\ast X^L) \text{ since $X^L$ is LI} \\
		&= (0, (\Ad(h^{-1}) \cdot X)^L) \\
		&= \xi_{\Ad(h)^{-1}X}
	}
\end{proof}

The Cartan atlas $(U_\alpha,\theta_\alpha)$ does not first just give $P \overset{\pi}{\to} M$, but also a one-form $\omega \in \Omega^1(P;\mf{g})$ defined locally by 
\eq{
	\omega : T_{(a,h)}(U_\alpha \times H) &\to T_a U_\alpha \times \mf{h}) \to \mf{g} \\
	(v,y) &\mapsto (v,\vartheta_H(y)) \mapsto \Ad(h^{-1}) \theta_\alpha(v) + \vartheta_H(y) \equiv \omega_\alpha(v,y)
}
On overlaps, we also have $\omega_\beta(v,y) = \Ad(h^{-1}) \theta_\beta(v) + \vartheta_H(y)$. The transition function is then $U_{\alpha\beta} \times H \overset{f_{\alpha\beta}}{\to} U_{\alpha\beta} \times H$ sending $(a,h) \mapsto (a,h_{\alpha\beta}(a)^{-1}h)$. 

We will claim that the $\omega_\alpha$ glue together properly to give a consistent $\omega$. To prove this we will need a preparatory lemma: 

\begin{lemma}
	Let $\mu: H \times H \to H$ and $i: H \to H$ denote multiplication and inversion as groups maps on $H$. Letting $\vartheta_H \in \Omega^1(H;\mf{h})$ be the LI MC one-form we have 
	\eq{
		\forall v \in T_{(h_1,h_2)}(H\times H), \, (\mu^\ast \vartheta_H)(v) &= \Ad(h_2^{-1}) \vartheta_H((pr_1)_\ast v) + \vartheta_H((pr_2)_\ast v) \\
		\forall v \in T_hH  , \, (i^\ast \vartheta_H)(v) &= -\Ad(h) \vartheta_H(v)
	}
\end{lemma}
\begin{proof}
	It is simpler notationally for matrix groups where $\ev{\vartheta_H}{h} = h^{-1}dh$. Hence 
	\eq{
		\ev{i^\ast \vartheta_H}{h}= h dh^{-1} = -hh^{-1} dh h^{-1} = -\Ad(h) \ev{\vartheta_H}{h}
	}
	Moreover we have 
	\eq{
		\mu^\ast \ev{\vartheta_H}{(h_1,h_2)} = (h_1 h_2)^{-1} d(h_1 h_2) = h_2^{-1} h_1^{-2}
		dh_1 h_2 + + h_2^{-1} dh_2  = \Ad(h_2^{-1}) \ev{\vartheta_H}{h_1} + \ev{\vartheta_H}{h_2}
	}
\end{proof}

Now we are ready to state what we want:

\begin{prop}
	The following diagram commutes:
	\begin{center}
		\begin{tikzpicture}[commutative diagrams/every diagram]
		\node (P0) at (30:2.5cm) {$T_a U_{\alpha\beta} \times T_{h_{\alpha\beta}(a)^{-1}h}H$};
		\node (P1) at (30+120:2.5cm) {$T_a U_{\alpha\beta} \times T_{h}H$};
		\node (P2) at (30+240:0.5cm) {$\mf{g}$};
		\path[commutative diagrams/.cd, every arrow, every label]
		(P1) edge node {$(f_{\alpha\beta})_\ast$} (P0)
		(P0) edge node {$\omega_\beta$} (P2)
		(P1) edge node[swap] {$\omega_\alpha$} (P2);
		\end{tikzpicture}
	\end{center}
\end{prop}
\begin{proof}
	We notice that $f_{\alpha\beta}(a,h) = (a,h_{\alpha\beta}(a)^{-1}h) = (\id \circ pr_1, \mu \circ (i \circ h_{\alpha\beta} \circ pr_1 \times pr_2))(a,h)$, so that if $(v,y) \in T_a U_{\alpha\beta} \times T_h H$, $(f_{\alpha\beta})_\ast (v,y) = (v,\mu_\ast(i_\ast \circ (h_{\alpha\beta})_\ast v,y)) \in T_a U_{\alpha\beta} \times T_{h_{\alpha\beta}(a)^{-1}h}H$. Hence 
	\eq{
		(\omega_\beta \circ (f_{\alpha\beta})_\ast)(v,y) &= \omega_\beta (v,\mu_\ast(i_\ast \circ (h_{\alpha\beta})_\ast v,y)) \\
		&= \Ad(h_{\alpha\beta}(a)^{-1}h)^{-1} \theta_\beta(v) + \vartheta_H(\mu_\ast (i_\ast \circ (h_{\alpha\beta})_\ast v,y)) 
	}
	Using the lemma we have that 
	\eq{
		\vartheta_H(\mu_\ast (i_\ast \circ (h_{\alpha\beta})_\ast v,y))  &= (\mu^\ast \vartheta_H)(i_\ast (h_{\alpha\beta})_\ast v,y) \\
		&= \Ad(h^{-1}) \vartheta_H (i_\ast (h_{\alpha\beta})_\ast v) + \vartheta_H(y)  \\
		\vartheta_H(i_\ast (h_{\alpha\beta})_\ast v) &= (i^\ast \vartheta_H)(h_{\alpha\beta}^\ast v) \\
		&= -\Ad(h_{\alpha\beta}(a))(h_{\alpha\beta}^\ast \vartheta_H)(v)
	}
	Hence 
	\eq{
		(\omega_\beta \circ (f_{\alpha\beta})_\ast)(v,y) &= \Ad(h)^{-1} \Ad(h_{\alpha\beta}(a)) \theta_\beta(v) - \Ad(h)^{-1}\Ad(h_{\alpha\beta}(a))(h_{\alpha\beta}^\ast \vartheta_H)(v) + \vartheta_H(y) \\
		&= \Ad(h)^{-1} \Ad(h_{\alpha\beta}(a)) \psquare{\theta_\beta(v) -(h_{\alpha\beta}^\ast \vartheta_H)(v)} + \vartheta_H(y) \\
		&= \Ad(h)^{-1} \circ \theta_\alpha(v) + \vartheta_H(y) \\
		&= \omega_\alpha(v,y)
	}
\end{proof}


\begin{definition}
	The one-form $\omega \in \Omega^1(P;\mf{g})$ is called a \bam{Cartan connection}
\end{definition}

\begin{prop}
	The Cartan connection $\omega \in \Omega^1(P;\mf{g})$ obeys the following:
	\begin{enumerate}
		\item $\forall p \in P, \, \omega_p : T_pP \to \mf{g}$ is a vector space isomorphism 
		\item $\forall h \in H, \, r_h^\ast \omega = \Ad(h^{-1})\circ \omega$
		\item $\forall X \in \mf{h}, \, \omega(\xi_X) = X$
	\end{enumerate}
\end{prop}
\begin{proof}
	We may separate the proof:
	\begin{enumerate}
		\item $\dim P = \dim H + \dim M = \dim \mf{h} + \dim \faktor{\mf{g}}{\mf{h}} = \dim \mf{g}$, so it suffices to show that $\omega_p$ is injective. Now if $(v,y) \in T_a U \times T_h H$ is such that $\omega(v,y) = \Ad(h^{-1})\theta(v) + \vartheta_H(y)=0$, we have $\Ad(h^{-1}) \theta(v) = -\vartheta_H(y) \in \mf{h}$ and hence $\theta(v) \in \Ad(h)\mf{h} = \mf{h} \Rightarrow pr_{\faktor{\mf{g}}{\mf{h}}}\theta(v) = 0$. By the regularity property of $\theta$, $v=0$. Hence $\vartheta_H(y) = 0$, but as $\vartheta_H$ is injective, we have $y=0$ 
		\item It is sufficient to check in a Cartan gauge $(U,\theta)$. Let $(v,y)\in T_aU \times T_hH$. Then for $k \in H$:
		\eq{
			(r_k^\ast \omega)(v,y) = \omega(v,(R_k)_\ast y) = \Ad(hk)^{-1}\circ \theta(v) + \vartheta_H((R_k)_\ast y) 
		}
		and using $R_k^\ast \vartheta_H = \Ad(k^{-1})\circ \vartheta_H$
		\eq{
			(r_k^\ast \omega)(v,y) &= \Ad(k^{-1}) \Ad(h)^{-1} \theta(v) + \Ad(k^{-1}) \vartheta_H(y) \\
			&= \Ad(k^{-1}) \psquare{\Ad(h)^{-1}\theta(v) + \vartheta_H(y)} \\
			&= \Ad(k^{-1}) \omega(v,y)
		}
		\item In a Cartan chart $\xi_X = (0,X^L) \in \mf{X}(U \times H)$, hence 
		\eq{
			\omega(\xi_X) = \Ad(h)^{-1} \theta(0) + \vartheta_H(X^L) = 0 + X = X
		}
	\end{enumerate}
\end{proof}

\begin{remark}
	Properties 2 and 3 are reminiscent of an Ehresmann connection except that $\omega$ takes values in $\mf{g}$ not $\mf{h}$. 
\end{remark}

Notice that if $\pbrace{(U_\alpha,\theta_\alpha)}$ is a Cartan atlas trivialising $P$, then if $s_\alpha : U_\alpha \to \ev{P}{U_\alpha}$ are the canonical sections, $s_\alpha(a) = [(a,e)]$, $(s_\alpha^\ast \omega)(v) = \omega(v,0) = \theta_\alpha(v)$. So $\theta_\alpha$ are the 'gauge fields' of the Cartan connection. Let $\Omega = d\omega + \frac{1}{2} \comm[\omega]{\omega}\in \Omega^2(p;\mf{g})$ denote the \bam{curvature} of the Cartan connection. Then $s_\alpha^\ast \Omega = d\theta_\alpha + \frac{1}{2} \comm[\theta_\alpha]{\theta_\alpha}$. Hence bundle automorphisms of $P$ (covering the identity) are the \bam{gauge symmetries} of the Cartan geometry. 

\begin{remark}
	$\omega$ parallelises $P$, just like $\vartheta_G$ parallelises $G$ in the Klein model. Given $X \in \mf{g}$ we get a vector field $\xi_X \in \mf{X}(P)$ defined by $\ev{\xi_X}{p} = \omega^{-1}_p(X)$, but unlike the case of $(G,\vartheta_G)$. this is not a Lie algebra morphism. This is despite that for $X \in \mf{h}, Y \in \mf{g}$ we do have $\comm[\xi_X]{\xi_Y} = \xi_{\comm[X]{Y}}$. The curvature $\omega$ is the obstruction to $X \mapsto \xi_X$ defining a Lie algebra morphism $\mf{g} \to \mf{X}(P)$. To see this, calculate
	\eq{
		\omega(\xi_{\comm[X]{Y}} - \omega(\comm[\xi_X]{\xi_Y}) &= \comm[X]{Y} + \pround{d\omega(\xi_X,\xi_Y) - \xi_X \omega(\xi_Y) + \xi_Y \omega(\xi_X)} \\
		&= \comm[X]{Y} + \pround{d\Omega(\xi_X,\xi_Y) - \comm[\omega(\xi_X)]{\omega(\xi_Y)}} + \xi_X Y - \xi_Y X  \\
		&= \comm[X]{Y} + \Omega(\xi_X,\xi_Y) - \comm[X]{Y} \\
		&= \Omega(\xi_X,\xi_Y)
	}
\end{remark}

We can now give the standard definition of a Cartan geometry modelled on a Klein geometry:

\begin{definition}
	A \bam{Cartan geometry} $(P,\omega)$ on $M$ modelled on $\faktor{G}{H}$ consists of the following:
	\begin{enumerate}
		\item a principal $H$-bundle $P \to M$
		\item $\omega \in \Omega^1(P;\mf{g})$ satisfying 
		\begin{enumerate}
			\item $\forall p \in P\, \omega_p : T_pP \to \mf{g}$ is a vector space isomorphism 
			\item $\forall h \in H, \, r_h^\ast \omega = \Ad(h^{-1}) \omega$
			\item $\forall X \in \mf{h}, \, \omega(\xi_X) = X$
		\end{enumerate}
	\end{enumerate}
\end{definition}

\begin{definition}
	Let $\Omega = d\omega + \frac{1}{2} \comm[\omega]{\omega}\in \Omega^2(P;\mf{g})$ be the curvature of $\omega$. Then projection $pr_{\faktor{\mf{g}}{\mf{h}}}\circ \Omega \in \Omega^2(P;\faktor{\mf{g}}{\mf{h}})$ is the \bam{torsion} of $\omega$. The Cartan geometry is $\bam{torsion free}$ if $\Omega \in \Omega^2(P;\mf{h})$
\end{definition}

\begin{lemma}
	Let $(P,\omega)$ be a Cartan geometry on $M$ modelled on $\faktor{G}{H}$. Let $\psi : P \to H$ be a smooth and $f : P \to  P $ be such that $f(p) = r_{\psi(p)}(p)$. Then $f^\ast \omega = \Ad(\psi^{-1})\omega + \psi^\ast \vartheta_H$ and $f^\ast \Omega = \Ad(\psi) \circ \Omega$.
\end{lemma}
\begin{proof}
	The expression for $f^\ast \Omega$ follows from that of $f^\ast \omega$. To calculate $f^\ast \omega$, we work relative to a Cartan gauge $(U,\theta)$ on $U \times H$. Then $f: U \times H \to U \times H$ by $f(a,h) = (a,h\psi(a,h))$ can be written as $f = (\id \circ pr_1, \mu \circ (pr_2 \times \psi))$. Hence if $(v,y) \in T_aU \times T_hH$
	\eq{
		f_\ast (v,y) &= (v,\mu_\ast(y,\psi_\ast(v,y))) \in T_a U \times T_{h \psi(a,h)} H \\
		\Rightarrow (f^\ast \omega)(v,y) &= \omega(v,\mu_\ast(y,\psi_\ast(v,y))) \\
		&= \Ad(h\psi(a,h))^{-1} \circ \theta(v) + \vartheta_H(\mu_\ast (y,\psi_\ast(v,y))) \\
		&= \Ad(\psi^{-1}) \circ \Ad(h^{-1}) \circ \theta(v) + (\mu^\ast \vartheta_H)(y,\psi_\ast(v,y)) \\
		&= \Ad(\psi^{-1}) \circ \Ad(h^{-1}) \circ \theta(v) + \Ad(\psi^{-1}) \circ \vartheta_H(y) + \vartheta_H(\psi_\ast(v,y)) \\
		&= \Ad(\psi^{-1}) \circ \psquare{\Ad(h^{-1}) \circ \theta(v) + \vartheta_H(y)} + (\psi^\ast \vartheta_H)(v,y) \\
		&= \psquare{\Ad(\psi^{-1}) \circ \omega + \psi^\ast \vartheta_H}(v,y)
	}
\end{proof}

\begin{corollary}
	$\Omega$ is horizontal, i.e. if either $u,v$ are tangent to the fibre, $\Omega(u,v) = 0$.
\end{corollary}
\begin{proof}
	Let $u,v \in T_pP$ and $v$ tangent to the fibre. Let $\psi : P \to H$ be any smooth map sending  $p \mapsto e$ s.t. $(\psi_\ast)_p v = -\omega_p(v) \in \mf{h}$. define $f: P \to P$ by $f(q) = q \cdot \psi(q)$. Then from the previous lemma we have that $p \in P$ 
	\eq{
		f^\ast \omega &= \Ad(\psi^{-1})\omega + \psi^\ast \vartheta_H = \omega + \psi^\ast \vartheta_H \\
		f^\ast \Omega &= \Omega
	}
	Hence 
	\eq{
		\omega_p(f_\ast v) &= \omega_p(v) + \vartheta_H(\psi_\ast v) = \omega_p(v) - \omega_p(v) = 0 \\
		\Rightarrow f_\ast v &= 0 \\
		\Rightarrow \Omega(u,v) &= \Omega(f_\ast u, f_\ast v) = \Omega(f_\ast u,0) = 0
	}
\end{proof}
It follows that $\Omega$ defines a 2-form on $\faktor{TP}{\ker \pi_\ast} \cong \pi^\ast  TM$. \\
Note that each fibre $F$ of $P$ is identified with $H$ up to left multiplication by some element of $H$. Since $\vartheta_H$ is left-invariant, it defines a "Maurer-Cartan" form $\vartheta_F$ on the fibre. The fact that $\forall X \in \mf{h}, \, \vartheta_F(\xi_X) = X$ shows that $\vartheta_F = \ev{\omega}{F}$. It then follows that $\Omega$ vanishes when restricted to any fibre. As such we can interpret a Cartan geometry $(P,\omega)$ as deforming $(G,\vartheta_G)$ in a way that fibrewise we still have $(H,\vartheta_H)$. \\
The tangent bundle of $\faktor{G}{H}$ is a vector bundle associated to $G \to \faktor{G}{H}$ via the linear isotropy representation $\Ad_{\faktor{\mf{g}}{\mf{h}}} : H \to GL(\faktor{\mf{g}}{\mf{h}}$ s.t. $T(\faktor{G}{H}) \cong G \times_h \faktor{\mf{g}}{\mf{h}}$. In a similar way, the tangent bundle of a Cartan geometry $(P,\omega)$ modelled on $\faktor{G}{H}$ is isomorphic to an associated vector bundle $P \times_H \faktor{\mf{g}}{\mf{h}}$. 

\begin{prop}
	Let $(P,\omega)$ be a Cartan geometry on $M$ modelled on $\faktor{G}{H}$. There is a canonical bundle isomorphism $\varphi : TM \overset{\cong}{\to} P \times_H \faktor{\mf{g}}{\mf{h}}$ such that $\forall p \in \pi^{-1}(x), \, \exists \varphi_p : T_xM \to \faktor{\mf{g}}{\mf{h}}$ a $H$-equivariant vector space isomorphism s.t. $\forall h \in H, \, \varphi_{p\cdot h} = \Ad(h^{-1})\circ \varphi_p$  
\end{prop}
\begin{proof}
	Consider the diagram 
	\begin{tkz}
		0 \arrow[r] & T_p(F_x) \arrow[r] \arrow[d,"\vartheta_H","\cong"'] & T_pP \arrow[r,"(\pi_\ast)_p"] \arrow[d,"\omega","\cong"'] & T_xM \arrow[r] \arrow[d,"\exists ! \varphi_p","\cong"',dashed,red] & 0 \\ 
		0 \arrow[r] & \mf{h} \arrow[r] & \mf{g} \arrow[r,"\rho"] & \faktor{\mf{g}}{\mf{h}} \arrow[r] & 0   
	\end{tkz}
	If $v \in T_x M$, we may write $v = (\pi_\ast)_p(u) = (\pi_\ast)_{ph}((r_h)_\ast u)$ for some $ u \in T_pP$. Thus 
	\eq{
		\varphi_{ph}(v) &= \varphi_{ph}((\pi_\ast)_{ph}((r_h)_\ast u)) \\
		&= \rho(\omega_{ph}((r_h)_\ast u)) \\
		&= \rho(\Ad(h)^{-1} \circ \omega_p(u)) \\
		&= \Ad(h)^{-1}(\varphi_p((\pi_\ast)_p u)) \\
		&= \Ad(h)^{-1} \varphi_p(v)
	}
	This allows us to define a bundle map 
	\eq{
		q : P \times \mf{g} &\to TM \\
		(p,X) &\mapsto (\pi(p),\varphi_p^{-1}(\rho(X))) 
	}
	Then 
	\eq{
		q(ph,\Ad(h)^{-1}X) &= (\pi(ph),\varphi^{-1}_{ph}(\rho(\Ad(h)^{-1}X))) \\
		&= (\pi(p), (\Ad(H) \varphi_{ph})^{-1} \rho(X)) \\
		&= (\pi(p), \varphi_p^{-1}(\rho(X))) \\
		&= q(p,X)
	}
	Hence $q$ induces $\bar{q} : P \times_H \faktor{\mf{g}}{\mf{h}} \to TM$, which covers the identity and is a linear iso on the fibres. 
\end{proof}

\begin{corollary}
	Let $(P,\omega)$ be a Cartan geometry on $M$ modelled on $\faktor{G}{H}$. Then vector fields $\xi \in \mf{X}(M)$ are in bijective correspondence with functions $\bar{\xi} : P \to \faktor{\mf{g}}{\mf{h}}$ such that $\forall p \in P, h \in H, \, \bar{\xi}(ph) = \Ad(h^{-1}) \circ \bar{\xi}(p)$ by 
	\eq{
		\xi \mapsto \bar{\xi} = \pbrace{p \in P :\mapsto \varphi_p(\xi_{\pi(p)}) \in \faktor{\mf{g}}{\mf{h}}}
	}
\end{corollary}

\begin{definition}
	The \bam{curvature function} $K : P \to \Hom(\Lambda^2 \faktor{\mf{g}}{\mf{h}},\mf{g})$ of a Cartan connection $\omega$ is defined by 
	\eq{
		\forall p \in P, \, \forall X,Y \in \mf{g}, \, K(p)(X,Y) \equiv \Omega_p(\omega_p^{-1}(X),\omega_p^{-1}(Y))
	}
\end{definition}

\begin{lemma}
	The curvature function is well defined and is $H$-equivariant, i.e. 
	\eq{
		\forall h \in H, \, K(ph)(X,Y) = \Ad(h^{-1}) K(p)(\Ad(h)X, \Ad(h)Y)
	}
\end{lemma}
\begin{proof}
	Fix $p \in P $ and let $\tilde{X} = X + W, \, \tilde{Y} = Y+Z$ for some $W,Z \in \mf{h}$. Then $\Omega_p(\omega_p^{-1}(\tilde{X}),\omega_p^{-1}(\tilde{Y})) = \Omega_p(\omega_p^{-1}(X),\omega_p^{-1}(Y))$ since $\omega_p^{-1}(Z),\omega_p^{-1}(W)$ are tangent to the fibres and $\Omega$ is horizontal. Therefore $K(p) \in \Hom(\Lambda^2 \faktor{\mf{g}}{\mf{h}},\mf{g})$. The equivariance follows from the equivariance of $\omega, \Omega$. 
\end{proof}

It follows that the curvature of a Cartan connection defines a \bam{curvature section} of the bundle\\
${P \times_H \Hom(\Lambda^2 \faktor{\mf{g}}{\mf{h}},\mf{g})}$. 

\begin{prop}
	A Cartan connection is torsion free iff the curvature function takes values in \\ $\Hom\pround{\Lambda^2 \faktor{\mf{g}}{\mf{h}},\mf{h}} \subset \Hom\pround{\Lambda^2 \faktor{\mf{g}}{\mf{h}},\mf{g}}$. 
\end{prop}

\begin{ex}
	Show that $K(p)(X,Y) = \comm[X]{Y} - \omega_p(\comm[\omega_p^{-1}X]{\omega_p^{-1}Y})$
\end{ex}

\begin{lemma}[Bianchi identity]
	$d\Omega = \comm[\Omega]{\omega}$
\end{lemma}
\begin{proof}
	Follows \textit{Mutatis Mutandis} as for Ehresmann connections. 
\end{proof}

Let $V$ be a vector space and $f: P \to V$ a function. A Cartan connection $\omega \in \Omega^1(P;\mf{g})$ defines a universal covariant derivative as follows: if $X \in \mf{g}$ and if $\xi_X = \omega^{-1}(X)$, then $\tilde{D}_X f \equiv \xi_X f$. Since this is linear in $X \in \mf{g}$, we get 
\eq{
	\tilde{D} : \Omega^0(P;V) &\to \Omega^0(P;V \otimes \mf{g}^\ast) \\
	f &\mapsto \tilde{D}f
}
where we define $\tilde{D}f$ by $(i_X)_\ast \tilde{D}f = \tilde{D}_X f$ for 
\eq{
	i_X : V \otimes \mf{g}^\ast &\to V \\
	v \otimes \eta &\mapsto \eta(X)v
}

\begin{definition}
	Let $\rho: H \to GL(V)$ be a representation. We define 
	\eq{
		\Omega^k(P;\rho) \equiv \pbrace{\alpha \in \Omega^k(P;V) \, | \, \forall h \in H, \, r_h^\ast \alpha = \rho(h^{-1}) \circ \alpha}
	}
	the \bam{k-forms on $P$ transforming according to $\rho$}. 
\end{definition}

\begin{prop}
	$\tilde{D} : \Omega^0(P;\rho) \to \Omega^1(P;\rho) \cong \Omega^0(P;\rho \otimes \Ad^\ast)$
\end{prop}
\begin{proof}
	Let $p \in P, \, X \in \mf{g}, \, f \in \Omega^0(P;\mf{g})$. Then 
	\eq{
		(i_X)_\ast (r_h^\ast (\tilde{D}f))(p) &= (i_X)_\ast (\tilde{D}f(ph)) \\
		&= (\tilde{D}_X f)(ph) \\
		&= \omega_{ph}^{-1}(X)f
	}
	Now $r_h^\ast \omega= \Ad(h^{-1}) \circ \omega \Rightarrow \omega_{ph} \circ (r_h)_\ast = \Ad(h^{-1}) \circ \omega_p \Rightarrow (r_{h^{-1}})_\ast \circ \omega_{ph}^{-1} = \omega_p^{-1} \circ \Ad(h)$ so
	\eq{
		(i_X)_\ast (r_h^\ast (\tilde{D}f))(p) &= \psquare{(r_h)_\ast \omega_p^{-1}(\Ad(h)X)}f
	}
	If $Y \in \mf{X}(P)$ we have 
	\eq{
		((r_h)_\ast Y)f = Y(r_h^\ast f) = Y(\rho(h^{-1}) \cdot f) = \rho(h^{-1}) Yf 
	}
	so taking $Y = \omega_p^{-1} (\Ad(h)X)$ yields 
	\eq{
		\psquare{(r_h)_\ast \omega_p^{-1}(\Ad(h)X)}f = \rho(h^{-1}) \omega_p^{-1}(\Ad(h)X) f = \rho(h^{-1}) \tilde{D}_{\Ad(h)X}f
	}
	and so 
	\eq{
		(i_X)_\ast (r_h^\ast \tilde{D}f)(p) = \rho(h^{-1}) \tilde{D}_{\Ad(h)X} f
	}
\end{proof}
Even if $(V,\rho)$ is irreducible, $(V \otimes \mf{g}^\ast, \rho \otimes \Ad^\ast)$ need not be. Decomposing $V \otimes \mf{g}^\ast$ into irreducibles decomposes $\tilde{D}$ and in this way we get 'famous' differential operators such as $\del,\bar{\del}, \divergence, \curl$.

\begin{lemma}
	Let $X \in \mf{h} $ and $f \in \Omega^0(P;\mf{g})$. Then $(i_X)_\ast \tilde{D}f = -\rho_\ast(X) f$ where $\rho_\ast : \mf{h} \to \End(V)$ is the LA hom induced by $\rho : H \to GL(V)$
\end{lemma}
\begin{proof}
	\eq{
		(i_X)_\ast (\tilde{D}f)(p) &= \omega_p^{-1}(X) f \\
		&= \ev{\frac{d}{dt} f(pe^{tX})}{t=0} \\
		&= \ev{\frac{d}{dt} \rho(e^{-tX}) f(p)}{t=0} \\
		&= -\rho_\ast(X) f(p)
	}
\end{proof}


%%%%%%%%%%%%%%%%%%%%%%%%%%%%%%%%%%%%%%%%%%%%%%%%%%%%%%%%%
\subsubsection{Reductive Cartan geometries}

Now assume that $(P,\omega)$ is reductive, s.t. $\mf{g} = \mf{h} \oplus \mf{m}$ with $\Ad(H)\mf{m} \subseteq \mf{m}$. Then the Cartan connection decomposes as $\omega = \omega_{\mf{h}} + \omega_{\mf{m}}$, so does the Cartan gauge $\theta = \theta_{\mf{h}}+ \theta_{\mf{m}}$, and so does $\tilde{D} = \tilde{D}_{\mf{h}} + \tilde{D}_{\mf{m}}$. If for $X \in \mf{h}, \, \tilde{D}_X f = - \rho(X) f$, then $\tilde{D}_{\mf{h}} = -\rho$. As we will see below, $\tilde{D}_{\mf{m}}$ defines a Kozul connection on any associated vector bundle $P \times_H V$. \\
It follows from the defining properties of a Cartan connection that $\omega_{\mf{h}} \in \Omega^1(P;\mf{h})$ is the connection one-form for an Ehresmann connection on the principal $H$-bundle $P \to M$. In contrast, the component $\omega_{\mf{m}} \in \Omega^1(P;\mf{m})$ satisfies 
\begin{enumerate}
	\item It is horizontal, i.e. $\forall X \in \mf{h}, \, \omega_{\mf{m}}(\xi_X) = 0$
	\item $r_h^\ast \omega_{\mf{m}} = \Ad(h^{-1}) \circ \omega_{\mf{m}}$
\end{enumerate}
The above two mean that $\omega_{\mf{m}}$ induces a one-form on $M$ with values in the associated vector bundle $P \times_H \mf{m}$, which is isomorphic to $TM$. Thus $\omega_{\mf{m}}$ is a \bam{soldering form} on $P$. \\
As $\omega$ splits, so does $\Omega = \Omega_{\mf{h}} + \Omega_{\mf{m}}$ where the structure equation $\Omega = d\omega + \frac{1}{2}\comm[\omega]{\omega}$ gives 
\eq{
	\Omega_{\mf{h}} &= d\omega_{\mf{h}} + \frac{1}{2} \comm[\omega_{\mf{h}}]{\omega_{\mf{h}}} + \frac{1}{2} \comm[\omega_{\mf{m}}]{\omega_{\mf{m}}}_{\mf{h}} \\
	\Omega_{\mf{m}} &= d\omega_{\mf{m}} +  \comm[\omega_{\mf{h}}]{\omega_{\mf{m}}} + \frac{1}{2} \comm[\omega_{\mf{m}}]{\omega_{\mf{m}}}_{\mf{m}}
}

Therefore, the $\mf{h}$-component of the curvature of the Cartan connection is not necessarily the curvature of the Ehresmann connection, but receives a correction from the soldering form:
\eq{
	\Omega^{\text{Cartan}}_{\mf{h}} = \Omega^{\text{Ehresmann}} + \frac{1}{2} \comm[\omega_{\mf{m}}]{\omega_{\mf{m}}}_{\mf{h}}
}
whereas the torsion of the Cartan connection is not necessarily the torsion of the affine connection defined by $\omega_{\mf{h}}$:
\eq{
	\Theta^{\text{Cartan}} = \Omega^{\text{Cartan}}_{\mf{m}}  = \Theta + \frac{1}{2} \comm[\omega_{\mf{m}}]{\omega_{\mf{m}}}_{\mf{m}}
}
Let's now consider the universal covariant derivative $\tilde{D} = \tilde{D}_{\mf{h}} + \tilde{D}_{\mf{m}}$. The $\mf{m}$-component defines a Kozul connection on any associated vector bundle $E \equiv P \times_H V$ for $(V, \rho)$ a representation of $H$. Indeed, let $\psi : \Gamma(E) \to \Omega^0(P;\rho)$ be the $C^\infty(M)$-modules isomorphism. We define $\nabla_\zeta : \Gamma(E) \to \Gamma(E)$ by the commutativity of the following square:
\begin{tkz}
	\Gamma(E) \arrow[r,"\nabla_\zeta"] \arrow[d,"\psi"',"\cong"] & \Gamma(E) \arrow[d,"\cong"',"\psi"] \\ \Omega^0(P;\rho) \arrow[r,"\tilde{\zeta}"'] & \Omega^0(p;\rho)
\end{tkz}
i.e. $\psi(\nabla_\zeta s) = \tilde{\zeta}\psi(s)$, where $\tilde{\zeta}$ is the \bam{horizontal lift} of $\zeta$, i.e the  unique\footnote{Is it clear why this vector field is unique} vector field on $P$ s.t. $(\pi_\ast)_p \tilde{\zeta} = \zeta_{\pi(p)}$ and $\omega_{\mf{h}} (\tilde{\zeta})=0$

\begin{prop}
	$\nabla$ defines a Kozul connection on $E$
\end{prop}
\begin{proof}
	$\nabla_\zeta$ is $\mbb{R}$-linear and if $f \in C^\infty(M)$, $(\pi^\ast f)\tilde{\zeta}$ is the horizontal lift of $f\zeta$, so we have $\nabla_{f \zeta}s = f\nabla_\zeta s$. Finally, to get that $\nabla$ is a derivation, see 
	\eq{
		\psi(\nabla_\zeta(fs)) &= \tilde{\zeta} \psi(fs) = \tilde{\zeta}(\pi^\ast f \psi(s)) = \tilde{\zeta} (\pi^\ast f) \psi(s) + (\pi^\ast f) \tilde{\zeta} \psi(s) \\
		&= \pi^\ast (\zeta f) \psi(s) + (\pi^\ast f) \psi(\nabla_\zeta s) = \psi((\zeta f)s) + \psi(f \nabla_\zeta s) \\
		&= \psi((\zeta f)s + f \nabla_\zeta s)
	}
\end{proof}

\begin{prop}
	Let $(U,\theta)$ be a gauge for a reductive Cartan geometry, $\sigma : U \to \ev{P}{U}$ the section such that $\theta = \sigma^\ast \omega$, $\zeta \in \mf{X}(U)$, and $\phi = \sigma^\ast \Phi$ where $\Phi \in \Omega^0(P;\rho)$. Then 
	\eq{
		\nabla_\zeta \phi \equiv \zeta(\phi) -\rho_\ast(\theta_{\mf{h}}(\zeta))\phi
	}
	is the expression of the covariant derivative of $\Phi$ in the gauge $(U,\theta)$. 
\end{prop}

%%%%%%%%%%%%%%%%%%%%%%%%%%%%%%%%%%%%%%%%%%%%%%%%%%%%%%%%%
\subsubsection{Special geometries}

We may define 'special geometries' via curvature constraints 

\begin{lemma}
	Let $V \subset \mf{g}$ be the vector subspace spanned by the values of the curvature form $\Omega$. Then $V$ is a $H$-submodule
\end{lemma}
\begin{proof}
	Let $v = \Omega_p(\xi_p,\eta_p)$. Then 
	\eq{
		\Ad(h^{-1})v &= \Ad(h^{-1})(\Omega_p(\xi_p,\eta_p)) \\
		&= (r_h^\ast \Omega_p)(\xi_p,\eta_p) \\
		&= \Omega_{ph}((r_h)_\ast \xi_p,(r_h)_\ast \eta_p)
	}
	which is a value of $\Omega$
\end{proof}

In particular if $V \subset \mf{h}$ is s.t. the Cartan geometry is torsion-free, then $V$ is is an ideal. If the geometry is torsion-free and the action of $H$ on $\mf{h}$ is irreducible, there are no special geometries arising from $\mf{g}$-curvature conditions. However, the $H$-modules $\Hom\pround{\Lambda^2 \pround{\faktor{\mf{g}}{\mf{h}}},\mf{h}}$ need not be irreducible and we can define special geometries by damnding that the curvature function $K : P \to \Hom\pround{\Lambda^2 \pround{\faktor{\mf{g}}{\mf{h}}},\mf{h}}$ takes values in a $H$-submodule. \\
If $H$ is compact, then $\Hom\pround{\Lambda^2 \pround{\faktor{\mf{g}}{\mf{h}}},\mf{h}}$ is fully reducible
\begin{example}
	$\mf{g} = \mf{so}_n \ltimes \mbb{R}^n$ and $\mf{h} = \mf{so}_n$. Have $\Hom\pround{\Lambda^2 \pround{\faktor{\mf{g}}{\mf{h}}},\mf{h}} = \Hom(\Lambda^2 \mbb{R}^n,\mf{so}_n)$. 
\end{example}
The subspace corresponding to those curvature functions obeying the (algebraic) Bianchi identity breaks up into three submodules: scalar, trace-free Ricci, and Weyl. \\
Cartan connections are special types of Ehresmann connections. Let $P \to M$, $G \to \faktor{G}{H}$, be principal $H$-bundles. There is an associated fibre bundle $Q = P\times_H G$ where $H$ acts on $G$ by left multiplication. This is a (right) principal $G$-bundle and $M$, and we have a natural inclusion $P \subset Q$ sending $p \mapsto (p,e)$. An Ehresmann connection on $Q$ is a $\mf{g}$-valued one-form and its restriction to $P$ gives a candidate for a Cartan connection on $P$.

\begin{theorem}
	Let $\faktor{G}{H}$ be a Klein geometry and let $P,Q$ be principal $H,G$-bundles respectively over a manifold $M$. Assume that $\dim P = \dim G $ and $\varphi : P \to Q$ is a $H$-bundle map. Then there is a bijection of sets 
	\eq{
		\pbrace{\text{Ehresmann connections on $Q$, kernels not $\varphi_\ast(TP)$}} \overset{\varphi^\ast}{\to} \pbrace{\text{Cartan connections on $P$}}
	}
\end{theorem}
\begin{proof}
	Let $\varpi \in \Omega^1(Q;\mf{g})$ be an Ehresmann connection s.t. $\varpi_\ast(TP) \cap \ker \varpi = 0$. It follows that $\omega = \varphi^\ast \varpi \in \Omega^1(p;\mf{g})$ with zero kernel. Since $\dim P = \dim \mf{g}$, $\omega_p : T_pP \to \mf{g}$ is injective and so an isomorphism. \\
	Since $\varphi : P \to Q$ is a $H$-bundle map, $\forall X \in \mf{h}$ the vector fields $\xi_X$ on $P$ and $\zeta_X$ on $Q$ are $\varphi$-related : i.e
	\eq{
		\forall p \in P, \, (\varphi_\ast)_p \xi_X(p) = \zeta_X(\varphi(p))
	}
	Also, 
	\eq{
		r_h^\ast \omega &= r_h^\ast \varphi^\ast \varpi = \varphi^\ast r_h^\ast \varpi = \varphi^\ast (\Ad(h^{-1}) \circ \varpi) = \Ad(h^{-1}) \circ \varphi^\ast \varpi = \Ad(h^{-1}) \circ \omega
	}
	so $\omega$ is a Cartan connection. Next we define a correspondence 
	\eq{
		\pbrace{\text{Cartan connections on $P$}} \overset{j}{\to} \pbrace{\parbox{3in}{Ehresmann connections on $Q$, kernels not $\varphi_\ast(TP)$}} 
	}
	Given a Cartan connection $\omega$ on $P$ we extend it to a form $\varpi = j(\omega)$ on $P\times G$ by 
	\eq{
		\varpi_{(p,g)} = \Ad(g^{-1}) \circ  \pi^\ast_P \omega_p + \ev{\pi^\ast_G \vartheta_G}{g} 
	}
	where $\pi_{P/G}:P\times G \to P/G$ are the canonical projections. We notice that $\forall X \in \mf{g}, \, \varpi(0,X^L) = X$. Also, if $i: P \to P\times G$ is the injection $p \mapsto (p,e)$ then $i^\ast \varpi = \omega$. In particular, $\varpi$ does not vanish on $T(P \times \pbrace{e})$. Let $\gamma \in G$ and consider $\id \times R_\gamma: P\times G \to P \times G$:
	\eq{
		(\id \times R_\gamma)^\ast \varpi_{(p,g\gamma)} &= \varpi_{(p,g\gamma)} \circ (\id \times R_\gamma)_\ast \\
		&= \pround{\Ad(g\gamma)^{-1} \circ  \pi^\ast_P \omega_p +\pi^\ast_G \vartheta_G} \circ (\id \times R_\gamma)_\ast \\
		&= \Ad(g \gamma)^{-1} \circ \omega \circ (\pi_P)_\ast \circ (\id \times R_\gamma)_\ast + \vartheta_G \circ (\pi_G)_\ast  \circ (\id \times R_\gamma)_\ast \\
		&= \Ad(g\gamma)^{-1} \circ \omega \circ (\pi_P)_\ast + \vartheta_G \circ (R_\gamma)_\ast \circ (\pi_G)_\ast \\
		&= \Ad(\gamma)^{-1} \pround{\Ad(g)^{-1} \circ \pi_P^\ast \omega + \pi_G^\ast \vartheta_G} \\
		&= \Ad(\gamma)^{-1} \circ \varpi_{(p,g)}
	}
	We now check that $\varpi$ is basic for $P \times G \to P\times_H G$ which means that it is both horizontal and 'invariant'. The latter condition requires that for $\alpha_h : P \times G \to P \times G, \, (p,g) \mapsto (ph h^{-1}g)$, we have $\alpha_h^\ast \varpi = \varpi$. We calculate 
	\eq{
		(\alpha_h^\ast \varpi)_{(p,g)} &= \varpi_{(ph,h^{-1}g)} \circ (\alpha_h)_\ast \\
		&= \Ad(h^{-1}g)^{-1} \pi_P^\ast \omega \circ (\alpha_h)_\ast + \pi_G^\ast \vartheta_G \circ (\alpha_h)_\ast \\
		&= \Ad(h^{-1}g)^{-1} \omega \circ (\pi_P)_\ast \circ (\alpha_h)_\ast + + \vartheta_G \circ (\pi_G)_\ast \circ (\alpha_h)_\ast \\
		&= \Ad(g^{-1}) \circ \Ad(h) \circ \omega \circ (R_h)_\ast \circ (\pi_P)_\ast + \vartheta_G \circ (L_{h^{-1}})_\ast \circ (\pi_G)_\ast \\
		&= \Ad(g^{-1}) \circ \pi_P^\ast \omega + \pi_G^\ast \vartheta_G \quad (\text{as $R_h^\ast \omega = \Ad(h)^{-1}\omega$ and $\vartheta_G$ is LI}) \\
		&= \varpi_{(p,g)}
	}
	To show $\varpi$ is horizontal, let $X \in \mf{h}$ and $\xi_X \in \mf{X}(P \times G)$ corresponding to the right $H$-action on $P \times G$:
	\eq{
		P \times G \times H &\to P \times G \\
		(p,g,h) &\mapsto (ph,h^{-1}g) = \pround{(\mu_P \times \mu_G) \circ (\id \times \id \times \i \times \id) \circ (\id \times \Delta \times \id) \circ \varrho)}(p,g,h)
	}
	where we have 
	\eq{
		\varrho : P \times G \times H &\to P \times H \times G \\
		(p,g,h) &\mapsto (p,h,g) \\
		& \phantom{=} \\
		\id \times \Delta \times \id : P \times G \times H &\to P \times H \times H \times G \\
		(p,h,g) & \mapsto (p,h,h,g) \\
		& \phantom{=} \\
		\id \times \id \times \i \times \id : P \times H \times H \times G &\to P \times H \times H \times G \\
		(p,h,h,g) &\mapsto (p,h,h^{-1},g) \\
		&\phantom{=} \\
		\mu_P \times \mu_G : P \times H \times H \times G &\to P \times G \\
		(p,h,h^{-1},g) &\mapsto (ph,h^{-1}g)
	}
	Then 
	\eq{
		(\xi_X)_{(p,g)} &= \pround{(\mu_P \times \mu_G) \circ (\id \times \id \times \i \times \id) \circ (\id \times \Delta \times \id) \circ \varrho)}_{\ast,(p,g,e)} (0,0,X) \\
		&= (\mu_P \times \mu_G)_\ast \circ (\id \times \id \times \i \times \id)_\ast \circ (\id \times \Delta \times \id)_{\ast,(p,e,g)}(0,X,0) \\
		&= (\mu_P \times \mu_G)_\ast \circ (\id \times \id \times \i \times \id)_{\ast,(p,e,e,g)}(0,X,X,0) \\
		&= (\mu_P \times \mu_G)_{\ast,(p,e,e,g)}(0,X,-X,0) \\
		&= (\mu_P)_{\ast,(p,e)}(0,X),(\mu_G)_{\ast,(e,g)}(-X,0) \\
		&= (\omega_p^{-1}(X),-(\vartheta_G)^{-1}_g(\Ad(g^{-1})X)) \\
		\Rightarrow \varpi_{(p,g)}(\xi_X) &= \varpi_{(p,g)}(\omega_p^{-1}(X),-(\vartheta_G)^{-1}_g(\Ad(g^{-1})X)) \\
		&= (\Ad(g^{-1})\cdot (\pi_P^\ast \circ \omega) + \pi_G^\ast \vartheta_G)(\omega_p^{-1}(X),-(\vartheta_G)^{-1}_g(\Ad(g^{-1})X)) \\
		&= \Ad(g^{-1})X = \Ad(g^{-1})X = 0
	}
	Therefore $\varpi$ descends to $\varpi \in \Omega^1(P\times_H G,\mf{g})$ and satisfies the properties of an Ehresmann connection which in addition obeys $\ker \varpi \cap \varphi_\ast(TP)=0 $. \\
	Finally, we need to show that $\varphi^\ast$ and $j$ are mutual inverses:
	\eq{
		\varphi^\ast (j(\omega_p)) = \varphi^\ast \varpi_{(p,e)} &= \Ad(e)^{-1}\circ \varphi^\ast \pi_P^\ast \omega_p + \varphi^\ast \pi_G^\ast {\vartheta_G}_e \\
		&= (\pi_P \circ \varphi)^\ast \omega_p + 0 \quad (\text{since $\pi_G \circ \varphi$ is constant})
		&= \omega_p
	}
	shows that $\varphi^\ast \circ j = \id$. To do the other direction, it suffices to show $\varphi^\ast$ is injective. Now if $\varphi^\ast \varpi = \varphi^\ast \varpi_2$ then $\varpi_1,\varpi_2$ agree on the image $\varphi_\ast(TP)$ and hence on all the right translations. But $\varpi_1, \varpi_2$ agree on $\xi_X$ and these two kinds of vectors span $TQ$
\end{proof}

%%%%%%%%%%%%%%%%%%%%%%%%%%%%%%%%%%%%%%%%%%%%%%%%%%%%%%%%
%%%%%%%%%%%%%%%%%%%%%%%%%%%%%%%%%%%%%%%%%%%%%%%%%%%%%%%%
%%%%%%%%%%%%%%%%%%%%%%%%%%%%%%%%%%%%%%%%%%%%%%%%%%%%%%%%
%%%%%%%%%%%%%%%%%%%%%%%%%%%%%%%%%%%%%%%%%%%%%%%%%%%%%%%%
\part{Geometry of Gauge Fields}

%%%%%%%%%%%%%%%%%%%%%%%%%%%%%%%%%%%%%%%%%%%%%%%%%%%%%%%%
%%%%%%%%%%%%%%%%%%%%%%%%%%%%%%%%%%%%%%%%%%%%%%%%%%%%%%%%
\section{Review of Manifold Theory}
You will definitely need this knowledge to carry on into the connection material. Some possible reading if this is not sufficient (as is possible)
\begin{itemize}
	\item Taubes, "Differential geometry"
	\item Warner, "Foundations of differential manifolds and Lie groups
	\item Lee, "Introduction to smooth manifolds"
\end{itemize}
%%%%%%%%%%%%%%%%%%%%%%%%%%%%%%%%%%%%%%%%%%%%%%%%%%%%%%%%
\subsection{Initial definitions}
%%%%%%%%%%%%%%%%%%%%%%%%%%%%%%%%%%%%%%%%%%%%%%%%%%%%%%%%
\subsubsection{The Starting Point}
\begin{definition}
A \bam{topological manifold} of dimension $n$, $M$, is a topological space that is Hausdorff, second countable, paracompact, and locally homeomorphic to $\mbb{R}^n$. i.e. $\forall m \in M, \, \exists U \ni m$ such that $U$ open and homeomorphic to $\mbb{R}^n$. 
\end{definition}
Recall that topological manifolds can have further properties such as \bam{compact} or \bam{orientable} (Note that orientability is a well defined topological property, but for manifolds not general topological spaces). \\
Now we want to be able to do calculus on manifolds, but the standard definition of derivative of a real valued function $f$ does not immediately parse. Namely the concept of $f(x+\eps)$ leads us to want a vector space structure, and so we are led to think locally, where we know our space looks like $\mbb{R}^n$, which is a vector space. Now we have a new problem, that our answer for the derivative may be ambiguous depending on which homeomorphism we chose 
\begin{example}
This manifests itself in dimension 1. Let $M = \mbb{R}$, but take the homeomorphisms $\id$ or $x^3$ in a neighbourhood of $m=0$. Take the function $f(x) = x^\frac{1}{3}$. If we ask whether this function is differentiable, that depends on which homeomorphism we are using. 
\end{example}
To solve this issue, we require extra information, and this turns a topological manifold into a smooth manifold:

\begin{definition}
A \bam{smooth manifold} is a topological manifold with a preferred set of homeomorphisms $\pbrace{\varphi_\alpha: U_\alpha \overset{\cong}{\to} \varphi_\alpha(U_\alpha) \subset \mbb{R}^n}$ such that 
\begin{itemize}
    \item $\cup_\alpha U_\alpha = M$
    \item $\ev{\varphi_\beta \circ \varphi_\alpha^{-1}}{\varphi_\alpha(U_\alpha \cap U_\beta)}$ is a diffeomorphism
\end{itemize}
This is called a \bam{smooth atlas}, and the $\varphi_\alpha$ are called charts. 
\end{definition}

%%%%%%%%%%%%%%%%%%%%%%%%%%%%%%%%%%%%%%%%%%%%%%%%%%%%%%%%
\subsubsection{Other viewpoints}
We can define $C^0(M)$, the $\mbb{R}$-algebra of real valued continuous functions on a topological manifolds naturally, but we require the smooth structure of transition functions in order to define $C^\infty(M)$, the subalgebra of smooth functions. An atlas is said to be compatible with a local homeomorphism $\varphi$ if $\pbrace{\varphi} \cup \pbrace{\varphi_\alpha}$ is still an atlas. This allows us to talk of \bam{maximal atlases}, which are those which contain all possible compatible charts. Each atlas has a unique compatible maximal atlas, and the corresponding equivalence classes of atlases are called \bam{smooth structure}

\begin{example}
Consider again $M = \mbb{R}$. The atlases given by $id$ and $x^3$ are incompatible as atlases, and so correspond to different smooth structures. However, the homeormorphism $x^\frac{1}{3}:M \to M$ becomes a diffeomorphism wrt the corresponding smooth structure.
\end{example}
We can, from this example, begin to ask the question of what smooth structures are possible up to diffeomorphism. This is a question answered by learning from gauge theory, which leads to the concept of exotic $\mbb{R}^4$. 

%%%%%%%%%%%%%%%%%%%%%%%%%%%%%%%%%%%%%%%%%%%%%%%%%%%%%%%%
\subsection{Objects on manifolds}

\subsubsection{Vector Fields}
We can take different viewpoints towards what is a vector field. Suppose we have a chart $\varphi_\alpha$ with coordinates $x^1, \dots, x^n$. \\
A physicist approach would be to define a vector field as such:
\begin{definition}[vector field - physicist way]
A \bam{vector field} is an assignment of $n$ smooth functions $(X^1, \dots, X^n)$ to each chart s.t when switching to a chart with coordinates $\tilde{x}^i$ the smooth functions $\tilde{X}^i$ are given by 
\eq{
\tilde{X}^i = \pd[\tilde{x}^i]{x^j} X^j
}
\end{definition}
A mathematical definition would be
\begin{definition}[Vector field - mathematician way]
 A \bam{vector field} is a derivation of $C^\infty(M)$. Recall a derivation is a map $X:C^\infty(M) \to C^\infty(M)$ s.t. $X$ is $\mbb{R}$ linear and that $X(fg) = X(f)g + fX(g)$. 
 \end{definition}
The connection between the two definitions is that $X^i \pd{x^i}$ is a derivation of $C^\infty(U_\alpha)$

$\forall m \in M$, we have a \bam{tangent space} to $M$ at $m$, written $T_m M$. This requires a smooth structure in order to define. Given the tangent spaces we can define 
\eq{
TM = \bigsqcup_m T_m M
}
the \bam{tangent bundle}. It is a manifold in its own right, and moreover it comes with the projection 
\eq{
\pi : TM & \to M \\
v \in T_m M &\mapsto m
}
With this we can then say
\begin{definition}
A \bam{vector field} is a section of $\pi$, i.e. a smooth map $X:M \to TM$ s.t. $\pi \circ X = id_M$. 
\end{definition}


Given a vector field $X \in \mf{X}(M)$, $\exists$ a 1-parameter group of diffeomorphisms $\phi_t^X:M \to M$ for $t \in \mbb{R}$ (we will sometimes denote this as $\exp(tX)$) such that 
\begin{itemize}
	\item $\phi_t^X \circ \phi_s^X = \phi_{t+s}^X$
	\item $\ev{\frac{d}{dt} \phi_t^X(m)}{t=0} = X_m$
\end{itemize}

If $M$ is not compact, $\phi_t^X$ will only exists 'locally'. If they do exists $\forall t$ we say that $X$ is \bam{complete}. 

\begin{example}
	Take $M = (0,1) \times \mbb{R}$, we can have $X = \del_x$ a horizontal vector field and this is not complete. 
\end{example}

\begin{definition}
	If $X$ is a (complete) vector field, the \bam{Lie derivative} of anything which can pullback under the diffeomorphisms $\phi_t^X$ by 
	\eq{
		\mc{L}_X T= \ev{\frac{d}{dt} (\phi_t^X)^\ast T }{t=0}  
	}
\end{definition}

\begin{example}
	Some simple cases are 
	\begin{enumerate}
		\item for $f \in C^\infty(M)$, $\mc{L}_X f = X(f)$
		\item for $\alpha \in \Omega^\ast(M)$, $\mc{L}_X \alpha = i_X d\alpha + d(i_X \alpha)$
		\item for $Y \in \mf{X}(M)$, $\mc{L}_X Y = \comm[X]{Y}$. This bracket is the \bam{Lie bracket}
	\end{enumerate}
\end{example}
\begin{lemma}
	We can calculate $\phi^{\comm[X]{Y}}_t = \phi^X_{-\sqrt{t}} \circ \phi^Y_{-\sqrt{t}} \circ \phi^X_{\sqrt{t}} \circ \phi^Y_{\sqrt{t}}$
\end{lemma}

%%%%%%%%%%%%%%%%%%%%%%%%%%%%%%%%%%%%%%%%%%%%%%%%%%%

\subsubsection{One forms}
Each tangent space is a vecotr space, and so has dual vector space $T_m^\ast M$. We can hence similarly define the \bam{cotangent bundle} 
\eq{
T^\ast M = \bigsqcup_m T_m^\ast M
}

\begin{definition}
A \bam{one form} is a section of $T^\ast M \to M$
\end{definition}
we can also apply operations from multilinear algebra, e.g. $TM \oplus TM, \dots$

\begin{definition}[Tensor field - physicist way]
A \bam{tensor} is a section of a bundle of the form $(p,q)$
\eq{
\underbrace{TM \otimes \dots \otimes TM}_{\times p} \otimes \underbrace{T^\ast M \otimes \dots \otimes T^\ast M}_{\times q} \to M 
}
\end{definition}

\begin{definition}
A \bam{k-form} is an element of 
\eq{
\Omega^k(M) = \Gamma(\Lambda^k(T^\ast M))
}
where $\Gamma$ is the space of sections. These are $\mbb{R}$-vector spaces, and further $C^\infty(M)$-modules. 
\end{definition}

%%%%%%%%%%%%%%%%%%%%%%%%%%%%%%%%%%%%%%%%%%%%%%%%%%%%%%%%
\subsubsection{Exterior algebra}
We can define a \bam{wedge operator} 
\eq{
\wedge : \Omega^k(M) \times \Omega^l(M) &\to \Omega^{k+l}(M) \\
(\alpha,\beta) &\mapsto \alpha \wedge \beta
}
We can thus get the graded algebra 
\eq{
\Omega^\ast(M) = \bigoplus_{i=0}^n \Omega^i(M)
}
We can further define an \bam{exterior derivative} by 
\eq{
d: \Omega^k(M) &\to \Omega^{k+1}(M)
}
Whereas the wedge operation was done pointwise, the exterior derivative requires local information. It has the property
\eq{
d^2 = 0
}
This gives a natural motivation to 
\begin{definition}
The \bam{i\textsuperscript{th} De Rham Cohomology class} of $M$ is 
\eq{
H_{dR}^i(M) = \faktor{\ker(d: \Omega^i(M) \to \Omega^{i+1}(M))}{\image(d: \Omega^{i-1}(M) \to \Omega^i(M))}
}
\end{definition}
Often (e.g when $M$ is compact) this is finite dimensional. Note that, despite the fact that we needed smooth structure in order to define this, it is in fact a topological property. 

%%%%%%%%%%%%%%%%%%%%%%%%%%%%%%%%%%%%%%%%%%%%%%%%%%%%%%%%
\subsubsection{The Hodge star operator}
The Hodge star is an operator fraught with minus sign problems. Check \href{https://ncatlab.org/nlab/show/Hodge+star+operator}{nlab} to try be consistent. 

\begin{definition}
	A \bam{volume form} of a differentiable manifold $M$ is a top-degree form (i.e. form of degree $\dim M$). 
\end{definition}

\begin{definition}
	An \bam{oriented} manifold is one equipped with a nowhere-vanishing volume form. 
\end{definition}

\begin{lemma}
	Every Riemannian manifold $(M,g)$ has a canonical choice of volume form given in local coordinates $x^1, \dots, x^n$ by 
	\eq{
		\omega_{vol} = \sqrt{\abs{g}} dx^1 \wedge \dots \wedge dx^n
	} 
\end{lemma}

\begin{lemma}
	If $V$ is a vector space with bilinear $\pangle{\cdot, \cdot}$ then $\bigwedge^k V$ inherits a bilinear by the Grammian determinant. 
	\eq{
		\pangle{\alpha_1 \wedge \dots \wedge \alpha_k, \beta_1 \wedge \dots \wedge \beta_k} = \det \pround{\pangle{\alpha_i,\beta_j}_{i,j=1}^k}	
	}
\end{lemma}

\begin{definition}
	If $M$ is a $n$-dimensional Riemannian manifold, the \bam{Hodge star} is the operator $\star : \Omega^r(M) \to \Omega^{n-r}(M)$ given by 
	\eq{
		\alpha \wedge (\star \beta) = \pangle{\alpha,\beta} \omega_{vol}
	}
	for $\alpha,\beta \in \Omega^r(M)$, where $\pangle{\cdot,\cdot}$ is the Euclidean structure on $\Omega^r(M)$. 
\end{definition}

\begin{lemma}
	This property defines $\star$ completely. 
\end{lemma}
\begin{proof}
	\hl{exercise}
\end{proof}

\begin{prop}
	If $M$ has determinantal sign $s$ of the inner product, and $\alpha \in \Omega^r(M)$ then 
	\eq{
		(\star)^2 \alpha = (-1)^{r(n-r)}s \alpha	
	}
\end{prop}
\begin{corollary}
	On Euclidean $\mbb{R}^4$, $(\star)^2 = 1$, so $\star$ has eigenvalues $\pm 1$. 
\end{corollary}
\begin{definition}
	Decompose, as eigenspaces of $\star$, $\Omega^2(M) = \Omega^2_+(M) \oplus \Omega^2_-(M)$. We call these the \bam{self-dual} and \bam{anti-self-dual} parts.
\end{definition}

\begin{lemma}
	Elements of $\Omega^2_+(M), \, \Omega^2_-(M)$ are orthogonal. 
\end{lemma}
\begin{proof}
	We have 
	\eq{
		\pangle{\omega_+, \omega_-}\omega_{vol} &= \omega_+ \wedge \star \omega_- = -\omega_+ \wedge \omega_- \\
		\pangle{\omega_-,\omega_+} \omega_{vol} &= \omega_- \wedge \star \omega_+ = \omega_+ \wedge \omega_-
	}
	but $\pangle{\omega_+, \omega_-} = \pangle{\omega_-, \omega_+}$ so must be 0. 
\end{proof}

\begin{prop}
	Take coordinates $\pbrace{x^\mu}$ on an $n$-dimensional manifold $M$. Let $g$ be a metric with $g_{\mu\nu} = g\pround{\pd{x^\mu}, \pd{x^\nu}}$. Then  
	\eq{
		\star (dx^{\mu_1} \wedge \dots \wedge dx^{\mu_p}) &= \frac{\sqrt{\abs{ g}}}{(n-p)!} g^{\mu_1 \nu_1} \dots g^{\mu_p \nu_p} \epsilon_{\nu_1 \dots \nu_n} dx^{\nu_{p+1}} \wedge \dots \wedge dx^{\nu_n} \\
		&=  \frac{1}{(n-p)!}\epsilon\indices{^{\mu_1}^\dots^{\mu_p}_{\mu_{p+1}}_\dots_{\mu_n}} dx^{\mu_{p+1}} \wedge \dots \wedge dx^{\mu_n} 
	}
Here we are using the Levi-Civita tensor, which means that, for example, in Minkowski 
\eq{
\eps_{\mu_1 \dots \mu_p \mu_{p+1} \dots \mu_n} \eps^{\nu_1 \dots \nu_p \mu_{p+1} \dots \mu_n} = -(n-p)! \delta^{\nu_1 \dots \nu_p}_{\mu_1 \dots \mu_p}
}
\end{prop}

%%%%%%%%%%%%%%%%%%%%%%%%%%%%%%%%%%%%%%%%%%%%%%%%%%%%%%%%
\subsection{Lie Theory}
\subsubsection{Lie Group Basics}

\begin{definition}
A \bam{Lie group} $G$ is a group object in the category of manifolds, i.e. 
\begin{itemize}
    \item $G$ is a manifold
    \item $\mu : G \times G \to G$ multiplication is smooth 
    \item $i : G \to G$ inverse is smooth
\end{itemize}
s.t. multiplication is associative, inverses commute, etc. We may write the Lie group as a \bam{pointed} manifold $(G,e)$ to make the identity explicit. 
\end{definition}

\begin{example}
We have $GL_n(\mbb{R})\subset M_n(\mbb{R}) \cong \mbb{R}^{n^2}$
\end{example}

\begin{lemma}
	$GL_n(k)\subset M_n(k)$ is open if $k=\mbb{R},\mbb{C}$.
\end{lemma}
\begin{proof}
	Defining $\det : M_n(k) \to k$ we have $GL_n(k) = \det^{-1}(k\setminus 0)$.
\end{proof}

\begin{remark}
	We cannot do the above proof for quaternions as there is no quaternionic determinant. 
\end{remark}

\begin{example}
	We can define $Sp(1) = \pbrace{q \in \mbb{H} \, | \, \bar{q}q = 1}$, where $\mbb{H}$ are the \bam{quaternions}. 
\end{example}

\begin{theorem}[Cartan]
Any closed subgroup of a Lie group is a Lie group itself. 
\end{theorem}

\begin{example}
A non-example of the above is given by 
\eq{
H = \pbrace{(e^{it},e^{iat}) \, | \, t \in \mbb{R}, \, a \in \mbb{R}\setminus \mbb{Q}} \subset U(1) \times U(1)
}
\end{example}

\begin{example}
$SL_n(\mbb{R}) = \det^{-1}(1) \subset GL_n(\mbb{R})$. By differentiable manifold theory we know $\dim SL_n(\mbb{R}) = n^2-1$. 
\end{example}



\begin{remark}
All of these are matrix groups. Not all Lie groups are matrix groups but most are. 
\end{remark}

%%%%%%%%%%%%%%%%%%%%%%%%%%%%%%%%%%%%%%%%%%%%%%%%%%%%%%%%
\subsubsection{Lie Group Homomorphisms}

\begin{definition}
	A \bam{Lie group homomorphism} $\phi:G \to H$ is a pointed smooth map obeying $\forall a,b \in G, \, \phi(ab) = \phi(a)\phi(b)$. 
\end{definition}

\begin{example}
	$\det : GL_n(\mbb{R}) \to \mbb{R}^\times$ is a Lie group hom. 
\end{example}

\begin{definition}
	Given $g \in G$ we have two diffeomorphisms $L_g, R_g : G \to G$ given by 
	\eq{
L_g(h) = gh, \quad R_g(h) = hg	
}
$L:G \to \Diff(G), \, g \mapsto L_g$ is a group homomorphism, and $R$ is an antihomomorphism respectively. 
\end{definition}

\begin{lemma}
	$L_g$ and $R_h$ commute. 
\end{lemma}
%%%%%%%%%%%%%%%%%%%%%%%%%%%%%%%%%%%%%%%%%%%%%%%%%%%%%%%%
\subsubsection{Compact Lie Groups}
In standard applications of gauge theory to particle physics phenomenology, it turns out that we typically want compact Lie groups. We will define some compact Lie groups as subgroups of $GL_n(k)$ preserving a positive definite inner product:

\begin{definition}
	The \bam{classical Lie groups} are the subgroups of $LG_n(k)$ ($k=\mbb{R},\mbb{C},\mbb{H}$) preserving:
	\begin{itemize}
		\item ($\mbb{R}$) the Euclidean inner product, this is the \bam{orthogonal group}
		\item ($\mbb{C}$) the Hermitian inner product, this is the \bam{unitary group}
		\item ($\mbb{H}$) quaternionic Hermitian inner product, this is the \bam{quaternionic unitary group}.
	\end{itemize} 
\end{definition} 

\begin{example}
	$O(n) = \pbrace{A \in GL(n,\mbb{R}) \, \mid \, A^T A = I}$
\end{example}

\begin{example}
	$Sp(2n,\mbb{R}) = \pbrace{A \in GL(2n,\mbb{R}) \, \mid \, A^T J A = J}$ where $J$ is block diagonal with blocks $\begin{psmallmatrix} 0 & -I_n \\ I_n & 0 \end{psmallmatrix}$
\end{example}

\begin{example}
	We have $GL(n,\mbb{C}) = \pbrace{A \in GL(2n,\mbb{R}) \, \mid \, AJ=JA}$
\end{example}

\begin{example}
	$U(n) = \pbrace{A \in GL(n,\mbb{C}) \, \mid \, A^\dagger A = I}$
\end{example}
\begin{remark}
	$U(1)$ will turn out to be the gauge group for electromagnetism. 	
\end{remark}


\begin{lemma}
	\eq{
		U(n) &= GL(n,\mbb{C}) \cap Sp(2n,\mbb{R}) \cap O(2n,\mbb{R}) \\
		&= GL(n,\mbb{C}) \cap Sp(2n,\mbb{R}) \\ 
		&= GL(n,\mbb{C}) \cap O(2n,\mbb{R}) \\
		&= Sp(2n,\mbb{R}) \cap O(2n,\mbb{R})
	}
\end{lemma}

\begin{lemma}
	$SO(2) \cong U(1) \cong S^1$. 
\end{lemma}
\begin{proof}
	View $\begin{psmallmatrix} \cos \theta & -\sin\theta \\ \sin\theta & \cos\theta \end{psmallmatrix} \to e^{i\theta}$. 
\end{proof}

\begin{lemma}
	$SO(3) \cong \mbb{RP}^3 \cong \faktor{S^3}{\sim},$ and $SU(3) \cong S^3$.
\end{lemma}
\begin{remark}
	We actually have a double cover $SU(2) \twoheadrightarrow SO(3)$
\end{remark}

\begin{lemma}
	$SO(4) \cong \faktor{(S^3 \times S^3)}{\sim}$.
\end{lemma}


%%%%%%%%%%%%%%%%%%%%%%%%%%%%%%%%%%%%%%%%%%%%%%%%%%%%%%%%
\subsubsection{Lie Algebra}

\begin{definition}
A (real) \bam{Lie algebra} is a $\mbb{R}$-vector space $\mf{g}$ with a bilinear, antisymmetric map $\comm[\cdot]{\cdot} : \mf{g} \times \mf{g} \to \mf{g}$ that satisfies the Jacobi identity. 
\end{definition}

\begin{definition}
	We say $X \in \mf{X}(G)$ is \bam{left invariant} if $\forall h \in G,\, (dL_g)_h X_h = X_{gh}$.
\end{definition}
 
\begin{lemma}
	LIVFs are determined by $X_e\in T_e G = \mf{g}$.
\end{lemma}
\begin{proof}
	Simply $X_g = (dL_g)_e X_e$ 
\end{proof}
 
\begin{lemma}
If $X,Y$ are LI then $\comm[X]{Y}$ is LI, where $\comm[\cdot]{\cdot}$ is the usual Lie bracket of vector fields. 
\end{lemma}

\begin{remark}
	Given any Lie group $(G,e)$ we have an associated Lie algebra $\mf{g} = \Lie(G) = T_eG$. Note $\dim\mf{g} = \dim G$. Given the above results, we can give $\mf{g}$ a bracket by 
	\eq{
		\comm[X_e]{Y_e} = \comm[X]{Y}_e
	}
\end{remark}


Now recall we have the exponential map $\exp : \mf{g} \to G$, and for matrix groups this is given by $\exp(X) = \sum_{n=0}^\infty \frac{X^n}{n!}$. 

\begin{remark}
All the LIVFs on $G$ are complete, and so we can define the exponential map by 
\eq{
\exp : \mf{g} &\to G \\
X &\mapsto \phi_1^X(e)
}
We can find $X$ from this by using $X = \ev{\frac{d}{dt} \exp(tX)}{t=0}$
\end{remark}

\begin{example} 
One can ask what the Lie algebras for the corresponding classical matrix groups are. 
\begin{itemize}
    \item $GL(n,\mbb{R})$ is a vector space, and so $T_e GL(n,\mbb{R}) = GL(n,\mbb{R})$. The bracket turns out to be the standard matrix commutator. 
    \item As $SL(n) = \pbrace{A \, \mid \, \det A = 1}$ we find $\mf{sl}_n = \pbrace{B \, | \, \det(\exp(tB)) = 1} = \pbrace{B \, \mid \, \tr(B) = 0}$
    \item $\mf{o}_n = \pbrace{B  \mid  \ev{\frac{d}{dt} \exp(tB)^T \exp(tB)}{t=0}} = \pbrace{B  \mid B^T + B = 0 }$
\end{itemize}
This process can be continues as similarly. 
\end{example}

\begin{definition}
We have the \bam{adjoint representation} of a group $G$ acting on $\mf{g}$ by 
\eq{
\Ad_g(X) = \ev{ \, \frac{d}{dt} g \exp(tX)^{-1} g^{-1}}{t=0}
}
\end{definition}

\begin{prop}
	An equivalent definition of $\Ad$ is that $\Ad_g = \pround{L_g \circ R_{g^{-1}}}_\ast $. 
\end{prop}
\begin{proof}
	We start by verifying that this other definition could make sense. $(L_g)_\ast : T_eG \to T_g G$ and $(R_{g^{-1}})_\ast : T_g G \to T_e G$ so 
	\eq{
	(R_{g^{-1}})_\ast \circ (L_g)_\ast =\pround{L_g \circ R_{g^{-1}}}_\ast  : T_eG \to T_e G
}
\hl{finish this off}.
\end{proof}

\begin{prop}
	$\Ad : G \to GL(T_e G)$ is a Lie group hom
\end{prop}
\begin{proof}
If we use the definition of $\Ad_g = d\pround{L_g \circ R_{g^{-1}}} =\pround{L_g \circ R_{g^{-1}}}_\ast$ we have 
\eq{
\Ad_{gh} &= \pround{L_{gh} \circ R_{(gh)^{-1}}}_\ast \\
	&= \pround{ L_g \circ R_{g^{-1}} \circ L_h \circ R_{h^{-1}}}_\ast \\
	&= \pround{L_g \circ R_{g^{-1}}}_\ast \circ \pround{L_h \circ R_{h^{-1}}}_\ast \\
	&= \Ad_g \circ \Ad_h
}
\end{proof}

\begin{lemma}
For a matrix group, $\Ad_A(B) = ABA^{-1}$. 
\end{lemma}

\begin{definition}
We also have an adjoint rep of $\mf{g}$ acting on $\mf{g}$ by 
\eq{
\ad_X(Y) = \ev{\frac{d}{dt} \Ad_{\exp(tx)}(Y)}{t=0} = \comm[X]{Y}
}
\end{definition}

\begin{ex}
	Show that if $G$ is a matrix Lie group, the Lie bracket is indeed the commutator of the matrices. 
\end{ex}

\begin{definition}
	A group rep $\rho$ is \bam{faithful} if $\ker \rho = e$
\end{definition}

\begin{fact}
	$\Ad$ need not be a faithful rep. 
\end{fact}

\begin{definition}
	Given a basis $\pbrace{t_i}$ of a Lie algebra, the \bam{structure constants} are the coefficients $c_{ij}^k$ defined by 
	\eq{
\comm[t_i]{t_j} = \sum_k c_{ij}^k t_k	
}
\end{definition}

\begin{prop}
	Let $\phi:G \to H$ be a Lie group hom. Then $(\phi_\ast)_e : \mf{g} \to \mf{h}$ is a Lie algebra hom
\end{prop}
\begin{proof}
	See 
	\eq{
\phi(ab) = \phi(a)\phi(b) &\Leftrightarrow (\phi \circ L_a)(b) = (L_{\phi(a)} \circ \phi)(b) \\
&\Leftrightarrow 	\phi \circ L_a = L_{\phi(a)} \circ \phi \\
&\Rightarrow (\phi \circ L_a)_\ast = (L_{\phi(a)} \circ \phi)_\ast \\
&\Rightarrow \phi_\ast \circ (L_a)_\ast = (L_{\phi(a)})_\ast \circ \phi_\ast \\
& \Rightarrow \forall X \in \mf{g}, \, (X)_a = 
}
\end{proof}

\begin{example}
	We have $\Ad:G \to GL(\mf{g})$ so there is an associated $\Ad_\ast : \mf{g} \to gl(\mf{g})$. This coincides with $ad : X \mapsto \ad_X$, as should be clear. 
\end{example}

%%%%%%%%%%%%%%%%%%%%%%%%%%%%%%%%%%%%%%%%%%%%%%%%%%%%%%%%
\subsubsection{Maurer Cartan Forms}

\begin{definition}
	The \bam{Maurer Cartan one form} is $\theta \in \Omega^1(G;\mf{g})$ given by 
	\eq{
\theta_g = (L_{g^{-1}})_\ast : T_gG \to T_e G = \mf{g}	
}
\end{definition}

\begin{lemma}
	$\theta$ is left invariant, i.e $L_g^\ast \theta = \theta$.
\end{lemma}
\begin{proof}
	See
	\eq{
(L_g^\ast \theta)_h &= \theta_{gh} \circ (L_g)_\ast \\
&= (L_{(gh)^{-1}})_\ast \circ (L_g)_\ast \\
&= (L_{(gh)^{-1}}\circ L_g)_\ast \\
&= (L_{h^{-1}})_\ast = \theta_h 	
}
\end{proof}

\begin{prop}
	$\theta$ obey the structure equation 
	\eq{
d\theta + \frac{1}{2} \comm[\theta]{\theta} = 0	
}
i.e. $\forall \xi, \eta \in \mf{X}(G), \, d\theta(\xi,\eta) + \comm[\theta(\xi)]{\theta(\eta)} = 0$.
\end{prop}
\begin{proof}
	Note $\theta(\xi) = \xi_e$ as 
	\eq{
\theta_g (\xi_g) = (L_{g^{-1}})_\ast (L_g)_\ast \xi_e = \xi_e	
}
and then 
\eq{
d\theta(\xi,\eta) = \xi \theta(\eta) - \eta \theta(\xi) - \theta(\comm[\xi]{\eta})
}

\end{proof}

\begin{ex}
	Show that for matrix groups $\theta_g = g^{-1}dg$.
\end{ex}

%%%%%%%%%%%%%%%%%%%%%%%%%%%%%%%%%%%%%%%%%%%%%%%%%%%%%%%%
\subsubsection{Invariant inner products on \secmath{\mf{g}}}
	
\begin{definition}
	For a $\mbb{R}$ LA $\mf{g}$, a bilinear form $\pangle{\cdot,\cdot}:\mf{g} \times \mf{g} \to \mbb{R}$ is said to be \bam{invariant} if 
	\eq{
\forall X,Y,Z \in \mf{g}, \, \pangle{\comm[X]{Y},Z} + \pangle{Y,\comm[X]{Z}} = 0	
}
\end{definition}

\begin{definition}
	The \bam{Killing form} is $\kappa:\mf{g} \times \mf{g} \to \mbb{R}$ given by 
	\eq{
\kappa(X,Y) = \tr(\ad_X \circ \ad_Y)	
}
\end{definition}

\begin{lemma}
	$\kappa$ is a symmetric, invariant, bilinear inner product. 
\end{lemma}

\begin{fact}[Cartan's Criterion]
	$\kappa$ is non-degenerate $\Leftrightarrow$ $\mf{g}$ is semisimple. 
\end{fact}

\begin{fact}
	$\kappa$ is negative definite $\Leftrightarrow$ $\mf{g}$ is the LA of a compact semisimple $G$.
\end{fact}

\begin{fact}
	$\mf{g}$ admits a positive definite invariant inner product only when $\mf{g}$ is the LA of a compact $G$. 
\end{fact}
%%%%%%%%%%%%%%%%%%%%%%%%%%%%%%%%%%%%%%%%%%%%%%%%%%%%%%%%
%%%%%%%%%%%%%%%%%%%%%%%%%%%%%%%%%%%%%%%%%%%%%%%%%%%%%%%%
\subsection{Symplectic Geometry}
\subsubsection{Symplectic Manifolds}

\begin{definition}
If $M$ is an even-dimensional manifold we say $M$ is symplectic if it is  equipped with $\omega \in \Omega^2(M)$ which is 
\begin{itemize}
    \item closed: $d\omega = 0$
    \item non-degenerate: $\forall m \in M,\, i_\cdot : T_m M \overset{\cong}{\to} T_m^\ast M$ is an isomorphism. 
\end{itemize}
\end{definition}

\begin{definition}
If $(M,\omega)$ is symplectic and $f \in C^\infty(M)$, then $df \in \Omega^1(M)$ and $\exists ! \, X^f \in \mf{X}(M)$ s.t $i_{X^f}\omega = df$. $X^f$ is called the \bam{Hamiltonian function associated to $f$}. 
\end{definition}

\begin{lemma}
The Hamiltonian function is constant along the flow lines of its vector field, i.e 
\eq{
t \mapsto (f \circ \phi_t^{X^f}(m))
}
is constant.
\end{lemma}

Given a smooth manifold $N$, we have the following result:

\begin{theorem}
The cotangent bundle $M= T^\ast N$ has a canonical symplectic structure. $\omega = - d\theta$ where $\theta$ is the tautological one form 
\end{theorem}

\begin{definition}
The \bam{tautological one form} $\theta$ on the cotangent bundle is given by noting 
\eq{
\pi : T^\ast N & \to N \\
\Rightarrow d\pi : T(T^\ast N) &\to TN
}
so if $ m \in M, \, m \in T_{\pi(m)}^\ast N$ and so we have 
\eq{
m : T_{\pi(m)}N \to \mbb{R}
}
and 
\eq{
\theta_m \equiv m \circ d\pi: T_m M \to \mbb{R}
}
\end{definition}

\begin{lemma}
$\omega = -d\theta$ is (obviously) closed and moreover non-degenerate, hence symplectic. 
\end{lemma}
\begin{proof}
If $q^1, \dots, q^n$ are coordinates on $N$, then $\pd{q^1}, \dots, \pd{q^n}$ are vector fields on $N$. Letting $p_i$ be the one form associated with $\pd{q^i}$ we get coordinates $(q^i,p_i)$ on $M$ in which 
\eq{
\theta &= p_i dq^i \\
\omega &= dq^i \wedge dp_i
}
\end{proof}

\begin{aside}
If $f \in C^\infty(M)$ we get associated Hamiltonian vector fields $X^f$. We then define the \bam{Poisson bracket} of $f$ and $g \in C^\infty(M)$ by 
\eq{
\acomm[f]{g} \equiv X^f(g) = \omega(X^f,X^g)
}
\end{aside}

\begin{theorem}[Darboux]
	On any symplectic manifold there exist local coordinates $(q,p)$ s.t. $\omega = dq \wedge dp$. 
\end{theorem}
\begin{corollary}
	There is no local invariant in symplectic geometry. 
\end{corollary}

\begin{remark}
	When we think of symplectic geometry vs complex geometry vs Riemannian geometry what we are really doing is thinking of the frame bundle of the tangent bundle of $M$, which when $\dim M = 2n$ is a $GL(2n,\mbb{R})$ bundle, and then asking about reduction of structure gropu to either $SP(n), \, GL(n,\mbb{C}),$ or $O(2n,\mbb{R})$.  
\end{remark}


%%%%%%%%%%%%%%%%%%%%%%%%%%%%%%%%%%%%%%%%%%%%%%%%%%%%%%%%
\subsubsection{Symmetries and Moment Maps}
\begin{definition}
	A group action of $G$ on a symplectic manifold $(M,\omega)$ is \bam{symplectic} if $\forall g \in G, \, g^\ast \omega = \omega$. 
\end{definition}
\begin{remark}
There is a mismatch of minus signs in the convention of the literature, namely that given a group action of $G$ on $(M,\omega)$ constructing a map 
\eq{
	\mf{g} &\to \mf{X}(M) \\
	X &\mapsto \tilde{X}
}
can be done in one of two ways 
\begin{enumerate}
	\item Set $\tilde{X}_m = \ev{\frac{d}{dt} \exp(tX) \cdot m}{t=0}$ as is natural, but then the map is a Lie algebra \bam{antihomomorphism}
	\item Set $\tilde{X}_m = \ev{\frac{d}{dt} \exp(-tX) \cdot m}{t=0}$, and then the map is  Lie algebra homomorphism. 
\end{enumerate}
For the purpose of this course the second convention is used.
\end{remark}



Now let us think about 
\begin{tkz}
	C^\infty(M) \arrow[r,"d"] & \Omega^1(M) \arrow[r] & \mf{X}(M) \\
	\mf{g} \arrow[u,dashed,"\mu^\ast"] \arrow[rru] & &
\end{tkz}
\begin{definition}
	If such a map $\mu^\ast$ exists, we say the action is \bam{Hamiltonian} and call the map the \bam{co-moment map}.
\end{definition}
\begin{definition}
	Given a co-moment map $\mu^\ast$ the corresponding \bam{moment map} is 
\eq{
	\mu : M &\to \mf{g}^\ast \\
	\mu(m)(X) &=\mu^\ast(X)(m)
}
\end{definition}

\begin{example}
	Let $U(1)$ act on $T^2$ by rotation around the central axis. If we view $T^2 \cong \faktor{\mbb{R}^2}{\mbb{Z}^2}$ then we can take the 2-from $dx\wedge dy$ on $\mbb{R}^2$ which descends to $T^2$. The vector field corresponding to this action is nowhere 0. This action then cannot be Hamiltonian, as any smooth map on a compact manifold must have critical points (see Morse theory) and so 1-form must have critical points, and hence the associated Hamiltonian vector field would be somewhere 0. 
\end{example}

\begin{example}
	If $G$ is action on coadjoint orbits of $\mf{g}^\ast$, $\mu$ is inclusion. 
\end{example}

\begin{prop}
	If $G$ acting on $M$ is semi-simple, moment maps always exist. 
\end{prop}

\begin{prop}
	If $G$ acts on $(M,\omega)$ and $0$ is a regular value of $\mu$, then $\faktor{\mu^{-1}(0)}{G}$ is a new symplectic manifold with natural symplectic form $\omega_{red}$ characterised by $i^\ast \omega = \pi^\ast \omega_{red}$ as below 
	\begin{tkz}
		\mu^{-1}(0) \arrow[r,"i",hook] \arrow[d,"\pi"'] & (M,\omega) \\
		(\faktor{\mu^{-1}(0)}{G},\omega_{red})
	\end{tkz}
This process is \bam{symplectic reduction}.
\end{prop}

%%%%%%%%%%%%%%%%%%%%%%%%%%%%%%%%%%%%%%%%%%%%%%%%%%%%%%%%
%%%%%%%%%%%%%%%%%%%%%%%%%%%%%%%%%%%%%%%%%%%%%%%%%%%%%%%%
\section{Bundles Theory}

%%%%%%%%%%%%%%%%%%%%%%%%%%%%%%%%%%%%%%%%%%%%%%%%%%%%%%%%
\subsection{Bundles}
\subsubsection{Principle Bundles}
\begin{definition}
A group action $G \lact M$ is \bam{free} if $\forall m \in M, g \in G, \, e \neq g \Rightarrow g\cdot m \neq m$
\end{definition}

\begin{definition}
A group action $G \lact M$ is \bam{proper} if the map between topological spaces 
\eq{
G \times M &\to M \times M \\
(g,m) &\mapsto (m,g\cdot m)
}
has compact preimages of compact subsets. 
\end{definition}

\begin{prop}
If $G$ is proper, any continuous group action is proper. 
\end{prop}

\begin{lemma}
If $G \lact M$ is smooth, free, and proper. Then $\faktor{M}{G}$ exists and is a smooth manifold
\end{lemma}

\begin{definition}
If $G$ is a Lie group, $M$ a smooth manifold, a \bam{G principal bundle over M} is given by 
\begin{itemize}
    \item $P$ a smooth manifold with projection $\pi:P \to M$
    \item a smooth right action of $G$, $P \times G \to P$
    \item a $G$-equivariant local trivialisation, i.e $\pbrace{U_\alpha}$ an open covering s.t $\Psi_\alpha : \pi^{-1}(U_\alpha) \overset{\cong}{\to} U_\alpha \times G, \, \Psi_\alpha(p) = (\pi(p),\psi_\alpha(p))$ and $(m,g) \cdot h = (m,gh)$
\end{itemize}
This is often noted as 
\begin{tkz}
G \arrow[r] & P \arrow[d,"\pi"] \\ & M
\end{tkz}
\end{definition}

\begin{remark}
By $G$-equivariance $\pi^{-1}(U_\alpha)$ are all $G$-orbits, and moreover the action is free and proper. This means we have $M \cong \faktor{P}{G}$. As such a $G$-principal bundle is given by a smooth, free, and proper action $P \ract G$. 
\end{remark}

If $U_\alpha,U_\beta$ are part of the open cover s.t $U_{\alpha\beta} \equiv U_\alpha \cap U_\beta \neq \emptyset$ then we can ask about the diagram
\begin{center}
\begin{tikzpicture}[commutative diagrams/every diagram]
\node (P0) at (90:1cm) {$\pi^{-1}(U_{\alpha\beta})$};
\node (P1) at (90+120:1.5cm) {$U_{\alpha\beta} \times G$};
\node (P2) at (90+240:1.5cm) {$U_{\alpha\beta} \times G$};
\path[commutative diagrams/.cd, every arrow, every label]
(P0) edge node[swap] {$\Psi_\alpha$} (P1)
(P0) edge node {$\Psi_\beta$} (P2)
(P1) edge node[swap] {} (P2);
\end{tikzpicture}
\end{center}

\begin{comment}
\begin{tkz}
\pi^{-1}(U_{\alpha\beta}) \arrow[r,"\Psi_\beta"] \arrow[d,"\Psi_\alpha"] & U_{\alpha\beta} \times G \\
U_{\alpha\beta}\times G \arrow[ur] & 
\end{tkz}
\end{comment}
which sends $(m,g) \mapsto (m,\phi_{\beta\alpha}(m)g)$ where the map $\phi_{\beta\alpha}$ satisfies 
\begin{itemize}
    \item $\phi_{\alpha\beta}(m) = \phi_{\beta\alpha}(m)^{-1}$ 
    \item $\phi_{\alpha\alpha} : U_\alpha \to \pbrace{e}\leq G $
    \item on triple intersects $\phi_{\gamma\beta}\phi_{\beta\alpha} = \phi_{\gamma\alpha}$. 
\end{itemize}

\begin{remark}
If a covering $M = \cup_\alpha U_\alpha$ is given with transition function $\phi_{\beta\alpha}:U_{\alpha\beta} \to G$ satisfying the above conditions, then we can recover $P$. This is done as 
\eq{
P = \faktor{\pround{\bigsqcup_\alpha U_\alpha \times G}}{\sim}
}
where $(m,g) \sim (m, \phi_{\beta\alpha}(m) g)$
\end{remark}

\begin{example}[Trivial Bundle]
Given $P=M \times G$ we get a $G$ action by right multiplication. 
\end{example}

\begin{example}[Hopf Fibration]
Take $P = S^{2n-1} \subset \mbb{C}^n$ and $G = U(1)$. Let $G$ act on $\mbb{C}^n$ by $\bm{v} \cdot g = g^{-1} \bm{v}$. We then get $M = \mbb{CP}^{n-1}$
\end{example}

\begin{example}
The edge of a Mobius strip is a $\mbb{Z}_2$-principal bundle over $S^1$.
\end{example}

\begin{remark}
All of the fibres are isomorphic to $G$ as manifolds, but they do not have an intrinsic group structure. As such the fibres are $G$-torsors. 
\end{remark}

\begin{definition}
A \bam{section} of $P \overset{\pi}{\to}M$ is a right inverse to $\pi$, i.e $s:M \to P$ s.t. $\pi \circ s = \id_M$. 
\end{definition}

\begin{lemma}
A PB has a global section iff is is trivial. 
\end{lemma}

\begin{definition}
A \bam{vector bundle} $E \to M$ is a (smooth) morphism s.t. all the fibres are vector spaces of the same dimension $r\equiv\rank E$. 
\end{definition}

\begin{definition}
If $E \to M$ is a vector bundle, the associated \bam{frame bundle} is 
\eq{
P \equiv \Fr_E =\pbrace{\text{bases of fibres of $E$}}
}
This is a $GL(r,\mbb{R})$-bundle. The action is given by 
\eq{
(\bm{\sigma}_1, \dots, \bm{\sigma}_r) \mapsto  (\bm{\sigma}_1, \dots, \bm{\sigma}_r) \cdot A = (\bm{\sigma}_1A, \dots, \bm{\sigma}_rA)
}
\end{definition}

%%%%%%%%%%%%%%%%%%%%%%%%%%%%%%%%%%%%%%%%%%%%%%%%%%%%%%%
\subsubsection{Bundles from Bundles}
\begin{definition}
	Given $P \to M$ a ppal $G$-bundle and $f:N \to M$ smooth we defin the \bam{pullback bundle} to be 
	$f^\ast \pi : f^\ast P \to N$ the fibered product 
	\eq{
f^\ast P = \pbrace{(y,p) \in N \times P \, | \, f(y) = \pi(p)}	
}
\end{definition}

If we are given a principal bundle $G \to P \overset{\pi}{\to} M$ and a left action $G \lact F$ we can get the action $(P \times F) \ract G$ by $(p,f) \cdot g = (p\cdot g, g^{-1} \cdot f)$. This action is always smooth, free, and proper. 

\begin{definition}
The \bam{associated fibre bundle} to $G \to P \overset{\pi}{\to} M$ with $G \lact F$ is 
\eq{
P_F \equiv \faktor{(P \times F)}{G} \to M
}
quotienting by orbits under the group action. All the fibres are diffeomorphic to $F$. 
\end{definition}

If $G \lact F$ preserves extra structure (e.g. if $F=V$ is a vector space and the action is linear) then all the fibres have this structure canonically. 

\begin{example}
Suppose $F=V$ is a vector space and $G \lact V$ is linear (i.e. $V$ is a rep space for $G$). Then $P_V$ is a vector bundle over $M$. 
\end{example}

\begin{prop}
The assignment $P+V \to P_V$ is the "inverse" of taking the frame bundle.  
\end{prop}
\begin{proof}
Using the defining rep of $GL(r,k) \lact k^r$ we get $(\Fr_E)_{k^r} \cong E$
\end{proof}

\begin{example}
If $G \lact \mf{g}$ via the adjoint action $\Ad$, we get $\ad(P) = \faktor{(P \times \mf{g})}{G}$
\end{example}

\begin{example}
Let $G \lact G$ by conjugation, i.e. $C_g(h) = ghg^{-1}$. This preserves group structure, so we have $C : G \to \Aut(G)$ and 
\eq{
\Ad(P) \equiv P_{G,C} = \faktor{(P\times G)}{G}
}
where $(p,g) \sim (ph,h^{-1}gh)$.
\end{example}

\begin{definition}
	Given $G \to P \to M$ and a Lie group hom $\rho : G \to H$, letting $G \lact H$ by $g \cdot h = \rho(g)h$ and noting $H \ract H$ by right multiplication, making it into a $H$-torsor, is preserved by the action of $G$ and so we have  
	\eq{
		P_H = \faktor{(P \times H)}{G}
	}
	a $H$-torsor bundle, or a principal $H$ bundle. This provides an \bam{extension of structure group}.
\end{definition}

\begin{definition}
Given a $H$-principal bundle $H \to \tilde{P} \to M$ and a Lie group hom $G \to H$, if $\exists G \to P \to M$ s.t.  $\tilde{P} \cong P$ as $H$-principal bundle, we say that it is a \bam{reduction of structure group} of $\tilde{P}$.
\end{definition}

\begin{example}
If $E \to M$ is a Euclidean vector bundle of rank $r$, we can look at $\Fr_E$ as a $GL(r)$-principal bundle, or at $\Fr_E^\perp= \pbrace{\text{bundle of orthonormal frames}}$ which is a $O(r)$-principal bundle. $\Fr_E^\perp$ is a reduction of structure group of $\Fr_E$ using $O(r) \hookrightarrow GL(r)$
\end{example}


\begin{prop}
There is a 1:1 correspondence 
\eq{
\pbrace{\text{reduction of structure groups of $\Fr_E$ using $O(r) \hookrightarrow GL(r)$}} \leftrightarrow \pbrace{\text{Choice of Euclidean structure on $E$}}
}
\end{prop}

\begin{example}
If $E$ is a rank $r$ bundle, and $\Lambda^2 E$ is a line bundle. Then $\Lambda^2 E$ is the vector bundle associated to $\Fr_E$ with $GL(2,k) \overset{\det}{\to} k^\times$. Hence we get a $k^\times$-principal bundle by extension. 
\end{example}

\begin{example}
There is a 1:1 correspondence 
\eq{
\pbrace{\text{reduction of structure group of $\Fr_E$ using $GL(r) \to SL(r)$}} \leftrightarrow \pbrace{\text{trivialisations $\Lambda^2 E \cong M \times \mbb{R}$}}
}
If we do not have $\Lambda^2 E \cong M \times \mbb{R}$ then reduction might not always be possible. 
\end{example}


\begin{example}
If $(M,g)$ is an oriented Riemannian Manifold of dimension $n$, then $\Fr_{TM}$ has a reduction of structure group from $GL_n(\mbb{R}) \to SO(n)$. $ n \geq 3 \Rightarrow \pi_1(SO(n)) = \mbb{Z}_2$ and so there is a 2:1 "universal cover" $\Spin(n) \to SO(n)$
\end{example}

\begin{definition}
A Spin structure on $M$ is a reduction of structure group from $SO(n)$ to $\Spin(n)$. 
\end{definition}

If a spin structure is given, then you can look at irreducible representations of the spin group that do not come from $SO(n)$. 


%%%%%%%%%%%%%%%%%%%%%%%%%%%%%%%%%%%%%%%%%%%%%%%%%%%%%%%%
\subsubsection{Gauge transformations}

\begin{definition}
Given $G \to P \to M$, a \bam{gauge transformation} for $P$ is a $G$-equivariant diffeomorphism $\psi$ s.t. 

\begin{center}
\begin{tikzpicture}[commutative diagrams/every diagram]
\node (P0) at (30:1cm) {$P$};
\node (P1) at (30+120:1cm) {$P$};
\node (P2) at (30+240:1cm) {$M$};
\path[commutative diagrams/.cd, every arrow, every label]
(P1) edge node {$\psi$} (P0)
(P0) edge node {$\pi$} (P2)
(P1) edge node[swap] {$\pi$} (P2);
\end{tikzpicture}
\end{center}
commutes. As the $\psi$ compose, we get a group $\mc{G}(P)$
\end{definition}

\begin{remark}
Note both $G$ and $\mc{G}(P)$ act on $P$ by right and left action respectively. 
\end{remark}

\begin{theorem}
There are canonical group isomorphisms 
\eq{
\mc{G}(P) \cong \Gamma(\Ad(P)) \cong \pbrace{f:P \to G \, | \, f \text{ smooth }, \, f(p \cdot g) = g^{-1}f(p) g}
}
\end{theorem}
\begin{proof}
We have 
\begin{center}
\begin{tikzpicture}[commutative diagrams/every diagram]
\node (P0) at (30:1.5cm) {$\Gamma(\Ad(P))$};
\node (P1) at (30+120:1.5cm) {$\mc{G}(P)$};
\node (P2) at (30+240:1cm) {$\pbrace{f:P \to G}$};
\path[commutative diagrams/.cd, every arrow, every label]
(P0) edge node[swap] {$C$} (P1)
(P2) edge node {$B$} (P0)
(P1) edge node {$A$} (P2);
\end{tikzpicture}
\end{center}
with 
\eq{
A: \psi &\mapsto (f: p \mapsto g \text{ if } \psi(p) = pg) \\
}
we may check 
\eq{
\psi(p\cdot\tilde{g}) = \psi(p)\cdot \tilde{g} = pg \tilde{g} = (p\tilde{g}) (\tilde{g}^{-1} g \tilde{g})
}
so $f$ obeys the necessary condition. We then have 
\eq{
B : f \mapsto (s : m \mapsto [p,f(p)]\in \faktor{(P \times G)}{G} \text{ for any }p \in \pi^{-1}(m))
}
Again we can check if $\tilde{p} \in \pi^{-1}(m)$, then 
\eq{
[p\cdot g,f(p\cdot g) ] = [p\cdot g, g^{-1}f(p) g] = [p,f(p)]
}
as this is the action we are quotienting out by. Finally we have 
\eq{
C : s \mapsto (\psi : p \mapsto p\cdot g \text{ where } s(\pi(p)) = [p, g])
}
and we can see $s(\pi(p \cdot \tilde{g})) = s(\pi(p)) = [p,g] = [p\tilde{g},\tilde{g}^{-1}g \tilde{g}]$ so
\eq{
\psi(p\cdot \tilde{g}) = (p \cdot \tilde{g}) \cdot (\tilde{g}^{-1}g \tilde{g}) = p \cdot g \cdot \tilde{g} = \psi(p) \cdot \tilde{g}
}
It is necessary to check that $A,B,C$ are group isomorphisms now. 
\end{proof}

\begin{prop}
	$\Lie(\mc{G}(P)) = \Gamma(\ad(P))$. 
\end{prop}
%%%%%%%%%%%%%%%%%%%%%%%%%%%%%%%%%%%%%%%%%%%%%%%%%%%%%%%%
%%%%%%%%%%%%%%%%%%%%%%%%%%%%%%%%%%%%%%%%%%%%%%%%%%%%%%%%
\subsection{Connections}

%%%%%%%%%%%%%%%%%%%%%%%%%%%%%%%%%%%%%%%%%%%%%%%%%%%%%%%%
\subsubsection{Kozul connections}
Let $ E \to M$ be a vector bundle. We want to be able to take directional derivatives of sections of $E$

\begin{definition}[Kozul connection]
A \bam{Kozul connection} on $E$ is a map 
\eq{
\nabla : \Gamma(E) \to \Gamma(E \otimes T^\ast M)
}
For $s \in \Gamma(E), \, X \in \mf{X}(M) = \Gamma(TM)$,
$(\nabla s)(X) \in \Gamma(E)$ is denoted as $\nabla_X s$. 
\end{definition}

We want this to satisfy a product rule that for $f \in C^\infty(M)$ 
\eq{
\nabla_X (fs) = X(f) s + f \nabla_X(s)
}
and linearity in $\Gamma(E), \mf{X}(M)$, i.e. $\forall c_i \in \mbb{R}, f_i \in C^\infty(M)$
\eq{
\nabla_X (c_1 s_1 + c_2 s_2) &= c_1 \nabla_X s_1 + c_2 \nabla_X s_2 \\
\nabla_{(f_1 X_1 + f_2 X_2)} s &= f_1 \nabla_{X_1} s + f_2 \nabla_{X_2} s 
}

If $E$ is a Euclidean vector bundle with inner product $\pangle{\cdot, \cdot}$, we may ask that a connection respects this additional structure. 

\begin{definition}
$\nabla$ respects  $\pangle{\cdot, \cdot}$ if 
\eq{
X(\pangle{s_1, s_2}) = \pangle{\nabla_X s_1, s_2} + \pangle{s_1, \nabla_X s_2}
}
\end{definition}

%%%%%%%%%%%%%%%%%%%%%%%%%%%%%%%%%%%%%%%%%%%%%%%%%%%%%%%%
\subsubsection{Ehresmann connections}

\begin{notation}
Let $G \to P \overset{\pi}{\to} M$ be a $G$-principal bundle. Denote the fibre through $p \in P$ by $F_p = p\cdot G$.
\end{notation}

\begin{definition}
	We call $T_p F_p = \ker d\pi_p$ the \bam{subspace of vertical tangent vectors at $p \in P$} and denote it by $V_p$. The collection of $V_p$ gives a smooth distribution. 
\end{definition}

\begin{definition}
	A differential form on $P$ is called \bam{horizontal} if it is zero on $V$. 
\end{definition}

\begin{definition}
	A form on $P$ is called \bam{basic} if it is horizontal and $G$-invariant 
\end{definition}

\begin{definition}
An \bam{Ehresmann connection} on $P$, $H$, is a smooth choice of complement $H_p \hookrightarrow T_pP$ s.t. 
\begin{itemize}
    \item $\forall p \in P, \, \dim H_p = \dim M$ 
    \item $\forall p \in P, \, T_pP = T_p F_p \oplus H_p$
    \item $H$ is $G$-equivariant, i.e. $dR_g H_p = H_{p \cdot g}$
\end{itemize}
The distrubution is called \bam{horizontal}.
\end{definition}

\begin{remark}
$d\pi_p : T_p P \to T_{\pi(p)}M$ gives a isomorphism $T_{\pi(p)} M \overset{\cong}{\to} H_p$, or equivalently a splitting of the SES
\eq{
0 \to T_p F_p \to T_pP \to T_{\pi(p)} M \to 0
}
\end{remark}

\begin{remark}
Smooth connections always exists. This is as they certainly exist on local trivialisations, and then they can be glued with partitions of unity. 
\end{remark}



%%%%%%%%%%%%%%%%%%%%%%%%%%%%%%%%%%%%%%%%%%%%%%%%%%%%%%%%
\subsubsection{Holonomy}

\begin{lemma}
Given $G \to P \overset{\pi}{\to} M$, a smooth connection $H$, and a smooth closed curve in the base $C: I \to M$, we can lift tangent vectors to $\pi^{-1}(C(I)) \subset M$ to tangent vectors to $P$ to get a vector field on $\pi^{-1}(p)$, and there is a unique such lift that is horizontal.
\end{lemma}
We can then flow along this vector field starting at $\pi^{-1}(C(0))$. This will give a curve $\tilde{C}:I \to P$, where it is not necessary that $\tilde{C}(0) = \tilde{C}(1)$. However, it will be true that $\tilde{C}(1) \in \pi^{-1}(C(0))$ so $\exists g \in G, \, \tilde{C}(1) = p \cdot g$. 

\begin{definition}
$g =g(p)$ is the \bam{holonomy} of $C$ w.r.t $H$ starting at $p$.  
\end{definition}

\begin{definition}
For $X \in \mf{g}$ we have $\tilde{X}\in T_pP$ s.t. 
\eq{
\tilde{X}(f) = \ev{\frac{d}{dt} f(p \cdot \exp(tX))}{t=0}
}
\end{definition}

\begin{definition}
A \bam{principal connection} is a connection form $\omega \in \Gamma(T^\ast P \otimes \mf{g})= \Gamma(\Hom(TP,\mf{g}))$ s.t. 
\begin{itemize}
    \item $\omega(\tilde{X}) = X$
    \item $(R_g)^\ast \omega = \Ad_{g^{-1}}\omega$
\end{itemize}
\end{definition}

\begin{example}
	On the Hopf bundle, $\pbrace{(z,w) \in \mbb{C}^2 \, | \, \abs{z}^2+\abs{w}^2=1} \to \mbb{CP}^1$, $(z,w) \mapsto [z:w]$, as a $U(1)$-bundle acting by multiplication, we can have $\omega = i \Im \bar{z}dx + \bar{w}dw$.
\end{example}

%%%%%%%%%%%%%%%%%%%%%%%%%%%%%%%%%%%%%%%%%%%%%%%%%%%%%%%%
\subsubsection{Relations between viewpoints}
Let $V$ be a vector space and $S \subset V$ a subspace. 

\begin{definition}
A \bam{complement} to $S$ is a choice $\tilde{S}$ s.t. $V = S \oplus \tilde{S}$. \end{definition}

\begin{lemma}
A complement to $S$ is equivalent to a projection $p: V \to S$ s.t. $\ev{p}{S} = \id_S$
\end{lemma}
\begin{corollary}
$\tilde{S}=\ker p$ 
\end{corollary}

Now, noting that $\omega$ gives a projection $\omega_p : T_pP \to \mf{g}$ for each $p \in P$, we get the following relation:

\begin{prop}
$\omega$ gives an Ehresmann connection by letting $H_p = \ker \omega_p$. 
\end{prop} 

We now recall some facts from the workshops: Given a vector bundle $E \to M$, $\forall U \subset M$ open, we have a short exact sequence 
\eq{
0 \to \End(E)(U) \hookrightarrow \mc{D}^{\leq 1}(E)(U) \overset{\sigma}{\to} \psquare{TM \otimes \End(E)}(U) \to 0
}
where $\mc{D}^{\leq 1}(E)(U)$ are the first order differential operators on $\ev{E}{U}$. Now for $D \in \mc{D}^{\leq 1}(E)(U)$, $s \in \Gamma(U,E)$, $f \in C^\infty(U)$, we have 
\eq{
D(fs) = \sigma(D)(f)(s)
}
Putting $\mc{D}^{\leq 1}_{\diag}(E)(U) = \sigma^{-1}(TM \otimes \id_E)$ we have SES 
\eq{
0 \to \End(E)(U) \to \mc{D}^{\leq 1}_{\diag}(E)(U) \to \ev{TM}{U} \to 0
}
A Kozul connection is now a splitting of this SES, $\nabla : \ev{TM}{U} \to \mc{D}^{\leq 1}_{\diag}(E)(U)$, i.e. given $X \in \mf{X}(U)$, $\nabla_X$ is a 1st order differential operator satisfying $\sigma(\nabla_X) = X$. \\
Now, if we have a principal bundle $G \to P \to M$, the \bam{Atiyah sequence} associated with it is a SES
\eq{
0 \to \ad(P) \to \faktor{TP}{G} \to TM \to 0 \, .
}
This is split by the map $\pround{\ev{d\pi}{H_p}}^{-1}:TM \to \faktor{TP}{G}$, given by an Ehresmann connection. Now with $G \lact V$ via a representation, we have the follwing result:

\begin{prop}
$\forall U \subset M$ we have 
\begin{tkz}
0 \arrow[r] & \ad(P)(U) \arrow[r] \arrow[d] & \faktor{TP}{G}(U) \arrow[r] \arrow[d] & \ev{TM}{U} \arrow[r] \arrow[d,"\id"] \arrow[l,red, bend left = 15] & 0 \\
0 \arrow[r] & \End(E)(U) \arrow[r] & \mc{D}^{\leq 1}_{\diag}(E)(U) \arrow[r,"\sigma"] & \ev{TM}{U} \arrow[r]  \arrow[l,red, bend left = 15] & 0 
\end{tkz}
commutes. 
\end{prop}
\begin{proof}
$G \lact V$ so we get $\mf{g} \to \End(V)$, a morphism of Lie algebras that is $G$-equivariant. Then we have 
\eq{
\faktor{(P \times \mf{g})}{G} \to \faktor{(P \times \End(V))}{G} = \End(E)
}
Further, the space of sections of $E$ is the space of $G$-equivariant functions $P \to V$. This means we can say that $G$-equivariant vector fields on $P$ also act on $G$-equivariant functions $P \to V$, i.e. sections of $E$. It can then be checked that these all commute. 
\end{proof}
\begin{corollary}
A splitting of the Atiyah sequence gives a Kozul connection. 
\end{corollary}

\begin{remark}[On the Atiyah sequence]
	We can send $P \times \mf{g} \to TP$ by sending $(p,\xi)$ to the vector field whose value at $p$ is given by that generated by the action of $G$ on $P$. As this action is fibre-wise, the vector field in the image must lie in the vertical component $\ker d\pi$. This gives the SES 
	\eq{
0 \to P \times \mf{g} \to TP \to TM \to 0	
}
As all these maps are $G$ invariant we can quotient out by the action of $G$ to get the Atiyah sequence. 
\end{remark}

%%%%%%%%%%%%%%%%%%%%%%%%%%%%%%%%%%%%%%%%%%%%%%%%%%%%%%%%
\subsubsection{The space of all connections}

\begin{definition}
If $P \to M$ is a $G$-principal bundle we can denote the \bam{space of all connections on $P$} as $\mc{C}_P$
\end{definition} 

\begin{remark}
	Some will denote this space as $\mc{A}_P$. 
\end{remark}

\begin{prop}
	$\mc{C}_P \neq \emptyset$, i.e. connections always exist. 
\end{prop}
\begin{proof}
	First note 
	\begin{itemize}
		\item A convex combination of connection 1-forms is a 1-form
		\item We can get trivial connection 1-forms on each local trivialisation. 
	\end{itemize}
	With these two facts, we can make a connection by using a partition of unity subordinate to the open cover giving the trivialisation. 
\end{proof}

\begin{prop}
	$\mc{C}_P$ is an affine space modelled on the vector space $\Gamma(T^\ast M \otimes \ad(P))$. 
\end{prop}
\begin{proof}
	Given $P\overset{\pi}{\to} M$ and $\ad P \to M$ we can form the pullpack bundle $\pi^\ast \ad P$ and 
	\eq{
\pi^\ast \ad P &\cong P \times \mf{g} \\
(p,[p,X]) &\mapsto (p,X)	
}
We can check though that 
\eq{
\image \pround{ \pi^\ast : \Omega^1(M;\mf{g}) \to \Omega^1(P;\mf{g})} = \pbrace{ \alpha \, | \, \alpha_p(\tilde{X})=0}
}
i.e the image is zero on $V_p$
\end{proof}

\begin{remark}
	$\mc{C}_P$ is not a vector space, but this is the next best thing. This means that $\Gamma(T^\ast M \otimes \ad(P))$ acts as an additive group on $\mc{C}_P$ freely and transitively, i.e. $\mc{C}_P$ is a torsor for the additive group. Alternatively this can be seen as saying that the difference between two connections gives an element of $\Gamma(T^\ast M \otimes \ad(P))$. 
\end{remark}

Let us try and understand this result in terms of the three ways to view a connection. 
\begin{enumerate}
	\item \bam{Splitting of the Atiyah sequence}: We can think of a section of $T^\ast M \otimes \ad(P)$ as a $C^\infty(M)$-linear map from $TM$ to $\ad(P)$. 
	\begin{tkz}
		0 \arrow[r] & \ad(P) \arrow[r] & \faktor{TP}{G} \arrow[r] & TM \arrow[r] \arrow[ll,dashed, bend left=20] & 0 
	\end{tkz}
$\ad(P)$ is a vector sub-bundle of $\faktor{TP}{G}$, and as it is exactly the kernel of $\faktor{TP}{G} \to TM$ we can add it to any splitting of such to get a new splitting. Similarly, given any two splittings of $\faktor{TP}{G} \to TM$ the difference has to be in the kernel of the map, i.e. takes values in $\ad(P)$, so gives a well defined map $TM \to \ad(P)$. 
\item \bam{Distribution on $TP$}: Take a vector space $V=V_1 \oplus V_2$. Then any linear map $f:V_1 \to V_2$ gives a linear subspace of $V$ 
\eq{
\text{graph}(f) = \pbrace{v \oplus f(v) \, | \, v \in V_1}
}
with $\dim \text{graph}(f) = \dim V_1$. This subspace lets us write 
\eq{
V = \text{graph}(f) \oplus V_2
}
Now suppose we have a distribution $H \subset TP$ and a section of $T^\ast M \otimes \ad(P)$. The latter can be pulled back to $P$ and using $H$ and using $H$ it gives a $G$-invariant linear map from $H$ to the vertical tangent space. This means we have $TP = H \oplus V$ and a linear map $H \to V$. As such we get a new distribution from $\text{graph}(H \to V)$. 
\item \bam{$\mf{g}$-valued one-from on $P$}: Let $\omega$ be the connection one-form and $\gamma:TP \to \ad(P)$ be the section of $T^\ast M \otimes \ad(P)$. From $\gamma$ we can build $\tilde{\gamma} = \gamma \circ d\pi : TP \to \mf{g}$. $\tilde{\gamma} + \omega$ is a new connection one-form. 
\end{enumerate}

%%%%%%%%%%%%%%%%%%%%%%%%%%%%%%%%%%%%%%%%%%%%%%%%%%%%%%%%
%%%%%%%%%%%%%%%%%%%%%%%%%%%%%%%%%%%%%%%%%%%%%%%%%%%%%%%%
\subsubsection{Curvature}

To any connection on $P$ we can associate a curvature - a section of $\wedge^2 T^\ast M \otimes \ad(P)$. To aid in the definition we need the following lemma:

\begin{lemma}
	If $E\to M$ is a vector bundle then a rule that assigns a section of $E$ to each $r$-tuple of vector fields $X^1, \dots, X^r$ which is alternating in the $X^i$ and $C^\infty(M)$-linear corresponds to a section of $\wedge^r T^\ast M \otimes E$. 
\end{lemma}

Now suppose we have a Kozul connection $\nabla$ on $E$

\begin{lemma}
	Given $s \in \Gamma(E), \, X,Y \in \Gamma(TM)$, the expression 
	\eq{
	\nabla_X \nabla_Y s - \nabla_Y \nabla_X s - \nabla_{\comm[X]{Y}}s
}
is 
\begin{itemize}
	\item $C^\infty$ linear in $s,X,Y$ 
	\item alternating in $X,Y$
\end{itemize}
\end{lemma}
\begin{proof}
	\hl{Exercise}
\end{proof}

\begin{corollary}
	The expression corresponds to $F_\nabla \in \Gamma(\wedge^2 T^\ast M \otimes \End(E))$ given by 
	\eq{
F_\nabla(X,Y)(s) = \nabla_X \nabla_Y s - \nabla_Y \nabla_X s - \nabla_{\comm[X]{Y}}s 	
}
\end{corollary}

We can again interpret this three ways via the different ways of thinking about a connection. We will need the following definition:

\begin{definition}
	Given a connection on $P$ given by a distribution $H \subset TP$, for $X \in \Gamma(TP)$ write $X^H$ for the horizontal component. Now for $\phi \in \Omega^r(P) \otimes V$, where $V$ is a vector space, define the \bam{covariant derivative} of $\phi$ to be the $V$-valued $(r+1)$-form $D\phi$ given by
	\eq{
D\phi(X_0, \dots, X_r) = d\phi(X_0^H, \dots, X_r^H)	
} 
\end{definition}
\begin{enumerate}
	\item \bam{Splitting of the Atiyah sequence}: Given $U\subset M$ open we can form the exact sequence 
	\begin{tkz}
		0 \arrow[r] & \Gamma(U,\ad(P)) \arrow[r] & \Gamma\pround{U,\faktor{TP}{G}} \arrow[r] &TU \arrow[r] & 0 
	\end{tkz}
This is in fact an exact sequence of Lie algebra - bunldes where the maps in the sequence respect the bracket structure. If we let $\gamma : TM \to \faktor{TP}{G}$ be the map that splits the Atiyah sequence then we define $F$ by 
\eq{
F(X,Y) = \comm[\gamma(X)]{\gamma(Y)} - \gamma(\comm[X]{Y})
}
As the Lie algebra structure is respected by the exact sequence we must have that $F(X,Y) \mapsto 0 \in TM$, and we have by exactness that is must take values in $\ad(P)$. 
	\item \bam{Distribution on $TP$}: Given a choice of horizontal subspace $H_p \subset T_pP$ we can lift $X \in \Gamma(TM)$ to a horizontal $\tilde{X} \in \Gamma(TP)$. Given $Y\in \Gamma(TM)$ we then get $\tilde{F}$ defined by 
	\eq{
	\tilde{F}(X,Y) = \comm[\tilde{X}]{\tilde{Y}} - \tilde{\comm[X]{Y}}
}
$\tilde{F}(X,Y) \in \Gamma(TP)$, and moreover $\tilde{F}(X,Y) \in \ker d\pi$, so it is vertical. Further $G$-invariance under the adjoint action means that $\tilde{F}$ to $F \in \Gamma(\wedge^2 T^\ast M \otimes \ad(P))$ a section on $M$ (it is clear that $F$ is alternating in $X,Y$ and is $C^\infty(M)$-linear). 
	\item \bam{$\mf{g}$-valued one-form on $P$}: Let $\omega$ be the connection one-form, with corresponding distribution $H$. We then get the curvature $\tilde{F} \in \Gamma(\wedge^2 T^\ast P \otimes \mf{g})$ by 
	\eq{
\tilde{F} = D \omega	
}
One can check (\hl{exercise}) that $\tilde{F}$ is $G$-equivariant in the sense that 
\eq{
R_g^\ast \tilde{F} = \Ad_{g^{-1}} \tilde{F} 
}
Becuase of the equivariance this descends to a section $F \in \Gamma(\wedge^2 T^\ast M \otimes \ad(P))$. 
\end{enumerate}

\begin{prop}
	In terms of just the connection one-form we can write 
	\eq{
\tilde{F}(X,Y) = d\omega(X,Y) + \comm[\omega(X)]{\omega(Y)}	
}
We often write this as the short hand 
\eq{
\tilde{F} = d\omega + \omega \wedge \omega 
}
which is \bam{Cartan's structure equation}. 
\end{prop}

\begin{theorem}[Bianchi identity]
	$D \tilde{F} = 0$
\end{theorem}
\begin{proof}
	We have
	\eq{
D \tilde{F}(X_1, X_2, X_3) &= d \tilde{F}(X_1^H, X_2^H, X_3^H) \\
&= d^2 \omega (X_1^H, X_2^H, X_3^H) + d\omega \wedge \omega (X_1^H, X_2^H, X_3^H) - \omega \wedge d\omega (X_1^H, X_2^H, X_3^H) \\
&= 0
}
as $d^2 = 0$ and $H=\ker \omega$ so $\omega(X^H)=0$. 
\end{proof}

\begin{definition}
	A connection is \bam{flat} (or \bam{integrable}) if its curvature is 0. 
\end{definition}

\begin{theorem}
	If $P$ is a $G$-principal bundle equipped with a flat connection then there exist local trivialisations such that all transition functions are constant. 
\end{theorem}
\begin{proof}
	\hl{exercise}
\end{proof}
%%%%%%%%%%%%%%%%%%%%%%%%%%%%%%%%%%%%%%%%%%%%%%%%%%%%%%%%
\subsubsection{Local expressions}

Recall that a bundle is trivial iff it has a global section $\sigma : M \to P$. This gives a canonical choice of connection given by 
\eq{
H = \ker(p_2 : M \times G \to G)
}
We refer to this induced connection as $d^\sigma$. \\
Now any two connections differ by a section of $T^\ast M \otimes \ad(P)$, and $\ad(P)$ is also trivialised as $\ad(P) \cong M \times \mf{g}$ if $P$ is, we can write any other connection as 
\eq{
d^\sigma + A
}
If the connection is given by one-form $\omega$ we have $A = \sigma^\ast \omega$. \\
Suppose we have expressed our $G$-principal bundle $P$ in terms of local trivialisations - that is we have written $M = \cup_\alpha U_\alpha$, taken $\Psi_\alpha : \pi^{-1}(U_\alpha) \overset{\cong}{\to} U_\alpha \times G$ with transition maps $\phi_{\alpha\beta}:U_{\alpha\beta} \to G$, and we have expressed this data as a collection of local sections $\sigma_\alpha : U_\alpha \to P$. In this case we may carry out the proceedure as before to write a connection locally as 
\eq{
d^{\sigma_\alpha} + A^\alpha
}
These are related according to the following results:
\begin{theorem}[compatibiltity conditions]
	Let $\theta$ be the LI MC one-form. Then 
	\eq{
A^\beta = \Ad_{\phi_{\beta\alpha}}(A^\alpha) + \phi_{\alpha\beta}^\ast \theta	
}
\end{theorem}
\begin{proof}
\hl{exercise}
\end{proof}

\begin{corollary}
	If $G$ is a matrix group, we have 
	\eq{
A^\beta = \phi_{\beta\alpha} A^{\alpha} \phi_{\beta\alpha}^{-1} + \phi_{\beta\alpha} d\phi_{\beta\alpha}	
}
\end{corollary}

\begin{remark}
	Physicists often refer to the $A^\alpha$ as \bam{local gauge potentials}. 
\end{remark}

\begin{prop}
	In terms of the local gauge potentials the cruvature is given by 
	\eq{
F = dA^\alpha + A^\alpha \wedge A^\alpha	
}
\end{prop}


Recall we know there is a group isomorphism between $\mc{G}(P)$, the group of gauge transformations, and $\Gamma(\Ad(P))$, the group of section of the bundle $\Ad(P)$. If we trivialise $P$ we trivialse $\Ad(P)$ making the sections into $G$-valued functions. The local trivialisations, $G$-valued functions $\psi_\alpha$, are related by 
\eq{
\psi_\beta = \phi_{\beta\alpha} \psi_\alpha \phi_{\beta\alpha}^{-1}
}
\begin{prop}
$\mc{G}(P)$ then acts locally by the compatibility condition on the space of all connections as 
\eq{
A^\alpha \mapsto \Ad_{\psi_\alpha} A^\alpha + \psi_\alpha^\ast \theta
}
\end{prop}

%%%%%%%%%%%%%%%%%%%%%%%%%%%%%%%%%%%%%%%%%%%%%%%%%%%%%%%%
%%%%%%%%%%%%%%%%%%%%%%%%%%%%%%%%%%%%%%%%%%%%%%%%%%%%%%%%
\subsection{Characteristic Classes in Chern-Weil Theory}

\begin{lemma}
Suppose we have 
\eq{
\phi : \underbrace{\mf{g} \otimes \dots \otimes \mf{g}}_{\times k} \to \mbb{R}
}
multilinear, symmetric, and ad-invariant. Think of $\phi(\tilde{F}, \dots, \tilde{F}) \in \Omega^{2k}(M,\mbb{R})$, then if $\omega, \omega^\prime$ are two connections
\eq{
\exists \alpha \in \Omega^{2k-1}, \, \phi(\tilde{F}_\omega, \dots, \tilde{F}_\omega) - \phi(\tilde{F}_{\omega^\prime}, \dots, \tilde{F}_{\omega^\prime}) = d\alpha 
}
\end{lemma}

\begin{corollary}
	$[\phi(\tilde{F}, \dots, \tilde{F})]\in H^{2k}_{dR}$ is a cohomology class. 
\end{corollary}

\begin{example}
	If $G=U(n)$, define $c_k$ by 
	\eq{
\sum_{k=0}^n c_k(X) t^k = \det \pround{I - \frac{t}{2\pi i}X}	
}
for $X \in \mf{u}(n)$ and then 
\eq{
c_k(P) = c_k (\tilde{F}, \dots, \tilde{F}) \in H^{2k}
}
for some $G$-bundle p, the \bam{Chern class}. 
\end{example}

\begin{remark}
	Here we are using the polarisation identity to define $c_k$ on multiple elements. 
\end{remark}

\begin{example}
	If $G= O(n), SO(n)$ then have 
	\eq{
\sum_k p_k(X) t^k = \det \pround{I-\frac{t}{2\pi}X} 	
}
for $X \in \mf{g}$, and 
\eq{
p_k(P) = p_k(\tilde{F}, \dots, \tilde{F}) \in H^{4k}
}
is the \bam{Ponytryagin class}
\end{example}

\begin{example}
	If $G=O(2n)$, define 
	\eq{
\Delta(X) = \pround{\frac{-1}{2\pi}}^m \Pfaff(X)	
}
The \bam{Euler class} 
\end{example}
%%%%%%%%%%%%%%%%%%%%%%%%%%%%%%%%%%%%%%%%%%%%%%%%%%%%%%%%
%%%%%%%%%%%%%%%%%%%%%%%%%%%%%%%%%%%%%%%%%%%%%%%%%%%%%%%%
\section{The Yang-Mills Theory}
%%%%%%%%%%%%%%%%%%%%%%%%%%%%%%%%%%%%%%%%%%%%%%%%%%%%%%%%
\subsection{Implicitly used results}

\hl{Maybe move this earlier}
We will first start by spelling out explicitly some results that we have been using but not making clear:

\begin{lemma}
	If $P \to M$ is a $G$-principal bundle, $V \to P$ is a vector bundle equipped with a linear lift of the action $P \ract G$ then $\faktor{V}{G}$ is a vector bundle over $M$ of the same rank as $V$ over $P$. 
\end{lemma}

\begin{lemma}
	Sections of the bundle $\faktor{V}{G} \to M$ correspond exactly to $G$-equivarant sections of $V \to P$, i.e. sections $s:V\to P$ s.t. 
	\begin{tkz}
	V \arrow[r,"g"] \arrow[d] & V \arrow[d] \\ P \arrow[r,"g"'] \arrow[u,bend left=20,"s"] & P \arrow[u,bend right=20, "s"']
	\end{tkz} 
\end{lemma}

\begin{lemma}
	pull-backs of sections of $\wedge^2 T^\ast M \otimes \faktor{(P \times V)}{G}$ are exactly basic $V$-valued forms on $P$. 
\end{lemma}

The reason we have made these explicit is as it's important to realise it is the case with $V= \mf{g}$ that is necessary for us, as it means we can descend $\tilde{F} \in \Gamma(\wedge^2 T^\ast P \otimes \mf{g})$ (a $\mf{g}$-valued form on $P$) to $F\in \Gamma(\wedge^2 T^\ast M \otimes \ad(P))$. 


%%%%%%%%%%%%%%%%%%%%%%%%%%%%%%%%%%%%%%%%%%%%%%%%%%%%%%%%
\subsection{Theory}
\subsubsection{Yang-Mills functional}
We are now going to restrict our mathematical picture. We will make the following assumptions:
\begin{itemize}
	\item $M$ is an oriented (pseudo-)Riemannian manifold
	\item $\mf{g}$ has a Euclidean inner product invariant under the adjoint action $G \lact \mf{g}$. (This gives a Eulidean structure on the bundle $\ad(P)$). 
\end{itemize}

\begin{remark}
For compact, semi-simple Lie groups an $\ad$-invariant Euclidean inner product on $\mf{g}$ is given by the Killing form. For other compact Lie groups we can take a faithful representation $\rho : G \to GL(V)$ and then have 
\eq{
\pangle{X,Y} = -\tr(d\rho(X) \circ d\rho(Y))
}
For simplicity we will notate this as 
\eq{
\pangle{X,Y} = -\tr(XY)
}
\end{remark}

What we have gained is that the bundle which the curvature takes values in has a Euclidean structure, as such we can make the following definition:

\begin{definition}
	The \bam{Yang-Mills function} is the functional $S_{YM}:\mc{C}_P \to \mbb{R}$ given by 
	\eq{
	S_{YM} = \int_M \abs{F}^2 \omega_{vol}
}
where $F$ is the curvature associated to the connection the functional is evalutated at. 
\end{definition}

\begin{lemma}
	$S_{YM}$ is invariant under the action of the gauge group $\mc{G}(P)$. 
\end{lemma}

\begin{corollary}
	The functional becomes a well-defined function $S_{YM} : \faktor{\mc{C}_P}{\mc{G}(P)} \to \mbb{R}$. 
\end{corollary}

\begin{lemma}
	The Yang-Mills functional can be written as
	\eq{
S_{YM} \propto \int_M \tr(F \wedge \star F)	
}
\end{lemma}
\begin{proof}
	The inner product on $\mf{g}$-valued forms is the tensor product of the inner products on $\mf{g}$ and on $\Omega^\cdot$. Hence by the definition of the Hodge star
	\eq{
\abs{F}^2 \omega_{vol} = \pangle{F,F}\omega_{vol} = \pangle{F \wedge \star F}	
}
We often denote the inner product on $\mf{g}$ as proportional to trace to highlight the fact that $\ad$-invariant implies cyclicity of trace on a matrix Lie algebra. 
\end{proof}

\begin{remark}
	If $M$ is compact, the integration over $M$ is always fine. If not, we need to restrict to connections for which $S_{YM}$ is well-defined. 
\end{remark}

%%%%%%%%%%%%%%%%%%%%%%%%%%%%%%%%%%%%%%%%%%%%%%%%%%%%%%%%
\subsubsection{The Yang-Mills equations}

We now want to derive equations that give us the stationary points for the action (as is standard for an action theory). We will work with a local trivialisation, and then for notation simplicity omit the index $\alpha$. For some generic $\tau \in \Gamma(T^\ast M \otimes \ad(P))$ we want to find $A$ s.t.  
\eq{
\ev{\frac{d}{dt}S_{YM}(A+t\tau)}= 0
}
We can calculate the associated curvature to this new connection is 
\eq{
F_t &= d(A + t\tau)+ (A + t\tau) \wedge (A + t\tau) \\
&= F + t\pround{d\tau + A \wedge \tau + \tau \wedge A} + t^2 \tau \wedge \tau
}
We now need the following lemma
\begin{lemma}
	The covariant derivative descends to an operator 
	\eq{
	d^A : \Gamma(\wedge^r T^\ast M \otimes \ad(P)) \to \Gamma(\wedge^{r+1}T^\ast M \otimes \ad(P))
}
\end{lemma}
\begin{proof}
	By the definition that for $\phi \in \Gamma(\wedge^r T^\ast P \otimes V)$ 
	\eq{
D\phi (X_0, \dots, X_r) = d\phi(X_0^H, \dots, X_r^H)	
}
we can see that $D\phi$ is horizontal. It is also $G$-invariant if $\phi$ is, and as such $D$ maps basic forms to basic forms. As such it descends as previously discussed.
\end{proof}

\begin{corollary}
	The Bianchi identity descends to $d^A F = 0$. 
\end{corollary}

\begin{remark}
	We will want to make use of the fact throughout that for forms 
	\eq{
d^A \tau = d\tau + \psquare{A \wedge \tau}	
}
where $\psquare{\omega \wedge \eta} = \omega \wedge \eta - (-1)^{pq} \eta \wedge \omega$ where the wedge without square brackets of forms should just take the multiplication of their algebra element coefficients. Be aware that the covariant derivative acts differently on connections and this is why we still have 
\eq{
F = d^A A = dA + \frac{1}{2}\psquare{A \wedge A} = dA + A \wedge A
}
To see more on this check out \href{https://en.wikipedia.org/wiki/Lie_algebra-valued_differential_form}{Lie algebra-valued differential forms} on Wikipedia and the 2006 gauge theory notes by Jose (pg 18).  
\end{remark}
Now using that $d^A \tau  = d\tau + A \wedge \tau$ we get 
\eq{
F_t &= F + t d^A \tau + t^2 \tau \wedge \tau \\
\Rightarrow \abs{F_t}^2 &= \abs{F}^2 + t \pangle{d^A \tau, F} + \mc{O}(t^2)
}
Hence the Yang-Mills equation is equivalent to 
\eq{
\int_M \pangle{d^A \tau, F} \omega_{vol} = 0
}
Using the formal adjoint $(d^A)^\ast$ and realising $\tau$ is generic we get 
\eq{
(d^A)^\ast F = 0
}
We need some lemmas:
\begin{lemma} $ d\pangle{\beta \wedge \star F} = \pangle{d^A\beta \wedge \star F} - \pangle{\beta \wedge d^A(\star F)} $
\end{lemma}
\begin{proof}
	We will show this in coordinates, by writing $\beta = \beta_\mu dx^\mu, \, F = \frac{1}{2}F_{\mu\nu} dx^\mu \wedge dx^\nu$. Then 
	\eq{
		\star F = \frac{1}{4} F_{\mu\nu} \eps\indices{^\mu^\nu_\rho_\sigma} dx^\rho \wedge dx^\sigma 	
	}
	and we get 
	\eq{
		d\pangle{\beta \wedge \star F} &= \frac{1}{4} \del_\mu \pangle{\beta_\nu, F_{\rho\sigma}} \eps\indices{^\rho^\sigma_\tau_\lambda} dx^\mu \wedge dx^\nu \wedge dx^\tau \wedge dx^\lambda \\
		\pangle{d^A\beta \wedge \star F} &= \frac{1}{4} \pangle{\del_\mu \beta_\nu + \comm[A_\mu]{\beta_\nu}, F_{\rho\sigma}} \eps\indices{^\rho^\sigma_\tau_\lambda} dx^\mu \wedge dx^\nu \wedge dx^\tau \wedge dx^\lambda \\
		\pangle{\beta \wedge d^A(\star F)} &= \frac{1}{4} \pangle{\beta_\nu, \del_\mu F_{\rho\sigma} + \comm[A_\mu]{F_{\rho\sigma}}} \eps\indices{^\rho^\sigma_\tau_\lambda} dx^\nu \wedge dx^\mu \wedge dx^\tau \wedge dx^\lambda
	}
	which means 
	\eq{
		d\pangle{\beta \wedge \star F} - \pangle{d^A\beta \wedge \star F} + \pangle{\beta \wedge d^A(\star F)} = \frac{-1}{4} \psquare{\pangle{\beta_\nu,\comm[A_\mu]{F_{\rho\sigma}}}+\pangle{\comm[A_\mu]{\beta_\nu},F_{\rho\sigma}}}\eps\indices{^\rho^\sigma_\tau_\lambda} dx^\mu \wedge dx^\nu \wedge dx^\tau \wedge dx^\lambda 
	}
	and we have asked as the inner product on $\mf{g}$ is $\ad$-invariant, which precisely says that $\forall X,Y,Z \in \mf{g}$, 
	\eq{
		\pangle{\comm[Z]{X},Y} + \pangle{X,\comm[Z]{Y}} = 0
	}
\end{proof}
\begin{remark}
	No part of the above calculation required properties of $\beta, F$, other than that they were $\mf{g}$-valued. This means that this result holds for general forms with the modification that if $\alpha \in \Omega^p(M), \, \beta \in \Omega^q(M)$, then 
	\eq{
d\pangle{\alpha \wedge \beta} = \pangle{d^A \alpha \wedge \beta} + (-1)^p \pangle{\alpha \wedge d^A \beta}	
}
\end{remark}

\begin{lemma}
	Let $d^\ast$ be the formal adjoint of $d$ (the \bam{codifferential}) wrt the inner product on the subset of $\Omega^k(M)$ that decays sufficiently if $M$ is non-compact given by
	\eq{
	\pangle{\pangle{\eta,\zeta}} = \int_M \pangle{\eta,\zeta} \omega_{vol}
}
that is, if $\eta \in \Omega^{k-1}(M), \, \zeta \in \Omega^{k}(M)$, $\pangle{\pangle{d\eta, \zeta}} = \pangle{\pangle{\eta,d^\ast \zeta}}$. We can express $d^\ast : \Omega^k(M) \to \Omega^{k-1}(M)$ as 
	\eq{
d^\ast = (-1)^{n(k-1)+1}s \star d \star	
}
where $n=\dim M$, $s$ is the parity of the signature of $M$, 
\end{lemma}
\begin{proof}
We prove this more generally for the covariant derivative. Using the previous lemma means 
\eq{
	\pangle{\pangle{d^A\eta, \zeta}} = \int_M \pangle{d^A\eta,\zeta} \omega_{vol} &= \int_M \pangle{d^A \eta \wedge \star \zeta} \\
	&= \int_M d\pangle{\eta \wedge \star \zeta} - (-1)^{k-1} \pangle{\eta \wedge d^A \star \zeta} \\
	&= \int_M (-1)^{(n-k+1)[n-(n-k+1)]+k}s\pangle{\eta,\star d^A \star \zeta}\omega_{vol} \\
	&= \pangle{\pangle{\eta,(-1)^{(n-k+2)(k-1)+1}s \star d^A \star \zeta }} \\
	&= \pangle{\pangle{\eta, (-1)^{n(k-1)+1}s \star d^A \star \zeta}}
}
where we have used the (abelian) Stokes' theorem and sufficient decay at infinity to set the boundary term to 0, and that as $d^A \star \zeta$ is a $(n-k+1)$-form. 
\end{proof} 

\begin{remark}
	This carefulness with signs is not important in the case of pure Yang-Mills, but if you consider Yang-Mills-Higgs this becomes necessary. Note another route of proof is possible if we use the non-abelian Stokes' theorem (see \cite{Schreiber2011} and nlab).
\end{remark}
As such we get the \bam{Yang-Mills equation}
\eq{
	d^A(\star F) = 0
}
%%%%%%%%%%%%%%%%%%%%%%%%%%%%%%%%%%%%%%%%%%%%%%%%%%%%%%%%
\subsection{Gauge theory in 4d}
\subsubsection{Abelian gauge theory and Electromagnetism}

Recall Maxwell's equations for electromagnetism 
\eq{
\bm{\nabla} \cdot \bm{E} &= 4\pi\rho \\
\curl \bm{B} &= \frac{4\pi}{c} \bm{J} + \frac{1}{c} \del_t \bm{E} \\
\curl \bm{E} &= -\frac{1}{c} \del_t \bm{B} \\
\bm{\nabla} \cdot \bm{B} &= 0 
}
We are going to see how this can be derived as a Yang-Mills gauge theory. \\
Restricting to the vacuum Maxwell equations where $\rho = 0, \, \bm{J} = 0$, we can build a two form 
\eq{
F = \tilde{B} - c dt \wedge E
}
where 
\eq{
E &= E_x dx + E_y dy + E_z dz \\
B &= B_x dx + B_y dy + B_z dz \\
 \tilde{B} = \star_3 B &= B_x dy \wedge dz + B_y dz \wedge dx + B_z dx \wedge dy 
}
are the natural one and two forms associated to a vector in 3d (using the 3d Hodge star to get the two form) written in cartesian coordinates. Now consider a $U(1)$-principal bundle $P$ over $\mbb{R}^4$ with Lorentzian signaure. As the Lie group is abelian, the commutator in the Lie algebra is 0 and $d^A = d$ is the standard differential for any connection. We now calculate
\eq{
dF &= d\tilde{B} + c dt \wedge dE \\
&= (\bm{\nabla}\cdot \bm{B})dx \wedge dy \wedge dz + \psquare{\del_t B_x + c(\del_y E_z - \del_z E_y)}dt \wedge dy \wedge dz \\
&\phantom{=} + \psquare{\del_t B_y + c(\del_z E_x - \del_x E_z) } dt \wedge dz \wedge dx + \psquare{\del_t B_z + c(\del_x E_y - \del_y E_x )} 
}
so this gives us two of the Maxwell equations, namely $\bm{\nabla} \cdot \bm{B}= 0$ and $\curl \bm{E} = -\frac{1}{c} \del_t \bm{B}$. Then
\eq{
\star_4 F &= c \tilde E + dt \wedge B
}
We can then see the duality between $E$ and $B$ and we get the other two equations from $d(\star F)=0$. \\
As the base manifold is simply connected it has trivial cohomology, and any closed two-form is exact. Hence $\exists \mc{A}$ s.t. $F = d\mc A$. Write 
\eq{
\mc A = -\phi dt + A
} 
where $A$ is the one form that corresponds to a vector $\bm{A}$ in 3d. We then recover the formula from electromagnetism that 
\eq{
\bm{B} &= \curl \bm{A} \\
\bm{E} &= - \grad\phi - \del_t \bm{A}
}
$\mc{A}$ is the connection for which $F$ is the curvature, and so we know we are free to apply a gauge transformation, under which (if $\phi_{\beta\alpha}=e^{i\psi}$) we get 
\eq{
\mc{A} \mapsto \mc{A} - i \, d\psi
}
%%%%%%%%%%%%%%%%%%%%%%%%%%%%%%%%%%%%%%%%%%%%%%%%%%%%%%%%
\subsubsection{Instantons}

We now want to consider $M=\mbb{R}^4$ with Euclidean signature (in practice this might be achieved though a Wick rotation of a Lorentzian metric if necessary). We will, as previously stated, be considering configurations which decay sufficiently quickly to infinity. This means we can add the point at infinity to $M$ to obtain $S^4$.

\begin{remark}
	Although $\mbb{R}^4$ does not have an isometric embedding into $S^4$, it does have a conformal embedding, and this will turn out to be sufficient, because the defining conditions (namely the self duality ones) are conformally invariant.  
\end{remark}
We can use the Hodge star to decompose the Yang-Mills functional as 
\eq{
S_{YM} = \int_M \abs{F}^2 \omega_{vol} = \int_M \abs{F_+}^2 \omega_{vol} + \int_M \abs{F_-}^2 \omega_{vol} 
}

Now we make the following definition
\begin{definition}
	We call 
	\eq{
c = \int_M \tr(F \wedge F)	
}
the \bam{second Chern number}
\end{definition}	
	
\begin{prop}
	$c$ depends only on the principal bundle and not the choice of connection. 
\end{prop}
	
\begin{lemma}
	$c = \int_M \abs{F_+}^2 \omega_{vol} - \int_M \abs{F_-}^2 \omega_{vol}$
\end{lemma}
\begin{proof}
Expanding $F = F_+ + F_-$ and using $\tr(F \wedge \star F) = \abs{F}^2 \omega_{vol}$ gives the result. 
\end{proof}
	
Hence we get the inequality 
\eq{
S_{YM} \geq \abs{c}
}
with equality iff either $F= F_+$ or $F=F_-$. Note in either case the Yang-Mills equations are trivially satisfied as 
\eq{
d^A (\star F) = \pm d^A F  = 0 \quad \text{by the Bianchi identity}
}

\begin{definition}
	An (anti-)instanton is a solution to the (anti-)self-dual euqations which can be extended to $S^4$. 
\end{definition}

We want to understand instantons up to gauge transformation, and the space parameterising these is the \bam{moduli space of instantons}. \\
If we now take recognise that $\mf{su}(2) \cong \mf{sp}(1) \cong \image \mbb{H}$ we can use the quaternion norm on our Lie algebra. In this normalisation we get the following results:
\begin{lemma}
	The \bam{1st Pontryagin Class} on the bundle $\ad P$ is 
	\eq{
P_1(\ad P) = \frac{c}{4\pi^2} \in \mbb{Z} 	
}
\end{lemma}

\begin{example}[BPST Instanton]
	Let us think of $M=\mbb{R}^3 = \mbb{H}$ by the identification $\bm{x} \in \mbb{R}^4 \leftrightarrow x = x_1 i + x_2 j + x_3 k + x_4 \in \mbb{H}$. Note $x\bar{x} = \abs{\bm{x}}^2$ and 
	\eq{
G = SU(2) \cong Sp(1) = \pbrace{u \in \mbb{H} \, | \, u\bar{u} = 1}	
}
We take a connection $A \in \Omega^1(\mbb{R}^4; \mf{g})$ given by 
\eq{
A(x) &= \frac{1}{\abs{x}^2+1} \image(x d\bar{x})  \\
\Rightarrow F &= \frac{1}{(\abs{x}^2+1)^2} dx \wedge d\bar{x}
}
We can check that $F$ is self-dual, by finding 
\eq{
dx \wedge d\bar{x} = -2 \psquare{(dx_{12} + dx_{34})i + (dx_{13}-dx_{24})j + (dx_{14}+dx_{23})k}
} 
and so corresponds to an instanton. The corresponding Pontryagin class is 
\eq{
P_1 &= \frac{1}{4\pi^2} \int_{\mbb{R}^4} \abs{F}^2 d^4x \\
&= \frac{1}{4\pi^2} \int_{\mbb{R}^4} \frac{4}{(\abs{x}^2+1)^4} \abs{(dx_{12} + dx_{34})i + \dots }^2 d^4x \\
&= \frac{1}{4\pi^2} \int_{\mbb{R}^4} \frac{4}{(\abs{x}^2+1)^4} \psquare{\abs{dx_{12} + dx_{34}}^2 + \dots } d^4 x 
}
We can check terms like 
\eq{
\abs{dx_{12}+dx_{34}}^2 d^4 x = (dx_{12} + dx_{34}) \wedge \star (dx_{12} + dx_{34}) = 2dx_{1234}
}
and so 
\eq{
P_1 = \frac{6}{\pi^2} \int_{\mbb{R}^4} \frac{d^4x}{(\abs{x}^2+1)^4} = \frac{6}{\pi^2} \int_0^\infty \frac{r^3 dr}{(r^2+1)^4} \cdot V(S^3) = 1
}
We can introduce parameters through a conformal transform $x \mapsto \lambda^{-1}(x-x_0)$ for $\lambda \in \mbb{R}^\times, \, x_0 \in \mbb{R}^4$. As $\star$ is a confomally invariant transform we have that the transformed $F$ is still self-dual. The transformed $A$ is 
\eq{
A^\prime(x) = \frac{1}{\abs{x-x_0}^2 + \lambda^2} \image \psquare{(x-x_0)d\bar{x}}
}
and by homotopy invariance, when $\lambda>0$ we must have that $A^\prime$ again has instanton number 1. If we consider the limit as $\lambda \to 0$, the small instanton limit, the instanton concentrates at $x_0$. This idea allows one to study the manifold the instanton lies on through 
\end{example}


%%%%%%%%%%%%%%%%%%%%%%%%%%%%%%%%%%%%%%%%%%%%%%%%%%%%%%%%
%%%%%%%%%%%%%%%%%%%%%%%%%%%%%%%%%%%%%%%%%%%%%%%%%%%%%%%%
\section{Chern-Simons Theory}
%%%%%%%%%%%%%%%%%%%%%%%%%%%%%%%%%%%%%%%%%%%%%%%%%%%%%%%%
\subsection{Primers}
\subsubsection{Symplectic Manifolds in Gauge Theory}
Let $\Sigma$ be a compact oriented surface, $G$ a compact Lie group, $P$ a principal $G$-bundle on $\Sigma$, $\pangle{\cdot,\cdot}$ a $G$-invariant inner product on $\mf{g}$.

\begin{prop}
	$\mc{C}_P$ is an $\infty$-dimensional symplectic manifold. 
\end{prop}
\begin{proof}
	As $\mc{C}_P$ is an affine space modelled on $\Gamma(T^\ast \Sigma \otimes \ad(P))$, we have that all the tangent spaces to $\mc{C}_P$ are canoncially isomorphic to $\Gamma(T^\ast \Sigma \otimes \ad(P))$. Here we can take 
	\eq{
\omega(A,B) = \int_\Sigma \pangle{A \wedge B}	
}
\begin{claim}
	$\omega$ is a well-defined symplectic form. 
\end{claim}
\end{proof}

\begin{theorem}[Atiyah, Bott]
	The action of $\mc{G}(P)$ on $\mc{C}_P$ is Hamiltonian with moment map given by curvature 
	\eq{
\nabla \mapsto F_\nabla \in \Gamma(\wedge^2 T^\ast \Sigma \otimes \ad(P)) \cong \Gamma(\ad(P))^\ast 	
}
\end{theorem}

\begin{definition}
	The \bam{moduli space of flat connections} is 
	\eq{
\mc{M}_P = \faktor{F^{-1}(0)}{\mc{G}(P)}	
}
\end{definition}

\begin{remark}
	$\mc{M}_P$ is not usually smooth everywhere, but smooth on an open and dense locus.
\end{remark}

\begin{prop}
	$\mc{M}_P$ is finite dimensional. 
\end{prop}

\begin{example}
	If $G$ is semisimple, then $\dim \mc{M}_P = (2g-2)\dim G$, where $g$ is the genus of $\Sigma$.  
\end{example}

\begin{remark}
	$\mc{M}_P$ also arises as 
	\begin{itemize}
		\item the moduli space of representations of $\pi_1(\Sigma)$ into $G$, i.e 
		\eq{
	\mc{M}_P = \faktor{\Hom(\pi_1(\Sigma),G)}{G}	
	}
\item the moduli space of semi-stable holomorphic $G_\mbb{C}$-bundle on $\Sigma$, now as a Riemann surface. 
	\end{itemize}
\end{remark}

\subsubsection{Geometric Quantisation}
Starting with symplectic manifold $(M,\omega)$, we want to create a Hilbert space. We will need 3 ingredients:
\begin{enumerate}
	\item Pre-quantum line bundle: $d\omega=0$ so we know $[\omega] \in H^\cdot_{dR}(M)$. We will want this to be integral, i.e. it lies in the lattice image of $H^\cdot_{dR}(M;\mbb{Z}) \to H^\cdot_{dR}(M;\mbb{R})$. Given this condition we have the following
	\begin{lemma}
		If $\omega$ satisfies the integrality condition, $\exists \mc{L}\to M$ a line bundle with connection whose curvature is $\omega$. This is called the \bam{pre-quantum line bundle}. 
	\end{lemma}
\item Choice of polarisation: e.g a Lagrangian foliation on $M$, or K\"ahler structure. 
\item Metaplectic correction: Given the canonical bundle $K$ we take $\sqrt{K}$. 
\end{enumerate}
Given the above, we will take the Hilbert space to be sections of $\mc{L} \otimes \sqrt{K}$ that respect polarisation (e.g. sections covariantly constant along leaves in the case of a Lagrangian foliation, or holomorphic sections when a K\"ahler structure specified).

%%%%%%%%%%%%%%%%%%%%%%%%%%%%%%%%%%%%%%%%%%%%%%%%%%%%%%%%
\subsubsection{Chern-Simons Forms}
We will now assume that we have a $G$-principal bundle $\pi : P\to M$. We will assume that $G=SU(n)$ and $P$ is trivialisable. 

\begin{remark}
	If $\dim M \leq 3$, $P$ is always trivialisable. 
\end{remark}
Wehave a correspondence between connections and $\mf{g}$-valued one forms $A \in \Omega^1(M;\mf{g})$,and so we can make the following definition:
\begin{definition}
	The \bam{Chern-Simons form} is 
	\eq{
CS : \mc{C}_P &\to \Omega^3(M) \\
 A &\mapsto \frac{1}{4\pi^2}\tr\pround{dA \wedge A + \frac{2}{3} A \wedge A \wedge A}
}
\end{definition}

\begin{lemma}
	$d(CS(A)) = \tr(F \wedge F)$. 
\end{lemma}
\begin{remark}
	Up to normalisation this is the second Chern class $c_2(P)$, and so if $P$ is trivial we have $c_2(P)=0 \Rightarrow CS(A)$ closed. 
\end{remark}

Alternatively we may pullback by $\pi$ to get the bundle $\pi^\ast P \to P$, and this has a tautological section, giving a tautological trivialisation. Now we can define a Chern-Simons form on $M$ if we have a connection on $P$ as 
\eq{
s^\ast \pround{CS(\pi^\ast \nabla)}
}
for a local section $s:M \to P$. Note here $CS$ is defined on $\mc{C}_{\pi^\ast P}$. We have the following idea:
\begin{idea}
	All characteristic classes being zero implies that all Chern-Weil forms are exact. The Chern-Simons forms are built to be exactly the forms s.t. the solve $d(CS) = CW$.  
\end{idea}
%%%%%%%%%%%%%%%%%%%%%%%%%%%%%%%%%%%%%%%%%%%%%%%%%%%%%%%%
\subsection{Topological Quantum Field Theory}
\begin{idea}
The idea now is the use the CS form as the Lagrangian for a quantum field theory. This will allow us to understand invariants of 3-manifolds. This is motivated by Jones' knot polynomial. \\
The rough idea of this will be that to a closed oriented 2-manifold $\Sigma$ we give a vector space $V_\Sigma$, and thn for oriented 3-manifolds we will construct $v \in V_{\del M}$. 
\end{idea}
%%%%%%%%%%%%%%%%%%%%%%%%%%%%%%%%%%%%%%%%%%%%%%%%%%%%%%%%
\subsubsection{Chern Simons Line Bundle}
Return to having $\Sigma$ be a compact oriented surface, $G$ a compact Lie group, $P$ a principal $G$-bundle on $\Sigma$, $\pangle{\cdot,\cdot}$ a $G$-invariant inner product on $\mf{g}$. \\
We want to get a line bundle on $\mc{M}_P$ s.t. the 1st chern class of $\mc{L}$ is the de-Rham cohomology class of $\omega$. \\
To do this construction, let $M$ be an oriented 3-manfiold s.t. $\del M = \Sigma$ 
\begin{lemma}
	Such an $M$ exists. 
\end{lemma}
\begin{lemma}
	$A \in \mc{C}_P, \, g \in \mc{G}(P)$ can extent to $\tilde{A} \in \mc{C}_M, \, \tilde{g} \in \mc{G}(M)$. 
\end{lemma}
We want to lif the action of $\mc{G}(P)$ on $\mc{C}_P$ to the trivial line bundle $\mc{C}_P \times \mbb{C}$ in a non-trivial way using the co-cycle
\eq{
\Theta(A,g) = \exp\pround{2\pi i \int_M CS(\tilde{A}^{\tilde{g}}) - CS(\tilde{A})}
}
\begin{prop}
	$\Theta$ is well defined and $\Theta(A,gh) = \Theta(A,g)\Theta(A^g,h)$. 
\end{prop}
\begin{corollary}
	We can define a lift of the action by $(A,z) \cdot g = (A^g,\Theta(A,g)z)$. 
\end{corollary}
This line bundle descends to a non-trivial line buundle $\mc{L}_{CS}$ on $\mc{M}_P$. 
\begin{prop}
	$\mc{L}_{CS}$ is a pre-quantum line bundle for the Atiyah-Bott symplectic structure on $\mc{M}_P$. 
\end{prop}

\subsubsection{Lagrangian Submanifolds in \secmath{\mc{M}_P}}
Now we want to thing about restricting $\mc{M}_M$ to $\mc{M}_{\del M}$. 
\begin{lemma}
	The map $\mc{G}(M) \to \mc{G}(\del M)$ is surjective
\end{lemma}
\begin{corollary}
	This induces a map $\mc{M}_M \to \mc{M}_{\del M}$.
\end{corollary}

\begin{theorem}
	The image of the induced map $\mc{M}_M \to \mc{M}_{\del M}$ is a Lagrangian submanifold in $\mc{M}_{\del M}$. 
\end{theorem}

\subsubsection{Axioms for TQFTs}
Atiyah-Segal gave an axiomatic approach to  2+1 TQFT: 
\begin{definition}
	The 2+1 TQFT axioms are 
	\begin{itemize}
		\item Assigns a vector space $Z(\Sigma)$ to every closed oriented 2-manifold $\Sigma$
		\item If $M$ is a 3-manfiodl s.t. $\del M = \Sigma$ assign $Z(M) \in Z(\Sigma)$
		\item If you change the orientation you take the dual, i.e $Z(\bar{\Sigma}) = Z(\Sigma)^\ast$
		\item $Z\pround{\Sigma_1 \sqcup \Sigma_2} = Z(\Sigma_1) \otimes Z(\Sigma_2)$
		\item $Z(\Sigma = \emptyset)=\mbb{C}$ and $Z(M=\emptyset) = 1 \in \mbb{C}$. 
	\end{itemize}
\end{definition}

\begin{example}
	Consider the 'cyclinder' $M = [0,1] \times \Sigma$. Then $\del M = \Sigma \sqcup \bar{\Sigma}$, $Z(\del M) = Z(\Sigma) \otimes Z(\Sigma)^\ast \cong \End(Z(\Sigma))$. We interpret this as saying that time evolution along this cylinder is a map from one boundary vector space to the other.
\end{example}
Geometrically quantised Chern-Simons satisfies these axioms, and so we can view it as a TQFT. 

%%%%%%%%%%%%%%%%%%%%%%%%%%%%%%%%%%%%%%%%%%%%%%%%%%%%%%%%
%%%%%%%%%%%%%%%%%%%%%%%%%%%%%%%%%%%%%%%%%%%%%%%%%%%%%%%%
\bibliographystyle{../../bib/custom-bib-style}
\bibliography{../../bib/jabref_library.bib}


\end{document}