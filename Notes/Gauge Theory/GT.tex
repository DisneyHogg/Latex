\documentclass{article}

\usepackage{../../header-colourful}
%%%%%%%%%%%%%%%%%%%%%%%%%%%%%%%%%%%%%%%%%%%%%%%%%%%%%%%%
%Preamble

\title{Gauge Theory Notes}
\author{Linden Disney-Hogg}
\date{January 2020}

%%%%%%%%%%%%%%%%%%%%%%%%%%%%%%%%%%%%%%%%%%%%%%%%%%%%%%%%
%%%%%%%%%%%%%%%%%%%%%%%%%%%%%%%%%%%%%%%%%%%%%%%%%%%%%%%%
\begin{document}

\maketitle
\tableofcontents

%%%%%%%%%%%%%%%%%%%%%%%%%%%%%%%%%%%%%%%%%%%%%%%%%%%%%%%%
%%%%%%%%%%%%%%%%%%%%%%%%%%%%%%%%%%%%%%%%%%%%%%%%%%%%%%%%
\section{Introduction}
These are lecture notes taken from "Topics in Mathematical Physics". It may go poorly trying to type this live but let's see. The course will be covering essentially differential geometric aspects of gauge theory
%%%%%%%%%%%%%%%%%%%%%%%%%%%%%%%%%%%%%%%%%%%%%%%%%%%%%%%%
\subsection{Historical Overview}
\subsubsection{Physics}
Starting with Maxwell's equations in the 19th century for describing classical electromagnetism, these are differential equations for some fields. These were reformulated by Weyl in the 1920-30s, and the term gauge theory was coined. Namely, electromagnetism was reformulated as a classical $U(1)$ gauge theory. The Aharonov-Bohm theory gave that this extra degree of freedom corresponding to the vector potential has a physical significance for quantum mechanics. In the 1950s, Yang \& Mills gave the equations for gauge theories with arbitrary non-abelian Lie groups. The physical relevance of the commutativity is that non-abelian groups - when quantised - lead to particles which self interact. Note that this is not the case for photons (which correspond to $U(1)$). In the 1960-70s it was shown that non-abelian gauge theories can be renormalised, so correctly quantised, and this allows the standard model to be developed. This is a quantum gauge theory with group $U(1)\times SU(2) \times SU(3)$.

\subsubsection{Maths}
In the 1930-60s the theory of the principal bundle was developed, and this included the concept of connections. In the 1970s it was realised that these are the same constructs as being developed in physics. In the late 70s the concept of moduli space of instantons was being developed, and in the 80s the Donaldson invariant theory was being worked on. These are invariants of differential manifolds, and led to the discovery of exotic $\mbb{R}^4$. This continued to develop ideas of Seiberg-Witten invariants in the 90s. 
%%%%%%%%%%%%%%%%%%%%%%%%%%%%%%%%%%%%%%%%%%%%%%%%%%%%%%%%
\subsection{Course direction}
The course will be slightly dictated by the will of the class, but the basic structure will be as follows:
\begin{itemize}
    \item Review of manifolds and De Rham cohomology
    \item Riemannian and symplectic geometry, including symplectic reduction
    \item Lie groups, Lie algebras, and basic rep theory
    \item Principal bundles, associated vector bundles, fibre bundles
    \item Connections from different points of view 
    \item Chern-Weil theory, characteristic classes, and classifying spaces
    \item Yang Mills functional, electromagnetism as a gauge theory, and the Aharonov-Bohm experiment
    \item Gauge theory in dimensions 2,3 and 4, instantons, Chern-Simons theory
\end{itemize}
%%%%%%%%%%%%%%%%%%%%%%%%%%%%%%%%%%%%%%%%%%%%%%%%%%%%%%%%
\subsection{Suggested Reading}
Possible texts to look at include: \\
Mathematically sound, though written for physicists - 
\begin{itemize}
    \item Nakahara, "Geometry, topology \& physics" 
    \item Nash, "Topology and geometry for physicists" 
    \item Frankel, "Geometry of physics" 
\end{itemize}
Written for mathematicians - 
\begin{itemize}
    \item Marcathe \& Martucci, "The mathematical formulation of gauge theory"
    \item Naber, "Topology, geometry, and gauge fields"
\end{itemize}
Other options:
\begin{itemize}
	\item Morgan, "Intro to gauge theory" in Gauge theory and the topology of 4-manifolds
	\item Nicolaescu, "Lectures on the geometry of manifolds". 
\end{itemize}
%%%%%%%%%%%%%%%%%%%%%%%%%%%%%%%%%%%%%%%%%%%%%%%%%%%%%%%%
%%%%%%%%%%%%%%%%%%%%%%%%%%%%%%%%%%%%%%%%%%%%%%%%%%%%%%%%
\subsection{Prelim Tools}
%%%%%%%%%%%%%%%%%%%%%%%%%%%%%%%%%%%%%%%%%%%%%%%%%%%%%%%%
\subsubsection{Quaternions}
Here we will cover just some separate material that will be necessary for some comprehension, but is very separate. 

\begin{definition}
	The \bam{quaternions}, denoted $\mbb{H}$, is the real associative algebra spanned by $\pbrace{1,i,j,k}$ subject to 
	\eq{
i^2=-1=j^2, \; ij=k = -ji	
}
For $q=xi+yj+zk+w\in \mbb{H}$ we define its \bam{real part} as $\real(q) = w$ and as such its \bam{quaternionic conjugate} by 
\eq{
\bar{q} = -xi-yj-zk+w
}
\end{definition}

\begin{lemma}
	We can deduce the following relations:
	\begin{itemize}
		\item $k^2=-1$
		\item $jk = i = -kj$
		\item $ki = j = -ik$
		\item $\bar{q_1 q_2} = \bar{q}_2 \bar{q}_1$. 
	\end{itemize}
\end{lemma}
\begin{proof}
	These will all be calculations:
	\begin{itemize}
		\item $k^2 = -ij^2i = i^2=-1$
		\item $jk = -j^2i = i$ and likewise for $kj$
		\item $ki = -ji^2 = j$ and likewise for $ik$.
		\item If we let $q_a = x_a i + y_a j + z_a k + w_a$ then 
		\eq{
q_1 q_2 &= x_1x_2 i^2 + y_1y_2 j^2 +	z_1 z_2 k^2 + w_1 w_2 + ij(x_1y_2 - x_2y_1) + jk(y_1 z_2 - y_2 z_1) + ki(z_1 x_2 - z_2 x_1) \\
&\phantom{=} + i(x_1 w_2 + x_2w_1) + j(y_1 w_2 + y_2 w_1) + k(z_1 w_2 + z_2 w_1) \\
&= w_1w_2 - x_1 x_2 - y_1 y_2 - z_1 z_2 + i [(x_1 w_2 + x_2w_1)+(y_1 z_2 - y_2 z_1)] \\
&\phantom{=} + j[(y_1 w_2 + y_2 w_1) + (z_1 x_2 - z_2 x_1)] + k[(z_1 w_2 + z_2 w_1) + (x_1y_2 - x_2y_1)]
	}
We can see that taking the negative of the $i,j,k$ coefficients is equivalent to taking the negative of the $x_a,y_a,z_a$ and swapping $1 \leftrightarrow 2$. 
	\end{itemize}
\end{proof}

\begin{definition}
	Define an inner product on the quaternion space by $\pangle{q_1,q_2} = \real(\bar{q}_1 q_2)$ and the norm induced of the quaternions as $\norm{q} = \sqrt{\pangle{q,q}}=\sqrt{\bar{q}q}$ 
\end{definition}

\begin{lemma}
	$\pangle{q_1,q_2} = x_1x_2 + y_1y_2 + z_1z_2 + q_1w_2$. 
\end{lemma}
\begin{proof}
	Using the above calculation of $q_1,q_2$ the answer is immediate. 
\end{proof}

\begin{lemma}
If $q = xi + yj + zk + w$ then $\norm{q} = \sqrt{x^2 + y^2 + z^2 + w^2}$. Further, $\norm{q_1 q_2} = \norm{q_1}\norm{q_2}$. 
\end{lemma}
\begin{proof}
	Using the proof of the above lemma the first result is immediate. To prove the latter note $\norm{q_1 q_2} = \sqrt{\bar{q_2} \bar{q}_1 q_1 q_2}$, and as $\bar{q}_1 q_1$ is a positive real it will pull out of the square root. 
\end{proof}

%%%%%%%%%%%%%%%%%%%%%%%%%%%%%%%%%%%%%%%%%%%%%%%%%%%%%%%%
%%%%%%%%%%%%%%%%%%%%%%%%%%%%%%%%%%%%%%%%%%%%%%%%%%%%%%%%
\section{Review of Manifold Theory}
You will definitely need this knowledge to carry on into the connection material. Some possible reading if this is not sufficient (as is possible)
\begin{itemize}
	\item Taubes, "Differential geometry"
	\item Warner, "Foundations of differential manifolds and Lie groups
	\item Lee, "Introduction to smooth manifolds"
\end{itemize}
%%%%%%%%%%%%%%%%%%%%%%%%%%%%%%%%%%%%%%%%%%%%%%%%%%%%%%%%
\subsection{Initial definitions}
%%%%%%%%%%%%%%%%%%%%%%%%%%%%%%%%%%%%%%%%%%%%%%%%%%%%%%%%
\subsubsection{The Starting Point}
\begin{definition}
A \bam{topological manifold} of dimension $n$, $M$, is a topological space that is Hausdorff, second countable, paracompact, and locally homeomorphic to $\mbb{R}^n$. i.e. $\forall m \in M, \, \exists U \ni m$ such that $U$ open and homeomorphic to $\mbb{R}^n$. 
\end{definition}
Recall that topological manifolds can have further properties such as \bam{compact} or \bam{orientable} (Note that orientability is a well defined topological property, but for manifolds not general topological spaces). \\
Now we want to be able to do calculus on manifolds, but the standard definition of derivative of a real valued function $f$ does not immediately parse. Namely the concept of $f(x+\eps)$ leads us to want a vector space structure, and so we are led to think locally, where we know our space looks like $\mbb{R}^n$, which is a vector space. Now we have a new problem, that our answer for the derivative may be ambiguous depending on which homeomorphism we chose 
\begin{example}
This manifests itself in dimension 1. Let $M = \mbb{R}$, but take the homeomorphisms $\id$ or $x^3$ in a neighbourhood of $m=0$. Take the function $f(x) = x^\frac{1}{3}$. If we ask whether this function is differentiable, that depends on which homeomorphism we are using. 
\end{example}
To solve this issue, we require extra information, and this turns a topological manifold into a smooth manifold:

\begin{definition}
A \bam{smooth manifold} is a topological manifold with a preferred set of homeomorphisms $\pbrace{\varphi_\alpha: U_\alpha \overset{\cong}{\to} \varphi_\alpha(U_\alpha) \subset \mbb{R}^n}$ such that 
\begin{itemize}
    \item $\cup_\alpha U_\alpha = M$
    \item $\ev{\varphi_\beta \circ \varphi_\alpha^{-1}}{\varphi_\alpha(U_\alpha \cap U_\beta)}$ is a diffeomorphism
\end{itemize}
This is called a \bam{smooth atlas}, and the $\varphi_\alpha$ are called charts. 
\end{definition}

%%%%%%%%%%%%%%%%%%%%%%%%%%%%%%%%%%%%%%%%%%%%%%%%%%%%%%%%
\subsubsection{Other viewpoints}
We can define $C^0(M)$, the $\mbb{R}$-algebra of real valued continuous functions on a topological manifolds naturally, but we require the smooth structure of transition functions in order to define $C^\infty(M)$, the subalgebra of smooth functions. An atlas is said to be compatible with a local homeomorphism $\varphi$ if $\pbrace{\varphi} \cup \pbrace{\varphi_\alpha}$ is still an atlas. This allows us to talk of \bam{maximal atlases}, which are those which contain all possible compatible charts. Each atlas has a unique compatible maximal atlas, and the corresponding equivalence classes of atlases are called \bam{smooth structure}

\begin{example}
Consider again $M = \mbb{R}$. The atlases given by $id$ and $x^3$ are incompatible as atlases, and so correspond to different smooth structures. However, the homeormorphism $x^\frac{1}{3}:M \to M$ becomes a diffeomorphism wrt the corresponding smooth structure.
\end{example}
We can, from this example, begin to ask the question of what smooth structures are possible up to diffeomorphism. This is a question answered by learning from gauge theory, which leads to the concept of exotic $\mbb{R}^4$. 

%%%%%%%%%%%%%%%%%%%%%%%%%%%%%%%%%%%%%%%%%%%%%%%%%%%%%%%%
\subsection{Objects on manifolds}

\subsubsection{Vector Fields}
We can take different viewpoints towards what is a vector field. Suppose we have a chart $\varphi_\alpha$ with coordinates $x^1, \dots, x^n$. \\
A physicist approach would be to define a vector field as such:
\begin{definition}[vector field - physicist way]
A \bam{vector field} is an assignment of $n$ smooth functions $(X^1, \dots, X^n)$ to each chart s.t when switching to a chart with coordinates $\tilde{x}^i$ the smooth functions $\tilde{X}^i$ are given by 
\eq{
\tilde{X}^i = \pd[\tilde{x}^i]{x^j} X^j
}
\end{definition}
A mathematical definition would be
\begin{definition}[Vector field - mathematician way]
 A \bam{vector field} is a derivation of $C^\infty(M)$. Recall a derivation is a map $X:C^\infty(M) \to C^\infty(M)$ s.t. $X$ is $\mbb{R}$ linear and that $X(fg) = X(f)g + fX(g)$. 
 \end{definition}
The connection between the two definitions is that $X^i \pd{x^i}$ is a derivation of $C^\infty(U_\alpha)$

$\forall m \in M$, we have a \bam{tangent space} to $M$ at $m$, written $T_m M$. This requires a smooth structure in order to define. Given the tangent spaces we can define 
\eq{
TM = \bigsqcup_m T_m M
}
the \bam{tangent bundle}. It is a manifold in its own right, and moreover it comes with the projection 
\eq{
\pi : TM & \to M \\
v \in T_m M &\mapsto m
}
With this we can then say
\begin{definition}
A \bam{vector field} is a section of $\pi$, i.e. a smooth map $X:M \to TM$ s.t. $\pi \circ X = id_M$. 
\end{definition}


Given a vector field $X \in \mf{X}(M)$, $\exists$ a 1-parameter group of diffeomorphisms $\phi_t^X:M \to M$ for $t \in \mbb{R}$ (we will sometimes denote this as $\exp(tX)$) such that 
\begin{itemize}
	\item $\phi_t^X \circ \phi_s^X = \phi_{t+s}^X$
	\item $\ev{\frac{d}{dt} \phi_t^X(m)}{t=0} = X_m$
\end{itemize}

If $M$ is not compact, $\phi_t^X$ will only exists 'locally'. If they do exists $\forall t$ we say that $X$ is \bam{complete}. 

\begin{example}
	Take $M = (0,1) \times \mbb{R}$, we can have $X = \del_x$ a horizontal vector field and this is not complete. 
\end{example}

\begin{definition}
	If $X$ is a (complete) vector field, the \bam{Lie derivative} of anything which can pullback under the diffeomorphisms $\phi_t^X$ by 
	\eq{
		\mc{L}_X T= \ev{\frac{d}{dt} (\phi_t^X)^\ast T }{t=0}  
	}
\end{definition}

\begin{example}
	Some simple cases are 
	\begin{enumerate}
		\item for $f \in C^\infty(M)$, $\mc{L}_X f = X(f)$
		\item for $\alpha \in \Omega^\ast(M)$, $\mc{L}_X \alpha = i_X d\alpha + d(i_X \alpha)$
		\item for $Y \in \mf{X}(M)$, $\mc{L}_X Y = \comm[X]{Y}$. This bracket is the \bam{Lie bracket}
	\end{enumerate}
\end{example}
\begin{lemma}
	We can calculate $\phi^{\comm[X]{Y}}_t = \phi^X_{-\sqrt{t}} \circ \phi^Y_{-\sqrt{t}} \circ \phi^X_{\sqrt{t}} \circ \phi^Y_{\sqrt{t}}$
\end{lemma}

%%%%%%%%%%%%%%%%%%%%%%%%%%%%%%%%%%%%%%%%%%%%%%%%%%%

\subsubsection{One forms}
Each tangent space is a vecotr space, and so has dual vector space $T_m^\ast M$. We can hence similarly define the \bam{cotangent bundle} 
\eq{
T^\ast M = \bigsqcup_m T_m^\ast M
}

\begin{definition}
A \bam{one form} is a section of $T^\ast M \to M$
\end{definition}
we can also apply operations from multilinear algebra, e.g. $TM \oplus TM, \dots$

\begin{definition}[Tensor field - physicist way]
A \bam{tensor} is a section of a bundle of the form $(p,q)$
\eq{
\underbrace{TM \otimes \dots \otimes TM}_{\times p} \otimes \underbrace{T^\ast M \otimes \dots \otimes T^\ast M}_{\times q} \to M 
}
\end{definition}

\begin{definition}
A \bam{k-form} is an element of 
\eq{
\Omega^k(M) = \Gamma(\Lambda^k(T^\ast M))
}
where $\Gamma$ is the space of sections. These are $\mbb{R}$-vector spaces, and further $C^\infty(M)$-modules. 
\end{definition}

%%%%%%%%%%%%%%%%%%%%%%%%%%%%%%%%%%%%%%%%%%%%%%%%%%%%%%%%
\subsubsection{Exterior algebra}
We can define a \bam{wedge operator} 
\eq{
\wedge : \Omega^k(M) \times \Omega^l(M) &\to \Omega^{k+l}(M) \\
(\alpha,\beta) &\mapsto \alpha \wedge \beta
}
We can thus get the graded algebra 
\eq{
\Omega^\ast(M) = \bigoplus_{i=0}^n \Omega^i(M)
}
We can further define an \bam{exterior derivative} by 
\eq{
d: \Omega^k(M) &\to \Omega^{k+1}(M)
}
Whereas the wedge operation was done pointwise, the exterior derivative requires local information. It has the property
\eq{
d^2 = 0
}
This gives a natural motivation to 
\begin{definition}
The \bam{i\textsuperscript{th} De Rham Cohomology class} of $M$ is 
\eq{
H_{dR}^i(M) = \faktor{\ker(d: \Omega^i(M) \to \Omega^{i+1}(M))}{\image(d: \Omega^{i-1}(M) \to \Omega^i(M))}
}
\end{definition}
Often (e.g when $M$ is compact) this is finite dimensional. Note that, despite the fact that we needed smooth structure in order to define this, it is in fact a topological property. 

%%%%%%%%%%%%%%%%%%%%%%%%%%%%%%%%%%%%%%%%%%%%%%%%%%%%%%%%
\subsection{Lie Theory}
\subsubsection{Lie Group Basics}

\begin{definition}
A \bam{Lie group} $G$ is a group object in the category of manifolds, i.e. 
\begin{itemize}
    \item $G$ is a manifold
    \item $\mu : G \times G \to G$ multiplication is smooth 
    \item $i : G \to G$ inverse is smooth
\end{itemize}
s.t. multiplication is associative, inverses commute, etc. We may write the Lie group as a \bam{pointed} manifold $(G,e)$ to make the identity explicit. 
\end{definition}

\begin{example}
We have $GL_n(\mbb{R})\subset M_n(\mbb{R}) \cong \mbb{R}^{n^2}$
\end{example}

\begin{lemma}
	$GL_n(k)\subset M_n(k)$ is open if $k=\mbb{R},\mbb{C}$.
\end{lemma}
\begin{proof}
	Defining $\det : M_n(k) \to k$ we have $GL_n(k) = \det^{-1}(k\setminus 0)$.
\end{proof}

\begin{remark}
	We cannot do the above proof for quaternions as there is no quaternionic determinant. 
\end{remark}

\begin{example}
	We can define $Sp(1) = \pbrace{q \in \mbb{H} \, | \, \bar{q}q = 1}$, where $\mbb{H}$ are the \bam{quaternions}. 
\end{example}

\begin{theorem}[Cartan]
Any closed subgroup of a Lie group is a Lie group itself. 
\end{theorem}

\begin{example}
A non-example of the above is given by 
\eq{
H = \pbrace{(e^{it},e^{iat}) \, | \, t \in \mbb{R}, \, a \in \mbb{R}\setminus \mbb{Q}} \subset U(1) \times U(1)
}
\end{example}

\begin{example}
$SL_n(\mbb{R}) = \det^{-1}(1) \subset GL_n(\mbb{R})$. By differentiable manifold theory we know $\dim SL_n(\mbb{R}) = n^2-1$. 
\end{example}



\begin{remark}
All of these are matrix groups. Not all Lie groups are matrix groups but most are. 
\end{remark}

%%%%%%%%%%%%%%%%%%%%%%%%%%%%%%%%%%%%%%%%%%%%%%%%%%%%%%%%
\subsubsection{Lie Group Homomorphisms}

\begin{definition}
	A \bam{Lie group homomorphism} $\phi:G \to H$ is a pointed smooth map obeying $\forall a,b \in G, \, \phi(ab) = \phi(a)\phi(b)$. 
\end{definition}

\begin{example}
	$\det : GL_n(\mbb{R}) \to \mbb{R}^\times$ is a Lie group hom. 
\end{example}

\begin{definition}
	Given $g \in G$ we have two diffeomorphisms $L_g, R_g : G \to G$ given by 
	\eq{
L_g(h) = gh, \quad R_g(h) = hg	
}
$L:G \to \Diff(G), \, g \mapsto L_g$ is a group homomorphism, and $R$ is an antihomomorphism respectively. 
\end{definition}

\begin{lemma}
	$L_g$ and $R_h$ commute. 
\end{lemma}
%%%%%%%%%%%%%%%%%%%%%%%%%%%%%%%%%%%%%%%%%%%%%%%%%%%%%%%%
\subsubsection{Compact Lie Groups}
In standard applications of gauge theory to particle physics phenomenology, it turns out that we typically want compact Lie groups. We will define some compact Lie groups as subgroups of $GL_n(k)$ preserving a positive definite inner product:

\begin{definition}
	The \bam{classical Lie groups} are the subgroups of $LG_n(k)$ ($k=\mbb{R},\mbb{C},\mbb{H}$) preserving:
	\begin{itemize}
		\item ($\mbb{R}$) the Euclidean inner product, this is the \bam{orthogonal group}
		\item ($\mbb{C}$) the Hermitian inner product, this is the \bam{unitary group}
		\item ($\mbb{H}$) quaternionic Hermitian inner product, this is the \bam{quaternionic unitary group}.
	\end{itemize} 
\end{definition} 

\begin{example}
	$O(n) = \pbrace{A \in GL(n,\mbb{R}) \, \mid \, A^T A = I}$
\end{example}

\begin{example}
	$Sp(2n,\mbb{R}) = \pbrace{A \in GL(2n,\mbb{R}) \, \mid \, A^T J A = J}$ where $J$ is block diagonal with blocks $\begin{psmallmatrix} 0 & -I_n \\ I_n & 0 \end{psmallmatrix}$
\end{example}

\begin{example}
	We have $GL(n,\mbb{C}) = \pbrace{A \in GL(2n,\mbb{R}) \, \mid \, AJ=JA}$
\end{example}

\begin{example}
	$U(n) = \pbrace{A \in GL(n,\mbb{C}) \, \mid \, A^\dagger A = I}$
\end{example}
\begin{remark}
	$U(1)$ will turn out to be the gauge group for electromagnetism. 	
\end{remark}


\begin{lemma}
	\eq{
		U(n) &= GL(n,\mbb{C}) \cap Sp(2n,\mbb{R}) \cap O(2n,\mbb{R}) \\
		&= GL(n,\mbb{C}) \cap Sp(2n,\mbb{R}) \\ 
		&= GL(n,\mbb{C}) \cap O(2n,\mbb{R}) \\
		&= Sp(2n,\mbb{R}) \cap O(2n,\mbb{R})
	}
\end{lemma}

\begin{lemma}
	$SO(2) \cong U(1) \cong S^1$. 
\end{lemma}
\begin{proof}
	View $\begin{psmallmatrix} \cos \theta & -\sin\theta \\ \sin\theta & \cos\theta \end{psmallmatrix} \to e^{i\theta}$. 
\end{proof}

\begin{lemma}
	$SO(3) \cong \mbb{RP}^3 \cong \faktor{S^3}{\sim},$ and $SU(3) \cong S^3$.
\end{lemma}
\begin{remark}
	We actually have a double cover $SU(2) \twoheadrightarrow SO(3)$
\end{remark}

\begin{lemma}
	$SO(4) \cong \faktor{(S^3 \times S^3)}{\sim}$.
\end{lemma}


%%%%%%%%%%%%%%%%%%%%%%%%%%%%%%%%%%%%%%%%%%%%%%%%%%%%%%%%
\subsubsection{Lie Algebra}

\begin{definition}
A (real) \bam{Lie algebra} is a $\mbb{R}$-vector space $\mf{g}$ with a bilinear, antisymmetric map $\comm[\cdot]{\cdot} : \mf{g} \times \mf{g} \to \mf{g}$ that satisfies the Jacobi identity. 
\end{definition}

\begin{definition}
	We say $X \in \mf{X}(G)$ is \bam{left invariant} if $\forall h \in G,\, (dL_g)_h X_h = X_{gh}$.
\end{definition}
 
\begin{lemma}
	LIVFs are determined by $X_e\in T_e G = \mf{g}$.
\end{lemma}
\begin{proof}
	Simply $X_g = (dL_g)_e X_e$ 
\end{proof}
 
\begin{lemma}
If $X,Y$ are LI then $\comm[X]{Y}$ is LI, where $\comm[\cdot]{\cdot}$ is the usual Lie bracket of vector fields. 
\end{lemma}

\begin{remark}
	Given any Lie group $(G,e)$ we have an associated Lie algebra $\mf{g} = \Lie(G) = T_eG$. Note $\dim\mf{g} = \dim G$. Given the above results, we can give $\mf{g}$ a bracket by 
	\eq{
		\comm[X_e]{Y_e} = \comm[X]{Y}_e
	}
\end{remark}


Now recall we have the exponential map $\exp : \mf{g} \to G$, and for matrix groups this is given by $\exp(X) = \sum_{n=0}^\infty \frac{X^n}{n!}$. 

\begin{remark}
All the LIVFs on $G$ are complete, and so we can define the exponential map by 
\eq{
\exp : \mf{g} &\to G \\
X &\mapsto \phi_1^X(e)
}
We can find $X$ from this by using $X = \ev{\frac{d}{dt} \exp(tX)}{t=0}$
\end{remark}

\begin{example} 
One can ask what the Lie algebras for the corresponding classical matrix groups are. 
\begin{itemize}
    \item $GL(n,\mbb{R})$ is a vector space, and so $T_e GL(n,\mbb{R}) = GL(n,\mbb{R})$. The bracket turns out to be the standard matrix commutator. 
    \item As $SL(n) = \pbrace{A \, \mid \, \det A = 1}$ we find $\mf{sl}_n = \pbrace{B \, | \, \det(\exp(tB)) = 1} = \pbrace{B \, \mid \, \tr(B) = 0}$
    \item $\mf{o}_n = \pbrace{B  \mid  \ev{\frac{d}{dt} \exp(tB)^T \exp(tB)}{t=0}} = \pbrace{B  \mid B^T + B = 0 }$
\end{itemize}
This process can be continues as similarly. 
\end{example}

\begin{definition}
We have the \bam{adjoint representation} of a group $G$ acting on $\mf{g}$ by 
\eq{
\Ad_g(X) = \ev{ \, \frac{d}{dt} g \exp(tX)^{-1} g^{-1}}{t=0}
}
\end{definition}

\begin{prop}
	An equivalent definition of $\Ad$ is that $\Ad_g = \pround{L_g \circ R_{g^{-1}}}_\ast $. 
\end{prop}
\begin{proof}
	We start by verifying that this other definition could make sense. $(L_g)_\ast : T_eG \to T_g G$ and $(R_{g^{-1}})_\ast : T_g G \to T_e G$ so 
	\eq{
	(R_{g^{-1}})_\ast \circ (L_g)_\ast =\pround{L_g \circ R_{g^{-1}}}_\ast  : T_eG \to T_e G
}
\hl{finish this off}.
\end{proof}

\begin{prop}
	$\Ad : G \to GL(T_e G)$ is a Lie group hom
\end{prop}
\begin{proof}
If we use the definition of $\Ad_g = d\pround{L_g \circ R_{g^{-1}}} =\pround{L_g \circ R_{g^{-1}}}_\ast$ we have 
\eq{
\Ad_{gh} &= \pround{L_{gh} \circ R_{(gh)^{-1}}}_\ast \\
	&= \pround{ L_g \circ R_{g^{-1}} \circ L_h \circ R_{h^{-1}}}_\ast \\
	&= \pround{L_g \circ R_{g^{-1}}}_\ast \circ \pround{L_h \circ R_{h^{-1}}}_\ast \\
	&= \Ad_g \circ \Ad_h
}
\end{proof}

\begin{lemma}
For a matrix group, $\Ad_A(B) = ABA^{-1}$. 
\end{lemma}

\begin{definition}
We also have an adjoint rep of $\mf{g}$ acting on $\mf{g}$ by 
\eq{
\ad_X(Y) = \ev{\frac{d}{dt} \Ad_{\exp(tx)}(Y)}{t=0} = \comm[X]{Y}
}
\end{definition}

\begin{ex}
	Show that if $G$ is a matrix Lie group, the Lie bracket is indeed the commutator of the matrices. 
\end{ex}

\begin{definition}
	A group rep $\rho$ is \bam{faithful} if $\ker \rho = e$
\end{definition}

\begin{fact}
	$\Ad$ need not be a faithful rep. 
\end{fact}

\begin{definition}
	Given a basis $\pbrace{t_i}$ of a Lie algebra, the \bam{structure constants} are the coefficients $c_{ij}^k$ defined by 
	\eq{
\comm[t_i]{t_j} = \sum_k c_{ij}^k t_k	
}
\end{definition}

\begin{prop}
	Let $\phi:G \to H$ be a Lie group hom. Then $(\phi_\ast)_e : \mf{g} \to \mf{h}$ is a Lie algebra hom
\end{prop}
\begin{proof}
	See 
	\eq{
\phi(ab) = \phi(a)\phi(b) &\Leftrightarrow (\phi \circ L_a)(b) = (L_{\phi(a)} \circ \phi)(b) \\
&\Leftrightarrow 	\phi \circ L_a = L_{\phi(a)} \circ \phi \\
&\Rightarrow (\phi \circ L_a)_\ast = (L_{\phi(a)} \circ \phi)_\ast \\
&\Rightarrow \phi_\ast \circ (L_a)_\ast = (L_{\phi(a)})_\ast \circ \phi_\ast \\
& \Rightarrow \forall X \in \mf{g}, \, (X)_a = 
}
\end{proof}

\begin{example}
	We have $\Ad:G \to GL(\mf{g})$ so there is an associated $\Ad_\ast : \mf{g} \to gl(\mf{g})$. This coincides with $ad : X \mapsto \ad_X$, as should be clear. 
\end{example}

%%%%%%%%%%%%%%%%%%%%%%%%%%%%%%%%%%%%%%%%%%%%%%%%%%%%%%%%
\subsubsection{Maurer Cartan Forms}

\begin{definition}
	The \bam{Maurer Cartan one form} is $\theta \in \Omega^1(G;\mf{g})$ given by 
	\eq{
\theta_g = (L_{g^{-1}})_\ast : T_gG \to T_e G = \mf{g}	
}
\end{definition}

\begin{lemma}
	$\theta$ is left invariant, i.e $L_g^\ast \theta = \theta$.
\end{lemma}
\begin{proof}
	See
	\eq{
(L_g^\ast \theta)_h &= \theta_{gh} \circ (L_g)_\ast \\
&= (L_{(gh)^{-1}})_\ast \circ (L_g)_\ast \\
&= (L_{(gh)^{-1}}\circ L_g)_\ast \\
&= (L_{h^{-1}})_\ast = \theta_h 	
}
\end{proof}

\begin{prop}
	$\theta$ obey the structure equation 
	\eq{
d\theta + \frac{1}{2} \comm[\theta]{\theta} = 0	
}
i.e. $\forall \xi, \eta \in \mf{X}(G), \, d\theta(\xi,\eta) + \comm[\theta(\xi)]{\theta(\eta)} = 0$.
\end{prop}
\begin{proof}
	Note $\theta(\xi) = \xi_e$ as 
	\eq{
\theta_g (\xi_g) = (L_{g^{-1}})_\ast (L_g)_\ast \xi_e = \xi_e	
}
and then 
\eq{
d\theta(\xi,\eta) = \xi \theta(\eta) - \eta \theta(\xi) - \theta(\comm[\xi]{\eta})
}

\end{proof}

\begin{ex}
	Show that for matrix groups $\theta_g = g^{-1}dg$.
\end{ex}

%%%%%%%%%%%%%%%%%%%%%%%%%%%%%%%%%%%%%%%%%%%%%%%%%%%%%%%%
\subsubsection{Invariant inner products on \secmath{\mf{g}}}
	
\begin{definition}
	For a $\mbb{R}$ LA $\mf{g}$, a bilinear form $\pangle{\cdot,\cdot}:\mf{g} \times \mf{g} \to \mbb{R}$ is said to be \bam{invariant} if 
	\eq{
\forall X,Y,Z \in \mf{g}, \, \pangle{\comm[X]{Y},Z} + \pangle{Y,\comm[X]{Z}} = 0	
}
\end{definition}

\begin{definition}
	The \bam{Killing form} is $\kappa:\mf{g} \times \mf{g} \to \mbb{R}$ given by 
	\eq{
\kappa(X,Y) = \tr(\ad_X \circ \ad_Y)	
}
\end{definition}

\begin{lemma}
	$\kappa$ is a symmetric, invariant, bilinear inner product. 
\end{lemma}

\begin{fact}[Cartan's Criterion]
	$\kappa$ is non-degenerate $\Leftrightarrow$ $\mf{g}$ is semisimple. 
\end{fact}

\begin{fact}
	$\kappa$ is negative definite $\Leftrightarrow$ $\mf{g}$ is the LA of a compact semisimple $G$.
\end{fact}

\begin{fact}
	$\mf{g}$ admits a positive definite invariant inner product only when $\mf{g}$ is the LA of a compact $G$. 
\end{fact}
%%%%%%%%%%%%%%%%%%%%%%%%%%%%%%%%%%%%%%%%%%%%%%%%%%%%%%%%
%%%%%%%%%%%%%%%%%%%%%%%%%%%%%%%%%%%%%%%%%%%%%%%%%%%%%%%%
\section{Riemannian and Symplectic Geometry}
\hl{Combine this section with the previous and neaten up }
%%%%%%%%%%%%%%%%%%%%%%%%%%%%%%%%%%%%%%%%%%%%%%%%%%%%%%%%
\subsection{The main stuff}

\begin{definition}
If $M$ is an even-dimensional manifold we say $M$ is symplectic if it is  equipped with $\omega \in \Omega^2(M)$ which is 
\begin{itemize}
    \item closed: $d\omega = 0$
    \item non-degenerate: $\forall m \in M,\, i_\cdot : T_m M \overset{\cong}{\to} T_m^\ast M$ is an isomorphism. 
\end{itemize}
\end{definition}

If $(M,\omega)$ is symplectic and $f \in C^\infty(M)$, then $df \in \Omega^1(M)$ and $\exists ! \, X^f \in \mf{X}(M)$ s.t $i_{X^f}\omega = df$. $X^f$ is called the Hamiltonian function associated to $f$. 

\begin{lemma}
The Hamiltonian function is constant along the flow lines of its vector field, i.e 
\eq{
t \mapsto (f \circ \phi_t^{X^f}(m))
}
is constant.
\end{lemma}

If $G$ is a Lie group with smooth action on $M$, we say that the action is \bam{symplectic} if
\begin{enumerate}
    \item $g^\ast \omega = \omega$
\end{enumerate}
Now let us think about 
\begin{tkz}
C^\infty(M) \arrow[r,"d"] & \Omega^1(M) \arrow[r] & \mf{X}(M) \\
\mf{g} \arrow[u,dashed,"\mu^\ast"] \arrow[rru] & &
\end{tkz}
If such a map exists, we say the action is Hamiltonian and call the map the \bam{co-moment map}. We can then repackage this into a \bam{moment map} 
\eq{
\mu : M &\to \mf{g}^\ast \\
\mu(m)(X) &=\mu^\ast(X)(m)
}
%%%%%%%%%%%%%%%%%%%%%%%%%%%%%%%%%%%%%%%%%%%%%%%%%%%%%%%%
\subsection{Left overs}
\subsubsection{Minus sign problems}
There is a mismatch of minus signs in the convention of the literature, namely that constructing a map 
\eq{
\mf{g} &\to \mf{X}(M) \\
X &\mapsto \tilde{X}
}
can be done in one of two ways 
\begin{enumerate}
    \item Set $\tilde{X}_m = \ev{\frac{d}{dt} \exp(tX) \cdot m}{t=0}$ as is natural, but then the map is a Lie algebra \bam{antihomomorphism}
    \item Set $\tilde{X}_m = \ev{\frac{d}{dt} \exp(-tX) \cdot m}{t=0}$, and then the map is  Lie algebra homomorphism. 
\end{enumerate}
For the purpose of this course the second convention is used. 

\subsubsection{Symplectic Manifolds}
Given a smooth manifold $N$, we have the following result:

\begin{theorem}
The cotangent bundle $M= T^\ast N$ has a canonical symplectic structure. $\omega = - d\theta$ where $\theta$ is the tautological one form 
\end{theorem}

\begin{definition}
The \bam{tautological one form} $\theta$ on the cotangent bundle is given by noting 
\eq{
\pi : T^\ast N & \to N \\
\Rightarrow d\pi : T(T^\ast N) &\to TN
}
so if $ m \in M, \, m \in T_{\pi(m)}^\ast N$ and so we have 
\eq{
m : T_{\pi(m)}N \to \mbb{R}
}
and 
\eq{
\theta_m \equiv m \circ d\pi: T_m M \to \mbb{R}
}
\end{definition}

\begin{lemma}
$\omega = -d\theta$ is (obviously) closed and moreover non-degenerate, hence symplectic. 
\end{lemma}
\begin{proof}
If $q^1, \dots, q^n$ are coordinates on $N$, then $\pd{q^1}, \dots, \pd{q^n}$ are vector fields on $N$. Letting $p_i$ be the one form associated with $\pd{q^i}$ we get coordinates $(q^i,p_i)$ on $M$ in which 
\eq{
\theta &= p_i dq^i \\
\omega &= dq^i \wedge dp_i
}
\end{proof}

\begin{aside}
If $f \in C^\infty(M)$ we get associated Hamiltonian vector fields $X^f$. We then define the \bam{Poisson bracket} of $f$ and $g \in C^\infty(M)$ by 
\eq{
\acomm[f]{g} \equiv X^f(g) = \omega(X^f,X^g)
}
\end{aside}
%%%%%%%%%%%%%%%%%%%%%%%%%%%%%%%%%%%%%%%%%%%%%%%%%%%%%%%%
%%%%%%%%%%%%%%%%%%%%%%%%%%%%%%%%%%%%%%%%%%%%%%%%%%%%%%%%
\section{Bundles}
\hl{Neaten up this section}
%%%%%%%%%%%%%%%%%%%%%%%%%%%%%%%%%%%%%%%%%%%%%%%%%%%%%%%%
\subsection{Principle Bundles}
\begin{definition}
A group action $G \lact M$ is \bam{free} if $\forall m \in M, g \in G, \, e \neq g \Rightarrow g\cdot m \neq m$
\end{definition}

\begin{definition}
A group action $G \lact M$ is \bam{proper} if the map between topological spaces 
\eq{
G \times M &\to M \times M \\
(g,m) &\mapsto (m,g\cdot m)
}
has compact preimages of compact subsets. 
\end{definition}

\begin{prop}
If $G$ is proper, any continuous group action is proper. 
\end{prop}

\begin{lemma}
If $G \lact M$ is smooth, free, and proper. Then $\faktor{M}{G}$ exists and is a smooth manifold
\end{lemma}

\begin{definition}
If $G$ is a Lie group, $M$ a smooth manifold, a \bam{G principal bundle over M} is given by 
\begin{itemize}
    \item $P$ a smooth manifold with projection $\pi:P \to M$
    \item a smooth right action of $G$, $P \times G \to P$
    \item a $G$-equivariant local trivialisation, i.e $\pbrace{U_\alpha}$ an open covering s.t $\Psi_\alpha : \pi^{-1}(U_\alpha) \overset{\cong}{\to} U_\alpha \times G, \, \Psi_\alpha(p) = (\pi(p),\psi_\alpha(p))$ and $(m,g) \cdot h = (m,gh)$
\end{itemize}
This is often noted as 
\begin{tkz}
G \arrow[r] & P \arrow[d,"\pi"] \\ & M
\end{tkz}
\end{definition}

\begin{remark}
By $G$-equivariance $\pi^{-1}(U_\alpha)$ are all $G$-orbits, and moreover the action is free and proper. This means we have $M \cong \faktor{P}{G}$. As such a $G$-principal bundle is given by a smooth, free, and proper action $P \ract G$. 
\end{remark}

If $U_\alpha,U_\beta$ are part of the open cover s.t $U_{\alpha\beta} \equiv U_\alpha \cap U_\beta \neq \emptyset$ then we can ask about the diagram
\begin{center}
\begin{tikzpicture}[commutative diagrams/every diagram]
\node (P0) at (90:1cm) {$\pi^{-1}(U_{\alpha\beta})$};
\node (P1) at (90+120:1.5cm) {$U_{\alpha\beta} \times G$};
\node (P2) at (90+240:1.5cm) {$U_{\alpha\beta} \times G$};
\path[commutative diagrams/.cd, every arrow, every label]
(P0) edge node[swap] {$\Psi_\alpha$} (P1)
(P0) edge node {$\Psi_\beta$} (P2)
(P1) edge node[swap] {} (P2);
\end{tikzpicture}
\end{center}

\begin{comment}
\begin{tkz}
\pi^{-1}(U_{\alpha\beta}) \arrow[r,"\Psi_\beta"] \arrow[d,"\Psi_\alpha"] & U_{\alpha\beta} \times G \\
U_{\alpha\beta}\times G \arrow[ur] & 
\end{tkz}
\end{comment}
which sends $(m,g) \mapsto (m,\phi_{\beta\alpha}(m)g)$ where the map $\phi_{\beta\alpha}$ satisfies 
\begin{itemize}
    \item $\phi_{\alpha\beta}(m) = \phi_{\beta\alpha}(m)^{-1}$ 
    \item $\phi_{\alpha\alpha} : U_\alpha \to \pbrace{e}\leq G $
    \item on triple intersects $\phi_{\gamma\beta}\phi_{\beta\alpha} = \phi_{\gamma\alpha}$. 
\end{itemize}

\begin{remark}
If a covering $M = \cup_\alpha U_\alpha$ is given with transition function $\phi_{\beta\alpha}:U_{\alpha\beta} \to G$ satisfying the above conditions, then we can recover $P$. This is done as 
\eq{
P = \faktor{\pround{\bigsqcup_\alpha U_\alpha \times G}}{\sim}
}
where $(m,g) \sim (m, \phi_{\beta\alpha}(m) g)$
\end{remark}

\begin{example}[Trivial Bundle]
Given $P=M \times G$ we get a $G$ action by right multiplication. 
\end{example}

\begin{example}[Hopf Fibration]
Take $P = S^{2n-1} \subset \mbb{C}^n$ and $G = U(1)$. Let $G$ act on $\mbb{C}^n$ by $\bm{v} \cdot g = g^{-1} \bm{v}$. We then get $M = \mbb{CP}^{n-1}$
\end{example}

\begin{example}
The edge of a Mobius strip is a $\mbb{Z}_2$-principal bundle over $S^1$.
\end{example}

\begin{remark}
All of the fibres are isomorphic to $G$ as manifolds, but they do not have an intrinsic group structure. As such the fibres are $G$-torsors. 
\end{remark}

\begin{definition}
A \bam{section} of $P \overset{\pi}{\to}M$ is a right inverse to $\pi$, i.e $s:M \to P$ s.t. $\pi \circ s = \id_M$. 
\end{definition}

\begin{lemma}
A PB has a global section iff is is trivial. 
\end{lemma}



\begin{definition}
A \bam{vector bundle} $E \to M$ is a (smooth) morphism s.t. all the fibres are vector spaces of the same dimension $r\equiv\rank E$. 
\end{definition}

\begin{definition}
If $E \to M$ is a vector bundle, the associated \bam{frame bundle} is 
\eq{
P \equiv \Fr_E =\pbrace{\text{bases of fibres of $E$}}
}
This is a $GL(r,\mbb{R})$-bundle. The action is given by 
\eq{
(\bm{\sigma}_1, \dots, \bm{\sigma}_r) \mapsto  (\bm{\sigma}_1, \dots, \bm{\sigma}_r) \cdot A = (\bm{\sigma}_1A, \dots, \bm{\sigma}_rA)
}
\end{definition}

%%%%%%%%%%%%%%%%%%%%%%%%%%%%%%%%%%%%%%%%%%%%%%%%%%%%%%%
\subsection{Bundles from Bundles}
\begin{definition}
	Given $P \to M$ a ppal $G$-bundle and $f:N \to M$ smooth we defin the \bam{pullback bundle} to be 
	$f^\ast \pi : f^\ast P \to N$ the fibered product 
	\eq{
f^\ast P = \pbrace{(y,p) \in N \times P \, | \, f(y) = \pi(p)}	
}
\end{definition}

%%%%%%%%%%%%%%%%%%%%%%%%%%%%%%%%%%%%%%%%%%%%%%%%%%%%%%%%
\subsubsection{Associated Fibre Bundle}
If we are given a principal bundle $G \to P \overset{\pi}{\to} M$ and a left action $G \lact F$ we can get the action $(P \times F) \ract G$ by $(p,f) \cdot g = (p\cdot g, g^{-1} \cdot f)$. This action is always smooth, free, and proper. 

\begin{definition}
The \bam{associated fibre bundle} to $G \to P \overset{\pi}{\to} M$ with $G \lact F$ is 
\eq{
P_F \equiv \faktor{(P \times F)}{G} \to M
}
quotienting by orbits under the group action. All the fibres are diffeomorphic to $F$. 
\end{definition}

If $G \lact F$ preserves extra structure (e.g. if $F=V$ is a vector space and the action is linear) then all the fibres have this structure canonically. 

\begin{example}
Suppose $F=V$ is a vector space and $G \lact V$ is linear (i.e. $V$ is a rep space for $G$). Then $P_V$ is a vector bundle over $M$. 
\end{example}

\begin{prop}
The assignment $P+V \to P_V$ is the "inverse" of taking the frame bundle.  
\end{prop}
\begin{proof}
Using the defining rep of $GL(r,k) \lact k^r$ we get $(\Fr_E)_{k^r} \cong E$
\end{proof}

\begin{example}
If $G \lact \mf{g}$ via the adjoint action $\Ad$, we get $\ad(P) = \faktor{(P \times \mf{g})}{G}$
\end{example}

\begin{example}
Let $G \lact G$ by conjugation, i.e. $C_g(h) = ghg^{-1}$. This preserves group structure, so we have $C : G \to \Aut(G)$ and 
\eq{
\Ad(G) \equiv P_{G,C} = \faktor{(P\times G)}{G}
}
where $(p,g) \sim (ph,h^{-1}gh)$.
\end{example}

\begin{definition}
	Given $G \to P \to M$ and a Lie group hom $\rho : G \to H$, letting $G \lact H$ by $g \cdot h = \rho(g)h$ and noting $H \ract H$ by right multiplication, making it into a $H$-torsor, is preserved by the action of $G$ and so we have  
	\eq{
		P_H = \faktor{(P \times H)}{G}
	}
	a $H$-torsor bundle, or a principal $H$ bundle. This provides an \bam{extension of structure group}.
\end{definition}

\begin{definition}
Given a $H$-principal bundle $H \to \tilde{P} \to M$ and a Lie group hom $G \to H$, if $\exists G \to P \to M$ s.t.  $\tilde{P} \cong P$ as $H$-principal bundle, we say that it is a \bam{reduction of structure group} of $\tilde{P}$.
\end{definition}

\begin{example}
If $E \to M$ is a Euclidean vector bundle of rank $r$, we can look at $\Fr_E$ as a $GL(r)$-principal bundle, or at $\Fr_E^\perp= \pbrace{\text{bundle of orthonormal frames}}$ which is a $O(r)$-principal bundle. $\Fr_E^\perp$ is a reduction of structure group of $\Fr_E$ using $O(r) \hookrightarrow GL(r)$
\end{example}


\begin{prop}
There is a 1:1 correspondence 
\eq{
\pbrace{\text{reduction of structure groups of $\Fr_E$ using $O(r) \hookrightarrow GL(r)$}} \leftrightarrow \pbrace{\text{Choice of Euclidean structure on $E$}}
}
\end{prop}

\begin{example}
If $E$ is a rank $r$ bundle, and $\Lambda^2 E$ is a line bundle. Then $\Lambda^2 E$ is the vector bundle associated to $\Fr_E$ with $GL(2,k) \overset{\det}{\to} k^\times$. Hence we get a $k^\times$-principal bundle by extension. 
\end{example}

\begin{example}
There is a 1:1 correspondence 
\eq{
\pbrace{\text{reduction of structure group of $\Fr_E$ using $GL(r) \to SL(r)$}} \leftrightarrow \pbrace{\text{trivialisations $\Lambda^2 E \cong M \times \mbb{R}$}}
}
If we do not have $\Lambda^2 E \cong M \times \mbb{R}$ then reduction might not always be possible. 
\end{example}


\begin{example}
If $(M,g)$ is an oriented Riemannian Manifold of dimension $n$, then $\Fr_{TM}$ has a reduction of structure group from $GL_n(\mbb{R}) \to SO(n)$. $ n \geq 3 \Rightarrow \pi_1(SO(n)) = \mbb{Z}_2$ and so there is a 2:1 "universal cover" $\Spin(n) \to SO(n)$
\end{example}

\begin{definition}
A Spin structure on $M$ is a reduction of structure group from $SO(n)$ to $\Spin(n)$. 
\end{definition}

If a spin structure is given, then you can look at irreducible representations of the spin group that do not come from $SO(n)$. 


%%%%%%%%%%%%%%%%%%%%%%%%%%%%%%%%%%%%%%%%%%%%%%%%%%%%%%%%
\subsection{Gauge transformations}

\begin{definition}
Given $G \to P \to M$, a \bam{gauge transformation} for $P$ is a $G$-equivariant diffeomorphism $\psi$ s.t. 

\begin{center}
\begin{tikzpicture}[commutative diagrams/every diagram]
\node (P0) at (30:1cm) {$P$};
\node (P1) at (30+120:1cm) {$P$};
\node (P2) at (30+240:1cm) {$M$};
\path[commutative diagrams/.cd, every arrow, every label]
(P1) edge node {$\psi$} (P0)
(P0) edge node {$\pi$} (P2)
(P1) edge node[swap] {$\pi$} (P2);
\end{tikzpicture}
\end{center}
commutes. As the $\psi$ compose, we get a group $\mc{G}(P)$
\end{definition}

\begin{remark}
Note both $G$ and $\mc{G}(P)$ act on $P$ by right and left action respectively. 
\end{remark}

\begin{theorem}
There are canonical group isomorphisms 
\eq{
\mc{G}(P) \cong \Gamma(\Ad(P)) \cong \pbrace{f:P \to G \, | \, f \text{ smooth }, \, f(p \cdot g) = g^{-1}f(p) g}
}
\end{theorem}
\begin{proof}
We have 
\begin{center}
\begin{tikzpicture}[commutative diagrams/every diagram]
\node (P0) at (30:1.5cm) {$\Gamma(\Ad(P))$};
\node (P1) at (30+120:1.5cm) {$\mc{G}(P)$};
\node (P2) at (30+240:1cm) {$\pbrace{f:P \to G}$};
\path[commutative diagrams/.cd, every arrow, every label]
(P0) edge node[swap] {$C$} (P1)
(P2) edge node {$B$} (P0)
(P1) edge node {$A$} (P2);
\end{tikzpicture}
\end{center}
with 
\eq{
A: \psi &\mapsto (f: p \mapsto g \text{ if } \psi(p) = pg) \\
}
we may check 
\eq{
\psi(p\cdot\tilde{g}) = \psi(p)\cdot \tilde{g} = pg \tilde{g} = (p\tilde{g}) (\tilde{g}^{-1} g \tilde{g})
}
so $f$ obeys the necessary condition. We then have 
\eq{
B : f \mapsto (s : m \mapsto [p,f(p)]\in \faktor{(P \times G)}{G} \text{ for any }p \in \pi^{-1}(m))
}
Again we can check if $\tilde{p} \in \pi^{-1}(m)$, then 
\eq{
[p\cdot g,f(p\cdot g) ] = [p\cdot g, g^{-1}f(p) g] = [p,f(p)]
}
as this is the action we are quotienting out by. Finally we have 
\eq{
C : s \mapsto (\psi : p \mapsto p\cdot g \text{ where } s(\pi(p)) = [p, g])
}
and we can see $s(\pi(p \cdot \tilde{g})) = s(\pi(p)) = [p,g] = [p\tilde{g},\tilde{g}^{-1}g \tilde{g}]$ so
\eq{
\psi(p\cdot \tilde{g}) = (p \cdot \tilde{g}) \cdot (\tilde{g}^{-1}g \tilde{g}) = p \cdot g \cdot \tilde{g} = \psi(p) \cdot \tilde{g}
}
It is necessary to check that $A,B,C$ are group isomorphisms now. 
\end{proof}

%%%%%%%%%%%%%%%%%%%%%%%%%%%%%%%%%%%%%%%%%%%%%%%%%%%%%%%%
%%%%%%%%%%%%%%%%%%%%%%%%%%%%%%%%%%%%%%%%%%%%%%%%%%%%%%%%
\section{Connections}

%%%%%%%%%%%%%%%%%%%%%%%%%%%%%%%%%%%%%%%%%%%%%%%%%%%%%%%%
\subsection{Kozul connections}
Let $ E \to M$ be a vector bundle. We want to be able to take directional derivatives of sections of $E$

\begin{definition}[Kozul connection]
A \bam{Kozul connection} on $E$ is a map 
\eq{
\nabla : \Gamma(E) \to \Gamma(E \otimes T^\ast M)
}
For $s \in \Gamma(E), \, X \in \mf{X}(M) = \Gamma(TM)$,
$(\nabla s)(X) \in \Gamma(E)$ is denoted as $\nabla_X s$. 
\end{definition}

We want this to satisfy a product rule that for $f \in C^\infty(M)$ 
\eq{
\nabla_X (fs) = X(f) s + f \nabla_X(s)
}
and linearity in $\Gamma(E), \mf{X}(M)$, i.e. $\forall c_i \in \mbb{R}, f_i \in C^\infty(M)$
\eq{
\nabla_X (c_1 s_1 + c_2 s_2) &= c_1 \nabla_X s_1 + c_2 \nabla_X s_2 \\
\nabla_{(f_1 X_1 + f_2 X_2)} s &= f_1 \nabla_{X_1} s + f_2 \nabla_{X_2} s 
}

If $E$ is a Euclidean vector bundle with inner product $\pangle{\cdot, \cdot}$, we may ask that a connection respects this additional structure. 

\begin{definition}
$\nabla$ respects  $\pangle{\cdot, \cdot}$ if 
\eq{
X(\pangle{s_1, s_2}) = \pangle{\nabla_X s_1, s_2} + \pangle{s_1, \nabla_X s_2}
}
\end{definition}

%%%%%%%%%%%%%%%%%%%%%%%%%%%%%%%%%%%%%%%%%%%%%%%%%%%%%%%%
\subsection{Ehresmann connections}

\begin{notation}
Let $G \to P \overset{\pi}{\to} M$ be a $G$-principal bundle. Denote the fibre through $p \in P$ by $F_p = p\cdot G$.
\end{notation}

\begin{definition}
	We call $T_p F_p = \ker d\pi_p$ the \bam{subspace of vertical tangent vectors at $p \in P$} and denote it by $V_p$. The collection of $V_p$ gives a smooth distribution. 
\end{definition}

\begin{definition}
	A differential form on $P$ is called \bam{horizontal} if it is zero on $V$. 
\end{definition}

\begin{definition}
	A form on $P$ is called \bam{basic} if it is horizontal and $G$-invariant 
\end{definition}

\begin{definition}
An \bam{Ehresmann connection} on $P$, $H$, is a smooth choice of complement $H_p \hookrightarrow T_pP$ s.t. 
\begin{itemize}
    \item $\forall p \in P, \, \dim H_p = \dim M$ 
    \item $\forall p \in P, \, T_pP = T_p F_p \oplus H_p$
    \item $H$ is $G$-equivariant, i.e. $dR_g H_p = H_{p \cdot g}$
\end{itemize}
The distrubution is called \bam{horizontal}.
\end{definition}

\begin{remark}
$d\pi_p : T_p P \to T_{\pi(p)}M$ gives a isomorphism $T_{\pi(p)} M \overset{\cong}{\to} H_p$, or equivalently a splitting of the SES
\eq{
0 \to T_p F_p \to T_pP \to T_{\pi(p)} M \to 0
}
\end{remark}

\begin{remark}
Smooth connections always exists. This is as they certainly exist on local trivialisations, and then they can be glued with partitions of unity. 
\end{remark}



%%%%%%%%%%%%%%%%%%%%%%%%%%%%%%%%%%%%%%%%%%%%%%%%%%%%%%%%
\subsection{Holonomy}

\begin{lemma}
Given $G \to P \overset{\pi}{\to} M$, a smooth connection $H$, and a smooth closed curve in the base $C: I \to M$, we can lift tangent vectors to $\pi^{-1}(C(I)) \subset M$ to tangent vectors to $P$ to get a vector field on $\pi^{-1}(p)$, and there is a unique such lift that is horizontal.
\end{lemma}
We can then flow along this vector field starting at $\pi^{-1}(C(0))$. This will give a curve $\tilde{C}:I \to P$, where it is not necessary that $\tilde{C}(0) = \tilde{C}(1)$. However, it will be true that $\tilde{C}(1) \in \pi^{-1}(C(0))$ so $\exists g \in G, \, \tilde{C}(1) = p \cdot g$. 

\begin{definition}
$g =g(p)$ is the \bam{holonomy} of $C$ w.r.t $H$ starting at $p$.  
\end{definition}

\begin{definition}
For $X \in \mf{g}$ we have $\tilde{X}\in T_pP$ s.t. 
\eq{
\tilde{X}(f) = \ev{\frac{d}{dt} f(p \cdot \exp(tX))}{t=0}
}
\end{definition}

\begin{definition}
A \bam{principal connection} is a connection form $\omega \in \Gamma(T^\ast P \otimes \mf{g})= \Gamma(\Hom(TP,\mf{g}))$ s.t. 
\begin{itemize}
    \item $\omega(\tilde{X}) = X$
    \item $(R_g)^\ast \omega = \Ad_{g^{-1}}\omega$
\end{itemize}
\end{definition}

\begin{example}
	On the Hopf bundle, $\pbrace{(z,w) \in \mbb{C}^2 \, | \, \abs{z}^2+\abs{w}^2=1} \to \mbb{CP}^1$, $(z,w) \mapsto [z:w]$, as a $U(1)$-bundle acting by multiplication, we can have $\omega = i \Im \bar{z}dx + \bar{w}dw$.
\end{example}

%%%%%%%%%%%%%%%%%%%%%%%%%%%%%%%%%%%%%%%%%%%%%%%%%%%%%%%%
\subsection{Relations between viewpoints}
Let $V$ be a vector space and $S \subset V$ a subspace. 

\begin{definition}
A \bam{complement} to $S$ is a choice $\tilde{S}$ s.t. $V = S \oplus \tilde{S}$. \end{definition}

\begin{lemma}
A complement to $S$ is equivalent to a projection $p: V \to S$ s.t. $\ev{p}{S} = \id_S$
\end{lemma}
\begin{corollary}
$\tilde{S}=\ker p$ 
\end{corollary}

Now, noting that $\omega$ gives a projection $\omega_p : T_pP \to \mf{g}$ for each $p \in P$, we get the following relation:

\begin{prop}
$\omega$ gives an Ehresmann connection by letting $H_p = \ker \omega_p$. 
\end{prop} 

We now recall some facts from the workshops: Given a vector bundle $E \to M$, $\forall U \subset M$ open, we have a short exact sequence 
\eq{
0 \to \End(E)(U) \hookrightarrow \mc{D}^{\leq 1}(E)(U) \overset{\sigma}{\to} \psquare{TM \otimes \End(E)}(U) \to 0
}
where $\mc{D}^{\leq 1}(E)(U)$ are the first order differential operators on $\ev{E}{U}$. Now for $D \in \mc{D}^{\leq 1}(E)(U)$, $s \in \Gamma(U,E)$, $f \in C^\infty(U)$, we have 
\eq{
D(fs) = \sigma(D)(f)(s)
}
Putting $\mc{D}^{\leq 1}_{\diag}(E)(U) = \sigma^{-1}(TM \otimes \id_E)$ we have SES 
\eq{
0 \to \End(E)(U) \to \mc{D}^{\leq 1}_{\diag}(E)(U) \to \ev{TM}{U} \to 0
}
A Kozul connection is now a splitting of this SES, $\nabla : \ev{TM}{U} \to \mc{D}^{\leq 1}_{\diag}(E)(U)$, i.e. given $X \in \mf{X}(U)$, $\nabla_X$ is a 1st order differential operator satisfying $\sigma(\nabla_X) = X$. \\
Now, if we have a principal bundle $G \to P \to M$, the \bam{Atiyah sequence} associated with it is a SES
\eq{
0 \to \ad(P) \to \faktor{TP}{G} \to TM \to 0 \, .
}
This is split by the map $\pround{\ev{d\pi}{H_p}}^{-1}:TM \to \faktor{TP}{G}$, given by an Ehresmann connection. Now with $G \lact V$ via a representation, we have the follwing result:

\begin{prop}
$\forall U \subset M$ we have 
\begin{tkz}
0 \arrow[r] & \ad(P)(U) \arrow[r] \arrow[d] & \faktor{TP}{G}(U) \arrow[r] \arrow[d] & \ev{TM}{U} \arrow[r] \arrow[d,"\id"] \arrow[l,red, bend left = 15] & 0 \\
0 \arrow[r] & \End(E)(U) \arrow[r] & \mc{D}^{\leq 1}_{\diag}(E)(U) \arrow[r,"\sigma"] & \ev{TM}{U} \arrow[r]  \arrow[l,red, bend left = 15] & 0 
\end{tkz}
commutes. 
\end{prop}
\begin{proof}
$G \lact V$ so we get $\mf{g} \to \End(V)$, a morphism of Lie algebras that is $G$-equivariant. Then we have 
\eq{
\faktor{(P \times \mf{g})}{G} \to \faktor{(P \times \End(V))}{G} = \End(E)
}
Further, the space of sections of $E$ is the space of $G$-equivariant functions $P \to V$. This means we can say that $G$-equivariant vector fields on $P$ also act on $G$-equivariant functions $P \to V$, i.e. sections of $E$. It can then be checked that these all commute. 
\end{proof}
\begin{corollary}
A splitting of the Atiyah sequence gives a Kozul connection. 
\end{corollary}

\begin{remark}[On the Atiyah sequence]
	We can send $P \times \mf{g} \to TP$ by sending $(p,\xi)$ to the vector field whose value at $p$ is given by that generated by the action of $G$ on $P$. As this action is fibre-wise, the vector field in the image must lie in the vertical component $\ker d\pi$. This gives the SES 
	\eq{
0 \to P \times \mf{g} \to TP \to TM \to 0	
}
As all these maps are $G$ invariant we can quotient out by the action of $G$ to get the Atiyah sequence. 
\end{remark}

%%%%%%%%%%%%%%%%%%%%%%%%%%%%%%%%%%%%%%%%%%%%%%%%%%%%%%%%
\subsection{The space of all connections}

\begin{definition}
If $P \to M$ is a $G$-principal bundle we can denote the \bam{space of all connections on $P$} as $\mc{C}_P$
\end{definition} 

\begin{remark}
	Some will denote this space as $\mc{A}_P$. 
\end{remark}

\begin{prop}
	$\mc{C}_P \neq \emptyset$, i.e. connections always exist. 
\end{prop}
\begin{proof}
	First note 
	\begin{itemize}
		\item A convex combination of connection 1-forms is a 1-form
		\item We can get trivial connection 1-forms on each local trivialisation. 
	\end{itemize}
	With these two facts, we can make a connection by using a partition of unity subordinate to the open cover giving the trivialisation. 
\end{proof}

\begin{prop}
	$\mc{C}_P$ is an affine space modelled on the vector space $\Gamma(T^\ast M \otimes \ad(P))$. 
\end{prop}
\begin{proof}
	Given $P\overset{\pi}{\to} M$ and $\ad P \to M$ we can form the pullpack bundle $\pi^\ast \ad P$ and 
	\eq{
\pi^\ast \ad P &\cong P \times \mf{g} \\
(p,[p,X]) &\mapsto (p,X)	
}
We can check though that 
\eq{
\image \pround{ \pi^\ast : \Omega^1(M;\mf{g}) \to \Omega^1(P;\mf{g})} = \pbrace{ \alpha \, | \, \alpha_p(\tilde{X})=0}
}
i.e the image is zero on $V_p$
\end{proof}

\begin{remark}
	$\mc{C}_P$ is not a vector space, but this is the next best thing. This means that $\Gamma(T^\ast M \otimes \ad(P))$ acts as an additive group on $\mc{C}_P$ freely and transitively, i.e. $\mc{C}_P$ is a torsor for the additive group. Alternatively this can be seen as saying that the difference between two connections gives an element of $\Gamma(T^\ast M \otimes \ad(P))$. 
\end{remark}

Let us try and understand this result in terms of the three ways to view a connection. 
\begin{enumerate}
	\item \bam{Splitting of the Atiyah sequence}: We can think of a section of $T^\ast M \otimes \ad(P)$ as a $C^\infty(M)$-linear map from $TM$ to $\ad(P)$. 
	\begin{tkz}
		0 \arrow[r] & \ad(P) \arrow[r] & \faktor{TP}{G} \arrow[r] & TM \arrow[r] \arrow[ll,dashed, bend left=20] & 0 
	\end{tkz}
$\ad(P)$ is a vector sub-bundle of $\faktor{TP}{G}$, and as it is exactly the kernel of $\faktor{TP}{G} \to TM$ we can add it to any splitting of such to get a new splitting. Similarly, given any two splittings of $\faktor{TP}{G} \to TM$ the difference has to be in the kernel of the map, i.e. takes values in $\ad(P)$, so gives a well defined map $TM \to \ad(P)$. 
\item \bam{Distribution on $TP$}: Take a vector space $V=V_1 \oplus V_2$. Then any linear map $f:V_1 \to V_2$ gives a linear subspace of $V$ 
\eq{
\text{graph}(f) = \pbrace{v \oplus f(v) \, | \, v \in V_1}
}
with $\dim \text{graph}(f) = \dim V_1$. This subspace lets us write 
\eq{
V = \text{graph}(f) \oplus V_2
}
Now suppose we have a distribution $H \subset TP$ and a section of $T^\ast M \otimes \ad(P)$. The latter can be pulled back to $P$ and using $H$ and using $H$ it gives a $G$-invariant linear map from $H$ to the vertical tangent space. This means we have $TP = H \oplus V$ and a linear map $H \to V$. As such we get a new distribution from $\text{graph}(H \to V)$. 
\item \bam{$\mf{g}$-valued one-from on $P$}: Let $\omega$ be the connection one-form and $\gamma:TP \to \ad(P)$ be the section of $T^\ast M \otimes \ad(P)$. From $\gamma$ we can build $\tilde{\gamma} = \gamma \circ d\pi : TP \to \mf{g}$. $\tilde{\gamma} + \omega$ is a new connection one-form. 
\end{enumerate}

%%%%%%%%%%%%%%%%%%%%%%%%%%%%%%%%%%%%%%%%%%%%%%%%%%%%%%%%
%%%%%%%%%%%%%%%%%%%%%%%%%%%%%%%%%%%%%%%%%%%%%%%%%%%%%%%%
\subsection{Curvature}

To any connection on $P$ we can associate a curvature - a section of $\wedge^2 T^\ast M \otimes \ad(P)$. To aid in the definition we need the following lemma:

\begin{lemma}
	If $E\to M$ is a vector bundle then a rule that assigns a section of $E$ to each $r$-tuple of vector fields $X^1, \dots, X^r$ which is alternating in the $X^i$ and $C^\infty(M)$-linear corresponds to a section of $\wedge^r T^\ast M \otimes E$. 
\end{lemma}

Now suppose we have a Kozul connection $\nabla$ on $E$

\begin{lemma}
	Given $s \in \Gamma(E), \, X,Y \in \Gamma(TM)$, the expression 
	\eq{
	\nabla_X \nabla_Y s - \nabla_Y \nabla_X s - \nabla_{\comm[X]{Y}}s
}
is 
\begin{itemize}
	\item $C^\infty$ linear in $s,X,Y$ 
	\item alternating in $X,Y$
\end{itemize}
\end{lemma}
\begin{proof}
	\hl{Exercise}
\end{proof}

\begin{corollary}
	The expression corresponds to $F_\nabla \in \Gamma(\wedge^2 T^\ast M \otimes \End(E))$ given by 
	\eq{
F_\nabla(X,Y)(s) = \nabla_X \nabla_Y s - \nabla_Y \nabla_X s - \nabla_{\comm[X]{Y}}s 	
}
\end{corollary}

We can again interpret this three ways via the different ways of thinking about a connection. We will need the following definition:

\begin{definition}
	Given a connection on $P$ given by a distribution $H \subset TP$, for $X \in \Gamma(TP)$ write $X^H$ for the horizontal component. Now for $\phi \in \Omega^r(P) \otimes V$, where $V$ is a vector space, define the \bam{covariant derivative} of $\phi$ to be the $V$-valued $(r+1)$-form $D\phi$ given by
	\eq{
D\phi(X_0, \dots, X_r) = d\phi(X_0^H, \dots, X_r^H)	
} 
\end{definition}
\begin{enumerate}
	\item \bam{Splitting of the Atiyah sequence}: Given $U\subset M$ open we can form the exact sequence 
	\begin{tkz}
		0 \arrow[r] & \Gamma(U,\ad(P)) \arrow[r] & \Gamma\pround{U,\faktor{TP}{G}} \arrow[r] &TU \arrow[r] & 0 
	\end{tkz}
This is in fact an exact sequence of Lie algebra - bunldes where the maps in the sequence respect the bracket structure. If we let $\gamma : TM \to \faktor{TP}{G}$ be the map that splits the Atiyah sequence then we define $F$ by 
\eq{
F(X,Y) = \comm[\gamma(X)]{\gamma(Y)} - \gamma(\comm[X]{Y})
}
As the Lie algebra structure is respected by the exact sequence we must have that $F(X,Y) \mapsto 0 \in TM$, and we have by exactness that is must take values in $\ad(P)$. 
	\item \bam{Distribution on $TP$}: Given a choice of horizontal subspace $H_p \subset T_pP$ we can lift $X \in \Gamma(TM)$ to a horizontal $\tilde{X} \in \Gamma(TP)$. Given $Y\in \Gamma(TM)$ we then get $\tilde{F}$ defined by 
	\eq{
	\tilde{F}(X,Y) = \comm[\tilde{X}]{\tilde{Y}} - \tilde{\comm[X]{Y}}
}
$\tilde{F}(X,Y) \in \Gamma(TP)$, and moreover $\tilde{F}(X,Y) \in \ker d\pi$, so it is vertical. Further $G$-invariance under the adjoint action means that $\tilde{F}$ to $F \in \Gamma(\wedge^2 T^\ast M \otimes \ad(P))$ a section on $M$ (it is clear that $F$ is alternating in $X,Y$ and is $C^\infty(M)$-linear). 
	\item \bam{$\mf{g}$-valued one-form on $P$}: Let $\omega$ be the connection one-form, with corresponding distribution $H$. We then get the curvature $\tilde{F} \in \Gamma(\wedge^2 T^\ast P \otimes \mf{g})$ by 
	\eq{
\tilde{F} = D \omega	
}
One can check (\hl{exercise}) that $\tilde{F}$ is $G$-equivariant in the sense that 
\eq{
R_g^\ast \tilde{F} = \Ad_{g^{-1}} \tilde{F} 
}
Becuase of the equivariance this descends to a section $F \in \Gamma(\wedge^2 T^\ast M \otimes \ad(P))$. 
\end{enumerate}

\begin{prop}
	In terms of just the connection one-form we can write 
	\eq{
\tilde{F}(X,Y) = d\omega(X,Y) + \comm[\omega(X)]{\omega(Y)}	
}
We often write this as the short hand 
\eq{
\tilde{F} = d\omega + \omega \wedge \omega 
}
which is \bam{Cartan's structure equation}. 
\end{prop}

\begin{theorem}[Bianchi identity]
	$D \tilde{F} = 0$
\end{theorem}
\begin{proof}
	We have
	\eq{
D \tilde{F}(X_1, X_2, X_3) &= d \tilde{F}(X_1^H, X_2^H, X_3^H) \\
&= d^2 \omega (X_1^H, X_2^H, X_3^H) + d\omega \wedge \omega (X_1^H, X_2^H, X_3^H) - \omega \wedge d\omega (X_1^H, X_2^H, X_3^H) \\
&= 0
}
as $d^2 = 0$ and $H=\ker \omega$ so $\omega(X^H)=0$. 
\end{proof}

\begin{definition}
	A connection is \bam{flat} (or \bam{integrable}) if its curvature is 0. 
\end{definition}

\begin{theorem}
	If $P$ is a $G$-principal bundle equipped with a flat connection then there exist local trivialisations such that all transition functions are constant. 
\end{theorem}
\begin{proof}
	\hl{exercise}
\end{proof}
%%%%%%%%%%%%%%%%%%%%%%%%%%%%%%%%%%%%%%%%%%%%%%%%%%%%%%%%
\subsection{Local expressions}

Recall that a bundle is trivial iff it has a global section $\sigma : M \to P$. This gives a canonical choice of connection given by 
\eq{
H = \ker(p_2 : M \times G \to G)
}
We refer to this induced connection as $d^\sigma$. \\
Now any two connections differ by a section of $T^\ast M \otimes \ad(P)$, and $\ad(P)$ is also trivialised as $\ad(P) \cong M \times \mf{g}$ if $P$ is, we can write any other connection as 
\eq{
d^\sigma + A
}
If the connection is given by one-form $\omega$ we have $A = \sigma^\ast \omega$. \\
Suppose we have expressed our $G$-principal bundle $P$ in terms of local trivialisations - that is we have written $M = \cup_\alpha U_\alpha$, taken $\Psi_\alpha : \pi^{-1}(U_\alpha) \overset{\cong}{\to} U_\alpha \times G$ with transition maps $\phi_{\alpha\beta}:U_{\alpha\beta} \to G$, and we have expressed this data as a collection of local sections $\sigma_\alpha : U_\alpha \to P$. In this case we may carry out the proceedure as before to write a connection locally as 
\eq{
d^{\sigma_\alpha} + A^\alpha
}
These are related according to the following results:
\begin{theorem}[compatibiltity conditions]
	Let $\theta$ be the LI MC one-form. Then 
	\eq{
A^\beta = \Ad_{\phi_{\beta\alpha}}(A^\alpha) + \phi_{\alpha\beta}^\ast \theta	
}
\end{theorem}
\begin{proof}
\hl{exercise}
\end{proof}

\begin{corollary}
	If $G$ is a matrix group, we have 
	\eq{
A^\beta = \phi_{\beta\alpha} A^{\alpha} \phi_{\beta\alpha}^{-1} + \phi_{\beta\alpha} d\phi_{\beta\alpha}	
}
\end{corollary}

\begin{remark}
	Physicists often refer to the $A^\alpha$ as \bam{local gauge potentials}. 
\end{remark}

\begin{prop}
	In terms of the local gauge potentials the cruvature is given by 
	\eq{
F = dA^\alpha + A^\alpha \wedge A^\alpha	
}
\end{prop}


Recall we know there is a group isomorphism between $\mc{G}(P)$, the group of gauge transformations, and $\Gamma(\Ad(P))$, the group of section of the bundle $\Ad(P)$. If we trivialise $P$ we trivialse $\Ad(P)$ making the sections into $G$-valued functions. The local trivialisations, $G$-valued functions $\psi_\alpha$, are related by 
\eq{
\psi_\beta = \phi_{\beta\alpha} \psi_\alpha \phi_{\beta\alpha}^{-1}
}
\begin{prop}
$\mc{G}(P)$ then acts locally by the compatibility condition on the space of all connections as 
\eq{
A^\alpha \mapsto \Ad_{\psi_\alpha} A^\alpha + \psi_\alpha^\ast \theta
}
\end{prop}

%%%%%%%%%%%%%%%%%%%%%%%%%%%%%%%%%%%%%%%%%%%%%%%%%%%%%%%%
%%%%%%%%%%%%%%%%%%%%%%%%%%%%%%%%%%%%%%%%%%%%%%%%%%%%%%%%
\section{Characteristic Classes in Chern-Weil Theory}

\begin{lemma}
Suppose we have 
\eq{
\phi : \underbrace{\mf{g} \otimes \dots \otimes \mf{g}}_{\times k} \to \mbb{R}
}
multilinear, symmetric, and ad-invariant. Think of $\phi(\tilde{F}, \dots, \tilde{F}) \in \Omega^{2k}(M,\mbb{R})$, then if $\omega, \omega^\prime$ are two connections
\eq{
\exists \alpha \in \Omega^{2k-1}, \, \phi(\tilde{F}_\omega, \dots, \tilde{F}_\omega) - \phi(\tilde{F}_{\omega^\prime}, \dots, \tilde{F}_{\omega^\prime}) = d\alpha 
}
\end{lemma}

\begin{corollary}
	$[\phi(\tilde{F}, \dots, \tilde{F})]\in H^{2k}_{dR}$ is a cohomology class. 
\end{corollary}

\begin{example}
	If $G=U(n)$, define $c_k$ by 
	\eq{
\sum_{k=0}^n c_k(X) t^k = \det \pround{I - \frac{t}{2\pi i}X}	
}
for $X \in \mf{u}(n)$ and then 
\eq{
c_k(P) = c_k (\tilde{F}, \dots, \tilde{F}) \in H^{2k}
}
for some $G$-bundle p, the \bam{Chern class}. 
\end{example}

\begin{remark}
	Here we are using the polarisation identity to define $c_k$ on multiple elements. 
\end{remark}

\begin{example}
	If $G= O(n), SO(n)$ then have 
	\eq{
\sum_k p_k(X) t^k = \det \pround{I-\frac{t}{2\pi}X} 	
}
for $X \in \mf{g}$, and 
\eq{
p_k(P) = p_k(\tilde{F}, \dots, \tilde{F}) \in H^{4k}
}
is the \bam{Ponytryagin class}
\end{example}

\begin{example}
	If $G=O(2n)$, define 
	\eq{
\Delta(X) = \pround{\frac{-1}{2\pi}}^m \Pfaff(X)	
}
The \bam{Euler class} 
\end{example}
%%%%%%%%%%%%%%%%%%%%%%%%%%%%%%%%%%%%%%%%%%%%%%%%%%%%%%%%
%%%%%%%%%%%%%%%%%%%%%%%%%%%%%%%%%%%%%%%%%%%%%%%%%%%%%%%%
\section{The Yang-Mills Theory}
%%%%%%%%%%%%%%%%%%%%%%%%%%%%%%%%%%%%%%%%%%%%%%%%%%%%%%%%
\subsection{Implicitly used results}

\hl{Maybe move this earlier}
We will first start by spelling out explicitly some results that we have been using but not making clear:

\begin{lemma}
	If $P \to M$ is a $G$-principal bundle, $V \to P$ is a vector bundle equipped with a linear lift of the action $P \ract G$ then $\faktor{V}{G}$ is a vector bundle over $M$ of the same rank as $V$ over $P$. 
\end{lemma}

\begin{lemma}
	Sections of the bundle $\faktor{V}{G} \to M$ correspond exactly to $G$-equivarant sections of $V \to P$, i.e. sections $s:V\to P$ s.t. 
	\begin{tkz}
	V \arrow[r,"g"] \arrow[d] & V \arrow[d] \\ P \arrow[r,"g"'] \arrow[u,bend left=20,"s"] & P \arrow[u,bend right=20, "s"']
	\end{tkz} 
\end{lemma}

\begin{lemma}
	pull-backs of sections of $\wedge^2 T^\ast M \otimes \faktor{(P \times V)}{G}$ are exactly basic $V$-valued forms on $P$. 
\end{lemma}

The reason we have made these explicit is as it's important to realise it is the case with $V= \mf{g}$ that is necessary for us, as it means we can descends $\tilde{F} \in \Gamma(\wedge^2 T^\ast P \otimes \mf{g})$ (a $\mf{g}$-valued form on $P$) to $F\in \Gamma(\wedge^2 T^\ast M \otimes \ad(P))$. 

%%%%%%%%%%%%%%%%%%%%%%%%%%%%%%%%%%%%%%%%%%%%%%%%%%%%%%%%
\subsection{The Hodge star operator}
\hl{Maybe move this to the Riemannian geometry section}

\begin{definition}
	A \bam{volume form} of a differentiable manifold $M$ is a top-degree form (i.e. form of degree $\dim M$). 
\end{definition}

\begin{definition}
	An \bam{oriented} manifold is one equipped with a nowhere-vanishing volume form. 
\end{definition}

\begin{lemma}
	Every Riemannian manifold $(M,g)$ has a canonical choice of volume form given in local coordinates $x^1, \dots, x^n$ by 
	\eq{
	\omega_{vol} = \sqrt{\abs{g}} dx^1 \wedge \dots \wedge dx^n
} 
\end{lemma}

\begin{lemma}
	If $V$ is a vector space with bilinear $\pangle{\cdot, \cdot}$ then $\bigwedge^k V$ inherits a bilinear by the Grammian determinant. 
	\eq{
\pangle{\alpha_1 \wedge \dots \wedge \alpha_k, \beta_1 \wedge \dots \wedge \beta_k} = \det \pround{\pangle{\alpha_i,\beta_j}_{i,j=1}^k}	
}
\end{lemma}

\begin{definition}
	If $M$ is a $n$-dimensional Riemannian manifold, the \bam{Hodge star} is the operator $\star : \Omega^r(M) \to \Omega^{n-r}(M)$ given by 
	\eq{
	\alpha \wedge (\star \beta) = \pangle{\alpha,\beta} \omega_{vol}
}
for $\alpha,\beta \in \Omega^r(M)$, where $\pangle{\cdot,\cdot}$ is the Euclidean structure on $\Omega^r(M)$. 
\end{definition}

\begin{lemma}
	This property defines $\star$ completely. 
\end{lemma}
\begin{proof}
	\hl{exercise}
	\end{proof}

\begin{prop}
	If $M$ has determinantal sign $\Delta$ of the inner product, then 
	\eq{
\star \star \alpha = (-1)^{r(n-r)} \Delta \alpha	
}
\end{prop}
%%%%%%%%%%%%%%%%%%%%%%%%%%%%%%%%%%%%%%%%%%%%%%%%%%%%%%%%
\subsection{Yang-Mills functional}
We are now going to restrict our mathematical picture. We will make the following assumptions:
\begin{itemize}
	\item $M$ is an oriented (pseudo-)Riemannian manifold
	\item $\mf{g}$ has a Euclidean inner product invariant under the adjoint action $G \lact \mf{g}$. (This gives a Eulidean structure on the bundle $\ad(P)$). 
\end{itemize}

\begin{remark}
For compact, semi-simple Lie groups an $\ad$-invariant Euclidean inner product on $\mf{g}$ is given by the Killing form. For other compact Lie groups we can take a faithful representation $\rho : G \to GL(V)$ and then have 
\eq{
\pangle{X,Y} = -\tr(d\rho(X) \circ d\rho(Y))
}
For simplicity we will notate this as 
\eq{
\pangle{X,Y} = -\tr(XY)
}
\end{remark}

What we have gained is that the bundle which the curvature takes values in has a Euclidean structure, as such we can make the following definition:

\begin{definition}
	The \bam{Yang-Mills function} is the functional $S_{YM}:\mc{C}_P \to \mbb{R}$ given by 
	\eq{
	S_{YM} = \int_M \abs{F}^2 \omega_{vol}
}
where $F$ is the curvature associated to the connection the functional is evalutated at. 
\end{definition}

\begin{lemma}
	$S_{YM}$ is invariant under the action of the gauge group $\mc{G}(P)$. 
\end{lemma}

\begin{corollary}
	The functional becomes a well-defined function $S_{YM} : \faktor{\mc{C}_P}{\mc{G}(P)} \to \mbb{R}$. 
\end{corollary}

\begin{lemma}
	The Yang-Mills functional can be written as
	\eq{
S_{YM} \propto \int_M \tr(F \wedge \star F)	
}
\end{lemma}

\begin{remark}
	If $M$ is compact, the integration over $M$ is always fine. If not, we need to restrict to connections for which $S_{YM}$ is well-defined. 
\end{remark}

%%%%%%%%%%%%%%%%%%%%%%%%%%%%%%%%%%%%%%%%%%%%%%%%%%%%%%%%
\subsection{The Yang-Mills equations}

We now want to derive equations that give us the stationary points for the action (as is standard for an action theory). We will work with a local trivialisation, and then for notation simplicity omit the index $\alpha$. For some generic $\tau \in \Gamma(T^\ast M \otimes \ad(P))$ we want to find $A$ s.t.  
\eq{
\ev{\frac{d}{dt}S_{YM}(A+t\tau)}= 0
}
We can calculate the associated curvature to this new connection is 
\eq{
F_t &= d(A + t\tau)+ (A + t\tau) \wedge (A + t\tau) \\
&= F + t\pround{d\tau + A \wedge \tau + \tau \wedge A} + t^2 \tau \wedge \tau
}
We now need the following lemma
\begin{lemma}
	The covariant derivative descends to an operator 
	\eq{
	d^A : \Gamma(\wedge^r T^\ast M \otimes \ad(P)) \to \Gamma(\wedge^{r+1}T^\ast M \otimes \ad(P))
}
\end{lemma}
\begin{proof}
	By the definition that for $\phi \in \Gamma(\wedge^r T^\ast P \otimes V)$ 
	\eq{
D\phi (X_0, \dots, X_r) = d\phi(X_0^H, \dots, X_r^H)	
}
we can see that $D\phi$ is horizontal. It is also $G$-invariant if $\phi$ is, and as such $D$ maps basic forms to basic forms. As such it descends as previously discussed.
\end{proof}

\begin{corollary}
	The Bianchi identity descends to $d^A F = 0$. 
\end{corollary}

\begin{remark}
	We will want to make use of the fact throughout that for forms 
	\eq{
d^A \tau = d\tau + \psquare{A \wedge \tau}	
}
where $\psquare{\omega \wedge \eta} = \omega \wedge \eta - (-1)^{pq} \eta \wedge \omega$ where the wedge without square brackets of forms should just take the multiplication of their algebra element coefficients. Be aware that the covariant derivative acts differently on connections and this is why we still have 
\eq{
F = d^A A = dA + \frac{1}{2}\psquare{A \wedge A} = dA + A \wedge A
}
To see more on this check out \href{https://en.wikipedia.org/wiki/Lie_algebra-valued_differential_form}{Lie algebra-valued differential forms} on Wikipedia and the 2006 gauge theory notes by Jose (pg 18).  
\end{remark}
Now using that $d^A \tau  = d\tau + A \wedge \tau$ we get 
\eq{
F_t &= F + t d^A \tau + t^2 \tau \wedge \tau \\
\Rightarrow \abs{F_t}^2 &= \abs{F}^2 + t \pangle{d^A \tau, F} + \mc{O}(t^2)
}
Hence the Yang-Mills equation is equivalent to 
\eq{
\int_M \pangle{d^A \tau, F} \omega_{vol} = 0
}
Using the formal adjoint $(d^A)^\ast$ and realising $\tau$ is generic we get 
\eq{
(d^A)^\ast F = 0
}
We make use of the following lemma:
\begin{lemma} Up to a sign 
	Let $d^\ast$ be the formal adjoint of $d$ wrt the inner product $\pangle{\pangle{\eta,\omega}} = \int_M \pangle{\eta,\omega} \omega_{vol}$ (the \bam{codifferential}). Then when acting on a $k$ form 
	\eq{
d^\ast = (-1)^{n(k-1)+1}s \star d \star	
}
where $n=\dim M$, $s$ is the parity of the signature of $M$, 
\end{lemma}
\begin{remark}
	The proof of the above lemma on the codifferential requires the application of Stokes' theorem on $M$. The same result can be shown for the covariant derivative if we are in a situation where the conditions for the non-abelian Stokes' theorem apply. This is the case when we are considering $(d^A)^\ast F$ (supposedly, see \cite{Schreiber2011} and nlab ).
\end{remark}
As such we get the \bam{Yang-Mills equation}
\eq{
d^A(\star F) = 0
}
%%%%%%%%%%%%%%%%%%%%%%%%%%%%%%%%%%%%%%%%%%%%%%%%%%%%%%%%
\subsection{Gauge theory in 4d - Abelian gauge theory and electromagnetism}

Recall Maxwell's equations for electromagnetism 
\eq{
\bm{\nabla} \cdot \bm{E} &= 4\pi\rho \\
\curl \bm{B} &= \frac{4\pi}{c} \bm{J} + \frac{1}{c} \del_t \bm{E} \\
\curl \bm{E} &= -\frac{1}{c} \del_t \bm{B} \\
\bm{\nabla} \cdot \bm{B} &= 0 
}
We are going to see how this can be derived as a Yang-Mills gauge theory. \\
Restricting to the vacuum Maxwell equations where $\rho = 0, \, \bm{J} = 0$, we can build a two form 
\eq{
F = \tilde{B} - c dt \wedge E
}
where 
\eq{
E &= E_x dx + E_y dy + E_z dz \\
B &= B_x dx + B_y dy + B_z dz \\
 \tilde{B} = \star_3 B &= B_x dy \wedge dz + B_y dz \wedge dx + B_z dx \wedge dy 
}
are the natural one and two forms associated to a vector in 3d (using the 3d Hodge star to get the two form) written in cartesian coordinates. Now consider a $U(1)$-principal bundle $P$ over $\mbb{R}^4$ with Lorentzian signaure. As the Lie group is abelian, the commutator in the Lie algebra is 0 and $d^A = d$ is the standard differential for any connection. We now calculate
\eq{
dF &= d\tilde{B} + c dt \wedge dE \\
&= (\bm{\nabla}\cdot \bm{B})dx \wedge dy \wedge dz + \psquare{\del_t B_x + c(\del_y E_z - \del_z E_y)}dt \wedge dy \wedge dz \\
&\phantom{=} + \psquare{\del_t B_y + c(\del_z E_x - \del_x E_z) } dt \wedge dz \wedge dx + \psquare{\del_t B_z + c(\del_x E_y - \del_y E_x )} 
}
so this gives us two of the Maxwell equations, namely $\bm{\nabla} \cdot \bm{B}= 0$ and $\curl \bm{E} = -\frac{1}{c} \del_t \bm{B}$. Then
\eq{
\star_4 F &= c \tilde E + dt \wedge B
}
We can then see the duality between $E$ and $B$ and we get the other two equations from $d(\star F)=0$. \\
As the base manifold is simply connected it has trivial cohomology, and any closed two-form is exact. Hence $\exists \mc{A}$ s.t. $F = d\mc A$. Write 
\eq{
\mc A = -\phi dt + A
} 
where $A$ is the one form that corresponds to a vector $\bm{A}$ in 3d. We then recover the formula from electromagnetism that 
\eq{
\bm{B} &= \curl \bm{A} \\
\bm{E} &= - \grad\phi - \del_t \bm{A}
}
$\mc{A}$ is the connection for which $F$ is the curvature, and so we know we are free to apply a gauge transformation, under which (if $\phi_{\beta\alpha}=e^{i\psi}$) we get 
\eq{
\mc{A} \mapsto \mc{A} - i \, d\psi
}
%%%%%%%%%%%%%%%%%%%%%%%%%%%%%%%%%%%%%%%%%%%%%%%%%%%%%%%%
\subsection{Gauge theory in 4d - Instantons}

We now want to consider $M=\mbb{R}^4$ with Euclidean signature (in practice this might be achieved though a Wick rotation of a Lorentzian metric if necessary). We will, as previously stated, being considering configurations which decay sufficiently quickly to infinity. This means we can add the point at infinity to $M$ to obtain $S^4$.

\begin{remark}
	Altough $\mbb{R}^4$ does not have an isometric embedding into $S^4$, it does have a conformal embedding, and this will turn out to be sufficient, because the defining conditions (namely the seul duality ones) are conformally invariant.  
\end{remark}

Let us now use some useful results in 4d: 

\begin{lemma}
	In 4d, $(\star)^2 = 1$, so $\star$ has eigenvalues $\pm 1$. 
\end{lemma}

\begin{definition}
	Decompose, as eigenspaces of $\star$, $\Omega^2(M) = \Omega^2_+(M) \oplus \Omega^2_-(M)$. We call these the \bam{self-dual} and \bam{anti-self-dual} parts.
\end{definition}

\begin{lemma}
	Elements of $\Omega^2_+(M), \, \Omega^2_-(M)$ are orthogonal. 
\end{lemma}
\begin{proof}
	We have 
	\eq{
\pangle{\omega_+, \omega_-}\omega_{vol} &= \omega_+ \wedge \star \omega_- = -\omega_+ \wedge \omega_- \\
\pangle{\omega_-,\omega_+} \omega_{vol} &= \omega_- \wedge \star \omega_+ = \omega_+ \wedge \omega_-
}
but $\pangle{\omega_+, \omega_-} = \pangle{\omega_-, \omega_+}$ so must be 0. 
\end{proof}

As such we can decompose the Yang-Mills functional as 
\eq{
S_{YM} = \int_M \abs{F}^2 \omega_{vol} = \int_M \abs{F_+}^2 \omega_{vol} + \int_M \abs{F_-}^2 \omega_{vol} 
}

Now we make the following definition
\begin{definition}
	We call 
	\eq{
c = \int_M \tr(F \wedge \highlight{\star}F)	
}
the \bam{second Chern number}
\end{definition}	
	
\begin{prop}
	$c$ depends only on the principal bundle and not the choice of connection. 
\end{prop}
	
\begin{lemma}
	$c = \int_M \abs{F_+}^2 \omega_{vol} - \int_M \abs{F_-}^2 \omega_{vol}$
\end{lemma}
\begin{proof}
Expanding $F = F_+ + F_-$ and using $\tr(F \wedge \star F) = \abs{F}^2 \omega_{vol}$ gives the result. 
\end{proof}
	
Hence we get the inequality 
\eq{
S_{YM} \geq \abs{c}
}
with equality iff either $F= F_+$ or $F=F_-$. Note in either case the Yang-Mills equations are trivially satisfied as 
\eq{
d^A (\star F) = \pm d^A F  = 0 \quad \text{by the Bianchi identity}
}

\begin{definition}
	An instanton is a solution to the (anti-)self-dual euqations which can be extended to $S^4$. 
\end{definition}

We want to understand instantons up to gauge transformation, and the space parameterising these is the \bam{moduli space of instantons}. 

%%%%%%%%%%%%%%%%%%%%%%%%%%%%%%%%%%%%%%%%%%%%%%%%%%%%%%%%
%%%%%%%%%%%%%%%%%%%%%%%%%%%%%%%%%%%%%%%%%%%%%%%%%%%%%%%%
\bibliographystyle{../../bib/custom-bib-style}
\bibliography{../../bib/library}


\end{document}