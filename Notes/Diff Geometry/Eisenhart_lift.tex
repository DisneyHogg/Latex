\documentclass{article}
\usepackage{../../header}

\title{The Eisenhart Lift}
\author{Linden Disney-Hogg \& Harry Braden}
\date{March 2020}

\begin{document}

\maketitle

%%%%%%%%%%%%%%%%%%%%%%%%%%%%%%%%%%%%%%%%%%%%%%%%%%
%%%%%%%%%%%%%%%%%%%%%%%%%%%%%%%%%%%%%%%%%%%%%%%%%%
\section{The Eisenhart Lift}
%%%%%%%%%%%%%%%%%%%%%%%%%%%%%%%%%%%%%%%%%%%%%%%%%%
\subsection{The metric and equations of motion}
Consider the $(d+2)$-dimensional line element,
\begin{equation}
\label{eislift}
ds^2=\hat g_{\mu\nu}\,dx\sp{\mu}\,dx\sp{\nu}
=h_{ij}\, dx\sp{i}\, dx\sp{j}+2dt \left( dv-\Phi dt+N_i dx\sp{i}\right),
\end{equation}
where $i,j=1,\ldots,d$, $x\sp{d+1}=t$,  $x\sp{d+2}=v$ and $\Phi$, $N_i$ and $h_{ij}$ are independent of the coordinate $v$. Then
$\xi = \partial_v$ is a Killing vector.
We have
$$\hat g=\begin{pmatrix}h_{ij}&N_i&0\\ N_j&-2\Phi&1\\ 0&1&0\end{pmatrix},\qquad
\hat g\sp{-1}=\begin{pmatrix}h\sp{ij}&0&-h\sp{ik}N_k\\ 0&0&1\\ -h\sp{jk}N_k &1&2 \Phi+ N_i h\sp{ij} N_j\end{pmatrix},
$$
where $h\sp{ij}$ is the inverse of $h_{ij}$. The geodesic Lagrangian is
$$\mathcal{ L}=\frac12 \hat  g_{\mu\nu}\,\dot x\sp{\mu}\,\dot x\sp{\nu}=
\frac12 h_{ij}\, \dot x\sp{i}\, \dot x\sp{j}+\dot t \dot v-\Phi {\dot t}^2+N_i \,\dot x\sp{i}\dot t
:= \tilde{L}+\dot t \dot v  ,
$$
where $\dot x\sp{\mu}=dx\sp{\mu}/d\lambda$ for an affine geodesic parameter $\lambda$ ($\tilde{L}$ is defined below).
Calculating the equations of motion from $\mathcal{L}$ enables a simple determination of (appropriate
combinations of) the  Christoffel symbols for $\hat g$. Recall
$$\hat \Gamma\sp{\mu}_{\, \nu \rho}=\frac12 \hat g\sp{\mu \delta}\left(
\hat g_{\delta \nu, \rho} + \hat g_{\delta \rho, \nu} - \hat g_{ \nu \rho,\delta}\right)
:=  \hat g\sp{\mu \delta}[ \nu \rho,\delta]_{\hat g}.
$$
and the equations of motion are
$$ 0=\ddot x\sp{\mu}+ \hat \Gamma\sp{\mu}_{\, \nu \rho} \dot x\sp{\mu}\dot x\sp{\rho}.
$$
Setting
$$A:=A_\mu dx\sp{\mu}=N_i dx\sp{i}-\Phi dt,\qquad
F=dA=\frac12 (\partial_\mu A_\nu- \partial_\nu A_\mu)\, dx\sp\mu \wedge dx\sp\nu =\frac12 F_{\mu\nu}
\, dx\sp\mu \wedge dx\sp\nu
$$
we get 
\eq{
F_{ij} &= \del_i N_j - \del_j N_i = -F_{ji} \\
F_{it} &= -(\del_t N_i + \del_i \Phi) = -F_{ti}
}
and using that the equations of motion for $v$, $x\sp{i}$ and $t$ (from $\frac{d}{d\lambda}\pd[\mc{L}]{\dot{x^\mu}} = \pd[\mc{L}]{x^\mu}$) yield
\eq{
0 = \frac{d}{d\lambda} \dot{t} =\ddot{t}
}
(for $v$) then 
\eq{
	\frac{1}{2} (\del_ih_{jk}) \dot{x}^j \dot{x}^k - (\del_i\Phi) \dot{t}^2 + (\del_i N_j) \dot{x}^j \dot{t} &= \frac{d}{d\lambda} \pround{h_{ij} \dot{x}^j + N_i \dot{t}} \\
	&=  h_{ij} \ddot{x}^j + (\del_kh_{ij}) \dot{x}^j\dot{x}^k + (\del_t h_{ij}) \dot{x}^j \dot{t} + (\del_j N_i) \dot{x}^j \dot{t}  +(\del_t N_i) \dot{t}^2 
}
(for $\dot{x}^i$) and 
\eq{
 \frac{1}{2}(\del_t h_{ij}) \dot{x}^i \dot{x}^j-(\del_t \Phi) \dot{t}^2 + (\del_t N_i) \dot{x}^i \dot{t} &= \frac{d}{d\lambda} \pround{\dot{v} - 2\Phi\dot{t} + N_i \dot{x}^i} \\
 &= \ddot{v} -2(\del_i \Phi) \dot{t} \dot{x}^i -2(\del_t \Phi) \dot{t}^2 -2\Phi \ddot{t} + (\del_j N_i) \dot{x}^i \dot{x}^j + (\del_t N_i) \dot{x}^i \dot{t} + N_i \ddot{x}^i
}
we get (collating them together)
\begin{align*}
0&=\ddot t,\\
0&= h_{ij}\,\ddot x\sp{j}+[jk,i]_h\, \dot x\sp{j} \dot x\sp{k} +
(\partial_t h_{ij}+\partial_j N_i-\partial_i N_j) \,\dot t \dot x\sp{j} +\left(\partial_i\Phi+\partial_t N_i\right) \dot t^2,\\
&= h_{ij}\,\ddot x\sp{j}+[jk,i]_h\, \dot x\sp{j} \dot x\sp{k}+ (\partial_t h_{ij} -F_{ij})\, \dot t \dot x\sp{j}
+F_{ti}\dot t\sp2,\\
0&=\ddot v+ N_i\ddot x\sp{i}+ \left[ \frac12 \left(\partial_j N_i+\partial_i N_j\right)- \frac12 \partial_t h_{ij}
\right]\,\dot x\sp{i} \dot x\sp{j} -2\partial_i\Phi \,\dot t \dot x\sp{i} - \partial_t\Phi\, \dot t^2,\\
&=\ddot v+
 \left[ \frac12 \left(\partial_j N_i+\partial_i N_j\right)-N_k \Gamma\sp{k}_{ij}- \frac12 \partial_t h_{ij}
\right]\,\dot x\sp{i} \dot x\sp{j} 
+\left[ -N\sp{k} (\partial_t h_{ki} -F_{ki}) -2\partial_i\Phi \right]\,\dot t \dot x\sp{i} 
+\left(- \partial_t\Phi +N\sp{i}F_{it} \right)\, \dot t^2
\end{align*}
where we have substituted $F$ in the latter, and usde the notation
\eq{
[jk,i]_h = \frac{1}{2}\pround{\del_j h_{ki} + \del_k h_{ji} - \del_i h_{jk}} \, .
}
Recall 
\eq{
\Gamma^i_{jk}= h^{il}[jk,l]_h \, .
}
Note that to raise the index of $N$ has required we recognise that 
\eq{
	N^i = \hat{g}^{ij} N_j = h^{ij} N_j
}
%%%%%%%%%%%%%%%%%%%%%%%%%%%%%%%%%%%%%%%%%%%%%%%%%%
\subsection{Equivalence of equations of motion}
The canonical momenta are given by $p_\mu =\partial \mc{L}/\partial \dot x\sp{\mu}=\hat{g}_{\mu\nu}\dot{x}^\nu$ giving
$$p_v =\dot t,\qquad p_i=h_{ij} \dot x\sp{j} +N_i \dot t, \qquad p_t=\dot v- 2\Phi \dot t +N_i \dot x\sp{i},
$$
and so
$$\dot t=p_v,\qquad  \dot x\sp{i}=h\sp{ij}(p_j- N_j p_v),\qquad \dot v=p_t-N\sp{i} p_i+[2\Phi +N^2]\, p_v.$$

Likewise, the geodesic Hamiltonian is
$$\mathcal{H}=p_\mu \dot x\sp{\mu}-\mathcal{L} = \frac12 \hat{g}\sp{\mu\nu}\,p_{\mu}\, p_{\nu}=
\frac12 h\sp{ij}\, (p_{i}-N_i p_v)(p_j-N_j p_v) +p_t  p_v+\Phi \,p_v^2 .
$$
The equations of motion are
\begin{align*}
	\frac{dt}{d\lambda}&=\frac{\partial \mathcal{H}}{\partial p_t}=p_v, &
	\frac{dv}{d\lambda}&=\frac{\partial \mathcal{H}}{\partial p_v},& 
	\frac{d x^i}{d\lambda}&=\frac{\partial \mathcal{H}}{\partial p_i}=h\sp{ij}\, (p_j-N_j p_v),
	\\
	\frac{d p_t}{d\lambda}&=-\frac{\partial \mathcal{H}}{\partial t},
	&
	\frac{d p_v}{d\lambda}&=-\frac{\partial \mathcal{H}}{\partial v}=0,&
	\frac{d p_i}{d\lambda}&=-\frac{\partial \mathcal{H}}{\partial x^i}.
\end{align*}
Because $v$ is a cyclic coordinate its conjugate momentum $p_v$ is conserved along geodesics:
thus $p_v=m$ is a constant and we may write
$$\mathcal{H}:=H+m\, p_t,\qquad H:=\frac12 h\sp{ij}\, (p_{i}-m N_i )(p_j-m N_j) +m^2\, \Phi.$$
We observe that we have the geodesics have the conserved quantities,
\begin{align*}
	\frac12 \hat g\sp{\mu\nu}\, p_\mu p_\nu&=m\left[ \frac{p\sp{i}p_i}{2m}- N\sp{i}p_i +m N\sp{i}N_i
	+p_t +m\Phi \right] :=- m E_0,\\
	\hat g\sp{\mu\nu}\, p_\mu \xi_\nu&=p_v=m.
\end{align*}
Following the identifications of \cite{Duval1991} we view $p_v=m$ as the mass, $-p_t=E$  as the energy,
$E_0$ as the internal energy, and $m\Phi=V$ as the potential energy. Taking the internal energy to vanish in the nonrelativistic limit the null geodesics of $\hat g$ may be identified with the motion in the $d$-dimensional space with potential energy $V$. We note that two conformally related metrics have the same null geodesics, and so the $d$-dimensional world lines will be the same. For $m\ne 0$ the equations of motion for $t$ then give $dt/d\lambda=m$, whence $dt =m\,d\lambda$ and we may eliminate the affine geodesic parameter $\lambda$
for $t$. The equations of motion are then precisely those coming from the standard mechanical system
$$\tilde L= \frac12 h_{ij}\, \dot x\sp{i}\, \dot x\sp{j}+N_i \,\dot x\sp{i}-\Phi $$
where $ \dot x\sp{i}$ is now the standard $d x\sp{i}/dt$ (and $\dot t = 1$). Now
\begin{enumerate}[(a)]
	\item in the case of a non-null geodesic, if we parameterised the curve by arc length, $\lambda=s$ and $t =ms$, then from (\ref{eislift}) we have 
	$$\frac{dv}{dt}= \frac1{2m^2} -\tilde L.$$
	The equations of motion for $v$ follow from this and
	$$v=\frac{t}{2m^2}-\int \tilde L\, dt +b.$$
	\item in the case of a null geodesics we have
	$$\frac{dv}{dt}=  -\tilde L,\qquad  v=-\int \tilde L\, dt +b.$$
\end{enumerate}
Thus we have for each $m\ne0$ and $b$ a bijection between the geodesics of $\hat g$ and the
equations of motion of $\tilde L$.
%%%%%%%%%%%%%%%%%%%%%%%%%%%%%%%%%%%%%%%%%%%%%%%%%%
\subsection{Connection and Curvature}
From the equations of motion we read that the nonvanishing Christoffel symbols for $\hat g$ are
\begin{align*}
\hat \Gamma\sp{i}_{jk}&=  \Gamma\sp{i}_{jk},
&\hat \Gamma\sp{i}_{jt}&=-\frac12 F\sp{i}_{\ j}+\frac12 h\sp{ik}\partial_t h_{kj},
&\hat \Gamma\sp{i}_{tt}&=h\sp{ik} \left( \partial_t N_k +\partial_{k} \Phi \right)= -F\sp{i}_{\ t},
\\
\hat \Gamma\sp{v}_{tt}&=-\partial_t \Phi+N\sp{k}F_{kt}, 
&\hat \Gamma\sp{v}_{ij}&=\frac14\left[\nabla\sp{(h)}_{i}N_{j}+\nabla\sp{(h)}_{j}N_{i}-\partial_t h_{ij}\right],
&\hat \Gamma\sp{v}_{ti}&=-\frac12 N\sp{k} (\partial_t h_{ki} -F_{ki}) -\partial_i\Phi .
\end{align*}
Recall now the equation for the Riemann tensor 
\eq{
\hat{R}\indices{^\mu_\nu_\rho_\sigma} = \del_\rho \hat{\Gamma}^\mu_{\nu\sigma} - \del_\sigma \hat{\Gamma}^\mu_{\nu\rho} + \hat{\Gamma}^\mu_{\rho\lambda}\hat{\Gamma}^\lambda_{\nu\sigma} - \hat{\Gamma}^\mu_{\sigma\lambda}\hat{\Gamma}^\lambda_{\nu\rho}
}
We immediately notice 
\eq{
\hat{R}\indices{^i_j_k_l} = R\indices{^i_j_k_l} + \hat{\Gamma}^i_{kt}\hat{\Gamma}^t_{jl} - \hat{\Gamma}^i_{lt}\hat{\Gamma}^t_{jk} + \hat{\Gamma}^i_{kv}\hat{\Gamma}^v_{jl} -\hat{\Gamma}^i_{lv}\hat{\Gamma}^v_{jk} = R\indices{^i_j_k_l}
}
as there are non-vanishing Christoffel symbols with $v$ as lower index, or $t$ as an upper index. \\
Further, as all Christoffel symbols are independent of $v$ (as the metric is) we can then say that $\hat{R}\indices{^\mu_\nu_v_\sigma} = 0$. As such $\hat{R}\indices{^\mu_\nu_\rho_\sigma}=0$ if any of $\nu,\rho,\sigma=v$. We can also see that $\hat{R}\indices{^t_\nu_\rho_\sigma}=0$ by the formula. so we now need only determine

\begin{multicols}{2}
\begin{enumerate}
	\centering
	\item $\hat{R}\indices{^i_j_t_l}$
	\item $\hat{R}\indices{^i_t_k_l}$
	\item $\hat{R}\indices{^i_t_t_l}$
	\item[\vspace{\fill}]
	\item $\hat{R}\indices{^v_j_k_l}$
	\item $\hat{R}\indices{^v_j_t_l}$
	\item $\hat{R}\indices{^v_t_k_l}$
	\item $\hat{R}\indices{^v_t_t_l}$
\end{enumerate}
\end{multicols}
Making the observation 
\eq{
\hat{R}\indices{^v_\nu_\rho_\sigma} = -h^{ik}N_k R_{i\nu\rho\sigma} + R_{t\nu\rho\sigma} 
}
and seeing that 
\eq{
	\hat{R}_{i\nu\rho\sigma} &= \hat{g}_{i\mu}\hat{R}\indices{^\mu_\nu_\rho_\sigma} \\
	&= h_{ij}\hat{R}\indices{^j_\nu_\rho_\sigma}
}
we can simplify
\eq{
\hat{R}\indices{^v_j_\rho_\sigma} &= -N_i\hat{R}\indices{^i_j_\rho_\sigma}-h_{ji}\hat{R}\indices{^i_t_\rho_\sigma}
}
and
\eq{
\hat{R}\indices{^v_t_\rho_\sigma} = -N_i\hat{R}\indices{^i_t_\rho_\sigma}
}
This lets us get the seoncd column of terms immediately after we have the first. 
Let us begin the slog:
\eq{
\hat{R}\indices{^i_j_t_l} &=\del_t \hat{\Gamma}^i_{jl} - \del_l \hat{\Gamma}^i_{jt} + \hat{\Gamma}^i_{t\mu}\hat{\Gamma}^\mu_{jl} - \hat{\Gamma}^i_{l\mu}\hat{\Gamma}^\mu_{jt}\\
&= \Gamma^i_{jl,t} - \nabla_l^{(h)} \hat{\Gamma}^i_{jt} \\
&= \Gamma^i_{jl,t} +\frac{1}{2}\nabla_l^{(h)}\psquare{h^{ik}\pround{F_{kj}-h_{kj,t}}} \\
&= \Gamma^i_{jl,t} - \frac{1}{2}h^{ik}\nabla_l^{(h)}h_{kj,t} +\frac{1}{2}\nabla_l^{(h)}F\indices{^i_j} 
}
Now note 
\eq{
h^{ik}\nabla_l^{(h)} (\del_t h_{kj}) &= h^{ik} \psquare{h_{kj,tl} - \Gamma^m_{lk}h_{mk,t} - \Gamma^m_{lj} h_{km.t}} \\
&= h^{ik} \psquare{h_{kj,lt} - \del_t (h_{mj}\Gamma^m_{lk}) + h_{mj}\Gamma^m_{lk,t} - \del_t(h_{km}\Gamma^m_{lj}) + h_{mk}\Gamma^m_{lj,t}} \\
&= h^{ik} \psquare{\del_t(h_{kj,l}-[lk,j]_h - [lj,k]_h)+h_{mj}\Gamma^m_{lk,t}} + \Gamma^i_{lj,t}
}
Calculating
\eq{
h_{kj,l}-[lk,j]_h - [lj,k]_h = h_{kj,l}-\frac{1}{2}(h_{lj,k}+h_{kj,l}-h_{lk,j}) - \frac{1}{2}(h_{lk,j}+h_{jk,l}-h_{jl,k}) = 0
}
we have
\eq{
	\hat{R}\indices{^i_j_t_l} = \frac{1}{2}\psquare{\nabla_l^{(h)}F\indices{^i_j} + \Gamma^i_{jl,t} - h^{ik}h_{jm}\Gamma^m_{lk,t}}
}
Further
\eq{
\hat{R}\indices{^i_t_k_l} &= \del_k \hat{\Gamma}^i_{tl} - \del_l \hat{\Gamma}^i_{tk} + \hat{\Gamma}^i_{k\mu}\hat{\Gamma}^\mu_{tl} - \hat{\Gamma}^i_{l\mu}\hat{\Gamma}^\mu_{tk} \\
&= 2\psquare{\nabla_{[k}^{(h)}\hat{\Gamma}^i_{l]t}+\Gamma^m_{[kl]}\hat{\Gamma}^i_{tm}} \\
&= -\nabla_{[k}^{(h)}\pround{F\indices{^i_l_]} - h^{ij}h_{l]j,t}} \\
&= h^{ij}h_{m[l}\Gamma^m_{k]j,t}-\nabla_{[k}^{(h)} F\indices{^i_l_]}
}
That is 
\eq{
\hat{R}\indices{^i_t_k_l} = 2\hat{R}\indices{^i_[_k_|_t_|_l_]} \, .
}

\eq{
\hat{R}\indices{^i_t_t_l} &=  \del_t \hat{\Gamma}^i_{tl} - \del_l \hat{\Gamma}^i_{tt} + \hat{\Gamma}^i_{t\mu}\hat{\Gamma}^\mu_{tl} - \hat{\Gamma}^i_{l\mu}\hat{\Gamma}^\mu_{tt} \\
&= -\frac{1}{2} \del_t \pround{F\indices{^i_l}-h^{ij}h_{jl,t}} +\frac{1}{4}\pround{F\indices{^i_j}-h^{ik}h_{kj,t}}\pround{F\indices{^j_l}-h^{jm}h_{ml,t}}  +\nabla_l^{(h)} F\indices{^i_t} \\
&= -\frac{1}{2} \del_t \pround{F\indices{^i_l}-h^{ij}h_{jl,t}} +\frac{1}{4}\pround{F\indices{^i_j}+h\indices{^i^k_,_t}h_{kj}}\pround{F\indices{^j_l}-h^{jm}h_{ml,t}}  +\nabla_l^{(h)} F\indices{^i_t} \\
&= -\frac{1}{2}  \pround{F\indices{^i_l_,_t}-h^{ij}h_{jl,tt}} +\frac{1}{4}\pround{F\indices{^i_j}F\indices{^j_l}-F^{ij}h_{jl,t} + F_{jl}h\indices{^i^j_,_t}+h\indices{^i^j_,_t}h_{jl,t}}  +\nabla_l^{(h)} F\indices{^i_t} 
}
With these three we can read off
\eq{
\hat{R}\indices{^v_j_k_l} &= -N_i R\indices{^i_j_k_l} -h_{ji}\psquare{h^{ia}h_{m[l}\Gamma^m_{k]a,t}-\nabla_{[k}^{(h)} F\indices{^i_l_]}} \\
&= -N_iR\indices{^i_j_k_l}- h_{m[l}\Gamma^m_{k]j,t} + \nabla_{[k}^{(h)} F_{|j|l]} \\
&\highlight{\overset{?}{=} -\frac{1}{2}R\indices{^i_j_k_l}N_i +\frac{1}{2}\nabla_j^{(h)} F_{kl}  - \frac{1}{2} h_{m[l}\Gamma^m_{k]j,t}}
}

\eq{
\hat{R}\indices{^v_j_t_l} &= -\frac{1}{2}N_i\psquare{\nabla_l^{(h)}F\indices{^i_j} + \Gamma^i_{jl,t} - h^{ik}h_{jm}\Gamma^m_{lk,t}} - h_{ji}\left\lbrace-\frac{1}{2}  \pround{F\indices{^i_l_,_t}-h^{ik}h_{kl,tt}} \right. \\
&\phantom{=}\left. +\frac{1}{4}\pround{F\indices{^i_k}F\indices{^k_l}-F^{ik}h_{kl,t} + F_{kl}h\indices{^i^k_,_t}+h\indices{^i^k_,_t}h_{kl,t}}  +\nabla_l^{(h)} F\indices{^i_t} \right\rbrace \\
&= -\frac{1}{2}N_i\psquare{\nabla_l^{(h)}F\indices{^i_j} + \Gamma^i_{jl,t} - h^{ik}h_{jm}\Gamma^m_{lk,t}}+\frac{1}{2}\pround{h_{ji}F\indices{^i_l_,_t}+h_{jl,tt}} \\
&\phantom{=} + \frac{1}{4} \pround{F_{jk}F\indices{^k_l} - F\indices{_j^k}h_{kl,t} + F_{kl}h_{ji}h\indices{^i^k_,_t} + h_{ji}h\indices{^i^k_,_t}h_{kl,t}} + \nabla_l^{(h)}F_{jt} \\
&= -\frac{1}{2}N_i\psquare{\nabla_l^{(h)}F\indices{^i_j} + \Gamma^i_{jl,t} - h^{ik}h_{jm}\Gamma^m_{lk,t}}+\frac{1}{2}\pround{h_{ji}F\indices{^i_l_,_t}+h_{jl,tt}} \\
&\phantom{=} + \frac{1}{4} \pround{F_{jk}F\indices{^k_l} - F\indices{_j^k}h_{kl,t} - F\indices{^i_l}h_{ji,t} + h_{ji}h\indices{^i^k_,_t}h_{kl,t}} + \nabla_l^{(h)}F_{jt}
}

\begin{remark}
When I try to calculate $\hat{R}\indices{^v_j_k_l}$ directly, I get
\eq{
\hat{R}\indices{^v_j_k_l} &= \del_k \hat{\Gamma}^v_{jl} - \del_l \hat{\Gamma}^v_{jk} + \hat{\Gamma}^v_{k\mu}\hat{\Gamma}^\mu_{jl} - \hat{\Gamma}^v_{l\mu}\hat{\Gamma}^\mu_{jk} \\
&= 2\psquare{\nabla_{[k}^{(h)} \hat{\Gamma}^v_{l]j}+\Gamma^m_{[kl]}\hat{\Gamma}^v_{jm}} \\
&= \frac{1}{2} \nabla_{[k}^{(h)} \psquare{\nabla\sp{(h)}_{l]}N_{j}+\nabla\sp{(h)}_{|j|}N_{l]}-h_{l]j,t}} \\
&= -\frac{1}{4}R\indices{^i_j_k_l}N_i + \frac{1}{2}\psquare{-R\indices{^i_[_l_k_]_j} N_i + \nabla_j^{(h)}\nabla_{[k}^{(h)}N_{l]}} -\frac{1}{2}\psquare{h_{mj}\Gamma^m_{[kl],t}+h_{m[l}\Gamma^m_{k]j,t}}\\
&= -\frac{1}{4}N_i \psquare{R\indices{^i_j_k_l} + R\indices{^i_l_k_j} - R\indices{^i_k_l_j}} + \frac{1}{4}\nabla_j^{(h)} F_{kl}  - \frac{1}{2} h_{m[l}\Gamma^m_{k]j,t}\\
&= -\frac{1}{2}R\indices{^i_j_k_l}N_i +\frac{1}{2}\nabla_j^{(h)} F_{kl}  - \frac{1}{2} h_{m[l}\Gamma^m_{k]j,t}
}
This does not (seemingly) agree with the previous calculation. \hl{Where is the error?}
\end{remark}

\begin{comment}	
\eq{
\hat{R}\indices{^v_j_t_l} &= \del_t \hat{\Gamma}^v_{jl} - \del_l \hat{\Gamma}^v_{jt} + \hat{\Gamma}^v_{t\mu}\hat{\Gamma}^\mu_{jl} - \hat{\Gamma}^v_{l\mu}\hat{\Gamma}^\mu_{jt} \\
&= \frac{1}{4}\del_t\psquare{\nabla\sp{(h)}_{l}N_{j}+\nabla\sp{(h)}_{j}N_{l}-h_{lj,t}} + \nabla_l^{(h)} \psquare{\frac12 N\sp{k} (h_{kj,t} -F_{kj}) +\Phi_{,j}} \\
&\phantom{=} +\frac{1}{2}\psquare{\nabla\sp{(h)}_{l}N_{i}+\nabla\sp{(h)}_{i}N_{l}-h_{li,t}}\psquare{F\sp{i}_{\ j} - h\sp{ik} h_{kj,t}} \\
&= \text{\hl{improve this somehow}}
}


\eq{
\hat{R}\indices{^v_t_t_l} &= \del_t \hat{\Gamma}^v_{tl} - \del_l \hat{\Gamma}^v_{tt} + \hat{\Gamma}^v_{t\mu}\hat{\Gamma}^\mu_{tl} - \hat{\Gamma}^v_{l\mu}\hat{\Gamma}^\mu_{tt} \\
&= -\del_t \psquare{\frac12 N\sp{k} (h_{kl,t} -F_{kl}) +\Phi_{,l}} - \del_l\psquare{-\Phi_{,t}+N^kF_{kt}} +\frac{1}{2} \psquare{\frac12 N\sp{k} (h_{ki,t} -F_{ki}) +\Phi_{,i}}\psquare{F\sp{i}_{\ l}- h\sp{ik} h_{kl,t}} \\
&\phantom{=} +F\indices{^i_t} \psquare{\nabla\sp{(h)}_{l}N_{i}+\nabla\sp{(h)}_{i}N_{l}-h_{li,t}}  \\
&= \text{\hl{improve this somehow}}
}
\end{comment}
	
\subsection{The Frame}
Given the metric (\ref{eislift}) we define the frame $\{ {\hat e}\sp{A}\}$,
\begin{equation*}
	ds^2=\hat g_{\mu\nu}\,dx\sp{\mu}\,dx\sp{\nu}
	=h_{ij}\, dx\sp{i}\, dx\sp{j}+2dt \left( dv-\Phi dt+N_i dx\sp{i}\right)
	= {\hat \eta}_{AB}\, {\hat e}\sp{A} {\hat e}\sp{B}
	= \eta_{ab}\, e\sp{a} e\sp{b}+  {\hat e}\sp{+} {\hat e}\sp{-} +  {\hat e}\sp{-} {\hat e}\sp{+}.
\end{equation*}
Here $A\in \{+,-,a,b,\ldots\}$, $ {\hat \eta}_{+-}={\hat \eta}_{-+}=1$, and we take
\begin{align*}
	{\hat e}\sp{+}:= dt,\qquad
	{\hat e}\sp{-} :=dv-\Phi\,dt +N_i\,dx\sp{i},\qquad
	{\hat e}\sp{a} := {\hat e}\sp{a}_{\mu} dx\sp{\mu}= { e}\sp{a}_{i} dx\sp{i}= { e}\sp{a},
\end{align*}
and 
$$e\sp{a}_i \eta_{ab} e\sp{b}_j =h_{ij}.$$
The coframe  $\{ {\hat E}_{A}\}$ with ${\hat e}\sp{A}(  {\hat E}_{B})=\delta\sp{A}_B$ is given by
\begin{align*}
	{\hat E}_{+}:= \partial_t +\Phi\,\partial_v,\qquad
	{\hat E}_{-} :=\partial_v,\qquad
	{\hat E}_{a} := {E}_{a}- N_a\,\partial_v,
\end{align*}
where $N_a=N_i E_a\sp{i}$ and similarly
$$ { e}\sp{a}(  { E}_{b})=\delta\sp{a}_b,\qquad { E}_{b}={ E}_{b}\sp{i}\,\partial_i.
$$
We emphasise that $N$, $\phi$ and $e\sp{a}$ may depend on $x\sp{i}$ \emph{and} $t$.

Denoting the structure constants $[\hat E_B, \hat E_C]=c\sp{A}_{\ BC} \hat E_A$ we have from
\begin{align*}
	d\alpha (X,Y)&=X( \alpha(Y))-Y(\alpha(X))-\alpha([X,Y])
	\intertext{for a one-form $\alpha$, then for the torsion free connection}
	d \hat e\sp{A}&= -\hat \omega\sp{A}_{\ B}\wedge e\sp{B}= \hat \omega\sp{A}_{\ B C} e\sp{B}\wedge e\sp{C}
	\intertext{we have}
	d \hat e\sp{A}(\hat E_B,\hat E_C)&=\hat \omega\sp{A}_{\ B C}-\hat \omega\sp{A}_{\  C B}
	= -\hat  e\sp{A}([\hat E_B,\hat  E_C])= -c\sp{A}_{\ BC},
	\intertext{from which}
	\hat \omega\sp{A}_{\ B C}&=\frac12 \hat \eta\sp{AF}( c_{CFB}+c_{BFC}-c_{FBC}).
\end{align*}
The $v$-independence of the metric means that
$$[\hat E_-,   \hat E_B]=0,\qquad  c\sp{A}_{\ - B}=0$$
while
\begin{align*}
	[\hat E_+,   \hat E_a]&= \partial_t E_a-  (\partial_t N_a) \partial_v -(E_a\Phi)  \partial_v\\
	&=(\partial_t E_a\sp{j})[\partial_j -N_j \partial_v] - E_a\sp{j})[\partial_t N_j+\partial_j\Phi    ]\partial_v\\
	&= (\partial_t E_a\sp{j} \, e\sp{b}_j)\, \hat E_b +F_{at}\, \hat E_-\\
	%&= c\sp{b}_{\ +a}\, \hat E_b+ c\sp{-}_{\ +a}\, \hat E_- ,
	\intertext{and}
	[\hat E_a,   \hat E_b]&=[ E_a,   E_b] -(E_a N_b- E_b N_a)\,  \partial_v\\
	&=  c\sp{f}_{\ ab}\, \hat E_f - F_{ab}\,\hat E_-
\end{align*}
giving the (possibly) non-vanishing structure constants as
$$ c\sp{f}_{\ ab},\ \  c\sp{-}_{\ ab}=- F_{ab},\ \ c\sp{b}_{\ +a}= (\partial_t E_a\sp{j} \, e\sp{b}_j),\ \
c\sp{-}_{\ +a}=F_{at}.$$
Now
\begin{align*}
	d {\hat e}\sp{+}&= 0,\\
	d {\hat e}\sp{-} &=\frac12 F_{ab}\, e\sp{a}\wedge e\sp{b} +F_{it}\, dx\sp{i}\wedge dt
	=\frac12 F_{ab}\, {\hat e}\sp{a}\wedge {\hat e}\sp{b} +F_{at}\, {\hat e}\sp{a}\wedge {\hat e}\sp{+}, \\
	d {\hat e}\sp{a} &=d\left(  {\hat e}\sp{a}_{\mu} dx\sp{\mu}\right)= { e}\sp{a}_{i} dx\sp{i}= { e}\sp{a}
	=(\partial_j { e}\sp{a}_{i})\, dx\sp{j} \wedge dx\sp{i} +(\partial_t  { e}\sp{a}_{i})\, dt\wedge dx\sp{i}\\
	&=\omega\sp{a}_{\ bc}\, e\sp{b}\wedge e\sp{c}-(E\sp{i}_b\,\partial_t  { e}\sp{a}_{i} )\,  dt\wedge{ e}\sp{b}
	=\omega\sp{a}_{\ bc}\, e\sp{b}\wedge e\sp{c}+(\partial_t E\sp{i}_b\, { e}\sp{a}_{i} )\,  dt\wedge{ e}\sp{b},
\end{align*}
from which we see
\begin{align*}
	\hat \omega\sp{a}_{\ B C}\, {\hat e}\sp{B}\wedge {\hat e}\sp{C}&= \omega\sp{a}_{\ bc}\, e\sp{b}\wedge e\sp{c}+(\partial_t E\sp{i}_b\, { e}\sp{a}_{i} )\,  {\hat e}\sp{+}\wedge {\hat e}\sp{b},\\
	\hat \omega\sp{-}_{\ B C}\, {\hat e}\sp{B}\wedge {\hat e}\sp{C}&= \frac12 F_{bc}\, {\hat e}\sp{b}\wedge {\hat e}\sp{c} +F_{at}\, {\hat e}\sp{a}\wedge {\hat e}\sp{+}.
\end{align*}
Set
$$\alpha\sp{a}_{\ b}:= { e}\sp{a}_{i} \,\partial_t E\sp{i}_b=c\sp{a}_{\ +b},\ \
\alpha_{ab}=-\alpha_{ba}, $$
Using the antisymmetry of the connection then $0=\hat\omega_{++A}=\hat\omega\sp{-}_{\ +A}$
and so
$$\hat\omega\sp{-}_{\ a+}= F_{at},\ \  \hat\omega\sp{-}_{\ ab}= \frac12 F_{ab},\ \
\hat\omega\sp{a}_{\ bc}=\omega\sp{a}_{\ bc}, \ \
\hat\omega_{a b+}=-\frac12 F_{a b} -\frac12\left[ \partial_t E\sp{i}_a \,E_{ib}
-  \partial_t E\sp{i}_b \,E_{ia}  \right]=-\frac12 F_{a b}+\alpha_{ab}.
$$

\subsection{Bargman Structures}
A Bargmann structure $(B, \hat g, \xi)$ is a principal bundle $\pi:\, B\rightarrow M$, where $\dim B=\dim M+1$,
equipped with a Lorentzian metric $\hat g$ and nowhere vanishing null vector field $\xi$ such that with respect to the usual Levi-Civita connection $\hat \nabla \xi=0$. Then $M:=B/\mathbb{R}\xi$ is equipped with
a Newton-Cartan geometry $(M, K, \theta, \nabla)$ where
$$ K=\pi_\ast {\hat g}\sp{-1}, \qquad \hat g(\xi)=\pi\sp\ast\theta,$$
$K$ is degenerate and $\pi\sp\ast\theta$ generates $\ker K$.

In our setting we have a metric of Brinkmann form
$$\hat g= h + dt\otimes \omega+\omega\otimes dt,\ \
\omega= dv -\Phi(x,t)\, dt+ N_i(x,t)\, dx\sp{i}, \ \  h=h_{ij}(x,t) dx\sp{i}\otimes dx\sp{j}.
$$
Then $\xi =\partial_v$, $\theta=dt$.

\section{Introduction}

Let us start with a bit of back story, so we can develop and go further. This will be built off of \cite{Duval1985}.
%%%%%%%%%%%%%%%%%%%%%%%%%%%%%%%%%%%%%%%%%%%%%%%%%%%%%%%%
\subsection{Galilei and Newton Structures}
We start with some more classical work.

\begin{definition}[Galilei group]
The \bam{Galilei group} is the matrix group
\eq{
G = \pbrace{\begin{pmatrix}
		R & b & c \\ 0 & 1 & e \\ 0 & 0 & 1
\end{pmatrix} \, | \, R \in SO(d), \ , b,c \in \mbb{R}^n, \, e \in \mbb{R} } \leq GL_{d+2}(\mbb{R})
}	
\end{definition}

We think of $G$ as acting on $(\bm{x},t,1)$ s.t. 
\eq{
\begin{pmatrix}R & b & c \\ 0 & 1 & e \\ 0 & 0 & 1 \end{pmatrix} \begin{pmatrix}
	x \\ t \\ 1
\end{pmatrix} = \begin{pmatrix}
Rx + tb + c \\ t+e \\ 1
\end{pmatrix}
}
with this action we see:
\begin{enumerate}
	\item $R$ are rotations in space
	\item $b$ are boosts
	\item $c,e$ are translations in space and time respectively
\end{enumerate}

With this interpretation we have

\begin{definition}
	The \bam{Homogeneous Galilei group/Euclidean group} $H$ is the group of Galilean transformations that preserve the spatio-temporal origin $(\bm{0},0,1)$. 
\end{definition}

\begin{prop}
	$H$ consists of matrices of the form 
	\eq{
	\begin{pmatrix}R & b & 0 \\ 0 & 1 & 0 \\ 0 & 0 & 1 \end{pmatrix} \, .
}
Moreover $H \cong SO(d)\ltimes \mbb{R}^d$ as a Lie group (\hl{not a as a Lie transformation group} \cite{Kunzle1972} ) is faithfully represented by matrices of the form 
\eq{
\begin{pmatrix}R & b  \\ 0 & 1  \end{pmatrix} \in GL_{d+1} \, .
}
\end{prop}
\begin{proof}
	See my CQIS notes for a more built up discussion of this fact. 
\end{proof}

We now recall the following def:

\begin{definition}
	The \bam{frame bundle} of a $k$-dimensional smooth manifold $M$ is $GL(M)$, the $GL_k$-principal fibre bundle with fibres at $x \in M$ given by the space of ordered bases of $T_xM$. 
\end{definition}

\begin{definition}
	A \bam{proper Galilei structure} $H(M)$ is a reduction of structure group of the frame bundle of a $(d+1)$-dimensional $M$ via $H \hookrightarrow GL_{d+1} $.
\end{definition}



%%%%%%%%%%%%%%%%%%%%%%%%%%%%%%%%%%%%%%%%%%%%%%%%%%%%%%%%
%%%%%%%%%%%%%%%%%%%%%%%%%%%%%%%%%%%%%%%%%%%%%%%%%%%%%%%%
\bibliographystyle{../../bib/custom-bib-style}
\bibliography{../../bib/library,../../bib/manual}

\end{document}
