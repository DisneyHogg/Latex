\documentclass{article}

\usepackage{../../header}
%%%%%%%%%%%%%%%%%%%%%%%%%%%%%%%%%%%%%%%%%%%%%%%%%%%%%%%%
%Preamble

\title{GR + Effective Field Theory}
\author{Linden Disney-Hogg}
\date{}

%%%%%%%%%%%%%%%%%%%%%%%%%%%%%%%%%%%%%%%%%%%%%%%%%%%%%%%%
%%%%%%%%%%%%%%%%%%%%%%%%%%%%%%%%%%%%%%%%%%%%%%%%%%%%%%%%
\begin{document}

\maketitle
\tableofcontents

%%%%%%%%%%%%%%%%%%%%%%%%%%%%%%%%%%%%%%%%%%%%%%%%%%%%%%%%
%%%%%%%%%%%%%%%%%%%%%%%%%%%%%%%%%%%%%%%%%%%%%%%%%%%%%%%%
%%%%%%%%%%%%%%%%%%%%%%%%%%%%%%%%%%%%%%%%%%%%%%%%%%%%%%%%
%%%%%%%%%%%%%%%%%%%%%%%%%%%%%%%%%%%%%%%%%%%%%%%%%%%%%%%%
\section{Introduction}
Let's note with some good sources to start with 
\begin{itemize}
	\item Donaghue : 1702.00319, 9512024
	\item Manohar : 1804.0563, 9606222
\end{itemize}
In GR we will take signature $(-, +, +, +)$. 
%%%%%%%%%%%%%%%%%%%%%%%%%%%%%%%%%%%%%%%%%%%%%%%%%%%%%%%%
%%%%%%%%%%%%%%%%%%%%%%%%%%%%%%%%%%%%%%%%%%%%%%%%%%%%%%%%
%%%%%%%%%%%%%%%%%%%%%%%%%%%%%%%%%%%%%%%%%%%%%%%%%%%%%%%%
%%%%%%%%%%%%%%%%%%%%%%%%%%%%%%%%%%%%%%%%%%%%%%%%%%%%%%%%
\section{GR+EFTs - Lecture 1}
%%%%%%%%%%%%%%%%%%%%%%%%%%%%%%%%%%%%%%%%%%%%%%%%%%%%%%%%
%%%%%%%%%%%%%%%%%%%%%%%%%%%%%%%%%%%%%%%%%%%%%%%%%%%%%%%%
\subsection{Effective Field Theories}

\begin{definition}[EFT]
The steps to define an \bam{effective field theory (EFT)} are 
\begin{enumerate}
	\item Write down the most general Lagrangian $L$ compatible with the symmetry of your problem. (Note if you are doing this in GR this will include a graviton)
	\item Keep all the terms with a fixed number of derivatives. 
	\item Fix dimensions of coefficients with dimensional analysis
	\item Do QFT with this action (loops)
	\item Identify regime of validity. 
\end{enumerate}
\end{definition}
Recall that really there is a step 0 here (for example read Carroll). In GR we have a massless spin 2 field with dynamical degrees of freedom $g_{\alpha\beta}$ (a metric) with 10 degrees of freedom (using symmetry). We then lose 4 degrees of freedom due to gauge transformation, i.e. spacetime diffeomorphisms. We again lose 4 degrees of freedom which are ``non-dynamical". This gives us the 2 we would expect for a spin 2 field. These latter 4 turn out to come from 4 of Einstein's equations which are degree 1 in time, and are the equivalent of Gauss' law in electrodynamics. 

\begin{remark}
	The above was in $d=4$ dimensions. In general we would get 
	\[
	\frac{1}{2}d(d+1) - d - d = \frac{1}{2}d(d-3)
	\]
	degrees of freedom for this field. This would mean that in $d=3$, gravity is non-dynamical (correct ``depending on what aspects of the theory you are interested in"). GR is only invariant under local diffeomorphisms, not global (or large diffeos), in fact if we are in a spacetime with a boundary we can get boundary gravitons (gravitons whose wavefunction has support on the boundary).  
\end{remark}

Recall that gravitons in the quantum theory correspond to gravitational waves in the classical theory. 

\begin{example}
	1) Let's take actions 
	\[
	S = \frac{1}{16 \pi G_N} \int d^4x \, \sqrt{-g} \pround{-2\Lambda + R + c+1 R^2 + c_2 R_{\mu \nu} R^{\mu\nu} + c_3 R_{\mu\nu\rho\sigma}  R^{\mu\nu\rho\sigma} +\dots } \, .
	\]
	2) Recall $R_{\mu\nu\rho\sigma} \sim \del \del g$. $g$ is our metric, $\Lambda$ is our cosmological constant. $\Lambda$ is not fixed in the theory, and has dimensions, but one in practice uses the experimental value so we can make predictions and move on. We could have also included matter fields. \\
	3) We can calculate the dimension of these constants to be 
	\[
	c_i \sim M_S^{-3} \, , \quad \frac{1}{G_N} \sim M_P^2 \, ,
	\]
	where $M_P$ is our planck mass and $M_S$ is \hl{\dots}. If we took $c_2 = 0 = c_3$ and $c_1 \neq 0$, then we get field equations that are approximately 
	\begin{equation}\label{eq: theory 1}
	\Box h + \frac{1}{M_S^2} \Box \Box h  = 8\pi G_N T \, .
	\end{equation}
	4) We are taking the perturbative expansion $g_{\mu\nu} =  \eta_{\mu\nu} + \frac{1}{M_P} h_{\mu\nu}$, so our action looks like 
	\[
	S \sim \int \underbrace{(\del h)^2 + \frac{1}{M_P} h(\del h)^2}_{R} + \underbrace{\frac{1}{M_S^2} \psquare{(\del^2 h)^2 + \frac{1}{M_P} h (\del^2 h)^2}}_{h.o.t} + \dots 
	\]
	This means our Feynmann rules associate a $\sqrt{G_N}$ to each vertex, so the contribution from the 4pt function is 
	\[
	\pround{\frac{E^2}{M_P^2}}^{4+\text{loops}} \, .
	\]
	5) Ultimately this makes this theory non-renormalisable as the coupling constant is not dimensionless. However if $E \ll M_P$ we are ok to do QFT, so we can work at lower-than-plank-scale energies we can make predictions. 
\end{example}
The propagator for \ref{eq: theory 1} looks like 
\[
\frac{1}{q^2 + \frac{q^4}{M_S^2}} = \frac{1}{q^2} - \frac{1}{q^2 + M_S^2}
\]
and because of the sign of the latter term, we call this a ghost, as it corresponds to a massive particle of imaginary mass. This means that extra degrees of freedom are necessary. From a phase space perspective, this is because the initial value problem of gravity represented by \ref{eq: theory 1} needs more initial values because it is a 4th order equation (as opposed to classical gravity, which is second order). We could also think of this semi-classically, as imagining the higher order terms in the actions as modifying the standard Newtonian potential to be 
\[
V(r) = -\frac{Gm_1 m_2}{r} \psquare{1-e^{-rM_S}} \, ,
\]
and so for distances $r \gg M_S^{-1}$, these corrections are suppressed. Ultimately this is saying GR is valid in ``low" energies, which is good. \\
So far we have been focusing on the effective part of EFT. 

%%%%%%%%%%%%%%%%%%%%%%%%%%%%%%%%%%%%%%%%%%%%%%%%%%%%%%%%
%%%%%%%%%%%%%%%%%%%%%%%%%%%%%%%%%%%%%%%%%%%%%%%%%%%%%%%%
\subsection{Gauge Field Theories - Symmetries and Hamiltonians}
%%%%%%%%%%%%%%%%%%%%%%%%%%%%%%%%%%%%%%%%%%%%%%%%%%%%%%%%
\subsubsection{The Classical Version}
Recall classical mechanics, where we have 
\[
I = \int dt \, L
\]
and we compute a Hamiltonian via Legendre transform 
\[
H = \ev{p\dot{q} - L}{p = \frac{\del L}{\del \dot{q}}} 
\]
which generates $t$-translations. Here $t$ is an independent variable, and we can parameterise by it. GR is different, because $t$ is one of our dependent variables, and so to construct a Hamiltonian we need a clock. In classical mechanics $t$ is this physical clock, but if we want to write our theory in terms of some non-physical (arbitrary parameter) clock $\tau$, one thing we can do is the following: \\
We might expect introducing our new arbitrary parameter to correspond to introducing a gauge freedom. We can use gauge symmetry to ``eat" degrees of freedom, so to reformulate our theory we know we will want to increase our degrees of freedom. Hence lets add to $q(t), p(t)$ the fields $T(\tau)$ (clock) and $\pi(\tau)$ (conjugate momentum to $T$). We now write down a new action (which we will justify in the posterior)
\[
\tilde{I} = \int d\tau \, \psquare{pq^\prime + \pi T^\prime -N\pround{ \pi  + H}}
\]
where $N$ is a Lagrange multiplier field, and $\phantom{q}^\prime = \frac{d}{d\tau}$. Not my conjugate momenta are (for example) $p = \frac{\del L}{\del q^\prime}$. The variation wrt $N$ imposes 
\[
\pi = -H(q,p)
\]
Then going on-shell wrt the field $N$ gives 
\[
\ev{\tilde{I}}{N=0} = \int d\tau \, \frac{dT}{d\tau} \psquare{p \dot{q} - H(p,q)}
\]
where $\dot{\phantom{q}} = \frac{d}{dT}$. This action $\tilde{I}$ is invariant under reparameterisation of $\tau$. The new Hamiltonian is 
\[
\tilde{H} = N(\pi + H) \, .
\]
This vanishes on-shell, which is exactly what we want. Recall we said that $\tilde{H}$ generates $\tau$-translations, but our theory is invariant under such reparameterisations (in the Hilbert space of physical states $\ket{\psi}$). \\
The reason for seeing this is to assure us that, despite this being a toy model, when we go through the same process to make GR a Hamiltonian theory, we should be ok. 
%%%%%%%%%%%%%%%%%%%%%%%%%%%%%%%%%%%%%%%%%%%%%%%%%%%%%%%%
\subsubsection{ADM Hamilton}
The paper by Arnowitt, Deser, Misner (ADM) is \cite{Arnowitt2008}. The idea is to formulate GR as an initial value problem, which can be related to the identification of the degrees of freedom as we mentioned earlier. We will follow their paper. \\
Lets take manifold $(M, g_{ab})$, assumed to be globally hyperbolic, which means that we can foliate our manifold by Cauchy surfaces $\Sigma_t$, defining a time function $t$ telling you which surface a point is on. Further define a unit normal to $\Sigma_t$, $n^a$, defined globally s.t. 
\[
n^a n_a = g_{ab} n^a n^b = -1 \, .
\]
As such we can write the induced metric on $\Sigma_t$ using standard differential geometry as 
\[
h_{ab} = g_{ab} + n_a n_b \, . 
\]
We will define a vector associated to $t$, confusingly named $t^a$, s.t. 
\[
t^a \nabla_a t = 1 \, . 
\]
We think of this as giving us a flow of $t$. Given this $t^a$ we can project it into our normal and parallel directions at a given $t$ to define 
\begin{align*}
	-t^a n_a &= N , \quad \text{(normal component, lapse function)}\, ,  \\
	h_{ab} t^b &= N_a, \quad \text{(tangential component, shift vector)} \, . 
\end{align*}
We can calculate 
\begin{align*}
	h\indices{^a_b} &= g^{ac}h_{cb} = \delta\indices{^a_b} + n^a n_b \\ 
	\Rightarrow h\indices{^a_b}t^b &= t^a - Nn^a \\
	\Rightarrow n^a &= \frac{1}{N}(t^a - N^a) \, .
\end{align*}
This means that we can write our full metric as 
\[
g^{ab} = h^{ab} - \frac{1}{N^2}(t^a - N^a)(t^b - N^b) \, ,
\]
which suggest that the degrees of freedom that appear in our Hamiltonian formulation of GR correspond to the freedom in the choice of foliation, because our metric is written in terms of the surface metrics and our choice of $t$. \\
We will now want to pick a particular coordinate chart where $t^a = (\del_t)^a$. In this gauge the metric is 
\[
ds^2 = -N^2 dt^2 + h_{ij}(dx^i + N^i dt)(dx^j + N^j dt) \, . 
\]
\begin{remark}
	Recall we spoke about a connection to the non-dynamical parts of Einstein's equations. Taking the usual Einstein tensor $G_{ab}$ we project to calculate 
	\begin{align*}
		G_{ab} n^a n^b = h\indices{^b_d} G_{bc} n^c = 0 \, . 
	\end{align*}
These equations have no second order derivatives in time, and there are four of them, so they represent non-dynamical constraints in the initial value problem we mentioned earlier. If we were to quantise this theory we would want these operators to act trivially on physical states of the Hilbert space. We may hope to replicate this trick from our classical analogue. \\
The action we will write is 
\[
I = \frac{1}{16 \pi G_N} \int_{M} d^4x \, \sqrt{-g}R - \frac{1}{8 \pi G_N} \int_{\del M} d^3x \, \sqrt{-\gamma} (K-K_0) \, . 
\]
This latter term is the Gibbons-Hawking-York term, where $K$ is the extrinsic curvature, $\gamma$ a metric on the boundary, and $K_0$ correspond to the curvature of the boundary in the vacuum. It turns out that $\delta I=0$ gives me GR. We can compute the Hamiltonian, the details of which are in Wald's book. We first rewrite in ADM variables as 
\[
I = \int_M d^4x \, \psquare{\pi^{ij}h-{ij}+ N \mathcal{H} - N^i \mathcal{H}_i} + \text{boundary term} \, ,
\]
where the $\mathcal{H}, \mathcal{H}_i$ are just some functions of our fields, and the we can write the Hamiltonian (which is a functional of $t^a$) as 
\[
H[t^a] = \int_{\Sigma_t} d^3x \, \psquare{N \mathcal{H} + N^i \mathcal{H}_i} + \text{boundary term} \, .
\]
This integral contains only Lagrange multipliers, so we get on-shell constrains of $\mathcal{H}=0=\mathcal{H}_i$ (of which there are 4), and this means that if we didn't have any boundary terms then our on-shell Hamiltonian acts as 0 on our Hilbert space. However, generically our boundary term will not be zero, and so we get a non-zero energy value. The boundary term looks like 
\[
\int_{\del \Sigma_t} d^2x \, \sqrt{\sigma} n^{\mu} T_{\mu \nu} t^\nu
\]
so we get 
\[
\delta I_{\text{on shell}} = \frac{1}{2} \int_{\del M} d^3 x \, \sqrt{-\gamma} T^{ij} \delta g_{ij}  \, .
\]
This calculation is in Brown-York gr-qc 9209012. This boundary term reproduces Komar's integral
\end{remark}


%%%%%%%%%%%%%%%%%%%%%%%%%%%%%%%%%%%%%%%%%%%%%%%%%%%%%%%%
%%%%%%%%%%%%%%%%%%%%%%%%%%%%%%%%%%%%%%%%%%%%%%%%%%%%%%%%
%%%%%%%%%%%%%%%%%%%%%%%%%%%%%%%%%%%%%%%%%%%%%%%%%%%%%%%%
%%%%%%%%%%%%%%%%%%%%%%%%%%%%%%%%%%%%%%%%%%%%%%%%%%%%%%%%
\section{Asymptotic Symmetry Groups (ASGs) - Lecture 2}
Recall we saw that the ADM Hamiltonian had boundary terms that did not vanish on shell. Indeed if one evaluates the energy on the vector $t^\mu = (1,0,0,0) = (\del_t)^\mu$ and take that $N \to 1, \, N^i \to 0$ on $\partial \Sigma$, then we explicitly get 
\[
E = \int_{\del \Sigma} d^2x \, \sqrt{\sigma} n^\mu T_{\mu t} \, . 
\]
We will now aim to put this in quite a general framework
%%%%%%%%%%%%%%%%%%%%%%%%%%%%%%%%%%%%%%%%%%%%%%%%%%%%%%%%
%%%%%%%%%%%%%%%%%%%%%%%%%%%%%%%%%%%%%%%%%%%%%%%%%%%%%%%%
\subsection{Local vs Large Diffeos}
If you compute 
\[
\delta_\xi \pround{\sqrt{-g} L } = \mathcal{L}_\xi \pround{\sqrt{-g} L} = \nabla_\mu (L \xi^\mu)
\] 
for some Lagrangian density $L$, we note that if $\xi$ acts non-trivially at the boundary the by Stokes' theorem we may have a boundary contribution to our action. Hence the assertion of gauge-invariance in a boundary-dependent statement. This gives us reason to introduce the following definition, 
\begin{definition}
	The ASG of a theory is 
	\[
	\text{ASG} = \frac{\text{symmetries}}{\text{trivial symmetries}} \, , 
	\]
where these are symmetries at the boundary, so the trivial symmetries are equivalently the local symmetries. 
\end{definition}

\begin{example}
	At null infinity in $d=4$ GR, we have the BMS group. 
\end{example}

\begin{example}
	In $AdS_3$ we have $Vir \oplus Vir$. 
\end{example}

%%%%%%%%%%%%%%%%%%%%%%%%%%%%%%%%%%%%%%%%%%%%%%%%%%%%%%%%
%%%%%%%%%%%%%%%%%%%%%%%%%%%%%%%%%%%%%%%%%%%%%%%%%%%%%%%%
\subsection{Quantum Gravity (QG)}
Treating now GR as an EFT for QG, we should have some UV completion, and this is the open problem. There are a few things we think we know.
\begin{itemize}
	\item QG has no local observables (gauge invariant quantities). If we were in a gauge QFT in flat space, then the local observables are 
	\begin{itemize}
		\item correlators $\pangle{O(x_1) \dots O(x_n)}$
		\item $S$-matrix elements
	\end{itemize}
   These observables however depend on spacetime points, so if we have a quantum theory of gravity where points fluctuate, these objects are no longer gauge invariant. If one were to do a semi-classical approximation and expand around a curved background, these may appear gauge-invariant, but in the full scale these arguments break down. One can overcome this fact by using what are called ``dressed operators", where we connect the operators to the boundary, thus fixing gauge invariance, but now this operator is non-local. 
   \item Gravitons are not a composite object. Note this is a valid question to ask, as we have particle theories where particles like mesons are the fundamental objects at low energies, but when we go to the UV completion we see that these are composite. This result is rigorous enough in the sense that we have the following theorem:
   \begin{theorem}
   	A theory containing a Poincar\'e covariant conserved stress tensor $T^{\mu\nu}$ forbids massless particles of spin $j>1$. 
   \end{theorem}
   \begin{proof}
   We sketch the idea. Labelling massless states with 4-momenta $p$ and spin $j_3$, consider the terms 
   \[
   \pangle{p^\prime, \pm j  | T^{\mu\nu} | p, \pm j}
   \]
   in the limit $p^\prime \to p$. Up to normalisations we can write 
   \[
   p^\mu = \int d^3x \, T^{0\mu}
   \]
   and 
   \[
   \pangle{p^\prime, \pm j  | p^\mu | p, \pm j} = p^\mu \delta(p-p^\prime)
   \]
   
but rewriting the LHS of the above using the integral gives 
\[
(2\pi)^3 \delta(p-p^\prime)\lim_{p^\prime \to p} \pangle{p^\prime, \pm j  | T^{\mu\nu} | p, \pm j} =  p^\mu \delta(p-p^\prime)
\]
and hence
\[
\lim_{p^\prime \to p} \pangle{p^\prime, \pm j  | T^{\mu\nu} | p, \pm j} = \frac{p^\mu p^\nu}{E (2\pi)^3} \neq 0 \, .
\]
Given 2 massless particles we have 
\[
(p+p^\prime)^2 = 2p\cdots p^\prime = 2 \abs{\bm{p}}\abs{\bm{p}^\prime} (\cos\phi-1) \leq 0
\]
and we will assume $aphi \neq 0$. Hence $\exists$ a frame s.t. $p = (\abs{\bm{p}}, \bm{p}), \, p^\prime = (\abs{\bm{p}}, -\bm{p})$. Consider now a rotation by angle $\theta$ about $\bm{p}$ and take $\theta^\prime = -\theta$. The bra and ket transform as 
\begin{align*}
	\ket{p, \pm j} &\mapsto e^{\pm i \theta j} \ket{p, \pm j} \\
	\ket{p^\prime, \pm j} &\mapsto e^{\mp i \theta j} \ket{p^\prime, \pm j} \, ,
\end{align*}
so we have 
\[
e^{\pm 2i\theta j} \lim_{p^\prime \to p} \pangle{p^\prime, \pm j  | T^{\mu\nu} | p, \pm j} = \lim_{p^\prime \to p} \Lambda(\theta)\indices{^\mu_\rho} \Lambda(\theta)\indices{^\rho_\sigma} \pangle{p^\prime, \pm j  | T^{\rho\sigma} | p, \pm j} \, .
\]
The $\Lambda$ are rotation matrices, so their eigenvalues can only be $\sim e^{\pm i \theta}$. This transform on the RHS does not depend on the quantum number, so we get a bound on $j$ because either matrix elements vanish (not possible) or $2j \in \pbrace{0, 1, 2}$. 
\end{proof}
It is important to note about this theory:
\begin{itemize}
	\item it does not remove gravitons in the UV, because GR does not have a conserved stress tensor that includes gravitons, we only have $\nabla_\mu T^{\mu\nu}_{\text{matter}}=0$, but what is important is that there in no local gauge-invariant energy momentum tensor in GR, which is connected to the fact that you shouldn't be able to measure energy locally (think about choosing a coordinate frame so locally Schwarzschild looks like Minkowski, but we know Schwarzschild should have a mass)
	\item if you thought that the graviton was composite as $\text{graviton} \sim 2\text{gluon}$, then because we have a conserved stress tensor for the gluons we could construct a stress tensor for the graviton, and this would be obstructed via the theorem. One might wonder how this intersects with the work of the double-copy community.
\end{itemize}
\end{itemize}

%%%%%%%%%%%%%%%%%%%%%%%%%%%%%%%%%%%%%%%%%%%%%%%%%%%%%%%%
\subsubsection{Summary}
The sum of the above is that 
\begin{itemize}
	\item gravitons appear in the UV together with some other particles
	\item it is possible that the graviton is an emergent degree of freedom when the UV theory is not an ordinary 4d QFT. Here emergent is in the sense that we have a QFT of sort on another manifold with associated tensor $\hat{T}_{\alpha \beta}$ from which the graviton $g_{\alpha \beta}$ arises on our original manifold under some correspondence. 
\end{itemize}

%%%%%%%%%%%%%%%%%%%%%%%%%%%%%%%%%%%%%%%%%%%%%%%%%%%%%%%%
%%%%%%%%%%%%%%%%%%%%%%%%%%%%%%%%%%%%%%%%%%%%%%%%%%%%%%%%
\subsection{BH Thermodynamics}
Let us list some main features of black holes.
\begin{itemize}
	\item $\exists$ an event horizon ($r=r_+$ when spherically symmetric) (this is a global requirement, looking at the penrose diagram we need to have global knowledge of spacetime to draw this),
	\item picking coordinates (e.g. $(t,r)$ for Schwarzschild), they flip at the horizon, so the interior geometry changes with ``real time which is $r$", 
	\item $\exists$ gravitational redshift, $\delta \tau  = \sqrt{-g_{tt}} \delta t$, $E_r = \frac{E_t}{\sqrt{-g_{tt}}}$. 
	\item $\exists$ BH uniqueness theorems
	\item we can phrase BHs locally in terms of trapped surfaces
	\item In Hawking 1971, it was proven that the area of the event hoizon $A$ always grows (i.e. $\delta A \geq 0$) under any classical process if 
	\begin{enumerate}
		\item the null energy condition holds (i.e. $T_{\mu\nu}k^\mu k^\nu \geq 0$ for any null $k^\mu$),
		\item cosmic censorship holds.
	\end{enumerate}
\end{itemize}
Let us take a warm-up example. 
\begin{example}
	Let's write 4d RN metric, Einstein coupled to Maxwell, 
	\begin{align*}
	 ds^2 &= -f(r)dt^2 + \frac{dr^2}{f(r)} + r^2 d\Omega_2^2 \\
	 A &= -\frac{Q}{r}dt 
	\end{align*}
where $f(r) = 1-\frac{2M}{r}+\frac{Q^2}{r^2}$, $M$ the BH mass and $Q$ the BH electric charge, $A$ a Maxwell field (not to be confused with area). Note this is a vacuum solution so $T^{\mu\nu}_{\text{matter}}=0$.
\begin{itemize}
	\item We can find $r_{\pm} = M \pm \sqrt{M^2-Q^2}$ s.t $f(r) \frac{(r-r_-)(r-r_+)}{r^2}$, and then the event horizon corresponds to $r=r_+$. 
	\item we can compute $A(r_+) = 4\pi r_+^2$. If we consider two solutions $(M, Q)$ and $(M+\delta M, Q+\delta Q)$ then 
	\[
	\delta A = 8 \pi r_+ \delta r_+ = 8 \pi r_+ \psquare{\pround{1+\frac{M}{\sqrt{M^2-Q^2}}} \delta M - \frac{Q}{\sqrt{M^2-Q^2}}\delta Q} \, .
	\]
	Using the result that $r_+^2 f^\prime (r_+) = r_+ - r_-$ one rewrites 
	\[
	\frac{f^\prime(r_+)}{16 \pi} \delta A = \delta M - \frac{Q}{r_+} \delta Q \, .
	\]
\end{itemize} 
Comparing this last equation with the 1st law of thermodynamics $TdS = dM -\Phi dQ$ ($\Phi$ a gauge potential, $dM$ a matter content) and the second $\delta S \geq 0$ we would be inclined to write $S = \eta A$ and then 
\begin{align*}
	\Phi &= -\ev{A_t}{r=r_+} \quad (\text{Maxwell}) \, , \\ 
	T_{BH} &= \frac{f^\prime(r_+)}{16 \pi \eta} \, .
\end{align*}
Hence the ``temperature" of the BH is constant over the entire horizon, which looks like the 0th law of thermodynamics. 
\end{example}
The rest of the analogy is completed in 
\begin{itemize}
	\item Israel, 1986, (3rd law)
	\item Bardeen, Carter, Hawking, 1973, (law of classical BH vs thermodynamics)
\end{itemize}
What remains is to calculate the appropriate $\eta$. Hawking did it first with a very complicated calculation, so next time we will see a simpler trick. 
%%%%%%%%%%%%%%%%%%%%%%%%%%%%%%%%%%%%%%%%%%%%%%%%%%%%%%%%
%%%%%%%%%%%%%%%%%%%%%%%%%%%%%%%%%%%%%%%%%%%%%%%%%%%%%%%%
%%%%%%%%%%%%%%%%%%%%%%%%%%%%%%%%%%%%%%%%%%%%%%%%%%%%%%%%
%%%%%%%%%%%%%%%%%%%%%%%%%%%%%%%%%%%%%%%%%%%%%%%%%%%%%%%%
\section{Trick to Compute \secmath{T_{BH}} - Lecture 3}

\begin{remark}
	Note that in QFT at vanishing temperature, we use $\mathbb{R}^{1,3}$, but at finite temperature ($\beta$ is inverse temperature) use $S_\beta^1 \times \mathbb{R}^3$. We use this to motivate going to Euclidean BHs to do this calculation. 
\end{remark}

Now take $t=i\tau$ to get line element 
\[
ds_E^2 = f(r) dt^2 + \frac{dr^2}{f(r)} + r^2 d\Omega _2^2 \, . 
\]
We wonder what happens at $r_+$, as $f(r_+)=0$, but note now this is not a horizon because we have changed signature. \\
Let $r-r_+ = \rho^2, \, \rho^2 \ll r_+$. Then as $\rho \to 0$ our line element has the limimt 
\[
ds_E^2 \to \frac{\beta}{4 \pi \eta} \pround{\rho^2 d\bar{\tau}^2 + d \rho^2} + r_+^2 d\Omega_2^2 
\]
where $\bar{\tau} = \pround{8 \pi \eta T_{BH}} \tau$ and $\beta = T_{BH}^{-1}$. This first term gives a smooth metric ``at the horizon" iff $\bar{\tau}$ is a polar coordinate, so has period $2\pi$ (otherwise we could have a conical singularity). Then 
\[
\bar{\tau} \sim \bar{\tau}+2 \pi \Rightarrow \tau \sim \tau + \frac{\beta}{4 \eta} \, .
\] 
If we take $r \to \infty$ in this limit metric, our asymptotic manifold is $S_\beta^1 \times \mathbb{R}^3$. Hence here we would want $\tau \sim \tau +\beta$, and this forces $\eta = \frac{1}{4}$. 
\begin{remark}
	One might wonder why we expect this transformed BH to not have a singularity, and this is valid (think about what gravitational instantons we get in Euclidean metric). We will talk about this more later.  
\end{remark}

%%%%%%%%%%%%%%%%%%%%%%%%%%%%%%%%%%%%%%%%%%%%%%%%%%%%%%%%
%%%%%%%%%%%%%%%%%%%%%%%%%%%%%%%%%%%%%%%%%%%%%%%%%%%%%%%%
\subsection{Covariant Phase Space}
We will now want to re-derive the thermodynamics of the black hole in covariant GR. As good references we have 
\begin{itemize}
	\item Iyer, Wald, 9403028
	\item Wald, Zoupas, 991095
	\item Wald, 9307038
\end{itemize}

\begin{example}
	Starting with $L = \mathcal{L}(q, \dot{q}) dt$ and the Euler-Lagrange eqns give 
	\[
	\delta_1  L = \pround{\frac{\partial \mathcal{L}}{\del q^a} - \frac{d}{dt} \frac{del \mathcal{L}}{\del \dot{q}^a}} dt \delta_1 q^a + d\Theta
	\]
	with $\Theta = \frac{\del \mathcal{L}}{\del \dot{q}^a} \delta_1 q^a = p_a \delta_1 q^a$. This gives a symplectic form 
	\[
	\Omega = \delta_2 \Theta(\delta_1 q^a , q^a, \dot{q}^a) - \delta_1 \Theta(\delta_2 q^a, q^a, \dot{q}^a) = \delta_2 p_a \delta_1 q^a - \delta_1 p_a \delta_2 q^a
	\]
	We can define a Noether current associated with a vector field $\xi$ 
	\[
	J_\xi = \Theta(q^a, \dot{q}^a ,\mc{L}_\xi q^a) - \underbrace{i_\xi L}_{\xi \cdot L} \, .
	\]
	For example if $\xi = \del_t$, then $\mc{L}_\xi q^a = \dot{q}^a$ and $i_\xi L = \mathcal{L}$ so 
	\[
	J_\xi = p_a \dot{q}^a - \mathcal{L} = H 
	\]
\end{example}
Think of this example as giving a blueprint for how to find the Noether current for a generic Lagrangian, as we will next attempt to do in GR. \\
Define:
\begin{itemize}
	\item our manifold $M$, globally hyperbolic, 
	\item $\phi = \phi(g_{ab}, \psi, \dots)$ our fields, $\psi$ matter fields,  
	\item $\mathcal{F}$ the space of kinematically allowed configurations, with requirements such as smoothness and asymptotic boundary conditions. 
	\item A bulk Lagrangian 
	\[
	L_{\text{bulk}}(g_{ab}, R_{abcd}, \nabla_a R_{abcd}, \dots, \psi, \nabla_a \psi, \dots)
	\]
	where we can can have arbitrary, but finite, number of derivatives in our terms. 
\end{itemize}
Our variation is defined as 
\[
\delta L_{\text{bulk}} = E[\phi] \delta \phi + d \Theta \, .
\]
\begin{remark}
	Note our asymptotic boudnary conditions will not affect our EL equations, but will affect $\Theta$. Hence there are potential ambiguities, discussed further in (Harlow, 1906.08616) (1912.06025). 
\end{remark}
In pure GR, we will take 
\[
(L_{\text{bulk}})_{abcd} = \frac{1}{16 \pi G_N} \epsilon_{abcd} R  \Rightarrow \Theta_{abc} = \frac{1}{16 \pi G_N} \epsilon_{abcd} g^{de} g^{fh} \pround{\nabla_f \delta g_{eh} -\nabla_e \delta g_{fh} } \, .
\]
Following our algorithm our symplectic current is 
\[
\omega(\phi ,\delta_1 \phi, \delta_2 \phi) = \delta_2 \Theta (\delta_1 \phi, \phi, \dots) - \delta_1 \Theta(\delta_2 \phi, \phi, \dots) \, ,
\]
(an $(n-1)$-form in general). The symplectic form is 
\[
\Omega_\Sigma = \int_\Sigma \omega (\phi, \delta_1 \phi, \delta_2 \phi) \, ,
\] 
depending on some hypersurface $\Sigma$. We now want to calculate the Noether current associated to a vector field $\xi$, with action on $\phi \in \mathcal{F}$ given by $\mathcal{L}_\xi \mathcal{F}$. We will define a tangent vector in $\mc{F}$ if it maps $\mc{F} \to \mc{F}$, thus taking $\mc{L}_\xi \phi = \delta_\xi \phi$. Let us also define $\bar{\mc{F}}$ to be the submanifold of on-shell configurations, and a projection onto physical phase space $\mc{F} \to \Gamma$ (with associated $\bar{\mc{F}} \to \bar{\Gamma}$).

\begin{definition}
	Suppose we have $\mc{F}$, $\xi$, and $\Sigma$ defining $\int_\Sigma \omega(\phi, \delta \phi , \delta_\xi \phi)$ that converges $\forall \phi \in \bar{\mc{F}}, \, \forall \delta \phi$ tangent to $\mc{F}$. Then a function $H_\xi : \mc{F} \to \mathbb{R}$ is called the \bam{Hamiltonian conjugate to $\xi$ on $\Sigma$} if 
	\[
	\forall \phi \in \bar{\mc{F}}, \; \forall \delta \phi \text{ tangent to }\mc{F}, \; \delta H_ \xi = \Omega_\Sigma(\phi, \delta \phi, \delta_\xi \phi) = \int_\Sigma \omega(\phi, \delta \phi, \delta_\xi \phi) \, . 
	\]
\end{definition}
If such a $H_\xi$ exists, it's value gives a conserved charge associated with transform by $\xi$ on $\Sigma$. \\
We now want to wonder when $H_\xi$ exists. Let us write our Noether current, an $(n-1)$-form $J_\xi = \Theta(\phi, \delta_\xi \phi) - i_\xi L_{\text{bulk}}$. Then 
\begin{align*}
	dJ_\xi &= d\Theta - d(i_\xi L_{\text{bulk}}) \\
	&= \psquare{\delta_\xi L_{\text{bulk}} - E[\phi] \delta \phi} - \psquare{(\mathcal{L}_\xi - i_\xi d) L_{\text{bulk}}} \\
	&= - E[\phi] \delta \phi \, , 
\end{align*}
where we knew $dL=0$ as $L$ is a top form. This vanishes on-shell, so we can write 
\[
J = dQ + \xi^a c_a
\]
where $c_a$ vanishes on-shell, and $Q$ is an $(n-2)$-form called the \bam{Noether charge}. In pure GR we can calculate 
\[
J_{abc} = \frac{1}{8 \pi G_N} \epsilon_{dabc} \nabla_e \nabla^{[e} \xi^{d]}
\]
and so 
\[
Q_{ab} = -\frac{1}{16 \pi G_N} \epsilon_{abcd} \nabla^c \xi^d \, . 
\]
Returning to generality, we can write the variation of our current as 
\begin{align*}
	\delta J_\xi &= \omega(\phi, \delta \phi, \mc{L}_\xi \phi) + di_\xi \Theta - i_\xi \pround{E[\phi]\delta \phi} \\
	\Rightarrow \delta J &= d(\delta Q) + \xi^a \delta c_a \quad (\text{as $\xi$ fixed}) \\
	\Rightarrow \omega(\phi, \delta \phi, \mc{L}_\xi \phi) &= \xi^a \delta x_a + d(\delta Q) - di_\xi \Theta \\
	\Rightarrow \delta H_\xi &= \int_\Sigma \xi^a \delta c_a + \int_{\del \Sigma} \psquare{\delta Q - i_\xi \Theta} \\
	\Rightarrow \ev{\delta H_\xi}{\text{on-shell}} &= \int_{\del \Sigma} \psquare{\delta Q - i_\xi \Theta} \, .
\end{align*}
Wald and Zoupas derive an integrability condition for the existence of $H_\xi$ that 
\begin{align*}
	0 = (\delta_1 \delta_2 - \delta_2 \delta_1) H_\xi =_{\text{on-shell}} - \int_{\del \Sigma } i_\xi \omega(\phi, \delta_1 \phi, \delta_2 \phi)
\end{align*}
It is worth remarking that the above was derived under very general conditions. 

%%%%%%%%%%%%%%%%%%%%%%%%%%%%%%%%%%%%%%%%%%%%%%%%%%%%%%%%
\subsubsection{Connection to ADM}
Recall on shell we have 
\[
\delta H_\xi = \delta \int_{\del \Sigma} Q - \int_{\del \Sigma} i_\xi \Theta
\]
The latter term is the tricky on for the existence of $H_\xi$. Pick $\del \Sigma$ to be at spacelike infinity, and assume $\exists B$ satisfying 
\[
\delta \int_{\del \Sigma} i_\xi B = \int_{\del \Sigma} i_\xi \Theta
\]
This would give us a solution given as 
\[
H_\xi = \int_{\del \Sigma} (Q-i_\xi B) \, .
\]
This is a boundary quantity, depending on $\xi$, and in pure GR we can compute this $B$. If we took $\xi = \del_t$ or $\xi=\del_\varphi$, we would reproduce ADM with $H=$ energy or angular momentum. \\
Let us now go to a BH situation. Here $\del \Sigma$ has two contributions, at infinity and at the horizon. Let us take the vector 
\[
\xi^a = t^a + \Omega_i \varphi_i^a \, ,
\]
a quantity vanishing at the horizon, meaning that the difficult term $\int_{\del \Sigma_{\text{Hor}}} i_\xi \Theta = 0$. Further, because $\omega(\phi, \delta \phi, \mc{L}_\xi \phi)=0$ because $\xi$ is Killing and so $\mc{L}_\xi \phi$, $\delta H_\xi =\int_\Sigma \omega = 0$. This combines to mean 
\[
\delta \int_{\del \Sigma_{\text{Hor}}} Q = \delta M - \Omega_i \delta J_i \, ,
\]
where $M, J_i$ are the ADM charges of energy and angular momentum. Assuming that the Lagrangian in the bulk is pure GR, we can use $Q_{ab} = -\frac{1}{16 \pi G_N} \epsilon_{abcd} \nabla^c \xi^d$. We will also want that 
\begin{itemize}
	\item the volume form on $\Sigma_{\text{Hor}}$ is 
	\[
	\epsilon_{ab} = \epsilon_{abcd} N^c \xi^d \, , 
	\]
	where $N$ is our future directed null normal to $\Sigma_{\text{Hor}}$ s.t $N^c \xi_c = -1$. 
	\item We contract $\epsilon^{ab}$ with the integrand for $Q$ to get 
	\[
	\epsilon^{ab} \epsilon_{abcd} \nabla^c \xi^d = 4 N_c \xi_d \nabla^d \xi^c \, .  
	\]
	\item As $\xi$ is Killing, $\ev{\xi^a \xi_a}{\Sigma_{\text{Hor}}} = 0 \Rightarrow \nabla_a \xi^b \xi_b = -2 \kappa \xi^a$ on the horizon (for some $\kappa$).Using that $\xi$ is Killing gives $\nabla_{(a}\xi_{b)}=0$ and hence 
	\[
	\xi^b \nabla_a \xi_b = -\xi^b \nabla_b \i_a = -\kappa \xi_a 
	\] 
	This is called the \bam{surface gravity}. 
\end{itemize}
Hence 
\[
\epsilon^{ab} \epsilon_{abcd} \nabla^c \xi^d = -4 \kappa 
\]
By covariance, we must have 
\[
\epsilon^{ab} \epsilon_{abcd} \nabla^c \xi^d = \gamma \epsilon_{ab} \epsilon^{ab} \, ,
\]
which gives in total that 
\[
\delta \int_{\Sigma_{\text{Hor}}} Q = \delta \pround{\int_{\Sigma_{\text{Hor}}}\frac{\kappa}{8 \pi }\epsilon_{ab}} = \delta M - \Omega_i \delta J_i
\]
\begin{remark}
	we may a few remarks. 
	\begin{itemize}
		\item The Euclidean $T_{BH} = \frac{\kappa}{2\pi}$ for RN. 
		\item One can show the 0th law of BH mechanics: 
		\[
		\xi_{[d]}\nabla_{c]}\kappa = -\xi_{[d]}R_{c]}^f \xi_f
		\]
		using $\nabla_a \nabla_b \xi_c = -R\indices{^d_b_c_a} \xi_d$. Now for RN we can write 
		\[
		\delta\int_{\Sigma_{\text{Hor}}} \frac{\kappa}{8 \pi} \epsilon^{ab} = \frac{\kappa}{2\pi}\delta \pround{\frac{A}{4}} \, , 
		\]
		and if the dominant energy condition holds, we can write $\xi_{[d]} \nabla_{c]} \kappa = 0$ this is true more generally, so 
		\[
		\delta M - \Omega_i \delta J_i = T_{E, RN} \frac{\delta A}{4} \, . 
		\]
		\item The sum effect of this is to show that entropy is a Noether charge, and is related to the surface gravity, a quantity interesting in differential geometry in its own right.
	\end{itemize}
\end{remark}

%%%%%%%%%%%%%%%%%%%%%%%%%%%%%%%%%%%%%%%%%%%%%%%%%%%%%%%%
%%%%%%%%%%%%%%%%%%%%%%%%%%%%%%%%%%%%%%%%%%%%%%%%%%%%%%%%
%%%%%%%%%%%%%%%%%%%%%%%%%%%%%%%%%%%%%%%%%%%%%%%%%%%%%%%%
%%%%%%%%%%%%%%%%%%%%%%%%%%%%%%%%%%%%%%%%%%%%%%%%%%%%%%%%
\section{Thermal States - Lecture 4}
Let's start with the question ``what is a thermal state?" Let's take two perspectives
\begin{itemize}
	\item From statistical mechanics, we have density $\rho_\beta = \frac{e^{-\beta H}}{Z(\beta)}$, $\beta=\frac{1}{T}, \, \pangle{X}_\beta = \operatorname{Tr}(\rho_\beta X)$. 
	\item Open subsystems, thermal states appear as a reduced density matrix of a closed Hilbert space. i.e. we have a Hilbert space $H = H_1 \otimes I_2 + I_1 \otimes H_2$, with state 
	\[
	\ket{\psi}_{TFD} = \frac{1}{\sqrt{Z}} \sum_i e^{-\beta E_i / 2 } \ket{i}_1 \otimes \ket{i}_2 \, , 
	\]
	and then we calculate 
	\[
	\rho = \ket{\psi}\bra{\psi} \Rightarrow \rho_\beta := \rho_1 = \operatorname{Tr}_{H_2} \rho \, .
	\]
	This means 
	\[
	\pangle{\psi \, | \, X_1 \otimes I_2 \, | \, \psi} = \operatorname{Tr}_{H_1}(\rho_1 X_1) \, . 
	\]
	Note `TFD' stands for thermal field double. 
\end{itemize}

\begin{remark}
	Consider 2 harmonic oscillators with Hilbert spaces $\mathcal{H}_1, \, \mathcal{H}_2$, corresponding creation an annihilation operators $a_i^\dagger, a_i$, with ground states $\ket{0}_i$ s.t. 
	\[
	a_i \ket{0}_i = 0 \, .
	\] 
	We thus have a natural vacuum state of $\mathcal{H} = \mathcal{H}_1 \otimes \mathcal{H}_2$ given by $\ket{0} = \ket{0}_1 \otimes \ket{0}_2$. The analogue of the TFD state here is 
	\[
	\ket{\psi} \propto \exp\psquare{e^{-w \beta/2} a^\dagger_1 a^\dagger_2} \ket{0} = \sum_n e^{-n\beta w/2 }\ket{nw}_1 \otimes \ket{nw}_2 \, .
	\]
	We observe that the operators 
	\begin{align*}
		b_1 &= \cosh(\theta) a_1 - \sinh(\theta) a_2^\dagger \, , \\
		b_2 &= \cosh(\theta) a_2 - \sinh(\theta) a_1^\dagger \, ,
	\end{align*}
where $\cosh(\theta) = \frac{1}{\sqrt{1-e^{-\beta w}}}$. This is a unitary transform (an example of a bogoluibov transform), and we have that 
\[
b_1 \ket{\psi} = 0 = b_2 \ket{\psi} \, . 
\]
Hence some unitarily related observes will see $\ket{\psi}$ as a vaccuum, where their annihilation and creation operators are $b_i^\dagger, b_i$. Later we will show this to be related to the fact that, implicitly we have picked a time to do quantum mechanics here, and by changing time we will change frequencies, hence energies. The Hamiltonian corresponding to this new set of energies will be the boosted Hamiltonian. This will somehow apply to the thermodynamics of BHs later, where we expect the time to change. 
\end{remark}

\begin{remark}
	Consider a Hilbert space $\mc{H} = \mc{H}_A \otimes \mc{H}_B$ with entangled state $\ket{\psi} \neq \ket{\chi}_A \otimes \ket{\tilde{\chi}}_B$. For example we could take the qubit Bell state 
\[
\ket{\psi} = \frac{1}{\sqrt{2}} \pround{\ket{0}_A \otimes \ket{0}_B + \ket{1}_A \otimes \ket{1}_B} \, .
\]
If we let $\rho = \ket{\psi} \bra{\psi}$, but we can can only see system $A$, then we have $\rho_A = \frac{1}{2} I_A$. This has maximum ignorance, in the sense that it maximises Shannon entropy, but also that as an operator it doesn't even know the basis wrt which it is defined.  
\end{remark}
%%%%%%%%%%%%%%%%%%%%%%%%%%%%%%%%%%%%%%%%%%%%%%%%%%%%%%%%
%%%%%%%%%%%%%%%%%%%%%%%%%%%%%%%%%%%%%%%%%%%%%%%%%%%%%%%%
\subsection{How do we measure entanglement}
We will want two properties of the entanglement we measure
\begin{enumerate}
	\item quantum entanglement $\Rightarrow$ quantum correlation
	\item if the system is in a pure state of $\mc{H}_A \otimes \mc{H}_B$, then $S(\rho_A) = - \operatorname{Tr}_{\mc{H}_A} \rho_A \log \rho_A$
\end{enumerate}
If $\mc{H}$ has finite dimensions, then 
\[
\forall O_A, O_B , \quad I(A, B) \geq \frac{\pround{\pangle{O_A O_B} - \pangle{O_A} \pangle{O_B}}}{2 \abs{O_A}^2 \abs{O_B}^2}
\]
where $I$ is the mutual information 
\[
I(A,B) := S(A) -S(A | B) \, , 
\]
which, despite how its definition looks, is symemtric. \\
In QFT, for a free massive scalar with vacuum $\ket{\Omega}$, we recall 
\[
\pangle{\Omega \, | \, \phi(0,x) \phi(0,y) \, | \, \Omega} = \frac{1}{4 \pi^2} \frac{m}{\abs{x-y}} k_A (m\abs{x-y}) \, .
\]
We hence get behaviour 
\begin{enumerate}
	\item $\abs{x-y} \ll m^{-1} \Rightarrow $ decays as $\abs{x-y}^{-2}$, 
	\item $\abs{x-y} \ll m^{-2} \Rightarrow $ decays as $e^{-m\abs{x-y}}$.
\end{enumerate}
If $m=0$, the decay is $\abs{x-y}^{-2}$ all the way. \\
Comparing with our dicsussion of the discrete case, we can think of $\mc{H}_A$ as a region of spacetime where a local observer can measure $\phi(0,x)$ by restricting $x \in A, \, y \in B$, where $A \cap B = \emptyset$ and $A \cup B$ is our Cauchy slice we are doing QFT on. This would make it look like our entanglement entropy diverges, as we could make $\abs{x-y}$ arbitrarily small by take points close to the split of the Cauchy surface.  Hence UV behaviour introduces a UB cutoff $\xi_{UV}$. \\
In gauge theories, there exist constraints (think of the Gauss law) which hold locally. Hence the split $\mc{H} = \mc{H}_A \otimes \mc{H}_B$ must be compatible with these. One proposal of how to deal with this is just to handle the physical phase space, and not deal with a splitting, while another is to increase the Hilbert space by adding ``edge modes", which are degrees of freedom living on the boundary (which can be measured in the lab via the Quantum Hall Effect).  
\begin{remark}
	In the discrete (i.e. the spin) story, where $\mc{H} = \mc{H}_A \otimes \mc{H}_B$, the Von-Neumann entropy $S_A$ is invariant under local unitary transformations in the subsytem, i.e $\rho_A^\prime = U_A^\dagger \rho_A U_A $. In a f.d. system these transforms are square matrices, but in QFT one type is time evolution with Hamiltonian, which deforms the Cauchy surface either side of the boundary. 
\end{remark}

%%%%%%%%%%%%%%%%%%%%%%%%%%%%%%%%%%%%%%%%%%%%%%%%%%%%%%%%
%%%%%%%%%%%%%%%%%%%%%%%%%%%%%%%%%%%%%%%%%%%%%%%%%%%%%%%%
\subsection{Review of path integrals, states, and operators in QFT}
%%%%%%%%%%%%%%%%%%%%%%%%%%%%%%%%%%%%%%%%%%%%%%%%%%%%%%%%
\subsubsection{Transition Amplitudes and Ground State Wave Functions in QM}
In QM, let's recall the position basis and operators $\hat{X} \ket{x} = x \ket{x}$ with 
\[
I = \int \ket{x} \bra{x} \, dx \, .
\]
The transition amplitude between $x_i$ as $t_i$, $i=1,2$ is 
\begin{align*}
G(x_2, t_2; x_1, t_1) &= \pangle{x_2, t_2 \, | \, x_1, t_1} = \pangle{x_2 \, | \, e^{-iH(t_2-t_1)} \, | \, x_1} \, , \\
&= \int_{\bar{x}(t_1)=x_1}^{\bar{x}(t_2) = x_2} e^{iS[\bar{x}(t)]} \psquare{D \bar{x}(t)} \, , 
\end{align*}
using Feynmann's path integral formulation. \\
To write the vacuum wave function introduce $\sum_n \ket{n} \bra{n}$, $\ket{n}$ the eigenstates of $H$, from which we can write 
\[
G(x,t; x_0, t_0) = \sum_n e^{-iE_n(t-t_0)} \pangle{x \, | \, n} \pangle{n \, \ \, x_0} = \sum_n e^{-iE_n(t-t_0)}  \psi_n(x) \psi_n^\ast (x_0) \, .
\] 
We observe that taking Euclidean time $t_E = it$ and the limit $t_E \to \infty$, this of $G$ is dominated by the contribution of the vacuum where $E=E_0$. Hence 
\[
\lim_{t_E \to \infty} G(x,t; 0,0) - \pround{e^{-E_0 t_E}\psi_0^\ast(0)} \psi_0(x) \, ,
\]
up to normalisation. This means we can compute $\psi_0(x)$, by taking the limit in Euclidean time of a quantity $G$ that we are taught to compute with path integrals. \\
Moving now to QFT, introduce operator $\hat{\phi}(x)$ s.t. $\hat{\phi}(x) \ket{\phi} = \phi(x) \ket{\phi}$, where $\ket{\phi} = \bigotimes_{y \in \Sigma} \ket{\phi(y)}$, $\Sigma$ our Cauchy slice. We then write the vacuum 
\[
\ket{\Omega} = \int [D\phi] \ket{\phi} \pangle{\phi \, | \, \Omega} \, .
\]
Transition amplitudes in Euclidean time $\tau$ between $\phi_1 = \phi(\tau_1=0, x)$ to $\phi_2=\phi(\tau_2, x)$ are
\begin{align*}
	\pangle{\phi_2 \, | \, e^{-\tau_2 H} \, | \, \phi_1} = \int_{\phi(\tau_1) = \phi_1}^{\phi(\tau_2) = \phi_2} [D\phi] e^{-S_E[\phi]}
\end{align*}

%%%%%%%%%%%%%%%%%%%%%%%%%%%%%%%%%%%%%%%%%%%%%%%%%%%%%%%%
%%%%%%%%%%%%%%%%%%%%%%%%%%%%%%%%%%%%%%%%%%%%%%%%%%%%%%%%
%%%%%%%%%%%%%%%%%%%%%%%%%%%%%%%%%%%%%%%%%%%%%%%%%%%%%%%%
%%%%%%%%%%%%%%%%%%%%%%%%%%%%%%%%%%%%%%%%%%%%%%%%%%%%%%%%
\section{Path Integrals - Lecture 5}
Recall we want to be handling transition amplitudes of the form $\pangle{\phi_2 \, | \, e^{-\tau H} \, | \, \phi_1}$. If we split this conceptually as the inner product $\pangle{\phi_2 \, | \, \Psi}$ where
\begin{itemize}
	\item $\ket{\Psi} = \ket{\phi_1(\tau)} := e^{-\tau H} \ket{\phi_1}$,
\item we let $\Phi(\phi_2) = \pangle{\phi_2 \, | \, \Psi}$ be the wavefunction evaluated at field configuration $\phi_2$. 
\end{itemize}
We can then have a formal description of the ket $\ket{\Psi}$ as 
\[
\int_{\phi(\tau_1) = \phi_1}^{\phi(\tau_2) = ??} [D\phi] e^{-S_E[\phi]} \, , 
\]
where the upper limit is free. 
\begin{remark}
	We might wonder why we are using the Euclidean path integral? States defined at the 2nd cut, i.e. $\phi_2(\tau_2)=??$, remain states in the Lorentzian theory, and so we are fine to Wick rotate. 
\end{remark}

%%%%%%%%%%%%%%%%%%%%%%%%%%%%%%%%%%%%%%%%%%%%%%%%%%%%%%%%
%%%%%%%%%%%%%%%%%%%%%%%%%%%%%%%%%%%%%%%%%%%%%%%%%%%%%%%%
\subsection{Wavefunction of the Vacuum}
We now want to compute the wavefunction of the vacuum using our transition amplitudes. The difficulty here is that we have an infinitely long interval of Euclidean time. 
\begin{align*}
	\pangle{\phi \, | \, \Omega} = \frac{1}{\pangle{\Omega \, | \, \chi}} \lim_{T_E \to \infty} \pangle{\phi \, | \, e^{-T_E H} \, | \, \chi} \propto \int_{\phi(T_E \to -\infty)=0}^{\phi(T_E = 0) = \phi} [D\phi] e^{-S_E[\phi]} \, . 
\end{align*}
We can look at certain correlation functions. 
\[
\pangle{\Omega \, | \, \Omega} = \sum_\phi \pangle{\Omega \, | \, \phi} \pangle{\phi \, | \, \Omega} 
\]
Inserting this $\sum_\phi \ket{\phi} \bra{\phi}$ is equivalent to gluing the path integral at at the finite limit at configuration $\phi$. Note that the complex conjugation occurring in going from $\ket{\Omega}$ to $\bra{\Omega}$ switches the sign of Euclidean time (as $T_E =it$), so the $T_E \to -\infty$ limit becomes a $T_E \to \infty$ limit. This is a useful perspective, as it means that we can now write correlation functions like $\pangle{ \Omega \, | \, \mathcal{O}(T_{E_1}, x_1) \dots \mathcal{O}(T_{E_n}, x_n)}$ by inserting these operators at points in our path integral over $T_E \in \mathbb{R}$. 

%%%%%%%%%%%%%%%%%%%%%%%%%%%%%%%%%%%%%%%%%%%%%%%%%%%%%%%%
%%%%%%%%%%%%%%%%%%%%%%%%%%%%%%%%%%%%%%%%%%%%%%%%%%%%%%%%
\subsection{Density Matrices}
Recall our density matrix is $\rho = \ket{\Psi} \bra{\Psi}$. This is an operator, so object of the form $\pangle{\phi_2 \, | \, \rho \, | \, \phi_1}$ gives us complex numbers. Hence the path integral associated to $\rho$ is a path integral of finite length in $T_E$ evolving between two cuts. Contracting with states $\ket{\phi_1}$ is the process of fixing states on those cuts.
\begin{example}
If we took the example of $\rho_\beta = Z^{-1} e^{-\beta H}$, then our partition function is $Z(\beta) = \operatorname{Tr}(e^{-\beta H})$. This expression is basis independent, so we can express it in our local basis as 
\[
Z(\beta) = \sum_{\phi_1} \pangle{\phi_1 \, | \, e^{-\beta H} \, | \, \phi_1} \, . 
\]
Here, this sum then becomes a sum over tori corresponding to our path integral. If we were to now compute correlation functions at finite temperature of the form 
\[
\pangle{\mc{O}(x_1) \dots \mc{O}(x_n)}_\beta = \tr(\rho_\beta \mc{O}(x_1) \dots \mc{O}(x_n))
\]
we then are seeing a sum over $M_{1,n}$, the moduli space of genus 1 curves with $n$ fixed punctures. 
\end{example}

%%%%%%%%%%%%%%%%%%%%%%%%%%%%%%%%%%%%%%%%%%%%%%%%%%%%%%%%
%%%%%%%%%%%%%%%%%%%%%%%%%%%%%%%%%%%%%%%%%%%%%%%%%%%%%%%%
\subsection{Unruh's Effect / QFT in Rindler Spactime}
%%%%%%%%%%%%%%%%%%%%%%%%%%%%%%%%%%%%%%%%%%%%%%%%%%%%%%%%
\subsubsection{Special Relativity}
Let's go back to special relativity for a minute (working in $\mathbb{R}^{1,1}$). Recall that the relation between the acceleration of two intertial observers is 
\[
a_x = \frac{a_x^\prime}{\gamma^3 \pround{1 + \frac{v u_x^\prime}{c^2}}} \, , 
\]
where $\gamma = \psquare{1-(v/c)^2}^{-1/2}$. In the particular case of instantaneous rest frame where $u_x^\prime=0$, $a_x^\prime = \gamma^3 a_x$. \\
	As $a_x = \frac{dv}{dt}$, we have 
	\[
	\gamma^3 \frac{dv}{dt} = \frac{d}{dt} \pround{\gamma v} \, .
	\]
	Assuming $a_x =\alpha = a_x^\prime$ a constant, this is a differential equation that we can solve for $\alpha t = \gamma(v) v$. Integrating again with $v = \frac{dx}{dt}$, gives 
	\[
	x(t) = \frac{c^2}{\alpha} \sqrt{1+\frac{\alpha^2 t^2}{c^4}} \, .
	\]
	Trajectories as above are hyperbolic, and define what is called \bam{Rindler observer}. Drawing these as trajectories in $\mathbb{R}^{1,1}$ shows why these are hyperbolic, and they have asymptotes $x \pm t=0$. Light emitted by this observer can only live in the \bam{right Rindler wedge}, the region $x\geq \abs{t}$. For time $t>t_c$, $t_c$ some critical time, light will not reach our observer. This means we have a horizon, but this horizon is observer dependent. This is called a \bam{Rindler horizon}. 

%%%%%%%%%%%%%%%%%%%%%%%%%%%%%%%%%%%%%%%%%%%%%%%%%%%%%%%%
\subsubsection{General Relativity}
We now write down the spacetime metric percieved by a Rindler observer covering the Rindler wedge. We will use proper time $\tau$, which must be given by 
\[
d\tau = \frac{dt}{\gamma} \Rightarrow t = \frac{c}{\alpha} \sinh \frac{\alpha \tau}{c} \Rightarrow x(|tau) = \frac{c^2}{\alpha} \cosh \frac{\alpha \tau}{c} \, .
\]
We also introduce coordinates $\rho, \eta$ ($\eta=\tau$) related by 
\begin{align*}
	ct &= \rho \sinh \frac{\alpha \eta}{c} \, , \\ 
	x &= \rho \cosh \frac{\alpha \eta}{c} \, ,
\end{align*}
which maker the metric 
\[
ds^2 = -dt^2 +dx^2 = -\rho^2 \psquare{d\pround{\frac{\alpha \eta}{c}}}^2 + d\rho^2 = -\rho^2 d \eta^2 + d\rho^2 \, .
\]
This metric looks like our BH metric very close to the Killing horizon, which can be seen geometrically by drawing the Penrose diagram. \\
Let's again Wick rotate by setting $T_E = it$ and $\theta = i\eta$ s.t. 
\[
ds^2 = dT_E^2 + dx^2 = \rho^2 d\theta^2 + d\rho^2 \, .
\]
\begin{remark}
	In the current normalisation, if $\theta$ has the periodicity condition $\theta \sim \theta+2\pi$, this is the metric of the Euclidean plane, which is the Wick rotation of Minkowski, and so both Euclidean continuations are equivalent. Hence Euclidean observables in QFT are identical. 
\end{remark}
We want to ask whether this interpretation holds when we continue back to Lorentzian QFT. We will aim to show the following:
\begin{itemize}
	\item In Minkowski, Euclidean correlation functions give rise to correlation functions in $\ket{\Omega} = \ket{0}_M$. 
	\item in Rindler's right wedge, continuation computes correlation functions in $\ket{0}_M$ for observable restrict to the wedge. 
\end{itemize}
\begin{remark}
	Note what we are working towards here is the idea that the fact that this spacetime is split into a left and right part, so our Cauchy surface $\Sigma$($=t=0$) is also split into a L/R part, will end up meaning that the Hilbert space $\mc{H}_M$ will factorise in to the tensor product $\mc{H}_L \otimes \mc{H}_R$. The problem we will then run up against is that $\ket{0}_M \neq \ket{0}_L \otimes \ket{0}_R$. \\
	To spell this out, imagine we have a QFT of scalar field (as we don't want to deal with gauge questions) with arbitrary (i.e. possibly interacting) Lagrangian. Our Minkowski Hilbert space $\mc{H}_M$ will contain states of the form $\Psi[\phi(x)]$, and we will have Hamiltonian $H_M$ conjugate to time $t$. Write $\Psi_0$ for the wavefunction corresponding to the vacuum $\ket{\Omega} = \ket{0}_M$. \\
	On the right Rindler wedge, the wavefunctions will be of the form $\Psi[\phi_R]$ where $\phi_R(x) = \phi(x>0, t=0)$. The Hamiltonian here $H_R$ will be conjugate to the clock $\eta$. Now because the information of $\ev{\phi}{\Sigma} = \phi(x,t=0)$ is contained in $(\phi_L(x), \phi_R(x)$, we get the factorisation $\mc{H}_M = \mc{H}_L \otimes \mc{H}_R$. Note that it is because our Hamiltonians have been take as conjugate to different coordinates, despite the fact that the metrics are the same, that they will be different and so the vacua will not agree on the intersect of the Cauchy surfaces. Indeed if one does the algebra, the vector field $\del_\eta$ is a boost generator in $x,t$ coordinates. 
\end{remark}

%%%%%%%%%%%%%%%%%%%%%%%%%%%%%%%%%%%%%%%%%%%%%%%%%%%%%%%%
%%%%%%%%%%%%%%%%%%%%%%%%%%%%%%%%%%%%%%%%%%%%%%%%%%%%%%%%
%%%%%%%%%%%%%%%%%%%%%%%%%%%%%%%%%%%%%%%%%%%%%%%%%%%%%%%%
%%%%%%%%%%%%%%%%%%%%%%%%%%%%%%%%%%%%%%%%%%%%%%%%%%%%%%%%
\section{Vacuum States - Lecture 6}
Define 
\[
\Psi_0(\phi) := \Psi_0[\phi_L, \phi_R]
\]
Using our polar coordinates $\rho, \theta$, we can do our path integral now, which goes from $\theta=-\pi$ to $\theta=\pi$. 
\begin{remark}
	We should have 
	\[
	\pangle{\phi(x_2, t_2) \, | \, \phi(x_1, t_1)} = \pangle{\phi_2 \, | \, e^{-i(t_2-t_1)H} \, | \, \phi_1} \, , 
	\]
	which should suggest to us that 
	\[
	\Psi_0(\phi) \propto \pangle{\phi_R \, | \, e^{-i(-i\pi)H_R} \, | \, \phi_L } = \pangle{\phi_R \, | \, e^{-\pi H_R} \, | \, \phi_L} \, . 
	\]
\end{remark}
We can introduce a resolution of the identity $\pbrace{\ket{n}_R}$. so 
\[
\Psi_0(\phi(x)) \propto \sum_n e^{-\pi n} \chi_n (\phi_R) \chi_n^\ast (\phi_L)
\]
where $\chi_n(\phi_R) = \pangle{\phi_R \, | \, n}_R$, and likewise for $L$. Rewrite this as 
\[
\sum_n e^{-\pi n } \chi_n(\phi_R) \tilde{\chi}_n(\phi_L)
\]
where we interpret $\tilde{\chi}$ as a wave function in a Hilbert space where evolution flows in the opposite direction to the right wedge (as it does in the left wedge). Note our $\Psi_0(\phi(x))$ is really $\pangle{\phi_L, \phi_R \, | \, 0}_M$, so what we have is that 
\[
\ket{\Omega} = \ket{0}_M \propto \sum_n e^{-\pi n} \ket{n}_R \otimes \ket{n}_L
\]
%%%%%%%%%%%%%%%%%%%%%%%%%%%%%%%%%%%%%%%%%%%%%%%%%%%%%%%%
%%%%%%%%%%%%%%%%%%%%%%%%%%%%%%%%%%%%%%%%%%%%%%%%%%%%%%%%
\subsection{Direct Density Matrix Calculation (Second Round)}
Recall $\rho = \ket{\Omega} \bra{\Omega}$. Taking two states in the right Hilbert space $\phi_i^R$, $i=1,2$, we can calculate the reduced density matrix on the right as 
\begin{align*}
	\pangle{\phi_2^R \, | \, \rho_R \, | \, \phi_1^R} = \sum_{\tilde{\phi}} \pangle{\tilde{\phi}, \phi_2^R \, | \, \rho_R \, | \, \phi_1^R , \tilde{\phi}}
\end{align*}
We will take the foliation $ds^2 = d\rho^2 + \rho^2 d \theta^2$, which means time translation is 
\[
\frac{1}{\hbar} \comm[H_R]{\mc{O}} = \del_\theta \mc{O} \, , 
\]
and using our transition amplitude intuition we write 
\[
\pangle{\phi_2^R \, | \, \rho_R \, | \, \phi_1^R} \propto \pangle{\phi_2 \, | \, e^{-2\pi H_R} \, | \, \phi_1} \Rightarrow \rho_R = \frac{e^{-2\pi H_R}}{Z} \, .
\]
\begin{remark}
	\begin{itemize}
		\item Note $\rho_R$ appears as a thermal state with $T = \frac{1}{2\pi}$ with respect to $H_R$.
		\item Note also this is a non-perturbative derivation independent of the Lagrangian.
		\item $H_R \sim K_x$ a boost generator after wick rotation, where 
		\[
		K_x \propto X \del_T - T \del_X \overset{T \to \pm i T_E}{\to} X \del_{T_E} + {T_E} \del_X = \del_\theta
		\] 
		\item $\ket{\Omega} = \ket{0}_M$ is annihilated by $H_R-H_L = H_R \otimes I_L - I_R \otimes H_L$. The minus sign is related to time running backward on the left Rindler wedge. That is 
		\[
		e^{-i\eta (H_R - H_L)} \ket{\Omega} = \ket{\Omega}
		\]
		\item Note that because $ds^2 = -\rho^2 d \eta^2 + d \rho^2$, we can write $d\tau_{obs} = R_{obs} d \eta = \frac{1}{a} d \eta$, so it is $R_{obs}$ that contains units, as $\tau$ is dimensionful and $\eta$ is dimensionless. These give $\eta = a \tau_{obs}$. If we were to compute temperature wrt $\tau_{obs}$, we would get dimensionful temperature 
		\[
		T_{\text{Unruh}} = \frac{a}{2 \pi} \frac{\hbar}{c k_b} 
		\] 
	\end{itemize}
\end{remark}

%%%%%%%%%%%%%%%%%%%%%%%%%%%%%%%%%%%%%%%%%%%%%%%%%%%%%%%%
%%%%%%%%%%%%%%%%%%%%%%%%%%%%%%%%%%%%%%%%%%%%%%%%%%%%%%%%
\subsection{Proper Analytic Derivation (Third Round)}
We know 
\[
\pangle{\phi_L, \phi_R \, | \, \Omega} = \frac{1}{Z} \pangle{\phi_R \, | \, e^{-\pi K_R} \Theta \, | \, \phi_L}
\]
where 
\begin{itemize}
	\item $K_R$ is the restriction of the analytic continuation of a boost to the right wedge, 
	\item $\Theta$ is the generator of a CPT transformation in QFT, i.e 
	\[
	\Theta^\dagger \Phi(t,x, y) \Theta = \Phi^\dagger(-t, -x, y) \, ,
	\]
	so it is antiunitary and 
	\[
	\pangle{u \, | \, \Theta^\dagger \, | \, v} = \pangle{v \, | \, \Theta \, | \, u} \, .
	\]
	Note $\Theta$ is necessary as it maps $\mc{H}_{R/L} \to \mc{H}_{L/R}$. If our states were charged under some gauge field, $\Theta$ would have to act on those quantum numbers too. 
\end{itemize}
Now using our expression 
\[
\pangle{\phi_L, \phi_R \, | \, \Omega} = \frac{1}{Z} \sum_n e^{- \pi n} \pangle{n \, | \, \Theta \phi_L} \pangle{ \phi_R \, | \, n}_R \, , 
\]
and introducing $\ket{n^\ast}_L = \Theta^\dagger \ket{n}_R$ we get 
\[
\pangle{\phi_L, \phi_R \, | \, \Omega} = \frac{1}{Z} \sum_n e^{- \pi n} \pangle{\phi_R \, \ \, n}_R \pangle{\phi_L \, | \, n^\ast}_L \, , 
\]
and so we deduce
\[
\ket{\Omega} = \frac{1}{Z} \sum_n e^{- \pi n} \ket{n^\ast}_L \ket{n}_R \, . 
\]
\begin{remark}
	Consider now a free field and take the coordinate change from $\mathbb{R}^{1,1}$ 
	\begin{align*}
		x &= e^{\xi_R} \cosh(\eta_R)  = -e^{-\xi_L} \cosh(\eta_L)\\ 
		t &= e^{\xi_R} \sinh(\eta_R)  = e^{-\xi_L} \sinh(\eta_L)
	\end{align*}
The corresponding metrices in the two wedges are 
\[
e^{-2\xi_L}(-d\eta_L^2 + d\xi_L^2) = -dt^2 +dx^2 = e^{2\xi_R}(-d\eta_R^2 + d\xi_R^2) \, .
\]
We have boost generator $K_x = \del_{\eta_R}$ in the right wedge (as we have seen previously) but we can also check in the left wedge this is $-\del_{\eta_L}$. 
\end{remark}

\begin{remark}
We make some philosophical remarks on the relevance of entanglement wrt the the smoothness of the horizon.
\begin{itemize}
	\item Note that if the state is $\ket{0}_R$, the right Rindler observer sees no particles, precisely because this is the vacuum state in $\mc{H}_R$.
	\item While to a Rindler observer there is a horizon corresponding to the wedge, any inertial observer will not see these horizons, and eventually will cross out of the region. 
	\item $\ket{0}_R$ is singular at the horizon. More precisely, if an inertial observer is in $\ket{0}_R$, and you compute the stress tensor, the expectation $\pangle{0 \, | \, T_{\mu \nu} \, | \, 0}_R$ diverges at the horizon. To interpret this, go back to our open subsystem perspective where $\rho = \rho_L \otimes \rho_R$. Here there are no correlations between $L$ and $R$, and the two subsystems do not know about the existence of the other. Hence a right wedge observer does not know about the existence of the left wedge.  
\end{itemize}
\end{remark}
\begin{remark}
	Note we started with classical black holes, and we showed that there was a formal analogy to classical thermodynamics, giving a `temperature'. We then went on a discussion where $G_N=0$, and saw that for non-inertial observers we could get situations where they perceive the vacuum state of the whole system as excited from a subsystem perspective - this was Unruh's effect. We ask - what can we learn about the physical reality of $T_{BH}$ from Unruh's effect?
	\begin{itemize}
		\item Unruh's effect is a purely QFT effect, so it cannot apply directly. 
		\item For BHs we have two scenarios, maximal Kruskal extensions, and the formation of BHs by gravitational collapse, which are two different systems. For the first, we do recall that the Schwarzschild metric is, close to the horizon, similar to Rindler, and so our previous work will apply. For the latter, we do not have white holes, and so it is harder (and this is what Hawking discussed in his work).  
	\end{itemize}
\end{remark}

%%%%%%%%%%%%%%%%%%%%%%%%%%%%%%%%%%%%%%%%%%%%%%%%%%%%%%%%
%%%%%%%%%%%%%%%%%%%%%%%%%%%%%%%%%%%%%%%%%%%%%%%%%%%%%%%%
\subsection{QFT in Curved Backgrounds}
Let's lay the groundwork to answer these questions raised at the end of the last section. In order to turn of $G_N$ without removing the black hole, we let $G_N \to 0$ s.t. $r_S = GM$ is constant. We do this by letting the dimensionless quantity $\frac{M}{M_P} \to \infty$. This keeps the dynamics of quantum matter, but kills graviton interactions (while still keeping free gravitons) while preserving the geometry. \\
Recall that when we used the Euclidean path integral to compute $\ket{0}_M$, in order to have the coordinates $(T_E, x)$ and $(\theta, \xi)$ be equivalent, we had to have $ \theta \sim \theta + 2\pi$ (Hartle-Hawking precsription). If $\theta$ was non-compact we have a degree of freedom, and this changes the vacuum. 

%%%%%%%%%%%%%%%%%%%%%%%%%%%%%%%%%%%%%%%%%%%%%%%%%%%%%%%%
%%%%%%%%%%%%%%%%%%%%%%%%%%%%%%%%%%%%%%%%%%%%%%%%%%%%%%%%
%%%%%%%%%%%%%%%%%%%%%%%%%%%%%%%%%%%%%%%%%%%%%%%%%%%%%%%%
%%%%%%%%%%%%%%%%%%%%%%%%%%%%%%%%%%%%%%%%%%%%%%%%%%%%%%%%
\section{HH Vacuum - Lecture 6}
\begin{remark}
	For some references to cover what we have seen, look at 
	\begin{itemize}
		\item Unruh, Notes on BH evaporation, 1976
		\item Fulling, 1973
		\item Davoies, 1975
		\item Unruh, Bisogano, Wichmann, 1975, 1976
		\item Crispins, Itiguchi, Metsas, 0710.5373(?)
	\end{itemize}
\end{remark}

\begin{remark}
	To make concrete a remark from before, recall the Schwarzschild metric 
	\[
	ds^2 = -f(r)dt^2 + \frac{dr^2}{f(r)} + r^2 d \Omega^2 \, , 
	\]
	and as $r \to \infty, \, f(r) \to 1$, and $f(r) \sim \rho^2$ when $r-r_+ := \rho^2 \ll r_+$. Hence we really do see the canonical asymptotic clock at infinity is the same clock up to constant rescalings as the on in the right Rindler wedge. 
\end{remark}
If we perform canonical quantisation in the Schwarzschild BH, the corresponding vacuum $\ket{0}_R$ has $t_E$ non-compact. In (\hl{cite, Hartle Hawking 1976}), HH discovered a vacuum state $\ket{0}_{HH}$ corresponding to $t_E \sim t_E + \beta_{BH}$. Taking again $t=0$ as the Cauchy slice, we have a Euclidean preparation for $\ket{0}_{HH}$ on the lower half plane $\times S^2$. We can thus write our state as 
\[
\ket{0}_{HH} \propto \sum_i e^{-\beta_{BH} E_i/2} \ket{i^\ast}_L \otimes \ket{i}_R \, , 
\]
and so our asymptotic observer only sees the $\rho_R$, a thermal state. Hence $\beta_{BH}$ is physical. 

%%%%%%%%%%%%%%%%%%%%%%%%%%%%%%%%%%%%%%%%%%%%%%%%%%%%%%%%
%%%%%%%%%%%%%%%%%%%%%%%%%%%%%%%%%%%%%%%%%%%%%%%%%%%%%%%%
\subsection{Towards 1-sided}
So far we have been looking at the Penrose diagram with both sides, but we will want to see if we can restrict to the 1-sided case (\hl{which corresponds to BH formed by collapse rather than always being there?})

Let's have a perturbative discussion. We have a field $\phi$ satisfying 
\[
\nabla_\mu \nabla^\mu \phi = m^2 \phi = \frac{1}{\sqrt{-g}} \del_\mu \pround{\sqrt{-g} g^{\mu\nu} \del_\nu} \phi \, .
\]
Taking the ansatz
\[
\phi = \frac{1}{r} Y_{lm}(\Omega) f(r,t)
\]
with tortoise coordinate 
\[
r_\ast = r+2M \log\pround{\frac{r}{2M}-1} \, ,
\]
this gives 
\[
\del_t^2 f = \del_{r_\ast}^2 f + V_{eff}(r) = 0
\]
where 
\[
V_{eff}(r) = \pround{1- \frac{2M}{r}} \psquare{\frac{l(l+1)}{r^2} + \frac{2M}{r^3} + m^2 } \, .
\]
This is a Schr\"odinger-like equation, which has solutions 
\[
f(r,t) = e^{-iwt} \psi_{wl}(r) \, .
\]
We have the regimes 
\begin{enumerate}
	\item $r \to 2M, \, r_\ast \to -\infty$ where $V_{eff} \to 0$ and we get free solutions
	\item $r \to \infty, \, r_\ast \to \infty$, in which case 
	\[
	V_{eff} \to \left \lbrace \begin{array}{cc}
	\frac{l(l+1)}{r^2} & m^2=0 \\ m^2 \pround{1-\frac{2M}{r}} & m^2 \neq 0 
	\end{array} \right .
	\]
\end{enumerate}
Importantly $\exists r_{max}$ around which $V_{eff} \sim \rho^2$ where $\rho = r-r_{max}$. We picture that we are scattering off this potential $V_{eff}$. \\
Consider first where $m^2=0$. We expect to get 
\[
\phi \sim \left \lbrace \begin{array}{cc}
f_+(u) [\text{support } J^+] + g_+(v) [\text{BH horizon}] & t \to \infty \\
f_-(u) [\text{support WH}] + g_-(v) [J^-] & t \to -\infty 
\end{array}\right. 
\]
where $u,v$ are the coordinates $t \mp r_\ast$ respectively. \\
Looking at the asymptotic past, consider a single particle Hilbert space $\mc{H}_{in} = \mc{H}_{in, \infty} \oplus \mc{H}_{in, WH}$. We need coordinates to reference states in this space, and we will use $t$ on $\mc{H}_{in, \infty}$, but we have ambiguity on $\mc{H}_{in, WH}$ (for example, use $t$ or Kruskal). \\
We can do the same thing for asymptotic future, and write $\mc{H}_{out} = \mc{H}_{out, \infty} \oplus \mc{H}_{out, BH}$. \\ 
Now for our 1-sided case, part of our Hilbert space decomposition will not be there, we will only have 
\[
L^2 \pround{\mc{H}_{out, BH}} \otimes L^2 \pround{ \mc{H}_{out, BH}} \, . 
\]
\begin{remark}
	For references on this section, see 
	\begin{itemize}
		\item Hawking, ``Particle creation ...", 1976
		\item Wald, ``On particle creation by BHs", 1975
	\end{itemize}
\end{remark}
The problem we need to solve, is that the wave equation in Minkowski is glued with what happened earlier on, which is time dependent and could depend on details of the collapse. Our strategy will be as follows:
\begin{enumerate}
	\item use late times, because frequencies suffer from gravitational redshift, making excitations very massive. Hence one can use geometric optics approximation to the KG equation $\phi \sim \phi_0 e^{iS}$, and so the field is determined by null geodesics $\gamma$, 
	\item $\gamma$ goes through the $J^-$ at $v=v_1$, close to $v=v_0$ (the last inward ray on the surface of the collapsing body, i.e. in the WH case $v=v_0$ corresponds with the WH horizon).
	\item Use geodesic deviation to $\gamma$, s.t this deviation can be parameterise by $\epsilon n^a$ where $n^a$ is the tangent to an ingoing null geodesic at the horizon, hence $n^a l_a = -1$, where $l^a$ is the null geodesic generating the horizon. 
	\item Let $p^a = \frac{\del u }{\del \lambda} \del u$, tangent to the ingoing null geodesic. $p^a = A^2 n^a$, $\lambda$ is an affine parameter $B^2 U = -B^2 e^{-\kappa u}$.  
	\item solve the deviation equation, which connects the behaviour at the horizon ($\lambda=0$) with $\gamma$ ($\lambda < 0$), 
	\[
	\frac{dp^a}{d\lambda} = \frac{d^2 x^a}{d \lambda^2} \Rightarrow \lambda p^a = x^a(\lambda) - x^a(0) = - \epsilon n^a \Rightarrow \epsilon = -\lambda A^2 \, .
	\]
	\item Following $\gamma$ up to $J^-$, 
	\[
	v_0-v = \epsilon = -\lambda A^2 = C^2 e^{-\kappa u}
	\]
\end{enumerate}
The solution to the scattering problem according to this strategy is 
\[
\phi \sim \left \lbrace \begin{array}{cc}
e^{-iwu} & J^+ \\
e^{i (w/\kappa) \log(v_0-v)/C^2} & v < v_0
\end{array}\right .
\]
Doing this analysis properly using Fourier analysis, Wald shows 
\[
\forall f >0, \quad \tilde{\phi}(-f) = -e^{-\pi w/\kappa} \tilde{\phi}(f) \, ,
\]
and so for any positive frequencies in the wavefunction, the negative frequency also shows up. Hence there will be non-trivial Bogoliubov transformations. 

%%%%%%%%%%%%%%%%%%%%%%%%%%%%%%%%%%%%%%%%%%%%%%%%%%%%%%%%
\subsection{Summary/End-result/Outcomes}
Hawking measured $\frac{dE}{dt}$, the outward energy flux at $\infty$. We think of this as radiation, and he showed that 
\[
\frac{dE_{\text{rad}}}{dt} = \frac{w dw}{2 \pi} \frac{P_{\text{absorbtion}}(w, l)}{e^{\beta_{BH}w}-1} \, .
\]
This is independent of the details of the collapse. We can make an approximation to this by using that when $wr_s \ll 1$, $P_{abs}(w,l) \sim (wr_s)^2$ ($r_s = 2GM$) for low $l$, and as higher powers for higher $l$. Hence we can approximate 
\[
\frac{dE_{\text{tot}}{dt} \approx \int_{0}^{1/r_s} frac{w dw}{2\pi} \frac{(w r_s)^2}{\beta_{BH} w} \sim \frac{c}{\beta_{BH} r_s} = \frac{\tilde{c}}{r_s^2} \, .
\]
Taking $E=M$, $r_S \sim G_N M$, then 
\[
M^2 dM = - \frac{\tilde{D}}{G_N^2} dt \, ,
\]
and so $t_{\text{evaporation}} \sim G_N^2 M^3$ ($d=4$). This is long, but finite. \\
In the paper by Wald, he computed the reduced density matrix to be 
\[
\rho \propto \bigotimes_{w,l,m} \psquare{\sum_n P_{abs}(w,l) e^{-\beta w n}\ket{n}_{w,l,m}\bra{n}} \, .
\] 
Hawking's conclusion was that, if $G_N \neq 0$, the time evolution of a pure state photon ends up in a mixed state. This violates unitarity in quantum mechanics. 
\begin{remark}
	At early times, some radiation is asymptotic, so entanglement is growing while the BH is becoming small. Hawking's process always operates, so any state is in a space $(\text{BH}) \times (\text{Asymp Rad})$. These two systems must have the same entanglement entropy at all time, but a small BH must have small entropy because it has a small number of states, unless quantum gravity does something here. \\
	One possible way of solving the problems occuring is to believe that BHs give a `coarse-graining', and so $\exists$ subtle quantum correction that may save the day. That is, suppose your state is pure, we probe it with correlators 
	\[
	\pangle{\psi_i \, | \, \mc{O}_1(x_1) \dots \mc{O}_N(x_N) \, \ \, \psi_i} = \operatorname{Tr}(\rho_\beta \mc{O}_1(x_1) \dots \mc{O}_N(x_N)) + \mc{O}\pround{e^{-S_{BH} - \tilde{N}}} \, ,
	\]
	and if $N$ is not large enough, then one cannot distinguish $\psi_i$ from another close state, so if you have an operator complex enough to distinguish the BH microstates then it may back-react and change your Hilbert space.  
	
\end{remark}


%%%%%%%%%%%%%%%%%%%%%%%%%%%%%%%%%%%%%%%%%%%%%%%%%%%%%%%%
%%%%%%%%%%%%%%%%%%%%%%%%%%%%%%%%%%%%%%%%%%%%%%%%%%%%%%%%
%%%%%%%%%%%%%%%%%%%%%%%%%%%%%%%%%%%%%%%%%%%%%%%%%%%%%%%%
%%%%%%%%%%%%%%%%%%%%%%%%%%%%%%%%%%%%%%%%%%%%%%%%%%%%%%%%
\section{The Holographic Principle - Lecture 7}
\hl{write this in}

%%%%%%%%%%%%%%%%%%%%%%%%%%%%%%%%%%%%%%%%%%%%%%%%%%%%%%%%
%%%%%%%%%%%%%%%%%%%%%%%%%%%%%%%%%%%%%%%%%%%%%%%%%%%%%%%%
%%%%%%%%%%%%%%%%%%%%%%%%%%%%%%%%%%%%%%%%%%%%%%%%%%%%%%%%
%%%%%%%%%%%%%%%%%%%%%%%%%%%%%%%%%%%%%%%%%%%%%%%%%%%%%%%%
\section{Partition Functions in GR - Lecture 8}
This was first considered by Gibbons and Hawking in 1977, the paper ``Action Integrals and Partition Functions in Quantum Gravity". Our partition function will be, in Euclidean signature, 
\[
Z = \int [Dg][D\phi] e^{-S_E[g, \phi]} \, ,
\]
where 
\[
S_E[g, \phi] = \frac{1}{16 \pi G_N} \int \sqrt{g}(R + \dots) + S_{\text{matter}} \, . 
\]
We will take finite temperature in order to make this object more tractable, and so have 
\[
t_E \sim t_E + \beta \, , \text{ and } \lim_{r \to \infty} g_{tt} = 1 \, ,
\]
to have asymptotic flatness. We will take what is called \textbf{semi-classical gravity}, where $\hbar =1$ but $G_N \to 0$. This means gravity is classical, but matter is quantum. \\
Moreover, we assume that our Euclidean action has the form 
\[
S_E = \frac{1}{\kappa^2}\psquare{\dots} \, , 
\]
and so the partition function can be approximated by saddles, $\bar{g}_i, \, \bar{\phi}_i$. Hence 
\[
Z \approx \sum_i e^{-S_E[\bar{g}_i, \bar{\phi}_i]} \pround{1+ \text{loop}} \, .
\]
%%%%%%%%%%%%%%%%%%%%%%%%%%%%%%%%%%%%%%%%%%%%%%%%%%%%%%%%
\subsubsection{Flat Space Thermodynamics} 
At finite $T$, we know that there is the saddle (which for us will just mean a stationary point of the action, so a solution to the Einstein equations) given by the Euclidean BH. Hence 
\[
\log Z \approx -S_E[\bar{g}_{BH}] = S-\beta E
\]
where $S = (1-\beta \del_\beta) \log Z$ and $E = - \del_\beta \log Z$ (recall $Z=Z[\beta]$).  

%%%%%%%%%%%%%%%%%%%%%%%%%%%%%%%%%%%%%%%%%%%%%%%%%%%%%%%%
%%%%%%%%%%%%%%%%%%%%%%%%%%%%%%%%%%%%%%%%%%%%%%%%%%%%%%%%
\subsection{Evaluation Euclidean Action} 
Introduce cutoff $r_0$ which we let $ \to \infty$. Taking $G_N = 1$ for now we get 
\[
s_E = -\frac{1}{16 \pi } \int_M \sqrt{g}R - \frac{1}{8 \pi} \int_{\del M} \sqrt{h} K \, ,
\]
where the boundary corresponds to $r_0$, $h$ the emtric induced on $\del M$, and $K = h^{ij} K_{ij}$ and $K_{ij} = \frac{1}{2}\mc{L}_n h_{ij}$, $N$ the inward pointing unit normal to $\del_M$. \\
We will impose the boundary condition $\ev{\delta g}{\del M}=0$, which means that we don't get a boundary term when integrating by parts the bulk term. The saddle is, as we have written, 
\[
ds^2 = \pround{1-\frac{2M}{r}} d\tau^2 + \frac{dr^2}{1-\frac{2M}{r}} + r^2 d \Omega^2 \, ,
\]
where $\tau$ is periodic $\tau \sim \tau+\beta$, where for the Schwarzschild BH $\beta = 8 \pi M$. On shell here $R=0$ as it is a vacuum solution, and we know 
\[
\ev{ds^2}{r_0} = \pround{1-\frac{2M}{r_0}} d\tau^2 + r_0^2 d \Omega_2^2 \, ,
\]
so letting $f = 1-\frac{2M}{r_0}$ our boundary has normal $n = \sqrt{f}\del_r$. This means we can calculate 
\[
K_tt = \sqrt{f} \frac{M}{r_0^2} \, , \quad K_{\theta_i \theta_j} = \sqrt{f} r_0 \hat{h}_{\theta_i \theta_j} \, ,
\]
hence $K = \frac{M}{\sqrt{f} r_0^2} + \sqrt{f} \frac{2}{r_0}$ and $\sqrt{h} = \sqrt{f} r_0^2 \sqrt{\hat{h}}$. This all combines to mean 
\[
\int_{\del M} \sqrt{h} K = 4 \pi \beta \psquare{8 \pi r_0 - 12 \pi M} \, .
\] 
This expression diverges when $r_0 \to \infty$, as one might expect. Like in QFT therefore, we add a counter term 
\[
S_{ct} = \frac{1}{8 \pi } \int_{\del M} \sqrt{h} K_0 \, ,
\]
where $K_0$ is the extrinsic curvature of an on-shell metric sharing the same boundary metric on the BH. This background metric will be 
\[
ds^2_{\text{background}} = \pround{1-\frac{2M}{r_0}} d\tau^2 + dr^2 + r^2 d\Omega_2^2 \, ,
\]
which it is not too hard to see is flat and will have the same boundary metric. One can work out that for this backgroun $K_0 = \frac{2}{r_0}$, and so 
\[
S_{ct} = \frac{1}{8 \pi} \psquare{8 \pi \beta \sqrt{f} r_0} \, .
\]
\hl{check factors here}. This is designed s.t now we get no divergence of the GHY action, but it modifies the finite piece. \\
According to our saddle point approximation, we now have 
\[
\lim_{r_0 \to \infty} \log Z \approx -\frac{\beta M}{2} = - 4 \pi M^2 = -\frac{\beta^2}{16 \pi} \, .  
\]
This gives 
\[
Z[\beta] \approx e^{-\frac{\beta^2}{16 \pi}} \, . 
\]

%%%%%%%%%%%%%%%%%%%%%%%%%%%%%%%%%%%%%%%%%%%%%%%%%%%%%%%%
\subsubsection{Check Thermodynamic Quantities} 
If our partition function calculation is to be good then we should have entropy 
\[
S = \log Z + \frac{\beta^2}{4 \pi} = 4 \pi M^2 
\]
which is equal to our $S_{BH}$, good. Moreover the energy $E$ should be 
\[
E = -\del_\beta \log Z = M \, .
\]
again agreeing with $E_{BH}$. 

%%%%%%%%%%%%%%%%%%%%%%%%%%%%%%%%%%%%%%%%%%%%%%%%%%%%%%%%
\subsubsection{Comparison with Wald} 
Recall the 1st law said that in Lorentzian signature there is a horzion, and it is necessarily cloaking a singularity. Our Euclidean partition function said that in Euclidean signature there does not exist as horizon that's a smooth saddle. \\
To compare, note in our calculation there is a boundary term. Wald's story, a covariant formalism, had ambiguity $L \to L+dB$. However, both Euclidean and Lorentzian spacetime are time translation invariant, and with the inclusion of the boundary term Wald also finds that 
\[
-S_E = S_{BH} - \beta (M - \Omega_i J_i)
\]
where the RHS really is the $\log Z$, or indeed the free energy $-\beta F$. The problem arises because when we go to Euclidean time we compactify the time dimension.

%%%%%%%%%%%%%%%%%%%%%%%%%%%%%%%%%%%%%%%%%%%%%%%%%%%%%%%%
\subsubsection{Final Remarks} 
\begin{itemize}
	\item Note we have other saddles, for example Euclidean Minkowski (often thought of as a thermal gas). When one is doing this calculation with multiple saddle, we have multiple free energies, and so there may be regimes where different saddles dominate at finite temperature.
	\item Gross, Perry, Yalle, proved in ``instability of flat space at finite $T$" (1982) that hot flat space is unstable due to what is called Jeans' instability. Moreover, the Schwarzschild BH has a negative mode (responsible for specific heat $<0$) also giving instability.
	\item Atick and Witten in ``Hagedorn transition and number of degrees of freedom in string theory" (1988) showed that closed strings can wind around $t_E$, giving a new quantum number (corresponding to the winding number). Tachyon instabilities arise in the spectrum below some temperature. Hence closed strings know about the difference between Euclidean and Lorentzian time. 
\end{itemize}

%%%%%%%%%%%%%%%%%%%%%%%%%%%%%%%%%%%%%%%%%%%%%%%%%%%%%%%%
%%%%%%%%%%%%%%%%%%%%%%%%%%%%%%%%%%%%%%%%%%%%%%%%%%%%%%%%
\subsection{de Sitter}
Recall de Sitter the maximally symmetric spacetime with $\Lambda>0$ given by 
\[
R_{\mu \nu} = \frac{d-1}{l^2} g_{\mu \nu} \, ,
\]
and we take $\Lambda = \frac{(d-1)(d-2)}{2 l^2}$. This can be viewed as a hyperboloid in $\mathbb{R}^{1,d}$ given by 
\[
-x_0^2 + \sum_i x_i^2 = l^2
\]
and so we take coordinates $x_0 = l \sinh(T/l)$, $x_i = l w_i\cosh(T/l)$, where $w_i$ are coordinates on a unit sphere. Then 
\[
ds^2 = -dT^2 + l^2 \cosh^2(T/l) d \Omega_{d-1}^2 \, .
\]
Alternatively if we let $\tan(\eta/2) = \tanh(T/2l)$ then on the boundary where $T \to \infty$ we get 
\[
ds^2 = \frac{l^2}{\cos^2 \eta} \pround{-d\eta^2 + d\Omega_{d-1}^2} \, .
\] 
\begin{remark}
	Following from causality, interial observers cannot observe full de Sitter. 
\end{remark}
Isolating out one particular $\theta$ on the $S^{d-1}$ we can talk about its north and south pole, and then we call the causally connected patch to the south pole the \textbf{causal diamond}. This has coordinates 
\begin{align*}
	x_0 &= \sqrt{l^2 - r^2} \sinh(t/l) \, , \\
	x_1 &= \sqrt{l^2 - r^2} \cosh(t/l) \, , \\
	x_i = r z_i \, , \quad (2 \leq i \leq d)
\end{align*} 
and this gives metric 
\[
ds^2 = -\pround{1-\frac{r^2}{l^2}} dt^2 + \frac{dr^2}{1-\frac{r^2}{l^2}} + r^2 d \Omega_{d-2}^2 \, .
\]
$r=l$ is horizon seen just by the south pole observer. 
%%%%%%%%%%%%%%%%%%%%%%%%%%%%%%%%%%%%%%%%%%%%%%%%%%%%%%%%
\subsubsection{de Sitter thermodynamics}
Now take $r = l(1-\rho^2)$ for $\rho^2 \ll 1$. This gives 
\[
ds^2 = 4l^2 (d \rho^2 + \frac{\rho^2}{l^2} dt^2)
\]
We identigy $t \sim t + 2\pi i l$ and so the Euclidean time has $t_E \sim t_E + 2 \pi l$ giving a temperature $T_{dS} = \frac{1}{2 \pi l}$, and a horizon area \\
In $d=4$, taking the saddle point approximation we have 
\[
S_{dS} = \pi  l^2 
\]
Euclidean de Sitter is a sphere, so has no boundary. and so we get 
\[
S_E = -\frac{1}{16 \pi} \int \sqrt{g}(R - 2\Lambda) = - pi l^2 
\]
Recall we would want to see $-S_E = \log Z = - \beta F = S_{dS} - \beta E_{dS}$, and the lack of boundary means $E_{dS} = 0$, so this story agrees with thermodynamics. 

%%%%%%%%%%%%%%%%%%%%%%%%%%%%%%%%%%%%%%%%%%%%%%%%%%%%%%%%
%%%%%%%%%%%%%%%%%%%%%%%%%%%%%%%%%%%%%%%%%%%%%%%%%%%%%%%%
\subsection{Remarks}
Let's collate what we know about QG so far. 
\begin{itemize}
	\item Recall the holographic principle slightly rephrased: the number of d.o.f of QG scale like the number of d.o.f in QFT of one dimension less. 
	\item Weinberg-Witten theory, which prevents gravitons in a relativistic QFT with a covariantly conserved $T_{\mu\nu}$. 
\end{itemize}
Some possible conclusions are 
\begin{itemize}
	\item gravitons may live in one extra dimension, 
	\item the QM description capturing the physics of graviton should be strongly coupled. 
\end{itemize}
We can ask some questions. 
\begin{itemize}
	\item How can we test these ideas? We might look to large $N$ limits of gauge theories, which look like string theories. It will turn out that pairing this with the holographic principle with CFTs will force you onto AdS. 
	\item How could we show that $A=B$? We could go to a low energy limit (sometimes called the decoupling limit), where $A$ (resp $B$) becomes $A_1, \, A_2$ (resp $B_1, \, B_2$), and in this limit show $A_i = B_i$. 
\end{itemize}

%%%%%%%%%%%%%%%%%%%%%%%%%%%%%%%%%%%%%%%%%%%%%%%%%%%%%%%%
%%%%%%%%%%%%%%%%%%%%%%%%%%%%%%%%%%%%%%%%%%%%%%%%%%%%%%%%
%%%%%%%%%%%%%%%%%%%%%%%%%%%%%%%%%%%%%%%%%%%%%%%%%%%%%%%%
%%%%%%%%%%%%%%%%%%%%%%%%%%%%%%%%%%%%%%%%%%%%%%%%%%%%%%%%
\section{Large N 't Hooft limit - Lecture 9}
The references for the following section will be 
\begin{itemize}
	\item 't Hooft, 1974, ``a planar diagram theory for strong interactions"
	\item review 9802419
\end{itemize}
Consider the QCD Lagrangian 
\[
\mc{L}_{QCD} = -\frac{1}{g_{YM}^2} \psquare{-\frac{1}{4}\tr (F_{\mu\nu}F^{\mu\nu}) - i \bar{\psi}(\slashed{D}-m) \psi} \, .
\]
This is strongly coupled at low energies. The fields are $SU(3)$ valued, we hope that if we change that to $SU(N)$, we get $1/N$ perturbative parameter. 

%%%%%%%%%%%%%%%%%%%%%%%%%%%%%%%%%%%%%%%%%%%%%%%%%%%%%%%%
%%%%%%%%%%%%%%%%%%%%%%%%%%%%%%%%%%%%%%%%%%%%%%%%%%%%%%%%
\subsection{Toy Model}
Consider first 
\[
\mc{L} = - \frac{1}{g^2} \psquare{ \frac{1}{2} \del_\mu \phi \del^\mu \Phi + \frac{1}{4} \Phi^4}
\]
where $\Phi \Phi\indices{^a_b}$ and $N\times N$ Hermitian matrix. We have global $U(N)$ symmetry given by $\Phi \mapsto U \Phi U^\dagger$. \\
The Feynmann rules are that we have propagator from $ab$ to $cd$ 
\[
\pangle{\Phi\indices{^a_b}(x) \Phi\indices{^c_d}(y)} = g^2 \delta^a_d \delta^c_b G(x-0)
\]
and 4-point function of interactions $ab$, $cd$, $ef$, $gh$ given by 
\[
\frac{1}{g^2} \delta^a_h \delta^c_b \delta^e_d \delta^g_f \, .
\]
To simplify the combinatorics of these, 't Hooft introduced \textbf{double line} notation, where instead of 1 line for a propagator we draw 2, where the upper index of one goes to the lower of the other, and they are given arrows in the opposite directions. 
%%%%%%%%%%%%%%%%%%%%%%%%%%%%%%%%%%%%%%%%%%%%%%%%%%%%%%%%
\subsubsection{Vacuum Diagrams}
We can now try to consider vacuum diagrams with a single vertex. One can work out that we get two diagrams, giving contributions 
\begin{enumerate}
	\item $g^2 N^3$ (a vertex with external legs on the same side joined - this can be drawn in the plane)
	\item $g^2 N$ (a vertex with external legs on the opposite side joined - this cannot be drawn in the plane) (note this looks like the M\"obius band)
\end{enumerate}
The diagrams we can draw without intersections are called \textbf{planar}. \\
One can then consider 2-vertex diagrams, and now we get contributions 
\begin{enumerate}
	\item g^4 N^4 (planar)
	\item g^4 N^2 (non planar)
\end{enumerate} 
One might wonder whether we can construct general rules to understand the behvaiour of these diagrams at any order. 
\begin{remark}
	Both non-planar diagrams can be drawm on a torus without crossing lines (the diagram has been `straightened out'). The power of $N$ is the number of faces in each diagram after being straightened out. 
\end{remark}
In order to generalise this, we use the following facts.
\begin{itemize}
	\item Any orientable 2d surface is topologically classified by an integer $h$ (the genus) 
	\item $\forall$ non-planar diagram, $\exists h$ s.t. the diagram can be straightened out on a surface of genus $h$ ($h \in \mathbb{Z}_{\geq 0}$), but not on a surface of smaller genus. 
	\item For any diagram, the power of $N$ is given by a number of faces $F$.  
	\item We have Euler's formula $V-E+F = 2-2h := \chi$. One can derive this by thinking of the diagram as a triangulation of the surface. 
\end{itemize}
Using these we can say that for any diagram 
\[
\mc{A} \sim N^F (g^2)^{E-V}
\]
where $V$ is the number of vertices and $E$ is the number of edges. 
%%%%%%%%%%%%%%%%%%%%%%%%%%%%%%%%%%%%%%%%%%%%%%%%%%%%%%%%
\subsubsection{'t Hooft Limit}
To make this tractable, we might have Euler's formula in mind, so we choose to take $N \to \infty, \, g \to 0$ s.t. $\lambda = g^2 N $ is fixed. This is called the \textbf{'t Hooft coupling / 't Hooft limit}. Then we get 
\[
\mc{A} \sim \lambda^{E-V} N^{V-E+F} \, .
\] 
Notes $E-V = L-1$, where $L$ is the number of loops in the diagram, i.e. the number of undetermined momenta (this is as edges correspond to momenta, vertices impose a delta conservation, and there is an overall conservation in the diagram). Hence we have 
\[
\mc{A} \sim \lambda^{L-1} N^{2-2h} \, . 
\]
What we learn is that our answer for the total vacuum amplitude is 
\[
N^2 \psquare{c_0 + c_1 \lambda + \dots} + \mathcal{O}(N^{2-2h}) \, ,
\]
and non-perturbatively, in the large $N$ limit, 
\[
\log Z = \sum_{h = 0}^\infty N^{2-2h} f_h(\lambda) \, .
\]
In our toy model approach, we had leading order term in our partition function $\log Z \sim N^2$. Recall that if we did a path integral to calculate $Z$ and took a saddle approximation, we would expect to get leading contribution to $\log Z$ of $S_E$ at the saddle. Because we had $\mc{L} = -\frac{N}{\lambda} \tr\psquare{\dots}$, we should then expect to get $\tr[\dots] \sim \mc{O}(N)$. This is a heuristic argument, but it will hold more generally that if we have $\mc{L} = -\frac{N}{\lambda}\tr[\dots]$, and we have matrix degrees of freedom, then we should expect a partition function of the form 
\[
\log Z \sim \sum_h N^{2-2h} f_h \, . 
\] 
The expansion becomes an expansion in the topology of Feynmann diagrams. 

%%%%%%%%%%%%%%%%%%%%%%%%%%%%%%%%%%%%%%%%%%%%%%%%%%%%%%%%
%%%%%%%%%%%%%%%%%%%%%%%%%%%%%%%%%%%%%%%%%%%%%%%%%%%%%%%%
\subsection{General Observables}
In QCD, we have gauge transform (in some convention)
\[
A_\mu \mapsto UAU^\dagger -i(\del_\mu U)U^\dagger \, .
\]
$\Phi^2$ was a good observable in our toy model, but it will not be so in QCD, and in that sense gauge theories are more restrictive than our toy model. If we have a Lagrangian $\mc{L}[A, \psi] = -\frac{N}{\lambda_{YM}} \tr[\dots]$. We will restrict to local operatrors, i.e involving trace. 
\begin{itemize}
	\item single trace $\mc{O}_i$ ($i \in I$ a label), 
	\item multi-trace operators $\mc{O}_{i_1}(x_1) \cdots \mc{O}_{i_n}(x_n)$. 
\end{itemize}
If we want to compute observables $\pangle{\mc{O}_{i_1}(x_1) \cdots \mc{O}_{i_n}(x_n)}$, we want to know if there is an $N$ expansion, similar to how there was for our vacuum correlator, but now we may have external legs. \\
Recall the way we calculate these objects with a generator 
\[
Z[J_1, \dots, J_n] = \int [DA_\mu] [D\psi] e^{i S_0 + i \int J_i(x) \mc{O}_i(x)} \, ,
\]
which gives the correlator as 
\[
\pangle{\mc{O}_{i_1}(x_1) \cdots \mc{O}_{i_n}(x_n)} = \ev{\frac{\delta^n \log Z}{i^n \delta J_1 \dots \delta J_n}}{J_i = 0} \, .
\]
For us $S_0 \sim \frac{N}{\lambda} \tr[\dots]$, and if we scale the $J_i$ by $N$ we get $iS_{eff} = N \tr[\dots]$, which is the right $N$ scaling. Hence 
\[
\pangle{\mc{O}_{i_1}(x_1) \cdots \mc{O}_{i_n}(x_n)} = \ev{\frac{\delta^n \log Z}{i^n N^n \delta J_1 \dots \delta J_n}}{J_i = 0} = \sum_{h=0}^\infty N^{2-n-2h} F_n^h(\dots) \, . 
\]
\begin{remark}
	We there are no operators, the dominant scaling of the vacuum is $\pangle{I} \sim \mc{O}(N^2)$, and likewise the dominant contribution with $k$ operators is $\mc{O}(N^{2-k})$.  
\end{remark}

%%%%%%%%%%%%%%%%%%%%%%%%%%%%%%%%%%%%%%%%%%%%%%%%%%%%%%%%
\subsubsection{Physical Consequences}
\begin{remark}
	Single and multi-trace operators at large $N$ behave like $n$-particle states, in the sense that 
	\[
	\mc{O}_i(x)\ket{0} \sim \text{ single particle}
	\]
\end{remark}


%%%%%%%%%%%%%%%%%%%%%%%%%%%%%%%%%%%%%%%%%%%%%%%%%%%%%%%%
%%%%%%%%%%%%%%%%%%%%%%%%%%%%%%%%%%%%%%%%%%%%%%%%%%%%%%%%
%%%%%%%%%%%%%%%%%%%%%%%%%%%%%%%%%%%%%%%%%%%%%%%%%%%%%%%%
%%%%%%%%%%%%%%%%%%%%%%%%%%%%%%%%%%%%%%%%%%%%%%%%%%%%%%%%
\bibliographystyle{../../bib/custom-bib-style}
\bibliography{../../bib/jabref_library.bib}

\end{document}
